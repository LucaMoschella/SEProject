\documentclass[11pt]{report}            %impostiamo il tipo di documento
%\documentclass[11pt]{scrreprt}         %alternativi, ha margini più larghi e altre minori differenze


%IMPORT PACKAGE & INIZIALIZZAZIONI & PERSONALIZZAZIONI PACKAGE- l'ordine ha importanza
%Compatibili input, output e lingua italiana
\usepackage[T1]{fontenc}                    
\usepackage[utf8]{inputenc}                 
\usepackage[italian]{babel} 
\usepackage{textcomp}

%Miglioramenti visivi - sempre necessario
\usepackage{microtype}                      

%Riferimenti intelligenti
\usepackage[italian]{varioref}              

%Gestione liste avanzata e semplificata
\usepackage[inline]{enumitem}           

%spazio vericale invece di indentazione nei paragrafi
%\usepackage{parskip}                       

%Gestione matematica
\usepackage{amsmath, amsthm, amssymb}   

%gestione indice analitico
\usepackage{makeidx}                        
\makeindex

%Gestione inserimente di codice
\usepackage{listings}

%Gestione tabelle avanzata
\usepackage{booktabs}           
\usepackage{longtable}      
\usepackage{tabularx}
\usepackage{multirow}
\usepackage{multicol}

%gestione delle immagini e dei colori
\usepackage[dvipsnames]{xcolor}
\usepackage{soulutf8}   
\usepackage{graphicx}   
\usepackage{rotating}
\usepackage{pdflscape}

                    %Inserire le immagini  dim relative a \textwidth
                    %es: 0.8\textwidth
                
%Gestione figura con sottofigura
\usepackage{subfig}     

%Gestione avvolgimento figure dal testo
\usepackage{wrapfig}    

%Gestione didascalie avanzata
\usepackage{caption}
\captionsetup{tableposition=top,figureposition=bottom,font=small,format=hang,labelfont={sf,bf}}

%Gestione didascalie a lato di una figura
\usepackage{sidecap}

%Gestione date
\usepackage{datetime}

%Miglioramento quoting                      
\usepackage{quoting}
\quotingsetup{font=small}

%Permette i commenti
\usepackage{comment}

%Gestione grafici
%\usepackage{pgfplots}
%\pgfplotsset{/pgf/number format/use comma,compat=newest}

%Gestione unità di misura
\usepackage[output-decimal-marker={,}]{siunitx}
                    %Per inserire numeri va usato \num{...}
    
%Gestione bibliografia e citazioni
\usepackage[autostyle,italian=guillemets]{csquotes}
\usepackage[backend=biber,style=philosophy-modern]{biblatex}
\addbibresource{bibliografia/bibliografia.bib}  %estensione necessaria (unico caso!)

%Roba per il titolo (strumenti grafici) 
\usepackage{tikz}
\usepackage{epigraph}
 
%Gestione dei link
\usepackage{hyperref}   
\hypersetup{
%   bookmarks=true,         % show bookmarks bar?
%   unicode=false,          % non-Latin characters in Acrobat’s bookmarks
%   pdftoolbar=true,        % show Acrobat’s toolbar?
%   pdfmenubar=true,        % show Acrobat’s menu?
%   pdffitwindow=false,     % window fit to page when opened
%   pdfstartview={FitH},    % fits the width of the page to the window
%   pdftitle={My title},    % title
%   pdfauthor={Author},     % author
%   pdfsubject={Subject},   % subject of the document
%   pdfcreator={Creator},   % creator of the document
%   pdfproducer={Producer}, % producer of the document
%   pdfkeywords={keyword1, key2, key3}, % list of keywords
%   pdfnewwindow=true,      % links in new PDF window
    colorlinks=true,       % false: boxed links; true: colored links
    linkcolor=Blue,          % color of internal links (change box color with linkbordercolor)
    citecolor=green,        % color of links to bibliography
    filecolor=magenta,      % color of file links
    urlcolor=cyan           % color of external links
}                   

%Crea il glossario ed imposta lo stile - dopo hyperref
\usepackage[xindy,toc,acronym]{glossaries}  
\makeglossaries
\setglossarystyle{altlist}
%Glossario
\newglossaryentry{anteprima}
{
    name={Anteprima},
    description={è una vista ridotta di un Contenuto. A seconda del tipo di contenuto cambia ciò che viene visualizzato},
    plural=Anteprime
}


\newglossaryentry{contenuto}
{
    name={Contenuto},
    description={è qualsiasi oggetto inserito nel sistema dagli utilizzatori dello stesso}
}

\newglossaryentry{riferimento}
{
    name={Riferimento},
    description={è una porzione di testo, un immagine o un elemento composto che contiene un link ad un altro oggetto presente nel sistema}
}

\newglossaryentry{cmsdef}
{
    name={Content management system},
    description={è uno strumento software, installato su un server web, il cui
        compito è facilitare la gestione dei contenuti di siti web, svincolando il
        webmaster da conoscenze tecniche specifiche di programmazione Web}
}


\newglossaryentry{account}
{
    name={Account},
    description={individua un singolo profilo all'interno del sistema, si assume che ogni utilizzatore abbia un singolo account. Un account ha dei privilegi di base, che gli permettono di accedere ad alcune funzionalità; l'Amministratore può aggiungere o rimuovere privilegi da un account}
}

\newglossaryentry{visitatore}
{
    name={Visitatore},
    description={è una Figura Pubblica non autenticata nel portale}
}

\newglossaryentry{figuraPubblica}
{
    name={Figura Pubblica},
    description={è il fruitore dei servizi del portale: Visitatore, Utente oppure Produttore}
}

\newglossaryentry{figuraAmministrativa}
{
    name={Figure Amministrativa},
    description={è colui che utilizza il portale con lo scopo di fornire servizi alle Figure Pubbliche}
}

\newglossaryentry{figuraPubblicaAutenticata}
{
    name={Figura Pubblica Autenticata},
    description={è una Figura Pubblica che ha effettuato il login nel portale: Utente oppure Produttore}
}

\newglossaryentry{produttore}
{
    name={Produttore},
    description={è una Figura Pubblica Autenticata. Modella un negozio o produttore di dolci a base di cioccolata (sia a livello industriale sia a livello artigianale) che ha una propria vetrina per l'esposizione}
}

\newglossaryentry{utente}
{
    name={Utente},
    description={è una Figura Pubblica Autenticata. Modella una persona interessata ai dolci a base di cioccolata}
}

\newglossaryentry{redattore}
{
    name={Redattore},
    description={è una Figura Amministrativa addetta alla gestione delle notizie presenti nel portale}
}

\newglossaryentry{assistente}
{
    name={Assistente},
    description={è una Figura Amministrativa addetta alla comunicazione con le Figure Pubbliche}
}

\newglossaryentry{moderatore}
{
    name={Moderatore},
    description={è una Figura Amministrativa addetta al controllo dei contenuti inseriti dalle Figure Pubbliche, garantendo la conformità degli stessi al regolamento della piattaforma}
}

\newglossaryentry{amministratore}
{
    name={Amministratore},
    description={è una Figura Amministrativa addetta alla gestione del portale e delle sue figure}
}

\newglossaryentry{follower}
{
    name={Follower},
    description={di \emph{X}, sono le figure pubblice autenticate che ricevono aggiornamenti da \emph{X}}
}

\newglossaryentry{followed}
{
    name={Followed},
    description={di \emph{X}, sono le figure pubblice autenticate dalle quali \emph{X} riceve aggiornamenti}
}

\newglossaryentry{vetrina}
{
    name={Vetrina},
    description={è uno spazio privato di ogni Produttore dedicato all'esposizione dei suoi prodotti}
}

\newglossaryentry{valutazione}
{
    name={Valutazione},
    description={di un prodotto è un voto in una scala limitata per esprimere la bontà dello stesso}
}

\newglossaryentry{giudizio}
{
    name={Giudizio},
    description={di una recensione è un voto che può essere positivo o negativo}
}

\newglossaryentry{recensione}
{
    name={Recensione},
    description={di un prodotto è una valutazione (obbligatoria) a cui è associata una descrizione}
}

\newglossaryentry{prodotto}
{
    name={Prodotto},
    description={indica un dolce a base di cioccolata}
}

\newglossaryentry{notizia}
{
    name={Notizia},
    description={indica un articolo o una novità proveniente dal mondo dolciario inserita da un Redattore}
}

%Acronimi
\newacronym{cms}{CMS}{Content Management System}

\newacronym{uml}{UML}{Unified Modeling Language}

\newacronym{rup}{RUP}{Rational Unified Process}

\newacronym{rmmm}{RMMM}{Risk Mitigation, Monitoring, Management}

\newacronym{ucp}{UCP}{Use Case Points}

\newacronym{api}{API}{Application Programming Interface}

\newacronym{gui}{GUI}{Graphical User Interface}

\newacronym{uaw}{UAW}{Unadjusted Actor Weight}
\newacronym{uucw}{UUCW}{Unadjusted Use Case Weight}
\newacronym{tcf}{TCF}{Technical Complexity Factor}
\newacronym{tf}{TF}{Technical Factor}
\newacronym{ecf}{ECF}{Environmental Complexity Factor}
\newacronym{ef}{EF}{Environmental Factor}
\newacronym{ee}{EE}{Estimated Effort}

\newacronym{crc}{CRC}{Classe Responsabilità Collaborazione}
\newacronym{ecb}{ECB}{Entity Control Boundary}
 %importiamo il glossario, va usato con \gls{<label>}


%ULTERIORI PERSONALIZZAZIONI

%cambia Chapter in Documento
\addto\captionsitalian{\renewcommand{\chaptername}{Documento}}

%cambia Bibliografia in Opere consultate - al maledetto non piace il comando sopra -___-
\DefineBibliographyStrings{italian}{%
    bibliography = {Opere consultate},
}


%toglie lo spazio allungato dopo i punti di fine frase.
\frenchspacing 

%non stiracchia la pagina per riempirla
\raggedbottom

%profondità numerazione
\setcounter{secnumdepth}{3}



%toglie la parola capitolo ad ogni capitolo
%\titleformat{\chapter}{\normalfont\bfseries\huge}{\thechapter.}{15pt}{\normalfont\bfseries\huge} %toglie la scritta "Capitolo N
               %importiamo la il preambolo: package, settings inizializzazioni
%PERCHE' USARE LE MACRO
%1) Usando le macro possiamo stabili la formattazione "logica" del documento,
%   rendendo possibile cambiare le convenzioni adottate mantenendo la coerenza
%2) Definire comandi abbreviati per semplificare comandi con opzioni complese ripetute più volte
%3) Senza esagerare, definire testo da inserire in automatico senza riscriverlo ogni volta


%%%%%%%%%%%%%%%%%%%%%%%%%%%%%%%%%%%%%%%%%%%%%%%%%%%%%%%%%%%%%%%%%%%%%%%%%%%%%%%%%%%%%%%%%%%%%%
%%%%%%%%%%%%%%%%%%%%%%%%%%%%%%%%%%%% FORMATTAZIONE LOGICA %%%%%%%%%%%%%%%%%%%%%%%%%%%%%%%%%%%%
%%%%%%%%%%%%%%%%%%%%%%%%%%%%%%%%%%%%%%%%%%%%%%%%%%%%%%%%%%%%%%%%%%%%%%%%%%%%%%%%%%%%%%%%%%%%%%

%Il documento indicato è quello in cui sono state definite per la prima volta le macro, poi riutilizzate anche in altri documenti. 
%(Un minimo di organizzazione senno impazzisco)

%%%%%%%%%%%%%%%%%%%%%%%%%%%%%%%%%%%% documento di richiesta %%%%%%%%%%%%%%%%%%%%%%%%%%%%%%%%%%%%

%%%%%%%%%%%%%%%%%%%%%%%%%%%%%%%%%%%% documento di visione %%%%%%%%%%%%%%%%%%%%%%%%%%%%%%%%%%%%

%ruolo nel sistema visitatori, etc. (visione, specifica sistema)
\newcommand{\ruolo}[1]{\emph{#1}} 

%autenticate o non autenticate (documenti di visione)
\newcommand{\stato}[1]{\emph{#1}} 

%intestazioni nelle tabelle generiche
\newcommand{\tabhead}[1]{\textsc{#1}} 



%%%%%%%%%%%%%%%%%%%%%%%%%%%%%%%%%%%% documento di fattibilità %%%%%%%%%%%%%%%%%%%%%%%%%%%%%%%%%%%%

%tecnologie necessarie
\newcommand{\tecnologia}[1]{\emph{#1}} 



%%%%%%%%%%%%%%%%%%%%%%%%%%%%%%%%%%%% documento di contratto %%%%%%%%%%%%%%%%%%%%%%%%%%%%%%%%%%%%

%matricola (e titolo)
\newcommand{\matricola}[1]{\textit{#1}}

%email (e titolo)
\newcommand{\email}[1]{\texttt{#1}}



%%%%%%%%%%%%%%%%%%%%%%%%%%%%%%%%%%%% documento di pianificazione %%%%%%%%%%%%%%%%%%%%%%%%%%%%%%%%%%%%

%fase di sviluppo (inception, etc.)
\newcommand{\fase}[1]{\textsf{#1}} 

%workflow di sviluppo (raccolta requisiti, etc.)
\newcommand{\workflow}[1]{\emph{#1}} 

%larghezza tabelle iterazione
\newcommand{\tabwidthiter}{\textwidth} 

%titolo tabelle iterazioni 
\newcommand{\tabtitleiter}[1]{\textbf{#1}} 

%campo tabelle iterazioni 
\newcommand{\tabheaditer}[1]{\textsc{#1}} 

%spazio fra tabelle iterazioni 
\newcommand{\tabitervspace}{\bigskip} 


%%%%%%%%%%%%%%%%%%%%%%%%%%%%%%%%%%%% documento di specifica sistema %%%%%%%%%%%%%%%%%%%%%%%%%%%%%%%%%%%%

%gruppi di ruoli (figure pubbliche, etc.)
\newcommand{\grupporuolo}[1]{\textsl{#1}} 

%campo di informazione nel prodotto
\newcommand{\infoProdotto}[1]{\emph{#1}} 



%%%%%%%%%%%%%%%%%%%%%%%%%%%%%%%%%%%% documento di specifica requisiti %%%%%%%%%%%%%%%%%%%%%%%%%%%%%%%%%%%%

%descrizione campi delle tabelle dei requisiti
\newcommand{\campiTabReq}[1]{\emph{#1}} 

%descrizione campi degli identificatori dei requisiti
\newcommand{\campiIdReq}[1]{\emph{#1}} 

%formattazione requisiti - non mette link
\newcommand{\reqsep}{.} 
\newcommand{\displayReq}[1]{\texttt{#1}} 
\newcommand{\req}[4]{\displayReq{#1\reqsep#2\reqsep#3\reqsep#4}} 
\newcommand{\subreq}[5]{\displayReq{#1\reqsep#2\reqsep#3\reqsep#4\reqsep#5}} 
\newcommand{\subsubreq}[6]{\displayReq{#1\reqsep#2\reqsep#3\reqsep#4\reqsep#5\reqsep#6}} 

%Creazione dei contatori per la gestione automatica dei requisiti
\newcounter{req}
\newcounter{sottoreq}[req]
\newcounter{sottosottoreq}[sottoreq]

%Funzione per resettare i contatori al cambio di categoria
\newcommand{\countReset}{%
\setcounter{req}{0}%
\setcounter{sottoreq}{0}%
\setcounter{sottosottoreq}{0}%
}


%Macro per linkare requisiti
%separatore id titolo
\newcommand{\sepIDTitle}{ - }

%macro per creare un nuovo requisito, usando i vari contatori. Necessita della crezione di una tabella per descriverla
%usage: \reqItem{reqkey}{\ereq{ ...ID... }}{titolo} 
\newcommand\reqItem[3]{\hyperref[desc:#1]{#2\phantomsection\label{id:#1}}\sepIDTitle\reqtit{#3}\label{title:#1}}

%macro per stampare (e linkare, tranne il titolo) informazioni del req.
%usage: \reqGet..{reqkey}
\newcommand\reqGetID[1]{\ref{id:#1}}
\newcommand\reqGetIDtodesc[1]{\hyperref[desc:#1]{\ref*{id:#1}}}
\newcommand\reqGetTitle[1]{\ref*{title:#1}}
\newcommand\reqGetIDTitle[1]{\reqGetID{#1}\sepIDTitle\reqGetTitle{#1}}

%fa ma in modo che la label attuale contenga il titolo (e non il numero di sezione)
\makeatletter
\newcommand{\reqtit}[1]{%
    \def\@currentlabel{#1}%
    #1%
}
\makeatother

%macro per gestire i requisiti, e sotto requisiti vari.
%Oltre a formattare il requisito con \req (è in pratica un wrapper di req), 
%fa in modo che anche la label contenga la giusta formattazione del requisito.
\makeatletter
\newcommand{\ereq}[3]{%
    \stepcounter{req}%
    \def\@currentlabel{\req{#1}{#2}{#3}{\arabic{req}}}%
    \req{#1}{#2}{#3}{\arabic{req}}%
}
\makeatother

\makeatletter
\newcommand{\esubreq}[3]{%
    \stepcounter{sottoreq}%
    \def\@currentlabel{\subreq{#1}{#2}{#3}{\arabic{req}}{\arabic{sottoreq}}}%
	\subreq{#1}{#2}{#3}{\arabic{req}}{\arabic{sottoreq}}%
}
\makeatother

\makeatletter
\newcommand{\esubsubreq}[3]{%
    \stepcounter{sottosottoreq}%
    \def\@currentlabel{\subsubreq{#1}{#2}{#3}{\arabic{req}}{\arabic{sottoreq}}{\arabic{sottosottoreq}}}%
	\subsubreq{#1}{#2}{#3}{\arabic{req}}{\arabic{sottoreq}}{\arabic{sottosottoreq}}%
}
\makeatother

%deprecated, inutilmente complicato. (Principio di funzionamento: sovrascrive il modo di riferirsi a contatori. Penso.)
%Funzione per creare il prossimo requisito, creare una label a cui punterà la defnizione, linkare la definizione.
%Usage: \command{tipo}{contesto}{categoria}{keyrequisito}
%\newcommand*\ereq[4]{\def\thereq{#1.#2.#3.\arabic{req}}\refstepcounter{req}\hyperref[definizione#4]{\thereq}\phantomsection\label{elenco#4}} 
%\newcommand*\esubreq[4]{\def\thesottoreq{#1.#2.#3.\arabic{req}.\arabic{sottoreq}}\refstepcounter{sottoreq}\hyperref[definizione#4]{\thesottoreq}\phantomsection\label{elenco#4}} 
%\newcommand*\esubsubreq[4]{\def\thesottosottoreq{#1.#2.#3.\arabic{req}.\arabic{sottoreq}.\arabic{sottosottoreq}}\refstepcounter{sottosottoreq}\hyperref[definizione#4]{\thesottosottoreq}\phantomsection\label{elenco#4}} 

%Funzione per stampare il nome del requisito associato alla label: keyrequisito
%Linka l'elemento nell'elenco, e fa in modo di essere linkato dall'elenco
%\newcommand{\refElencoFromDef}[1]{\ref{elenco#1}\phantomsection\label{definizione#1}}

%Funzione per stampare il nome del requisito associato alla label: keyrequisito
%Linka l'elemento nell'elenco.
%\newcommand{\refElenco}[1]{\ref{elenco#1}}

%titolo tabelle req 
\newcommand{\tabtitlereq}[1]{#1} 

%campo tabelle req 
\newcommand{\tabheadreq}[1]{\textsc{#1}} 

%spazio fra tabelle requisiti 
\newcommand{\tabreqvspace}{\smallskip} 

%spazio fra tabelle requisiti 
\newcommand{\tabwidthreq}{\textwidth} 

%Tabella per descrizione requisiti funzionali 
%usage: {reqkey}{descrizione}{priorità}{stato}
\newcommand{\reqTabRF}[4]{\reqTab{#1}{#2}{Requisito funzionale}{#3}{#4}}

%Tabella per descrizione requisiti non funzionali
%usage: {reqkey}{descrizione}{priorità}{stato}
\newcommand{\reqTabRNF}[4]{\reqTab{#1}{#2}{Requisito non funzionale}{#3}{#4}}

%Tabella per descrizione requisiti
%usage: {reqkey}{descrizione}{tipo}{priorità}{stato}
\newcommand{\reqTab}[5]{%
	\begin{center}
		\begin{tabularx}{\tabwidthreq}{ l  X }
			\toprule
				\multicolumn{2}{c}{\phantomsection\label{desc:#1}\tabtitlereq{\reqGetIDTitle{#1}}}  \\
			\cmidrule(l{\cmidrulekern}r{\cmidrulekern}){1-2}
				\tabheadreq{ID} & \reqGetID{#1} \\ 
			\addlinespace[0.5em] 
				\tabheadreq{Titolo} & \reqGetTitle{#1} \\
			\addlinespace[0.5em]
				\tabheadreq{Descrizione} &  #2 \\		
			\addlinespace[0.5em]
				\tabheadreq{Tipo} &  #3 \\
			\addlinespace[0.5em]
				\tabheadreq{Priorità} &  #4 \\
			\addlinespace[0.5em]
				\tabheadreq{Stato} &  #5 \\
			\bottomrule
		\end{tabularx}
	\end{center}
}

\newcommand{\prioritaH}{Alta} 
\newcommand{\prioritaM}{Media} 
\newcommand{\prioritaL}{Media}

\newcommand{\statoOK}{Approvato} 
\newcommand{\statoNOK}{Non approvato} 
\newcommand{\statoUn}{Da approvare} 




%%%%%%%%%%%%%%%%%%%%%%%%%%%%%%%%%%%%%%%%%%%%%%%%%%%%%%%%%%%%%%%%%%%%%%%%%%%%%%%%%%%%%%%%%%
%%%%%%%%%%%%%%%%%%%%%%%%%%%%%%%%%%%% INDICE ANALITICO %%%%%%%%%%%%%%%%%%%%%%%%%%%%%%%%%%%%
%%%%%%%%%%%%%%%%%%%%%%%%%%%%%%%%%%%%%%%%%%%%%%%%%%%%%%%%%%%%%%%%%%%%%%%%%%%%%%%%%%%%%%%%%%

%indiicizzare e inserire contemporaneamente
\newcommand{\iindex}[1]{#1\index{#1}} 



%%%%%%%%%%%%%%%%%%%%%%%%%%%%%%%%%%%%%%%%%%%%%%%%%%%%%%%%%%%%%%%%%%%%%%%%%%%%%%%%%%%%%
%%%%%%%%%%%%%%%%%%%%%%%%%%%%%%%%%%%% RIFERIMENTI %%%%%%%%%%%%%%%%%%%%%%%%%%%%%%%%%%%%
%%%%%%%%%%%%%%%%%%%%%%%%%%%%%%%%%%%%%%%%%%%%%%%%%%%%%%%%%%%%%%%%%%%%%%%%%%%%%%%%%%%%%

%a documenti
\newcommand{\docref}[1]{\nameref{#1} \vpageref{#1}} 



%%%%%%%%%%%%%%%%%%%%%%%%%%%%%%%%%%%%%%%%%%%%%%%%%%%%%%%%%%%%%%%%%%%%%%%%%%%%%%%%%%%%
%%%%%%%%%%%%%%%%%%%%%%%%%%%%%%%%%%%% MATEMATICA %%%%%%%%%%%%%%%%%%%%%%%%%%%%%%%%%%%%
%%%%%%%%%%%%%%%%%%%%%%%%%%%%%%%%%%%%%%%%%%%%%%%%%%%%%%%%%%%%%%%%%%%%%%%%%%%%%%%%%%%%

%insiemi numerici
\newcommand{\numberset}{\mathbb}
\newcommand{\N}{\numberset{N}}
\newcommand{\R}{\numberset{R}}

%lettere strane
\renewcommand{\epsilon}{\varepsilon}
\renewcommand{\theta}{\vartheta}
\renewcommand{\rho}{\varrho}
\renewcommand{\phi}{\varphi}



%%%%%%%%%%%%%%%%%%%%%%%%%%%%%%%%%%%%%%%%%%%%%%%%%%%%%%%%%%%%%%%%%%%%%%%%%%%%%%%%%%%%%%%%%%%
%%%%%%%%%%%%%%%%%%%%%%%%%%%%%%%%%%%% TABELLE GENERICHE %%%%%%%%%%%%%%%%%%%%%%%%%%%%%%%%%%%%
%%%%%%%%%%%%%%%%%%%%%%%%%%%%%%%%%%%%%%%%%%%%%%%%%%%%%%%%%%%%%%%%%%%%%%%%%%%%%%%%%%%%%%%%%%%

%larghezza tabelle revisione - non usata (se con poco testo viene brutta)
\newcommand{\tabwidthrev}{0.8\textwidth} 



%%%%%%%%%%%%%%%%%%%%%%%%%%%%%%%%%%%%%%%%%%%%%%%%%%%%%%%%%%%%%%%%%%%%%%%%%%%%%%%%%%%%%%%%
%%%%%%%%%%%%%%%%%%%%%%%%%%%%%%%%%%%%  AMBIENTI %%%%%%%%%%%%%%%%%%%%%%%%%%%%%%%%%%%%%%%%%
%%%%%%%%%%%%%%%%%%%%%%%%%%%%%%%%%%%%%%%%%%%%%%%%%%%%%%%%%%%%%%%%%%%%%%%%%%%%%%%%%%%%%%%%

%toglie il bold dalle label delle descrizioni
\renewcommand{\descriptionlabel}[1]{\textsf{#1}}


%lista descrption dove c'è una leggera indentazione.
\newenvironment{descriptionInd}{\begin{description}[leftmargin=1.5cm,labelindent=1cm]}{\end{description}}

%lista descrption dove la descrizione va a capo.
\newenvironment{descriptionNext}{\begin{description}[style=nextline]}{\end{description}}
        %signatura: \newenvironment{name}[num]{before}{after}

%lista numerata inseribile dentro tabelle
\newenvironment{enumWork}
{
	\begin{minipage}[t]{\linewidth}
		\begin{enumerate*}[itemjoin={\newline}]
		}
		{
		\end{enumerate*}
	\end{minipage}
}

%lista itemize inseribile dentro tabelle
\newenvironment{itemWork}
{
	\begin{minipage}[t]{\linewidth}
		\begin{itemize}[nosep, leftmargin=*]
		}
		{
		\end{itemize}
	\end{minipage}
}                  %importiamo le macro personalizzate

%%%%%%%%%%%%    PER COMPILARE   %%%%%%%%%%%%%%%%%
%1) Compilare
%2) Eseguire i programmi per generare indice, glossario e bibliografia {sotto strumenti}
%3) Compilare DUE volte (per aggiornare i riferimenti)


%%%%%%%%%%%%         TIPS       %%%%%%%%%%%%%%%%%
%-1) Il libro "L'arte di scrivere in LaTeX" è ottimo (vedi fra fonti) :D
%0) Le voci del glossario (e acronimi) e della libreria vanno aggiunte a mano nei database. (vedi esempi)
%1) Per citare una fonte bibliografica: \cite{label}
%2) Per riferirsi un termine nel glossario: \gls{label}
%3) Per riferirsi ad un acronimo: \gls{label}   -> verrà espanso solo la prima volta
%4) Per inserire un termine nell'indice analitico: \index{entry}
%   Se lo si vuole anche scrivere in contemporanea si può usare la macro \iindex{entry}
%5) Vedere documentazione: biblatex, makeidx, glossaries per roba più fica
%6) Inserire una label in un punto (occhio all'ambiente in cui si è) con: \label{key}
%7) Ci si può riferire ad una label con: \ref{label} o \vref{label} <- l'ultimo mette la p. in modo intelligente
%9) Per inserire link, oltre a \url{URL} e \ref{label}, vedere funzionalità del package hyperref. (testo -> qualcosa)
%10) Per inserire una note a pié di pagina: \footcite{bibid}    - Forse meglio fare una macro e chiamare quella
%11) Per inserire una note a margine:   \marginpar{}            - Forse meglio fare una macro e chiamare quella
%12) Per le tabelle usare (solo) lo stile booktabs: \toprule, \midrule, \bottomrule, \cmidrule{2-3}
%13) Per inserire immgaini: \includegraphics[width=x\textwidth]{imagefile} dove x è il fatto di scala
%14) Preferire linserimento di immagini o tabelle fluttuanti (decide latex dove metterle) piuttosto che in testo
%   Quelle fluttuanti sono fighe prchè c'è la gestione figa della numerazione, nome e didascalia. E tramite la
%   preferenza di posizionamento (che sono opzionali) H lo si può comunque forzare
%15) Utilizzare anche l'ambiente quoting, viene bene
%16) enumitem ha opzioni interessani per personalizzare le liste. (La macro descriptionNext lo usa)
%17) Per i comandi di base LaTex: tasto LaTeX in alto
%18) La tilde è lo "spazio non separabile" da un sare in casi come: Dr. Rossi
%19) tab spazi e a capo valgono uno spazio. Una linea vuota indica un nuovo paragrafo (\par)
%20) Per tabelle più lunghe di una pagina usare l'ambiente longtable (e non tabular)
%21) \\* command disallows page breaking on the next row. Your next row is the \midrule so the page can be broken again after the \midrule. You could also try \\*\midrule\\*


%%%%%%%%%%%%         EXAMPLES       %%%%%%%%%%%%%%%%%
%1) Tabella fluttuante
%
%\dots qui finisce un capoverso.
%
%\begin{table}
%   \caption{. . .}  <- didascalia automatica
%   \label{tab:esempio}
%   \centering
%   \begin{tabular}{. . .}
%       ...
%   \end{tabular}
%\end{table}
%
%La tabella~\ref{tab:esempio} è un esempio di tabella mobile.
%
%%%%%
%
%2) Figura fluttuante
%
%\dots qui finisce un capoverso.
%
%\begin{figure}
%\centering   <- attentione, non center
%\includegraphics[width=0.5\textwidth]{...}
%\caption{. . .} <- didascalia automatica
%\label{fig:esempio}
%\end{figure}
%
%La figura~\ref{fig:esempio} è un esempio di figura mobile.
%
%%%%%
%
%3) Figura in testo (normale figura che "deve stare proprio lì" -> no didsascalia)
%
%\begin{center}
%   \includegraphics[width= 0.5\textwidth]{Rettili}
%\end{center}
%
%%%%%
%
%4) Tabella in testo (normale tabella)
%
%\begin{center}
%   \begin{tabular}{ll} <- SE CONTIENE PRINCIPALMENTE MATEMATICA: ARRAY, NON TABULAR (sempre racchiusa in ambiente matematico)
%       \toprule
%       Alcaloide & Origine \\
%       \midrule
%       atropina & belladonna \\
%       morfina & papavero \\
%       nicotina & tabacco \\
%       \bottomrule
%   \end{tabular}
%\end{center}


%\includeonly{documents/1.PropostaDiProgetto} %Per compilazioni parziali

\begin{document} %inizio stesura documento
    %\%title{\emph{Ingegneria del Software}\\ \textbf{{Chochok}}}
%\author{Valentino Maiorca \\ Luca Moschella \\  Alessio Quercia}
%\date{ \today }
%\maketitle
\renewcommand\epigraphflush{flushright}
\renewcommand\epigraphsize{\normalsize}
\setlength\epigraphwidth{0.7\textwidth}
\definecolor{titlepagecolor}{cmyk}{1,.60,0,.40}
\DeclareFixedFont{\titlefont}{T1}{ppl}{b}{it}{0.5in}
\makeatletter                        
\def\printauthor{%                  
    {\large \@author}}              
\makeatother
\author{%
    Valentino Maiorca \\
    {\footnotesize\matricola{ matricola 1595881}} \\
    \email{\scriptsize maiorca.1595881@studenti.uniroma1.it}\vspace{20pt} \\
    Luca Moschella \\
    {\footnotesize\matricola{ matricola 1594551}} \\
    \email{\scriptsize moschella.1594551@studenti.uniroma1.it}\vspace{20pt} \\
    Alessio Quercia \\
    {\footnotesize\matricola{ matricola 1596270}} \\
    \email{\scriptsize quercia.1596270@studenti.uniroma1.it}
}
\newcommand\titlepagedecoration{%
    \begin{tikzpicture}[remember picture,overlay,shorten >= -10pt]
    
    \coordinate (aux1) at ([yshift=-15pt]current page.north east);
    \coordinate (aux2) at ([yshift=-410pt]current page.north east);
    \coordinate (aux3) at ([xshift=-4.5cm]current page.north east);
    \coordinate (aux4) at ([yshift=-150pt]current page.north east);
    
    \begin{scope}[titlepagecolor!40,line width=12pt,rounded corners=12pt]
    \draw
    (aux1) -- coordinate (a)
    ++(225:5) --
    ++(-45:5.1) coordinate (b);
    \draw[shorten <= -10pt]
    (aux3) --
    (a) --
    (aux1);
    \draw[opacity=0.6,titlepagecolor,shorten <= -10pt]
    (b) --
    ++(225:2.2) --
    ++(-45:2.2);
    \end{scope}
    \draw[titlepagecolor,line width=8pt,rounded corners=8pt,shorten <= -10pt]
    (aux4) --
    ++(225:0.8) --
    ++(-45:0.8);
    \begin{scope}[titlepagecolor!70,line width=6pt,rounded corners=8pt]
    \draw[shorten <= -10pt]
    (aux2) --
    ++(225:3) coordinate[pos=0.45] (c) --
    ++(-45:3.1);
    \draw
    (aux2) --
    (c) --
    ++(135:2.5) --
    ++(45:2.5) --
    ++(-45:2.5) coordinate[pos=0.3] (d);   
    \draw 
    (d) -- +(45:1);
    \end{scope}
    \end{tikzpicture}%
}
\begin{titlepage}
    \noindent
    \titlefont Chochok\par
    \epigraph{Progetto di Ingegneria del Software}%
    {\textit{\today \hfill Università La Sapienza}\\ \textsc{Informatica}}
    \null\vfill
    \vspace*{1cm}
    \noindent
    \hfill
    \begin{minipage}{0.75\linewidth}
        \begin{flushright}
            \printauthor
        \end{flushright}
    \end{minipage}
    %
    \begin{minipage}{0.02\linewidth}
        \rule{1pt}{160pt}
    \end{minipage}
    \titlepagedecoration
\end{titlepage}
    \tableofcontents
%
    \chapter{Richiesta sistema}

L’associazione italiana amatori di cioccolato\footnote{Raggiungibile al seguente indirizzo: \url{http://www.chococlub.com/}} è intenzionata a lanciare un nuovo portale.
Si richiede di ideare un portale che soddisfi le seguenti necessità indispensabili:
\begin{itemize}
    \item Realizzazione di un catalogo generale di prodotti ( cioccolato e dolci a base di esso ).
    \item Aggiornamento del catalogo.
    \item Ricerca sul catalogo dei prodotti, anche avanzata, sulla base delle caratteristiche degli stessi.
    \item Sezione notizie provenienti dal mondo dolciario.
    \item Funzionalità social per interazione fra l'utenza del sistema, fra cui la possibilità di recensire prodotti e seguire gli aggiornamenti di altri utenti.
\end{itemize}
Volontà dell'associazione è ottenere un sistema di facile consultazione e aggiornamento.

Le utenze del portale saranno i produttori, i quali potranno esporre i propri prodotti in uno spazio a loro dedicato, e gli utenti, che potranno visualizzare e recensire i vari prodotti.

    \chapter{Documento di visione} 
\label{cha:documento_di_visione}

\section{Introduzione}
\label{sec:introduzione}
Dopo un'attenta analisi della richiesta dell'associazione ChocoClub, abbiamo stilato il seguente documento di visione, che ha lo scopo di introdurre le principali figure del sistema proposto, assieme alle loro necessità e studiare la fattibilità dello stesso.

\section{Descrizione figure} 
\label{sec:descrizionefigure}
Le figure che potranno utilizzare il portale si dividono in tre categorie:
\begin{itemize}
	\item \ruolo{Visitatori}
	\item \ruolo{Utenza}
	\item \ruolo{Figure amministrative}
\end{itemize}

\subsection{Visitatori} % (fold)
\label{sub:}
I visitatori, coloro che non sono stati identificati dal portale, potranno usufruire dei suoi servizi pubblici.
\begin{descriptionInd}
    \item[Visitatori] useranno il portale per ricercare prodotti e per leggere notizie presenti nel sito.
\end{descriptionInd}



% subsection subsection_name (end)
\subsection{Utenza}
\label{sub:utenza}
L'utenza è rappresentata da coloro che, dopo aver effettuato la procedura di autenticazione, utilizzerà il portale per usufruire dei suoi servizi. Essi si dividono in due profili differenti:
\begin{descriptionInd}
    \item[Produttori] useranno il portale per pubblicizzare i propri prodotti.

    \item[Utenti] useranno il portale per ricercare prodotti, con la possibilità di recensirli e/o valutarli; per seguire altra utenza e ricevere aggiornamenti da essa e per leggere notizie presenti nel sito.
\end{descriptionInd}




\subsection{Figure amministrative}
\label{sub:figureamministrative}
Le figure amministrative sono coloro che utilizzeranno il portale con lo scopo di fornire servizi all'utenza.
\begin{descriptionInd}
   % \item[Web Admin] avrà il compito di configurare e mantenere il portale, sia lato backend che frontend.
    \item[Redattori] useranno il portale per pubblicare notizie e articoli inerenti al mondo dolciario.   
    \item[Assistenti] useranno il portale per fornire supporto all'utenza.
    \item[Moderatori] useranno il portale per gestire segnalazioni di utenti.
\end{descriptionInd}


\section{Tabelle necessità o esigenze} % (fold)
\label{sec:tabelle_necessita_o_esigenze}
Nella seguente tabella riassumiamo le necessità o esigenze delle diverse figure:
\begin{center}
	\begin{tabularx}{0.8\textwidth}{l X}
	\toprule 
		\tabhead{Figura} & \tabhead{Necessità o Esigenze} \\
	\midrule
		\ruolo{Visitatori} & ricercare prodotti, visualizzare prodotti, visualizzare notizie, autenticarsi o iscriversi.  \\
		\addlinespace
		\ruolo{Produttori} & inserire prodotti nello spazio dedicato, rimuovere prodotti dallo spazio dedicato, modificare i prodotti inseriti.  \\
		\ruolo{Utenti} & Cercare prodotti, visualizzare prodotti, valutare/recensire prodotti, visualizzare notizie, seguire altra utenza. \\
		\addlinespace
		\ruolo{Redattori} & pubblicare notizie, modificare notizie, eliminare notizie.  \\
		\ruolo{Assistenti} & rispondere alle richieste dell`utenza  \\
		\ruolo{Moderatori} & eliminare contenuti inappropriati, modificare contenuti inappropriati.  \\
	\bottomrule
	\end{tabularx}
\end{center}


\section{Studio di fattiblità}
\label{sec:studio_di_fatt}

\newdate{visuno}{22}{04}{2016}
\section{Revisioni}
\begin{center}
    \begin{tabular}{lll}
        \toprule
        Versione & Data & Descrizione \\
        %   \cmidrule{2-3} 
        \midrule
        1.0 & \displaydate{visuno} & Prima versione \\
        \bottomrule
    \end{tabular}
\end{center}




    %%Date versioni
%Versione 1
\newdate{versioneuno}{22}{04}{2016}
\chapter{Proposta di progetto} 

\section{Introduzione}
L’associazione italiana amatori di cioccolato\footnote{Raggiungibile al seguente indirizzo: \url{http://www.chococlub.com/}} è intenzionata a lanciare un nuovo portale.
Le richieste per questo sistema sono le seguenti:
\begin{itemize}
	\item possibilità da parte dei produttori di inserire (o modificare, o rimuovere) i propri prodotti (sia cioccolato che dolci a base di esso) nel catalogo generale.
	\item ricerca sul catalogo dei prodotti, anche avanzata, sulla base delle caratteristiche degli stessi.
	\item sezione notizie.
	\item funzionalità stile social (recensioni di prodotti e sistema di follower/follo\-wed).
\end{itemize}
Volontà dell'associazione è ottenere un sistema di facile consultazione e aggiornamento.
Esso dovrà permettere agli utenti del servizio di ricercare e recensire prodotti come pure di consigliarne altri.
L'interazione tra utenti invece, è richiesta tramite un sistema di follower/followed che faccia da contorno al sistema principale, quindi non invasivo, ma che permetta ai fruitori del servizio (i follower) di ottenere informazioni riguardo attività da parte di chi seguono (i followed).
Nel sistema follower/followed chiunque può essere un follower (sia gli utenti semplici, sia i produttori) nonché followed.

È stato deciso di non utilizzare i contenuti attualmente presenti sul sito dell’associazione, per le seguenti motivazioni: il sito è statico e di difficile consultazione e analisi, oltretutto la maggior parte dei contenuti non è aggiornata. Inoltre, i dati attualmente contenuti non riguardano principalmente i prodotti, ma le pasticcerie come attività commerciali, quindi non sono utili per gli obiettivi del nuovo portale.
 
Per questo motivo è stato deciso di realizzare un portale ex novo, avvalendosi dell’utilizzo di un \gls{cms}.

Dopo un’attenta analisi della richiesta ed uno studio del settore, abbiamo elaborato la seguente proposta di realizzazione della piattaforma. Va fatto notare che il punto di forza del servizio saranno le informazioni sui prodotti esclusivamente di cioccolata, ma è stato deciso di includere anche la catalogazione di dolci che la contengono, per rivolgersi ad una più ampia fetta di mercato.

\section{Descrizione generale}
Come da richiesta, il portale permetterà agli utenti di reperire informazioni ad alta affidabilità riguardo la cioccolata ed i suoi produttori.
Tali informazioni copriranno ogni aspetto riguardante il prodotto, visto che saranno anche arricchite dagli utenti attraverso un sistema di recensioni e commenti, così da rendere completa la descrizione dei prodotti (includendo l'opinione dell'utente stesso).
Per i produttori, il portale sarà un mezzo di pubblicità in quanto avranno una vetrina personalizzata per esporre i propri prodotti e saranno essi stessi ad inserire tutte le specifiche dei loro prodotti, ovvero le informazioni che gli utenti potranno reperire.
Il portale avrà diversi tipi di ruoli amministrativi e di utenze; segue una breve descrizione delle stesse, in seguito approfondita. 
Le \ruolo{Utenze}:
\begin{descriptionInd}
	\item[Produttore] si potrà iscrivere al portale solo tramite account verificati da siti terzi o tramite approvazione privata, contattando direttamente il supporto. Avrà il compito di creare e mantenere aggiornata la sua vetrina. Sarà a lui delegata l’aggiunta di ogni suo prodotto, anche in seguito a richieste da parte degli utenti.
	
	\item[Utente] si potrà iscrivere al portale, anche tramite account di siti terzi. La loro funzionalità principale, oltre la ricerca dei prodotti, sarà la possibilità di recensire e/o valutare i prodotti stessi. 
\end{descriptionInd}
Le \ruolo{Figure di gestione del portale}:
\begin{descriptionInd}
	\item[Web Admin] avrà accesso completo al pannello di controllo del \gls{cms}
	
	\item[Newser] avrà, oltre alle funzionalità base dell’utente, il compito di gestire gli articoli e le notizie dal mondo dolciario.
	
	\item[Support] gestirà l’e-mail del portale rispondendo alle richieste di supporto e avrà il compito di approvare le richieste di iscrizione da parte dei produttori (tali richieste saranno inviate tramite form per una più facile gestione di esse).
	
	\item[Moderatore] avrà il compito di gestire segnalazioni di utenti eliminando/modificando eventuali contenuti non consoni presenti in recensioni.
\end{descriptionInd}

\noindent
Alcune funzionalità che potranno essere implementate in futuro sono: 
\begin{itemize}
	\item l’implementazione della \funfutura{wishlist} per ricordarsi dei prodotti che si vorrebbero provare.

	\item l’aggiunta del \funfutura{produttore delegato}, una figura necessaria per aiutare le piccole realtà dolciarie che si occuperà di gestire le vetrine al loro posto.

	\item Aggiunta dei commenti alle \funfutura{notizie}.
\end{itemize}

Per la fase iniziale di riempimento del portale, sarà compito di ChocoClub stessa prendere contatti con le molte pasticcerie artigianali già affiliate e le case produttrici, affinché esse si iscrivano al portale, per avere un buon numero di contenuti di partenza. 

\section{Descrizione funzionalità principali}
Di seguito è presente una panoramica generale di alcune delle principali funzionalità che saranno presenti nel portale, tale elenco potrà essere soggetto a modifiche nel corso dello sviluppo.
\begin{enumerate}
	\item \bloccofun{Iscrizione \& Autenticazione} \label{bloccofun:iscAut}
		\begin{enumerate}        	
			\item	\begin{descriptionNext}
						\item[Iscrizione tramite portale] Permette l’iscrizione direttamente tramite il portale. 
					\end{descriptionNext} \label{fun:isc.portale}
						                      
			\item 	\begin{descriptionNext}
						\item[Iscrizione tramite account terzi verificati] Si tratta della possibilità di iscriversi tramite account di altri servizi come LinkedIn, Facebook e Twitter, purché risultino “verificati” anche su quelle piattaforme.							
					\end{descriptionNext} \label{fun:isc.terziv}
					                       
			\item 	\begin{descriptionNext}
						\item[Iscrizione tramite account terzi non verificati]	È possibile utilizzare un account di siti terzi Google o Facebook per iscriversi							 
					\end{descriptionNext} \label{fun:isc.terzinv}
					                       
			\item 	\begin{descriptionNext}
						\item[Iscrizione tramite approvazione] Si tratta della possibilità per i produttori di iscriversi in modo verificato al portale, senza possedere un account già verificato su siti terzi.
						               
					\end{descriptionNext} \label{fun:isc.appr}
					                       
			\item 	\begin{descriptionNext}
						\item[Login] Permette di fare il login nel sistema utilizzando le credenziali con cui ci si è registrati.							
					\end{descriptionNext} \label{fun:aut.login}			
		\end{enumerate}
		
	\item \bloccofun{Gestione iscrizione}	\label{bloccofun:gestisc}		
		\begin{enumerate}        
			\item	\begin{descriptionNext}
						\item[Approvazione iscrizione] Si tratta della funzione relativa alla \funref{fun:isc.appr}, permette di approvare le richieste di iscrizione.
					\end{descriptionNext}	\label{fun:appr.iscr}
					                      
			\item	\begin{descriptionNext}
						\item[Rimozione account altrui] Permette di eliminare un account, ma NON le recensioni e valutazioni ad esso correlate nel caso di utenti e NON rimuove i prodotti nel caso di account produttore.
					\end{descriptionNext}	\label{fun:rim.iscr}
		\end{enumerate}
		
	\item \bloccofun{Gestione spazio dedicato alla presentazione prodotti (vetrina)}	\label{bloccofun:gestvet}		
		\begin{enumerate}        
			\item	\begin{descriptionNext}
						\item[Inserimento descrizione e immagine vetrina] Permette di avere a disposizione una vetrina pubblica, in cui ci sarà la descrizione del produttore e in cui verranno presentati i suoi prodotti. La funzionalità permette di personalizzare la vetrina attraverso un campo testuale (descrizione) ed un’immagine di sfondo.
					\end{descriptionNext}	\label{fun:ins.vetr}
					                      
			\item	\begin{descriptionNext}
						\item[Inserimento prodotto in vetrina] Permette l’aggiunta di un prodotto alla vetrina. Questa funzionalità segue un rigido protocollo per far sì che le informazioni inserite siano complete e coerenti. 
						Durante l’inserimento di un prodotto è necessario immettere: il \campo{nome del prodotto} con una sua \campo{descrizione} ed almeno una \campo{foto}. 
						È inoltre necessario compilare un modulo che contiene campi a \dominiocampo{scelta multipla} e campi con \dominiocampo{tag dinamici}, di cui a seguire una breve descrizione: il sistema metterà a disposizione una grande lista dei possibili tag, che saranno suggeriti come auto-completamento all’utilizzatore durante la compilazione del campo correlato.
						L’utilizzo di un tag non in lista è permesso, ma esso verrà sottoposto ad una fase di verifica che se superata comporterà l’aggiunta in lista, in caso contrario verrà rimosso anche dal prodotto. Sarà specificato un limite massimo e minimo di tag da inserire per prodotto.
						Nel modulo verrà chiesta la \campo{marca} (a scelta fra quelle associate al produttore che sta inserendo il prodotto), la \campo{categoria} del prodotto che si sta inserendo (a scelta fra \dominiocampo{Dolce al cioccolato} oppure \dominiocampo{cioccolato}), \campo{particolarità} (risposta con \dominiocampo{tag dinamici}),  la \campo{consistenza} (a scelta fra \dominiocampo{solida}, \dominiocampo{liquida} oppure \dominiocampo{spalmabile}), gli \campo{ingredienti} (risposta con \dominiocampo{tag dinamici}).
						Durante la compilazione del modulo di un prodotto è possibile compilare ulteriori campi (oltre a quelli proposti di default) scegliendo fra quelli messi a disposizione dal sistema, per migliorare la descrizione del prodotto e facilitarne la ricerca.
					\end{descriptionNext}	\label{fun:ins.prod}

			\item	\begin{descriptionNext}
						\item[Modifica prodotto in vetrina] Permette la modifica delle caratteristiche dei prodotti presenti, secondo gli stessi campi della \funref{fun:ins.prod}. Permette, inoltre, di etichettare un prodotto come fuori produzione.
					\end{descriptionNext} \label{fun:mod.login}

			\item	\begin{descriptionNext}
						\item[Visualizzazione statistiche vetrina privata] Mostra statistiche della vetrina, in rapporto al tempo; riassume le recensioni dei prodotti in essa presenti, in modo filtrabile e ordinabile in base a diversi criteri e permette di visualizzare il numero di visite ricevute.
					\end{descriptionNext} \label{fun:vis.vetrPr}	 	  				
		\end{enumerate}			
		
	\item \bloccofun{Ricerca} \label{bloccofun:ric}
		\begin{enumerate}
			\item	\begin{descriptionNext}
						\item[Ricerca prodotti tramite diversi criteri] Permette di effettuare ricerche, le ricerche potranno basarsi su qualsiasi campo della scheda del prodotto e dettagli del produttore. Sarà possibile ordinare i risultati.
					\end{descriptionNext}	\label{fun:ric.prod}

			\item	\begin{descriptionNext}
						\item[Ricerca profili pubblici] Permette di ricercare e in seguito visualizzare tramite l’apposita funzionalità, i profili pubblici di utenti e/o produttori.
					\end{descriptionNext}	\label{fun:ric.prof}

			\item	\begin{descriptionNext}
						\item[Ricerca notizie] Permette di effettuare ricerche sulle notizie pubblicate nel portale.
					\end{descriptionNext}	\label{fun:ric.not}
		\end{enumerate}

	\item \bloccofun{Gestione recensioni} \label{bloccofun:gestrec}
		\begin{enumerate}
			\item	\begin{descriptionNext}
						\item[Valutazione prodotto] Permette di esprimere una valutazione su scala limitata di un prodotto.
					\end{descriptionNext}	\label{fun:val.prod}

			\item	\begin{descriptionNext}
						\item[Recensione prodotto esistente] Permette di aggiungere, alla semplice \funref{fun:val.prod}, anche una descrizione testuale della recensione, più completa.
					\end{descriptionNext}	\label{fun:rece.prod}

			\item	\begin{descriptionNext}
						\item[Richiesta inserimento di un prodotto a un produttore]Permette di segnalare ad un produttore la mancanza di un suo prodotto tra quelli presenti nel sistema.
					\end{descriptionNext}	\label{fun:ric.prodotto}

			\item	\begin{descriptionNext}
						\item[Richiesta inserimento di un prodotto e del suo produttore]Permette di segnalare al personale incaricato l’assenza di un produttore ed allo stesso tempo di uno suo specifico prodotto.
					\end{descriptionNext}	\label{fun:ric.produttore}

			\item	\begin{descriptionNext}
						\item[Modifica recensione propria] Permette di modificare il campo testuale menzionato nella \funref{fun:rece.prod} della propria recensione.
					\end{descriptionNext}		\label{fun:mod.recPropria}

			\item	\begin{descriptionNext}
						\item[Modifica recensione altrui] Permette di modificare il campo testuale menzionato nella  \funref{fun:rece.prod} di una qualsiasi recensione.
					\end{descriptionNext}		\label{fun:mod.recAltrui}

			\item	\begin{descriptionNext}
						\item[Modifica valutazione] Permette di modificare la valutazione sia per recensioni che per semplici valutazioni.
					\end{descriptionNext}		\label{fun:mod.valPropia}

			\item	\begin{descriptionNext}
						\item[Valutazione recensione] Permette di esprimere una valutazione positiva o negativa sulle recensioni. Ciò permetterà di stabilire quali recensioni sono più utili.
					\end{descriptionNext}		\label{fun:val.rec}

			\item	\begin{descriptionNext}
						\item[Elimina recensione propria] Permette di eliminare la propria recensione. 
					\end{descriptionNext}		\label{fun:del.recPropia}

			\item	\begin{descriptionNext}
						\item[Elimina recensione altrui] Permette di eliminare una qualsiasi recensione.
					\end{descriptionNext}	 \label{fun:del.recAltrui}
					
			\item	\begin{descriptionNext}
						\item[Commento a recensione] Permette di commentare una qualsiasi recensione.
					\end{descriptionNext}	\label{fun:comm.rec}

			\item	\begin{descriptionNext}
						\item[Segnalazione contenuti] Permette di segnalare contenuti come recensioni e commenti alle stesse, se non consoni (secondo il regolamento del servizio). 
					\end{descriptionNext}	\label{fun:report.cont}
		\end{enumerate}

	\item \bloccofun{Gestione notizie} \label{bloccofun:gestnot}
		\begin{enumerate}
			\item	\begin{descriptionNext}
						\item[Consultazione notizie] Permette la visualizzazione delle notizie inserite nel portale, in base a preferenze dell’utente e impostazioni del gestore.
					\end{descriptionNext}	\label{fun:vis.not}

			\item	\begin{descriptionNext}
						\item[Inserisci notizia] Permette di aggiungere una notizia visibile tramite la \funref{fun:vis.not}.
					\end{descriptionNext}	\label{fun:ins.not}
					
			\item	\begin{descriptionNext}
						\item[Modifica notizia] Permette di modificare una notizia già esistente.
					\end{descriptionNext}	 \label{fun:mod.not}

			\item	\begin{descriptionNext}
						\item[Elimina notizia] Permette di eliminare una notizia già esistente.
					\end{descriptionNext}	\label{fun:del.not}
		\end{enumerate}

	\item \bloccofun{Gestione follower/followed} \label{bloccofun:gestfoll}
		\begin{enumerate}
			\item	\begin{descriptionNext}
						\item[Visualizzazione aggiornamenti followed] Mostra gli aggiornamenti e le notizie relativi ai propri followed.
						Questo sistema permetterà di ottenere una vista sulle attività dei followed, ovvero: nuova recensione, nuovo followed, nuova valutazione se si tratta di un utente; nuovo prodotto, prodotto fuori produzione se invece di tratta di un produttore.
					\end{descriptionNext}	\label{fun:vis.foll}

			\item	\begin{descriptionNext}
						\item[Follow account] Permette di seguire un account e ricevere da esso notizie e aggiornamenti. 
					\end{descriptionNext}	\label{fun:foll.acc}
					
			\item	\begin{descriptionNext}
						\item[Unfollow account] Permette di non seguire più un account che si segue. 
					\end{descriptionNext} \label{fun:unfoll.acc}
		\end{enumerate}

	\item \bloccofun{Gestioni suggerimenti} \label{bloccofun:gestsugg}
		\begin{enumerate}
			\item	\begin{descriptionNext}
						\item[Ti potrebbero piacere\dots] La funzionalità permette di ottenere suggerimenti ad hoc, essi potranno essere basati su diversi parametri: affinità con prodotti per i quali ha già espresso preferenza, informazioni presenti nel profilo (come il luogo di residenza per prodotti artigianali) e recensioni di utenti seguiti.
					\end{descriptionNext}	\label{fun:cons.prod}

			\item	\begin{descriptionNext}
						\item[Prodotti simili a\dots] Permette di suggerire prodotti simili a quello che si sta visualizzando sulla base di diversi criteri.
					\end{descriptionNext}	\label{fun:prod.sim}
					
			\item	\begin{descriptionNext}
						\item[Notizie simili a\dots] Permette di suggerire articoli e notizie sulla base dell’articolo che si sta attualmente leggendo.
					\end{descriptionNext}	\label{fun:not.sim}
		\end{enumerate}

	\item \bloccofun{Gestione account} \label{bloccofun:gestacc}
		\begin{enumerate}
			\item	\begin{descriptionNext}
						\item[Visualizzazione profilo] Permette di visualizzare il proprio profilo.
					\end{descriptionNext}	\label{fun:vis.prof}

			\item	\begin{descriptionNext}
						\item[Accesso alle impostazioni] Permette di accedere alle impostazioni dell’account. Le impostazioni variano a seconda del tipo di account. Per gli utenti, fra le altre impostazioni, c’è la possibilità di stabilire le preferenze di prodotti e/o produttori, per ricerche più mirate.
					\end{descriptionNext}	\label{fun:gest.imp}
					
			\item	\begin{descriptionNext}
						\item[Logout] Permette di fare il logout del sistema.
					\end{descriptionNext}	\label{fun:aut.logout}

			\item	\begin{descriptionNext}
						\item[Rimuovi account] Permette di eliminare il proprio account, ma NON le recensioni e valutazioni ad esso correlate nel caso di utenti e NON rimuove i prodotti nel caso di account produttore.
					\end{descriptionNext}	\label{fun:del.acc}
		\end{enumerate}			


	\item \bloccofun{Visualizzazione} \label{bloccofun:gestvis}
		\begin{enumerate}
			\item	\begin{descriptionNext}
						\item[Visualizza vetrina] Permette di visualizzare la vetrina di un dato produttore, ordinando i prodotti in base a diversi criteri.
					\end{descriptionNext}	\label{fun:vis.vet}

			\item	\begin{descriptionNext}
						\item[Visualizza statistiche vetrina pubbliche] Permette di visualizzare le statistiche pubbliche di una vetrina.
					\end{descriptionNext}	\label{fun:vis.statvetPubbliche}
					
			\item	\begin{descriptionNext}
						\item[Visualizza prodotto] Permette di visualizzare un dato prodotto e le sue informazioni.
					\end{descriptionNext}	\label{fun:vis.prod}

			\item	\begin{descriptionNext}
						\item[Visualizza recensioni] Permette di visualizzare le recensioni associate ad un prodotto, ordinandole in base a diversi criteri.
					\end{descriptionNext}	\label{fun:vis.recs}

			\item	\begin{descriptionNext}
						\item[Visualizza profilo altrui] Permette di visualizzare un dato profilo; si differenziano le informazioni fornite se il profilo è associato ad un produttore o ad un utente.
					\end{descriptionNext}	\label{fun:vis.profAltrui}

			\item	\begin{descriptionNext}
						\item[Visualizza statistiche profilo] Permette di visualizzare le statistiche di un dato profilo; si differenziano le informazioni fornite se il profilo è associato ad un produttore o ad un utente.
					\end{descriptionNext}	\label{fun:vis.profStat}		
		\end{enumerate}	
\end{enumerate}

\section{Utenze del portale}
Di seguite descriviamo brevemente le principali utenze che possono interagire con il portale.

\subsection{Utente non autenticato}
Un utente non autenticato può effettuare un’iscrizione o l’autenticazione (blocco funzionalità \funref{bloccofun:iscAut}), può effettuare ricerche (blocco funzionalità \funref{bloccofun:ric}), può consultare notizie (funzionalità descritta nel punto \funref{fun:vis.not}), può vedere prodotti e notizie simili a quelle che sta visualizzando (funzionalità descritte nei punti \funref{fun:prod.sim} e \funref{fun:not.sim}) e può inoltre visualizzare tutte le sezioni pubbliche del sito (blocco funzionalità \funref{bloccofun:gestvis}).

\subsection{Produttore autenticato}
Un produttore autenticato può gestire lo spazio dedicato alla presentazione prodotti (blocco funzionalità \funref{bloccofun:gestvet}), può effettuare ricerche (blocco funzionalità \funref{bloccofun:ric}), può segnalare contenuti e commentare recensioni (funzionalità descritte nei punti \funref{fun:comm.rec} e \funref{fun:report.cont}), può consultare notizie (funzionalità descritta nel punto \funref{fun:vis.not}), può usufruire del sistema di follower/followed (blocco funzionalità \funref{bloccofun:gestfoll}), può usufruire delle funzionalità di gestione account (blocco funzionalità \funref{bloccofun:gestacc}) e può inoltre visualizzare tutte le sezioni pubbliche del sito (blocco funzionalità \funref{bloccofun:gestvis}).

\subsection{Utente autenticato}
Un utente autenticato può effettuare ricerche (blocco funzionalità \funref{bloccofun:ric}), può effettuare qualsiasi operazione sulle recensioni tranne l’eliminazione e la modifica delle recensioni di altri utenti (blocco funzionalità \funref{bloccofun:gestrec} tranne le funzionalità \funref{fun:del.recAltrui} e \funref{fun:mod.recAltrui}),  può consultare notizie (funzionalità descritta nel punto \funref{fun:vis.not}), può usufruire del sistema di follower/followed  (blocco funzionalità \funref{bloccofun:gestfoll}), può usufruire del sistema di suggerimenti automatici  (blocco funzionalità \funref{bloccofun:gestsugg}), può usufruire delle funzionalità di gestione account (blocco funzionalità \funref{bloccofun:gestacc}) e può anche visualizzare tutte le sezioni pubbliche del sito  (blocco funzionalità \funref{bloccofun:gestvis}).

\section{Figure di gestione del portale}
Di seguite descriviamo brevemente le principali figure di gestione del portale.

\subsection{Web Admin}
Il WebAdmin ha accesso a tutte le funzionalità, sia quelle principali sopra descritte che quelle secondarie che verranno descritte in seguito.

\subsection{Newser}
Si tratta di un utente autenticato (quindi ne eredita tutte le funzionalità), ma ha in più la completa gestione delle news (blocco funzionalità \funref{bloccofun:gestnot}).

\subsection{Support}
Si tratta di un utente autenticato (quindi ne eredita tutte le funzionalità), ma ha in più la gestione delle iscrizioni con approvazione (blocco funzionalità \funref{bloccofun:gestisc}).

\subsection{Moderatore}
Si tratta di un utente autenticato (quindi ne eredita tutte le funzionalità), ma ha in più la gestione totale del sistema recensioni (blocco funzionalità \funref{bloccofun:gestrec}).

\section{Informazioni sui prodotti}
Ogni prodotto conterrà le seguenti informazioni:
\begin{itemize}[noitemsep]
	\item Nome prodotto. 
	\item Almeno una foto dell’articolo.
	\item  Descrizione di presentazione.
	\item  Marca.
	\item Categoria.
	\item Particolarità.
	\item Consistenza.
	\item Ingredienti.
	\item Altri campi (a seconda dei quali ha compilato il produttore).
\end{itemize}
Ad ogni prodotto saranno associate le sue recensioni con annessa valutazione e voto medio. Saranno inoltre suggeriti prodotti simili.

\section{Revisioni}
\begin{center}
	\begin{tabular}{lll}
		\toprule
		Versione & Data & Descrizione \\
		%	\cmidrule{2-3} 
		\midrule
		1.0 & \displaydate{versioneuno} & Prima versione \\
		\bottomrule
	\end{tabular}
\end{center}

%
%
%Stampa il glossario - anche quelli non usati
    \glsaddallunused
    \printglossaries
%
%Stampa la bibliografia - anche quella non citata
    \nocite{*}
    \cleardoublepage
    \phantomsection
    \addcontentsline{toc}{chapter}{\bibname}
    \printbibliography
%
%Stampa l'indice analitico - se presente
    \cleardoublepage
    \phantomsection
    \addcontentsline{toc}{chapter}{\indexname}
    \printindex %Maaaaa serve l'indice se abbiamo il glossario?
%
\end{document}  %fine stesura doumento
