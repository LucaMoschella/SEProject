\documentclass[11pt]{report}            %impostiamo il tipo di documento
%\documentclass[11pt]{scrreprt}         %alternativi, ha margini più larghi e altre minori differenze


%IMPORT PACKAGE & INIZIALIZZAZIONI & PERSONALIZZAZIONI PACKAGE- l'ordine ha importanza
%Compatibili input, output e lingua italiana
\usepackage[T1]{fontenc}  					
\usepackage[utf8]{inputenc}					
\usepackage[italian]{babel}					

%Miglioramenti visivi - sempre necessario
\usepackage{microtype} 						

%Riferimenti intelligenti
\usepackage[italian]{varioref} 				

%Gestione liste avanzata e semplificata
\usepackage[inline]{enumitem} 			

%spazio vericale invece di indentazione nei paragrafi
%\usepackage{parskip} 						

%Gestione matematica
\usepackage{amsmath, amsthm, amssymb} 	

%gestione indice analitico
\usepackage{makeidx}						
\makeindex

%Gestione inserimente di codice
\usepackage{listings}

%Gestione tabelle avanzata
\usepackage{booktabs}					
\usepackage{tabularx}
\usepackage{multicol}

%gestione delle immagini e dei colori
\usepackage{xcolor}
\usepackage{graphicx} 	
					%Inserire le immagini  dim relative a \textwidth
					%es: 0.8\textwidth
				
%Gestione figura con sottofigura
\usepackage{subfig}		

%Gestione avvolgimento figure dal testo
\usepackage{wrapfig} 	

%Gestione didascalie avanzata
\usepackage{caption}
\captionsetup{tableposition=top,figureposition=bottom,font=small,format=hang,labelfont={sf,bf}}

%Gestione didascalie a lato di una figura
\usepackage{sidecap}

%Gestione date
\usepackage{datetime}

%Miglioramento quoting						
\usepackage{quoting}
\quotingsetup{font=small}

%Permette i commenti
\usepackage{comment}

%Gestione grafici
%\usepackage{pgfplots}
%\pgfplotsset{/pgf/number format/use comma,compat=newest}

%Gestione unità di misura
\usepackage[output-decimal-marker={,}]{siunitx}
					%Per inserire numeri va usato \num{...}
	
%Gestione bibliografia e citazioni
\usepackage[autostyle,italian=guillemets]{csquotes}
\usepackage[backend=biber]{biblatex}
\addbibresource{bibliografia/bibliografia.bib}	%estensione necessaria (unico caso!)

%Gestione dei link
\usepackage{hyperref} 						

%Crea il glossario ed imposta lo stile - dopo hyperref
\usepackage[xindy,toc]{glossaries} 	
\makeglossaries
\setglossarystyle{altlist}
%Glossario
\newglossaryentry{anteprima}
{
    name={Anteprima},
    description={è una vista ridotta di un Contenuto. A seconda del tipo di contenuto cambia ciò che viene visualizzato},
    plural=Anteprime
}


\newglossaryentry{contenuto}
{
    name={Contenuto},
    description={è qualsiasi oggetto inserito nel sistema dagli utilizzatori dello stesso}
}

\newglossaryentry{riferimento}
{
    name={Riferimento},
    description={è una porzione di testo, un immagine o un elemento composto che contiene un link ad un altro oggetto presente nel sistema}
}

\newglossaryentry{cmsdef}
{
    name={Content management system},
    description={è uno strumento software, installato su un server web, il cui
        compito è facilitare la gestione dei contenuti di siti web, svincolando il
        webmaster da conoscenze tecniche specifiche di programmazione Web}
}


\newglossaryentry{account}
{
    name={Account},
    description={individua un singolo profilo all'interno del sistema, si assume che ogni utilizzatore abbia un singolo account. Un account ha dei privilegi di base, che gli permettono di accedere ad alcune funzionalità; l'Amministratore può aggiungere o rimuovere privilegi da un account}
}

\newglossaryentry{visitatore}
{
    name={Visitatore},
    description={è una Figura Pubblica non autenticata nel portale}
}

\newglossaryentry{figuraPubblica}
{
    name={Figura Pubblica},
    description={è il fruitore dei servizi del portale: Visitatore, Utente oppure Produttore}
}

\newglossaryentry{figuraAmministrativa}
{
    name={Figure Amministrativa},
    description={è colui che utilizza il portale con lo scopo di fornire servizi alle Figure Pubbliche}
}

\newglossaryentry{figuraPubblicaAutenticata}
{
    name={Figura Pubblica Autenticata},
    description={è una Figura Pubblica che ha effettuato il login nel portale: Utente oppure Produttore}
}

\newglossaryentry{produttore}
{
    name={Produttore},
    description={è una Figura Pubblica Autenticata. Modella un negozio o produttore di dolci a base di cioccolata (sia a livello industriale sia a livello artigianale) che ha una propria vetrina per l'esposizione}
}

\newglossaryentry{utente}
{
    name={Utente},
    description={è una Figura Pubblica Autenticata. Modella una persona interessata ai dolci a base di cioccolata}
}

\newglossaryentry{redattore}
{
    name={Redattore},
    description={è una Figura Amministrativa addetta alla gestione delle notizie presenti nel portale}
}

\newglossaryentry{assistente}
{
    name={Assistente},
    description={è una Figura Amministrativa addetta alla comunicazione con le Figure Pubbliche}
}

\newglossaryentry{moderatore}
{
    name={Moderatore},
    description={è una Figura Amministrativa addetta al controllo dei contenuti inseriti dalle Figure Pubbliche, garantendo la conformità degli stessi al regolamento della piattaforma}
}

\newglossaryentry{amministratore}
{
    name={Amministratore},
    description={è una Figura Amministrativa addetta alla gestione del portale e delle sue figure}
}

\newglossaryentry{follower}
{
    name={Follower},
    description={di \emph{X}, sono le figure pubblice autenticate che ricevono aggiornamenti da \emph{X}}
}

\newglossaryentry{followed}
{
    name={Followed},
    description={di \emph{X}, sono le figure pubblice autenticate dalle quali \emph{X} riceve aggiornamenti}
}

\newglossaryentry{vetrina}
{
    name={Vetrina},
    description={è uno spazio privato di ogni Produttore dedicato all'esposizione dei suoi prodotti}
}

\newglossaryentry{valutazione}
{
    name={Valutazione},
    description={di un prodotto è un voto in una scala limitata per esprimere la bontà dello stesso}
}

\newglossaryentry{giudizio}
{
    name={Giudizio},
    description={di una recensione è un voto che può essere positivo o negativo}
}

\newglossaryentry{recensione}
{
    name={Recensione},
    description={di un prodotto è una valutazione (obbligatoria) a cui è associata una descrizione}
}

\newglossaryentry{prodotto}
{
    name={Prodotto},
    description={indica un dolce a base di cioccolata}
}

\newglossaryentry{notizia}
{
    name={Notizia},
    description={indica un articolo o una novità proveniente dal mondo dolciario inserita da un Redattore}
}

%Acronimi
\newacronym{cms}{CMS}{Content Management System}

\newacronym{uml}{UML}{Unified Modeling Language}

\newacronym{rup}{RUP}{Rational Unified Process}

\newacronym{rmmm}{RMMM}{Risk Mitigation, Monitoring, Management}

\newacronym{ucp}{UCP}{Use Case Points}

\newacronym{api}{API}{Application Programming Interface}

\newacronym{gui}{GUI}{Graphical User Interface}

\newacronym{uaw}{UAW}{Unadjusted Actor Weight}
\newacronym{uucw}{UUCW}{Unadjusted Use Case Weight}
\newacronym{tcf}{TCF}{Technical Complexity Factor}
\newacronym{tf}{TF}{Technical Factor}
\newacronym{ecf}{ECF}{Environmental Complexity Factor}
\newacronym{ef}{EF}{Environmental Factor}
\newacronym{ee}{EE}{Estimated Effort}

\newacronym{crc}{CRC}{Classe Responsabilità Collaborazione}
\newacronym{ecb}{ECB}{Entity Control Boundary}

\newacronym{aam}{AAM}{Advanced Access Manager}
\newacronym{ure}{URE}{User Role Editor}
\newacronym{wpum}{WPUF}{WP User Manager}
\newacronym{wpuf}{WPUF}{WP User Fronted}
\newacronym{wc}{WC}{WooCommerce}
\newacronym{wcv}{WCV}{WC Vendors}
\newacronym{bp}{BP}{BuddyPress}
\newacronym{bpf}{BPF}{BuddyPress Follow}
\newacronym{wpsprts}{WPSPRTS}{WP Support Plus Responsive Ticket System}
\newacronym{wprp}{WPRP}{WP Report Post}
\newacronym{nf}{NF}{Ninja Forms}
\newacronym{pbp}{PBP}{Profile Builder Pro}
\newacronym{mrf}{MRF}{Multiple Registration Forms}
\newacronym{nfc}{NFC}{Nextend Facebook Connect}
	%importiamo il glossario, va usato con \gls{<label>}


%ULTERIORI PERSONALIZZAZIONI

%cambia Chapter in Documento
\addto\captionsitalian{\renewcommand{\chaptername}{Documento}}

%cambia Bibliografia in Opere consultate - al maledetto non piace il comando sopra -___-
\DefineBibliographyStrings{italian}{%
	bibliography = {Opere consultate},
}


%toglie lo spazio allungato dopo i punti di fine frase.
\frenchspacing 

%non stiracchia la pagina per riempirla
\raggedbottom

%toglie la parola capitolo ad ogni capitolo
%\titleformat{\chapter}{\normalfont\bfseries\huge}{\thechapter.}{15pt}{\normalfont\bfseries\huge} %toglie la scritta "Capitolo N
               %importiamo la il preambolo: package, settings inizializzazioni
%PERCHE' USARE LE MACRO
%1) Usando le macro possiamo stabili la formattazione "logica" del documento,
%		rendendo possibile cambiare le convenzioni adottate mantenendo la coerenza
%2) Definire comandi abbreviati per semplificare comandi con opzioni complese ripetute più volte
%3) Senza esagerare, definire testo da inserire in automatico senza riscriverlo ogni volta

%MACRO COMANDI

\newcommand{\wemph}[1]{\emph{#1}} %strong emphatize 
\newcommand{\semph}[1]{\textbf{#1}} %weak emphatize

\newcommand{\iindex}[1]{#1\index{#1}} %Indiicizzare e inserire contemporaneamente

% forse da togleire - DA TOGLIRE SICURAMENTE, NON USARE ( meglio scrivere e usare ref/vref)
\newcommand{\funref}[1]{funzionalità descritta nel punto \texttt{\ref{#1}}}
\newcommand{\funrefs}[2]{funzionalità descritte nei punti \texttt{\ref{#1}} e \texttt{\ref{#2}} }
\newcommand{\bloccofunref}[1]{blocco funzionalità \texttt{\ref{#1}.}}


%Matematica

%insiemi numerici
\newcommand{\numberset}{\mathbb}
\newcommand{\N}{\numberset{N}}
\newcommand{\R}{\numberset{R}}

%lettere strane
\renewcommand{\epsilon}{\varepsilon}
\renewcommand{\theta}{\vartheta}
\renewcommand{\rho}{\varrho}
\renewcommand{\phi}{\varphi}

%MACRO AMBIENTI

%lista descrption dove la descrizione va a capo.
\newenvironment{descriptionNext}{\begin{description}[style=nextline]}{\end{description}}
%Def: \newenvironment{name}[num]{before}{after}                  %importiamo le macro personalizzate

%%%%%%%%%%%%    PER COMPILARE   %%%%%%%%%%%%%%%%%
%1) Compilare
%2) Eseguire i programmi per generare indice, glossario e bibliografia {sotto strumenti}
%3) Compilare DUE volte (per aggiornare i riferimenti)


%%%%%%%%%%%%         TIPS       %%%%%%%%%%%%%%%%%
%-1) Il libro "L'arte di scrivere in LaTeX" è ottimo (vedi fra fonti) :D
%0) Le voci del glossario (e acronimi) e della libreria vanno aggiunte a mano nei database. (vedi esempi)
%1) Per citare una fonte bibliografica: \cite{label}
%2) Per riferirsi un termine nel glossario: \gls{label}
%3) Per riferirsi ad un acronimo: \gls{label}   -> verrà espanso solo la prima volta
%4) Per inserire un termine nell'indice analitico: \index{entry}
%   Se lo si vuole anche scrivere in contemporanea si può usare la macro \iindex{entry}
%5) Vedere documentazione: biblatex, makeidx, glossaries per roba più fica
%6) Inserire una label in un punto (occhio all'ambiente in cui si è) con: \label{key}
%7) Ci si può riferire ad una label con: \ref{label} o \vref{label} <- l'ultimo mette la p. in modo intelligente
%9) Per inserire link, oltre a \url{URL} e \ref{label}, vedere funzionalità del package hyperref. (testo -> qualcosa)
%10) Per inserire una note a pié di pagina: \footcite{bibid}    - Forse meglio fare una macro e chiamare quella
%11) Per inserire una note a margine:   \marginpar{}            - Forse meglio fare una macro e chiamare quella
%12) Per le tabelle usare (solo) lo stile booktabs: \toprule, \midrule, \bottomrule, \cmidrule{2-3}
%13) Per inserire immgaini: \includegraphics[width=x\textwidth]{imagefile} dove x è il fatto di scala
%14) Preferire linserimento di immagini o tabelle fluttuanti (decide latex dove metterle) piuttosto che in testo
%   Quelle fluttuanti sono fighe prchè c'è la gestione figa della numerazione, nome e didascalia. E tramite la
%   preferenza di posizionamento (che sono opzionali) H lo si può comunque forzare
%15) Utilizzare anche l'ambiente quoting, viene bene
%16) enumitem ha opzioni interessani per personalizzare le liste. (La macro descriptionNext lo usa)
%17) Per i comandi di base LaTex: tasto LaTeX in alto
%18) La tilde è lo "spazio non separabile" da un sare in casi come: Dr. Rossi
%19) tab spazi e a capo valgono uno spazio. Una linea vuota indica un nuovo paragrafo (\par)
%20) Per tabelle più lunghe di una pagina usare l'ambiente longtable (e non tabular)
%21) \\* command disallows page breaking on the next row. Your next row is the \midrule so the page can be broken again after the \midrule. You could also try \\*\midrule\\*


%%%%%%%%%%%%         EXAMPLES       %%%%%%%%%%%%%%%%%
%1) Tabella fluttuante
%
%\dots qui finisce un capoverso.
%
%\begin{table}
%   \caption{. . .}  <- didascalia automatica
%   \label{tab:esempio}
%   \centering
%   \begin{tabular}{. . .}
%       ...
%   \end{tabular}
%\end{table}
%
%La tabella~\ref{tab:esempio} è un esempio di tabella mobile.
%
%%%%%
%
%2) Figura fluttuante
%
%\dots qui finisce un capoverso.
%
%\begin{figure}
%\centering   <- attentione, non center
%\includegraphics[width=0.5\textwidth]{...}
%\caption{. . .} <- didascalia automatica
%\label{fig:esempio}
%\end{figure}
%
%La figura~\ref{fig:esempio} è un esempio di figura mobile.
%
%%%%%
%
%3) Figura in testo (normale figura che "deve stare proprio lì" -> no didsascalia)
%
%\begin{center}
%   \includegraphics[width= 0.5\textwidth]{Rettili}
%\end{center}
%
%%%%%
%
%4) Tabella in testo (normale tabella)
%
%\begin{center}
%   \begin{tabular}{ll} <- SE CONTIENE PRINCIPALMENTE MATEMATICA: ARRAY, NON TABULAR (sempre racchiusa in ambiente matematico)
%       \toprule
%       Alcaloide & Origine \\
%       \midrule
%       atropina & belladonna \\
%       morfina & papavero \\
%       nicotina & tabacco \\
%       \bottomrule
%   \end{tabular}
%\end{center}


%\includeonly{documents/1.PropostaDiProgetto} %Per compilazioni parziali

\begin{document} %inizio stesura documento
    \title{\emph{Ignegneria del Software}\\ \textbf{{Chochok}}}
\author{Luca Moschella \and Valentino Maiorca \and Alessio Quercia}
\date{ \today }
\maketitle
    \tableofcontents
%
    \chapter{Richiesta sistema}

L’associazione italiana amatori di cioccolato\footnote{Raggiungibile al seguente indirizzo: \url{http://www.chococlub.com/}} è intenzionata a lanciare un nuovo portale.
Tale associazione ci richiede di ideare un portale che soddisfi le sue necessità, quelle per loro indispensabili sono le seguenti:
\begin{itemize}
	\item realizzazione di un catalogo generale di prodotti ( cioccolato e dolci a base di esso ) 
	\item ricerca sul catalogo dei prodotti, anche avanzata, sulla base delle caratteristiche degli stessi.
	\item aggiornamento del catalogo.
	\item sezione notizie provenienti dal mondo dolciario.
	\item funzionalità social per interazione fra l'utenza del sistema, fra cui la possibilità di recensire prodotti e seguire gli aggiornamenti di altri utenti.
\end{itemize}
Volontà dell'associazione è ottenere un sistema di facile consultazione e aggiornamento.

L'associazione vuole che il portale interessi le seguenti figure, ed ha individuato per loro le seguenti necessità:
\begin{descriptionInd}
	\item[Utente] possibilità di ricercare prodotti, possibilità di recensire e/o valutare i prodotti stessi, seguire altre utenze per ricevere aggiornamenti da esse, leggere notizie presenti nel sito
	\item[Produttore] possibilità di creare e mantenere uno spazio per pubblicizzare i suoi prodotti
\end{descriptionInd}





    \chapter{Documento di visione} 
\label{cha:documento_di_visione}

\section{Introduzione}
\label{sec:introduzione}
Dopo un'attenta analisi della richiesta dell'associazione ChocoClub, abbiamo stilato il seguente documento di visione, che ha lo scopo di introdurre le principali figure del sistema proposto, assieme alle loro necessità e studiare la fattibilità dello stesso.

\section{Descrizione figure} 
\label{sec:descrizionefigure}
Le figure che potranno utilizzare il portale si dividono in tre categorie:
\begin{itemize}
	\item \ruolo{Visitatori}
	\item \ruolo{Utenza}
	\item \ruolo{Figure amministrative}
\end{itemize}

\subsection{Visitatori} % (fold)
\label{sub:}
I visitatori, coloro che non sono stati identificati dal portale, potranno usufruire dei suoi servizi pubblici.
\begin{descriptionInd}
    \item[Visitatori] useranno il portale per ricercare prodotti e per leggere notizie presenti nel sito.
\end{descriptionInd}



% subsection subsection_name (end)
\subsection{Utenza}
\label{sub:utenza}
L'utenza è rappresentata da coloro che, dopo aver effettuato la procedura di autenticazione, utilizzerà il portale per usufruire dei suoi servizi. Essi si dividono in due profili differenti:
\begin{descriptionInd}
    \item[Produttori] useranno il portale per pubblicizzare i propri prodotti.

    \item[Utenti] useranno il portale per ricercare prodotti, con la possibilità di recensirli e/o valutarli; per seguire altra utenza e ricevere aggiornamenti da essa e per leggere notizie presenti nel sito.
\end{descriptionInd}




\subsection{Figure amministrative}
\label{sub:figureamministrative}
Le figure amministrative sono coloro che utilizzeranno il portale con lo scopo di fornire servizi all'utenza.
\begin{descriptionInd}
   % \item[Web Admin] avrà il compito di configurare e mantenere il portale, sia lato backend che frontend.
    \item[Redattori] useranno il portale per pubblicare notizie e articoli inerenti al mondo dolciario.   
    \item[Assistenti] useranno il portale per fornire supporto all'utenza.
    \item[Moderatori] useranno il portale per gestire segnalazioni di utenti.
\end{descriptionInd}


\section{Tabelle necessità o esigenze} % (fold)
\label{sec:tabelle_necessita_o_esigenze}
Nella seguente tabella riassumiamo le necessità o esigenze delle diverse figure:
\begin{center}
	\begin{tabularx}{0.8\textwidth}{l X}
	\toprule 
		\tabhead{Figura} & \tabhead{Necessità o Esigenze} \\
	\midrule
		\ruolo{Visitatori} & ricercare prodotti, visualizzare prodotti, visualizzare notizie, autenticarsi o iscriversi.  \\
		\addlinespace
		\ruolo{Produttori} & inserire prodotti nello spazio dedicato, rimuovere prodotti dallo spazio dedicato, modificare i prodotti inseriti.  \\
		\ruolo{Utenti} & Cercare prodotti, visualizzare prodotti, valutare/recensire prodotti, visualizzare notizie, seguire altra utenza. \\
		\addlinespace
		\ruolo{Redattori} & pubblicare notizie, modificare notizie, eliminare notizie.  \\
		\ruolo{Assistenti} & rispondere alle richieste dell`utenza  \\
		\ruolo{Moderatori} & eliminare contenuti inappropriati, modificare contenuti inappropriati.  \\
	\bottomrule
	\end{tabularx}
\end{center}


\section{Studio di fattiblità}
\label{sec:studio_di_fatt}

\newdate{visuno}{22}{04}{2016}
\section{Revisioni}
\begin{center}
    \begin{tabular}{lll}
        \toprule
        Versione & Data & Descrizione \\
        %   \cmidrule{2-3} 
        \midrule
        1.0 & \displaydate{visuno} & Prima versione \\
        \bottomrule
    \end{tabular}
\end{center}



    \chapter{Studio della fattibilità} % (fold)
\label{cha:studio_della_fattibilita}

\section{Introduzione} 
\label{sec:realizzabilita_sistema}
ChocoClub intende affrontare la realizzazione di un portale per adeguarsi alle nuove esigenze del mercato in 
termini di comunicazione fra utenti, integrazione con altri sistemi e con i socialmedia.

\section{Obiettivo} 
\label{sec:obiettivo}
Obiettivo di questo progetto è la definizione e realizzazione di una piattaforma tecnologica Web Based
che realizzi una vetrina e catalogo multimediale per la consultazione di informazioni provenienti dal mondo dolciario e che fornisca alle figure pubbliche autenticate metodi di interazione.

\section{Requisiti} 
\label{sec:requisiti}
I principali requisiti del sistema che si vuole realizzare sono i seguenti:
\begin{itemize}
	\item Verrà utilizzato un \gls{cms} commerciale, con eventuali estensioni e plugin.

	\item Ogni produttore potrà gestire la sua vetrina, inserendo e modificando i suoi prodotti.

	\item Le figure pubbliche potranno effettuare vari tipi di ricerche all'interno del portale. 

	\item Le figure pubbliche avranno accesso ad un sistema social di follower/followed.

	\item Gli utenti potranno recensire prodotti.

	\item Gli utenti potranno suggerire prodotti e/o produttori mancanti.

	\item Le figure pubbliche autenticate potranno commentare altre recensioni.

	\item Si potranno gestire dal portale le informazioni inviate ai social come Facebook, Twitter, Linkedin.
\end{itemize}



\section{Architettura del sistema} 
\label{sec:architettura_sistema}
Il portale che si intende realizzare offre funzionalità abbastanza comuni e già messe a dispozione da molti sofware, intendiamo quindi utilizzare un \gls{cms} commerciale che inglobi le funzionalità richieste.

Il \gls{cms} verrà installato su un server di proprietà di ChocoClub.
I fruitori del servizio potranno connettersi al server, il quale darà loro accesso alle funzionalità presenti nel portale, messe a disposizione dal \gls{cms}.
L'architettura del sistema è quindi riassumibile dalla seguente figura:
%METTERE FIGURA

\section{Competenze} 
\label{sec:competenze}
Le competenze necessarie per portare a termine con successo il progetto sono:
\begin{itemize}
	\item Progettazione di applicativi Web.
	\item Gestione infrastrutture informatiche.
	\item Gestione di un \gls{cms} e di eventuali estensioni e plugin per esso.
\end{itemize}

\section{Tecnologie}
\label{sec:tecnologie}
Le tecnologie richieste per lo sviluppo del portale sono attualmente presenti nel mercato, e utilizzate da sistemi simili, possiamo quindi convenire che esso sia effettivamente realizzabile.
Sarà necessario l'utilizzo di un \tecnologia{\gls{cms}} e la conoscenza di \tecnologia{CSS} per eventuali personalizzazioni dell'interfaccia.
Per quanto riguarda l'infrastruttura, visto l'approccio centralizzato, basterà un \tecnologia{server} per rendere disponibile l'accesso al servizio.

\section{Vantaggi} 
\label{sec:vantaggi}
L'uso di un \gls{cms} permetterà di abbattere i costi di sviluppo e implementazione in quanto non sarà necessario re--implementare tutte le funzionalità necessarie, poiché già disponibili nel \gls{cms}.

Inoltre la maggior parte dei \gls{cms} commerciali mette a disposizione una grandissima quantità di estensioni e plugin, che permette di usufruire di un notevole numero di funzionalità; le quali potranno essere utilizzate in futuro per ampliare il portale con nuovi servizi, ad esempio esistono plugin per l'intregazione con i principali social network o per la gestione del sistema di recensioni.

Grazie alla soluzione proposta, Chococlub potrà disporre di un portale più fruibile rispetto a quello attuale, con contenuti sempre aggiornati in quanto delegato ai produttori associati ed iscritti al portale. Disporrà inoltre di un portale con una completa integrazione con i sistemi Social. 

Chococlub inoltre, dopo un periodo di assestamento del portale, potrà valutare: 
\begin{itemize}
	\item La ricezione di compenso per la pubblicità garantita ai produttori iscritti al proprio portale.
	\item L'introduzione di un produttore delegato che, sotto compenso, potrà gestire le piccole realtà locali che non fanno utilizzo delle nuove tecnologie.
\end{itemize}

Il sistema è vantaggioso quindi sia per i produttori, che potranno pubblicizzare i propri prodotti, sia per gli utenti, che potranno interagire direttamente con i produttori, scoprire nuovi prodotti e comunicare con altri utenti e sia per Chococlub che, oltre al beneficio di un portale unificante, potrà valutare diversi modi per introdurre nuove fonti di guadagno.


\newdate{fattuno}{1}{05}{2016}
\section{Revisioni}
\begin{center}
    \begin{tabular}{lll}
        \toprule
        	\tabhead{Versione} & \tabhead{Data} & \tabhead{Descrizione} \\
		\cmidrule(l{\cmidrulekern}r{\cmidrulekern}){1-3}
        	1.0 & \displaydate{fattuno} & Prima versione \\
        \bottomrule
    \end{tabular}
\end{center}



% In documento: Project plan
%\section{Attività} 
%\label{sec:attivita}

% In documento: Project plan
%\section{Stima Tempi} 
%\label{sec:tempi}

% In documento: Project plan
%\section{Stima Costi} 
%\label{sec:costi}


%? Probabilmente non serve più, non abbiamo parlato di "rinnovare portali" ne di informazioni guà presenti:
%Per la fase iniziale di inserimento informazioni è stato deciso di non utilizzare i contenuti attualmente presenti sul sito dell’associazione, in quanto il sito è statico e di difficile consultazione; oltretutto, la maggior parte dei contenuti non è aggiornata e non riguarda principalmente i prodotti, ma le pasticcerie come attività commerciali, quindi non è utile per gli obiettivi del nuovo portale.

    \chapter{Contratto} 
\label{cha:contratto}

\section{Introduzione}
\label{sec:introduzione}
Il seguente contratto avrà validità previa approvazione delle specifiche del sistema e dell'analisi dei costi previsti.

\section{Caratteristiche del sistema} 
\label{sec:caratteristiche_del_sistema}
Il sistema che si intende realizzare è quello descritto ad alto livello nel \docref{cha:documento_di_visione}, esso verrà descritto nel dettaglio nei documenti di specifica. 

Possiamo affermare che tale sistema è realizzabile grazie allo \docref{cha:studio_della_fattibilita}.

Faremo in modo di curare con la massima attenzione design e usabilità del sistema, in modo da rendere più semplice possibile l'utilizzo del portale. Il sito sarà dinamico e la grafica personalizzabile secondo le direttive dell'acquirente.


\section{Tempi di realizzazione} 
\label{sec:tempi_di_realizzazione}

Ci impegneremo a realizzare il sistema in un tempo di tre mesi dall'approvazione delle specifiche.
%Dopo le specifiche saranno indicati anche i costi, tramite l'analisi dei costi.

\section{Assistenza e mantenimento} 
\label{sec:assistenza_e_mantenimento}
Includiamo nel budget previsto la manutenzione e il debug della versione consegnata per un periodo di due anni. Eventuali nuove versioni o funzionalità aggiuntive non sono incluse in questo contratto. 

Ci occuperemo dell’installazione del sistema nell'infrastruttura dell'acquirente e dell'eventuale assistenza tecnica.


\section{Costi} 
\label{sec:costi}
L'analisi dei costi verrà effettuata dopo la realizzazione delle specifiche del sistema e allegata in apposito documento.

\section{Divieto di cessione} 
\label{sec:divieto_di_cessione}
Il presente contratto non può essere ceduto, in tutto o in parte, a soggetti
terzi rispetto a quelli firmatari del medesimo, a meno che a ciò non risulti
espressa autorizzazione dell’altra parte.

\section{Legge applicabile e foro competente} 
\label{sec:legge_applicabile_e_foro_competente}
La legge applicabile al presente contratto è quella Italiana. Per qualsiasi
controversia relativa al presente contratto sarà esclusivamente competente il
Foro di Roma. Per quanto non espressamente previsto dal presente contratto
saranno applicabili le norme di legge vigenti.

%\section{Domicilio} 
%\label{sec:domicilio}

%\section{Risoluzione contrattuale} 
%\label{sec:risoluzione_contrattuale}

    \chapter{Pianificazione del progetto} 
\label{cha:pianificazione_del_progetto}	

\section{Introduzione}
In questo documento verrà descritto il processo di sviluppo utilizzato e verrà effettuata la pianificazione delle iterazioni e delle attività da svolgere.
Sarà inoltre fornita una pianificazione temporale del lavoro e sarà possibile verificare lo stato attuale dei lavori consultando questo documento. Infatti manterremo aggiornato questo documento tenendo traccia dell'avanzamento dello sviluppo del sistema.

\section{Processo di sviluppo}
\label{sec:processo_di_sviluppo}
Il progetto sarà realizzato secondo la metodologia di sviluppo \gls{rup}.
Tale motodologia prevede quattro fasi sequenziali che possono essere iterate:
\begin{enumerate}
	\item \fase{Inizio}: fase in cui vengono raccolti i requisiti principali.

	\item \fase{Elaborazione}: fase in cui vengono corretti i requisiti, viene effettuata una analisi ordinata, la progettazione e le prime implementazioni.

	\item \fase{Costruzione}: fase in cui viene completata la progettazione (con i dettagli), viene implementato il sistema e vengono effettuati verifiche e test di aggregazione.

	\item \fase{Transizione}: fase in cui vengono effettuati test finali di pre--produzione, messa in opera, manutenzione, archiviazione documentale e attivazione procedure di gestione.
\end{enumerate}

\noindent
In ognuna di queste fasi possono essere svolte le seguenti attività:
\begin{itemize}
	\item \workflow{Raccolta dei requisiti}
	\item \workflow{Analisi}
	\item \workflow{Progettazione}
	\item \workflow{Implementazione}
	\item \workflow{Test}
\end{itemize}

\noindent
A seconda della fase in cui ci si trova verranno svolte più o meno attività. 
Un tipico esempio di distribuzione delle attività nelle varie fasi è il seguente:
\begin{center}
   \includegraphics[width= \textwidth]{assets/fasiworkflow}
\end{center}

\section{Pianificazione iterazioni}
\label{sec:pianificazione_iterazioni}
Saranno di seguito elencate le iterazioni svolte durante il progetto, durante lo sviluppo compariranno le iterazioni già effettuate e la pianificazione dell'iterazione successiva.

\begin{center}
	\begin{tabularx}{\tabwidthiter}{ l  X } 
		\toprule
		\multicolumn{2}{c}{\tabtitleiter{Iterazione 01}}  \\
		\cmidrule(l{\cmidrulekern}r{\cmidrulekern}){1-2}
		\tabheaditer{Fase} & Inception \\ 
		\addlinespace[1em] 
		\tabheaditer{Milestone} & 
		    \begin{enumWork}
			        \item Stesura del documento di richiesta
			        \item Stesura del documento di visione
			        \item Stesura dello studio di fattibilità
			        \item Stesura del contratto
			        \item Inizio stesura del glossario
			        \item Inizio stesura del documento di pianificazione del progetto
		    \end{enumWork} \\
		\addlinespace[1em]
		\tabheaditer{Stato} &  In corso \\
		\bottomrule
	\end{tabularx}
\end{center}

\begin{landscape}
\section{Diagramma di Gantt}
\label{sec:diagramma_di_gantt}
Nel seguente diagramma di Gantt sono riportati i periodi di svolgimento di ogni attività effettuata per la realizzazione del sistema.

\begin{center}
	\includegraphics[width=\linewidth]{diagrammaGantt/DiagrammaGantt}
\end{center}
\end{landscape}


    %\include{documents/1-PropostaDiProgetto}
%
%
%Stampa il glossario - anche quelli non usati
    \glsaddallunused
    \printglossaries
%
%Stampa la bibliografia - anche quella non citata
    \nocite{*}
    \cleardoublepage
    \phantomsection
    \addcontentsline{toc}{chapter}{\bibname}
    \printbibliography
%
%Stampa l'indice analitico - se presente
    \cleardoublepage
    \phantomsection
    \addcontentsline{toc}{chapter}{\indexname}
    \printindex %Maaaaa serve l'indice se abbiamo il glossario?
%
\end{document}  %fine stesura doumento
