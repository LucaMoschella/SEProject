%Date versioni
%Versione 1
\newdate{versioneuno}{22}{04}{2016}
\chapter{Proposta di progetto}

\section{Introduzione}
L’associazione italiana amatori di cioccolato\footnote{Raggiungibile al seguente indirizzo: \url{http://www.chococlub.com/}} è intenzionata a lanciare un nuovo portale.
Le richieste per questo sistema sono le seguenti:
\begin{itemize}
	\item possibilità da parte dei produttori di inserire (o modificare, o rimuovere) i propri prodotti (sia cioccolato che dolci a base di esso) nel catalogo generale.
	\item ricerca sul catalogo dei prodotti, anche avanzata, sulla base delle caratteristiche degli stessi.
	\item sezione notizie.
	\item funzionalità stile social (recensioni di prodotti e sistema di follower/follo\-wed).
\end{itemize}
Volontà dell'associazione è ottenere un sistema di facile consultazione e aggiornamento.
Esso dovrà permettere agli utenti del servizio di ricercare e recensire prodotti come pure di consigliarne altri.
L'interazione tra utenti invece, è richiesta tramite un sistema di follower/followed che faccia da contorno al sistema principale, quindi non invasivo, ma che permetta ai fruitori del servizio (i follower) di ottenere informazioni riguardo attività da parte di chi seguono (i followed).
Nel sistema follower/followed chiunque può essere un follower (sia gli utenti semplici, sia i produttori) nonché followed.
 
È stato deciso di non utilizzare i contenuti attualmente presenti sul sito dell’associazione, per le seguenti motivazioni: il sito è statico e di difficile consultazione e analisi, oltretutto la maggior parte dei contenuti non è aggiornata. Inoltre, i dati attualmente contenuti non riguardano principalmente i prodotti, ma le pasticcerie come attività commerciali, quindi non sono utili per gli obiettivi del nuovo portale.
 
Per questo motivo è stato deciso di realizzare un portale ex novo, avvalendosi dell’utilizzo di un \semph{CMS}.

Dopo un’attenta analisi della richiesta ed uno studio del settore, abbiamo elaborato la seguente proposta di realizzazione della piattaforma. Va fatto notare che il punto di forza del servizio saranno le informazioni sui prodotti esclusivamente di cioccolata, ma è stato deciso di includere anche la catalogazione di dolci che la contengono, per rivolgersi ad una più ampia fetta di mercato.

\section{Descrizione generale}
Come da richiesta, il portale permetterà agli utenti di reperire informazioni ad alta affidabilità riguardo la cioccolata ed i suoi produttori.
Tali informazioni copriranno ogni aspetto riguardante il prodotto, visto che saranno anche arricchite dagli utenti attraverso un sistema di recensioni e commenti, così da rendere completa la descrizione dei prodotti (includendo l'opinione dell'utente stesso).
Per i produttori, il portale sarà un mezzo di pubblicità in quanto avranno una vetrina personalizzata per esporre i propri prodotti e saranno essi stessi ad inserire tutte le specifiche dei loro prodotti, ovvero le informazioni che gli utenti potranno reperire.
Il portale avrà diversi tipi di ruoli amministrativi e di utenze; segue una breve descrizione delle stesse, in seguito approfondita. 
\begin{description}
	\item Le \wemph{Utenze}: 
			\begin{description}
				\item[Produttore] si potrà iscrivere al portale solo tramite account verificati da siti terzi o tramite approvazione privata, contattando direttamente il supporto. Avrà il compito di creare e mantenere aggiornata la sua vetrina. Sarà a lui delegata l’aggiunta di ogni suo prodotto, anche in seguito a richieste da parte degli utenti.
				
				\item[Utente] si potrà iscrivere al portale, anche tramite account di siti terzi. La loro funzionalità principale, oltre la ricerca dei prodotti, sarà la possibilità di recensire e/o valutare i prodotti stessi. 
			\end{description}
			
	\item Le \wemph{Figure di gestione del portale}:
			\begin{description}
				\item[Web Admin] avrà accesso completo al pannello di controllo del CM
				
				\item[Newser] avrà, oltre alle funzionalità base dell’utente, il compito di gestire gli articoli e le notizie dal mondo dolciario.
				
				\item[Support] gestirà l’e-mail del portale rispondendo alle richieste di supporto e avrà il compito di approvare le richieste di iscrizione da parte dei produttori (tali richieste saranno inviate tramite form per una più facile gestione di esse).
				
				\item[Moderatore] avrà il compito di gestire segnalazioni di utenti eliminando/modificando eventuali contenuti non consoni presenti in recensioni.
			\end{description}
\end{description}

Alcune funzionalità che potranno essere implementate in futuro sono: 
\begin{itemize}
	\item l’implementazione della \wemph{wishlist} per ricordarsi dei prodotti che si vorrebbero provare.

	\item l’aggiunta del \wemph{produttore delegato}, una figura necessaria per aiutare le piccole realtà dolciarie che si occuperà di gestire le vetrine al loro posto.

	\item Aggiunta dei commenti alle \wemph{notizie}.
\end{itemize}

Per la fase iniziale di riempimento del portale, sarà compito di ChocoClub stessa prendere contatti con le molte pasticcerie artigianali già affiliate e le case produttrici, affinché esse si iscrivano al portale, per avere un buon numero di contenuti di partenza. 

\section{Descrizione funzionalità principali}
Di seguito è presente una panoramica generale di alcune delle principali funzionalità che saranno presenti nel portale, tale elenco potrà essere soggetto a modifiche nel corso dello sviluppo.
\begin{enumerate}
	\item \wemph{Iscrizione \& Autenticazione} \label{bloccofun:iscAut}
		\begin{enumerate}        	
			\item	\begin{descriptionNext}
						\item[Iscrizione tramite portale] Permette l’iscrizione direttamente tramite il portale. 
					\end{descriptionNext} \label{fun:isc.portale}
						                      
			\item 	\begin{descriptionNext}
						\item[Iscrizione tramite account terzi verificati] Si tratta della possibilità di iscriversi tramite account di altri servizi come LinkedIn, Facebook e Twitter, purché risultino “verificati” anche su quelle piattaforme.							
					\end{descriptionNext} \label{fun:isc.terziv}
					                       
			\item 	\begin{descriptionNext}
						\item[Iscrizione tramite account terzi non verificati]	È possibile utilizzare un account di siti terzi Google o Facebook per iscriversi							 
					\end{descriptionNext} \label{fun:isc.terzinv}
					                       
			\item 	\begin{descriptionNext}
						\item[Iscrizione tramite approvazione] Si tratta della possibilità per i produttori di iscriversi in modo verificato al portale, senza possedere un account già verificato su siti terzi.
						               
					\end{descriptionNext} \label{fun:isc.appr}
					                       
			\item 	\begin{descriptionNext}
						\item[Login] Permette di fare il login nel sistema utilizzando le credenziali con cui ci si è registrati.							
					\end{descriptionNext} \label{fun:aut.login}			
		\end{enumerate}
		
	\item \wemph{Gestione iscrizione}	\label{bloccofun:gestisc}		
		\begin{enumerate}        
			\item	\begin{descriptionNext}
						\item[Approvazione iscrizione] Si tratta della funzione relativa alla \funref{fun:isc.appr}, permette di approvare le richieste di iscrizione.
					\end{descriptionNext}	\label{fun:appr.iscr}
					                      
			\item	\begin{descriptionNext}
						\item[Rimozione account altrui] Permette di eliminare un account, ma NON le recensioni e valutazioni ad esso correlate nel caso di utenti e NON rimuove i prodotti nel caso di account produttore.
					\end{descriptionNext}	\label{fun:rim.iscr}
		\end{enumerate}
		
	\item \wemph{Gestione spazio dedicato alla presentazione prodotti (vetrina)}	\label{bloccofun:gestvet}		
		\begin{enumerate}        
			\item	\begin{descriptionNext}
						\item[Inserimento descrizione e immagine vetrina] Permette di avere a disposizione una vetrina pubblica, in cui ci sarà la descrizione del produttore e in cui verranno presentati i suoi prodotti. La funzionalità permette di personalizzare la vetrina attraverso un campo testuale (descrizione) ed un’immagine di sfondo.
					\end{descriptionNext}	\label{fun:ins.vetr}
					                      
			\item	\begin{descriptionNext}
						\item[Inserimento prodotto in vetrina] Permette l’aggiunta di un prodotto alla vetrina. Questa funzionalità segue un rigido protocollo per far sì che le informazioni inserite siano complete e coerenti. 
						Durante l’inserimento di un prodotto è necessario immettere: il \semph{nome del prodotto} con una sua \semph{descrizione} ed almeno una \semph{foto}. 
						È inoltre necessario compilare un modulo che contiene campi a \wemph{scelta multipla} e campi con \wemph{tag dinamici}, di cui a seguire una breve descrizione: il sistema metterà a disposizione una grande lista dei possibili tag, che saranno suggeriti come auto-completamento all’utilizzatore durante la compilazione del campo correlato.
						L’utilizzo di un tag non in lista è permesso, ma esso verrà sottoposto ad una fase di verifica che se superata comporterà l’aggiunta in lista, in caso contrario verrà rimosso anche dal prodotto. Sarà specificato un limite massimo e minimo di tag da inserire per prodotto.
						Nel modulo verrà chiesta la \semph{marca} (a scelta fra quelle associate al produttore che sta inserendo il prodotto), la \semph{categoria} del prodotto che si sta inserendo (a scelta fra \wemph{Dolce al cioccolato} oppure \wemph{cioccolato}), \semph{particolarità} (risposta con tag dinamici),  la \semph{consistenza} (a scelta fra \wemph{solida}, \wemph{liquida} oppure \emph{spalmabile}), gli \semph{ingredienti} (risposta con tag dinamici).
						Durante la compilazione del modulo di un prodotto è possibile compilare ulteriori campi (oltre a quelli proposti di default) scegliendo fra quelli messi a disposizione dal sistema, per migliorare la descrizione del prodotto e facilitarne la ricerca.
					\end{descriptionNext}	\label{fun:ins.prod}

			\item	\begin{descriptionNext}
						\item[Modifica prodotto in vetrina] Permette la modifica delle caratteristiche dei prodotti presenti, secondo gli stessi campi della \funref{fun:ins.prod}. Permette, inoltre, di etichettare un prodotto come fuori produzione.
					\end{descriptionNext} \label{fun:mod.login}

			\item	\begin{descriptionNext}
						\item[Visualizzazione statistiche vetrina privata] Mostra statistiche della vetrina, in rapporto al tempo; riassume le recensioni dei prodotti in essa presenti, in modo filtrabile e ordinabile in base a diversi criteri e permette di visualizzare il numero di visite ricevute.
					\end{descriptionNext} \label{fun:vis.vetrPr}	 	  				
		\end{enumerate}			
		
	\item \emph{Ricerca} \label{bloccofun:ric}
		\begin{enumerate}
			\item	\begin{descriptionNext}
						\item[Ricerca prodotti tramite diversi criteri] Permette di effettuare ricerche, le ricerche potranno basarsi su qualsiasi campo della scheda del prodotto e dettagli del produttore. Sarà possibile ordinare i risultati.
					\end{descriptionNext}	\label{fun:ric.prod}

			\item	\begin{descriptionNext}
						\item[Ricerca profili pubblici] Permette di ricercare e in seguito visualizzare tramite l’apposita funzionalità, i profili pubblici di utenti e/o produttori.
					\end{descriptionNext}	\label{fun:ric.prof}

			\item	\begin{descriptionNext}
						\item[Ricerca notizie] Permette di effettuare ricerche sulle notizie pubblicate nel portale.
					\end{descriptionNext}	\label{fun:ric.not}
		\end{enumerate}

	\item \wemph{Gestione recensioni} \label{bloccofun:gestrec}
		\begin{enumerate}
			\item	\begin{descriptionNext}
						\item[Valutazione prodotto] Permette di esprimere una valutazione su scala limitata di un prodotto.
					\end{descriptionNext}	\label{fun:val.prod}

			\item	\begin{descriptionNext}
						\item[Recensione prodotto esistente] Permette di aggiungere, alla semplice \funref{fun:val.prod}, anche una descrizione testuale della recensione, più completa.
					\end{descriptionNext}	\label{fun:rece.prod}

			\item	\begin{descriptionNext}
						\item[Richiesta inserimento di un prodotto a un produttore]Permette di segnalare ad un produttore la mancanza di un suo prodotto tra quelli presenti nel sistema.
					\end{descriptionNext}	\label{fun:ric.prodotto}


			\item	\begin{descriptionNext}
						\item[Richiesta inserimento di un prodotto e del suo produttore]Permette di segnalare al personale incaricato l’assenza di un produttore ed allo stesso tempo di uno suo specifico prodotto.
					\end{descriptionNext}	\label{fun:ric.produttore}

			\item	\begin{descriptionNext}
						\item[Modifica recensione propria] Permette di modificare il campo testuale menzionato nella \funref{fun:rece.prod} della propria recensione.
					\end{descriptionNext}		\label{fun:mod.recPropria}

			\item	\begin{descriptionNext}
						\item[Modifica recensione altrui] Permette di modificare il campo testuale menzionato nella  \funref{fun:rece.prod} di una qualsiasi recensione.
					\end{descriptionNext}		\label{fun:mod.recAltrui}

			\item	\begin{descriptionNext}
						\item[Modifica valutazione] Permette di modificare la valutazione sia per recensioni che per semplici valutazioni.
					\end{descriptionNext}		\label{fun:mod.valPropia}

			\item	\begin{descriptionNext}
						\item[Valutazione recensione] Permette di esprimere una valutazione positiva o negativa sulle recensioni. Ciò permetterà di stabilire quali recensioni sono più utili.
					\end{descriptionNext}		\label{fun:val.rec}

			\item	\begin{descriptionNext}
						\item[Elimina recensione propria] Permette di eliminare la propria recensione. 
					\end{descriptionNext}		\label{fun:del.recPropia}

			\item	\begin{descriptionNext}
						\item[Elimina recensione altrui] Permette di eliminare una qualsiasi recensione.
					\end{descriptionNext}	 \label{fun:del.recAltrui}
					
			\item	\begin{descriptionNext}
						\item[Commento a recensione] Permette di commentare una qualsiasi recensione.
					\end{descriptionNext}	\label{fun:comm.rec}

			\item	\begin{descriptionNext}
						\item[Segnalazione contenuti] Permette di segnalare contenuti come recensioni e commenti alle stesse, se non consoni (secondo il regolamento del servizio). 
					\end{descriptionNext}	\label{fun:report.cont}
		\end{enumerate}

	\item \wemph{Gestione notizie} \label{bloccofun:gestnot}
		\begin{enumerate}
			\item	\begin{descriptionNext}
						\item[Consultazione notizie] Permette la visualizzazione delle notizie inserite nel portale, in base a preferenze dell’utente e impostazioni del gestore.
					\end{descriptionNext}	\label{fun:vis.not}

			\item	\begin{descriptionNext}
						\item[Inserisci notizia] Permette di aggiungere una notizia visibile tramite la \funref{fun:vis.not}.
					\end{descriptionNext}	\label{fun:ins.not}
					
			\item	\begin{descriptionNext}
						\item[Modifica notizia] Permette di modificare una notizia già esistente.
					\end{descriptionNext}	 \label{fun:mod.not}

			\item	\begin{descriptionNext}
						\item[Elimina notizia] Permette di eliminare una notizia già esistente.
					\end{descriptionNext}	\label{fun:del.not}
		\end{enumerate}

	\item \wemph{Gestione follower/followed} \label{bloccofun:gestfoll}
		\begin{enumerate}
			\item	\begin{descriptionNext}
						\item[Visualizzazione aggiornamenti followed] Mostra gli aggiornamenti e le notizie relativi ai propri followed.
						Questo sistema permetterà di ottenere una vista sulle attività dei followed, ovvero: nuova recensione, nuovo followed, nuova valutazione se si tratta di un utente; nuovo prodotto, prodotto fuori produzione se invece di tratta di un produttore.
					\end{descriptionNext}	\label{fun:vis.foll}

			\item	\begin{descriptionNext}
						\item[Follow account] Permette di seguire un account e ricevere da esso notizie e aggiornamenti. 
					\end{descriptionNext}	\label{fun:foll.acc}
					
			\item	\begin{descriptionNext}
						\item[Unfollow account] Permette di non seguire più un account che si segue. 
					\end{descriptionNext} \label{fun:unfoll.acc}
		\end{enumerate}

	\item \wemph{Gestioni suggerimenti} \label{bloccofun:gestsugg}
		\begin{enumerate}
			\item	\begin{descriptionNext}
						\item[Ti potrebbero piacere \dots] La funzionalità permette di ottenere suggerimenti ad hoc, essi potranno essere basati su diversi parametri: affinità con prodotti per i quali ha già espresso preferenza, informazioni presenti nel profilo (come il luogo di residenza per prodotti artigianali) e recensioni di utenti seguiti.
					\end{descriptionNext}	\label{fun:cons.prod}

			\item	\begin{descriptionNext}
						\item[Prodotti simili a \dots] Permette di suggerire prodotti simili a quello che si sta visualizzando sulla base di diversi criteri.
					\end{descriptionNext}	\label{fun:prod.sim}
					
			\item	\begin{descriptionNext}
						\item[Notizie simili a \dots] Permette di suggerire articoli e notizie sulla base dell’articolo che si sta attualmente leggendo.
					\end{descriptionNext}	\label{fun:not.sim}
		\end{enumerate}

	\item \wemph{Gestione account} \label{bloccofun:gestacc}
		\begin{enumerate}
			\item	\begin{descriptionNext}
						\item[Visualizzazione profilo] Permette di visualizzare il proprio profilo.
					\end{descriptionNext}	\label{fun:vis.prof}

			\item	\begin{descriptionNext}
						\item[Accesso alle impostazioni] Permette di accedere alle impostazioni dell’account. Le impostazioni variano a seconda del tipo di account. Per gli utenti, fra le altre impostazioni, c’è la possibilità di stabilire le preferenze di prodotti e/o produttori, per ricerche più mirate.
					\end{descriptionNext}	\label{fun:gest.imp}
					
			\item	\begin{descriptionNext}
						\item[Logout] Permette di fare il logout del sistema.
					\end{descriptionNext}	\label{fun:aut.logout}

			\item	\begin{descriptionNext}
						\item[Rimuovi account] Permette di eliminare il proprio account, ma NON le recensioni e valutazioni ad esso correlate nel caso di utenti e NON rimuove i prodotti nel caso di account produttore.
					\end{descriptionNext}	\label{fun:del.acc}
		\end{enumerate}			


	\item \wemph{Visualizzazione} \label{bloccofun:gestvis}
		\begin{enumerate}
			\item	\begin{descriptionNext}
						\item[Visualizza vetrina] Permette di visualizzare la vetrina di un dato produttore, ordinando i prodotti in base a diversi criteri.
					\end{descriptionNext}	\label{fun:vis.vet}

			\item	\begin{descriptionNext}
						\item[Visualizza statistiche vetrina pubbliche] Permette di visualizzare le statistiche pubbliche di una vetrina.
					\end{descriptionNext}	\label{fun:vis.statvetPubbliche}
					
			\item	\begin{descriptionNext}
						\item[Visualizza prodotto] Permette di visualizzare un dato prodotto e le sue informazioni.
					\end{descriptionNext}	\label{fun:vis.prod}

			\item	\begin{descriptionNext}
						\item[Visualizza recensioni] Permette di visualizzare le recensioni associate ad un prodotto, ordinandole in base a diversi criteri.
					\end{descriptionNext}	\label{fun:vis.recs}

			\item	\begin{descriptionNext}
						\item[Visualizza profilo altrui] Permette di visualizzare un dato profilo; si differenziano le informazioni fornite se il profilo è associato ad un produttore o ad un utente.
					\end{descriptionNext}	\label{fun:vis.profAltrui}

			\item	\begin{descriptionNext}
						\item[Visualizza statistiche profilo] Permette di visualizzare le statistiche di un dato profilo; si differenziano le informazioni fornite se il profilo è associato ad un produttore o ad un utente.
					\end{descriptionNext}	\label{fun:vis.profStat}		
		\end{enumerate}	
\end{enumerate}

\section{Utenze del portale}
Di seguite descriviamo brevemente le principali utenze che possono interagire con il portale.

\subsection{Utente non autenticato}
Un utente non autenticato può effettuare un’iscrizione o l’autenticazione (blocco funzionalità \funref{bloccofun:iscAut}), può effettuare ricerche (blocco funzionalità \funref{bloccofun:ric}), può consultare notizie (funzionalità descritta nel punto \funref{fun:vis.not}), può vedere prodotti e notizie simili a quelle che sta visualizzando (funzionalità descritte nei punti \funref{fun:prod.sim} e \funref{fun:not.sim}) e può inoltre visualizzare tutte le sezioni pubbliche del sito (blocco funzionalità \funref{bloccofun:gestvis}).

\subsection{Produttore autenticato}
Un produttore autenticato può gestire lo spazio dedicato alla presentazione prodotti (blocco funzionalità \funref{bloccofun:gestvet}), può effettuare ricerche (blocco funzionalità \funref{bloccofun:ric}), può segnalare contenuti e commentare recensioni (funzionalità descritte nei punti \funref{fun:comm.rec} e \funref{fun:report.cont}), può consultare notizie (funzionalità descritta nel punto \funref{fun:vis.not}), può usufruire del sistema di follower/followed (blocco funzionalità \funref{bloccofun:gestfoll}), può usufruire delle funzionalità di gestione account (blocco funzionalità \funref{bloccofun:gestacc}) e può inoltre visualizzare tutte le sezioni pubbliche del sito (blocco funzionalità \funref{bloccofun:gestvis}).

\subsection{Utente autenticato}
Un utente autenticato può effettuare ricerche (blocco funzionalità \funref{bloccofun:ric}), può effettuare qualsiasi operazione sulle recensioni tranne l’eliminazione e la modifica delle recensioni di altri utenti (blocco funzionalità \funref{bloccofun:gestrec} tranne le funzionalità \funref{fun:del.recAltrui} e \funref{fun:mod.recAltrui}),  può consultare notizie (funzionalità descritta nel punto \funref{fun:vis.not}), può usufruire del sistema di follower/followed  (blocco funzionalità \funref{bloccofun:gestfoll}), può usufruire del sistema di suggerimenti automatici  (blocco funzionalità \funref{bloccofun:gestsugg}), può usufruire delle funzionalità di gestione account (blocco funzionalità \funref{bloccofun:gestacc}) e può anche visualizzare tutte le sezioni pubbliche del sito  (blocco funzionalità \funref{bloccofun:gestvis}).

\section{Figure di gestione del portale}
Di seguite descriviamo brevemente le principali figure di gestione del portale.

\subsection{Web Admin}
Il WebAdmin ha accesso a tutte le funzionalità, sia quelle principali sopra descritte che quelle secondarie che verranno descritte in seguito.

\subsection{Newser}
Si tratta di un utente autenticato (quindi ne eredita tutte le funzionalità), ma ha in più la completa gestione delle news (blocco funzionalità \funref{bloccofun:gestnot}).

\subsection{Support}
Si tratta di un utente autenticato (quindi ne eredita tutte le funzionalità), ma ha in più la gestione delle iscrizioni con approvazione (blocco funzionalità \funref{bloccofun:gestisc}).

\subsection{Moderatore}
Si tratta di un utente autenticato (quindi ne eredita tutte le funzionalità), ma ha in più la gestione totale del sistema recensioni (blocco funzionalità \funref{bloccofun:gestrec}).

\section{Informazioni sui prodotti}
Ogni prodotto conterrà le seguenti informazioni:
\begin{itemize}
	\item Nome prodotto. 
	\item Almeno una foto dell’articolo.
	\item  Descrizione di presentazione.
	\item  Marca.
	\item Categoria.
	\item Particolarità.
	\item Consistenza.
	\item Ingredienti.
	\item Altri campi (a seconda dei quali ha compilato il produttore).
\end{itemize}
Ad ogni prodotto saranno associate le sue recensioni con annessa valutazione e voto medio. Saranno inoltre suggeriti prodotti simili.

\section{Revisioni}
\begin{center}
	\begin{tabular}{lll}
		\toprule
		Versione & Data & Descrizione \\
		%	\cmidrule{2-3} 
		\midrule
		1.0 & \displaydate{versioneuno} & Prima versione \\
		\bottomrule
	\end{tabular}
\end{center}
