\chapter{Analisi del sistema}

\section{Introduzione}
In questo documento effetturemo l'analisi del sistema, ci avvarremo del linguaggio \gls{uml} per formalizzare i requisiti descritti nel documento \docref{cha:specifica_requisiti} e in parte espressi nei casi d'uso nella \docref{cha:usecase}.
Effettueremo la formalizzazione esclusivamente delle parti prinicpali del sistema, in particolare andremo a realizzare i seguenti diagrammi:
\begin{itemize}
	\item \nameref{cha:attivita}
	\item \nameref{cha:package}
	\item \nameref{cha:classi}
	\item \nameref{cha:sequenza}
\end{itemize}
Per capire come strutturare le classi, le interazioni tra di esse e le operazioni che devono svolgere, utilizzeremo le carte \gls{crc}.

\section{Diagrammi di attività} \label{cha:attivita}
I diagrammi di attività descrivono il comportamento nel tempo di un particolare elemento. Con essi si modella l’interazione e la fusione dei flussi principali con i flussi alternativi dei casi d’uso. Questo viene fatto per rendere più visibili le varie ramificazioni che i flussi possono avere all’interno di un caso d’uso. Di seguito sono elencati i diagrammi di attività, non verranno riportati tutti i diagrammi, ma solo quelli corrispondenti ai casi d'uso principali.
\begin{center}
	\begin{tabularx}{\textwidth}{ l X } 
		\toprule
		\formattaTitoloTab{ID} & \formattaTitoloTab{Caso d'uso di riferimento} \\
		\cmidrule(l{\cmidrulekern}r{\cmidrulekern}){1-2}
		\newAttivita{da:login}{\formattaAT}{Login} & \getIDTitletodesc{cu:login} \\ 
		\addlinespace[1em] 
		\newAttivita{da:logout}{\formattaAT}{Logout} & \getIDTitletodesc{cu:logout} \\ 
		\addlinespace[1em] 
		\newAttivita{da:iscrizione}{\formattaAT}{Iscrizione} & \getIDTitletodesc{cu:iscrizionePortale} \\
		 													 & \getIDTitletodesc{cu:iscrizioneSocial} \\
															 & \getIDTitletodesc{cu:iscrizioneApprovazione} \\ 
		\addlinespace[1em] 
		\newAttivita{da:approvazione}{\formattaAT}{Approvazione iscrizione} & \getIDTitletodesc{cu:approvazioneIscrizione} \\ 
		\addlinespace[1em]
		\newAttivita{da:schedaprodotto}{\formattaAT}{Inserimento scheda prodotto} & \getIDTitletodesc{cu:personalizzaVetrinaInsProd} \\
		\addlinespace[1em]
		\newAttivita{da:ricerche}{\formattaAT}{Ricerche} & \getIDTitletodesc{cu:ricercaProdotto} \\
														 & \getIDTitletodesc{cu:ricercaNotizia} \\
														 & \getIDTitletodesc{cu:ricercaProfilo} \\
		\addlinespace[1em]
		\newAttivita{da:valrec}{\formattaAT}{Inserimento valutazione e recensione} & \getIDTitletodesc{cu:inserisciValutazioneProdotto} \\
																				   & \getIDTitletodesc{cu:inserisciRecensioneProdotto}\\
		\addlinespace[1em]
		\newAttivita{da:oprec}{\formattaAT}{Operazioni recensione} & \getIDTitletodesc{cu:modificaRecensioneProdotto}\\
																   & \getIDTitletodesc{cu:eliminaRecensioneProdotto}\\
																   & \getIDTitletodesc{cu:commentoRecensione}\\
																   & \getIDTitletodesc{cu:giudizioRecensione}\\ 
																   & \getIDTitletodesc{cu:modificaGiudizioRecensione}\\ 
		\bottomrule
	\end{tabularx}
\end{center}

\linkedSubsection{da:login}
\begin{center}
			\includegraphics[width=\textwidth]{assets/visualParadigm/attivita/login}
\end{center}

\linkedSubsection{da:logout}
\begin{center}
	\includegraphics[width=\textwidth]{assets/visualParadigm/attivita/logout}
\end{center}

\linkedSubsection{da:iscrizione}
\begin{center}
	\includegraphics[width=\textwidth]{assets/visualParadigm/attivita/iscrizioni}
\end{center}

\linkedSubsection{da:approvazione}
\begin{center}
	\includegraphics[width=\textwidth]{assets/visualParadigm/attivita/approvazioneIscrizione}
\end{center}

\linkedSubsection{da:schedaprodotto}
\begin{center}
	\includegraphics[width=0.5\textwidth]{assets/visualParadigm/attivita/schedaprodotto}
\end{center}

\linkedSubsection{da:ricerche}
\begin{center}
	\includegraphics[width=0.94\textwidth]{assets/visualParadigm/attivita/ricerche}
\end{center}

\linkedSubsection{da:valrec}
\begin{center}
	\includegraphics[width=0.5\textwidth]{assets/visualParadigm/attivita/valutazioneRecensione}
\end{center}

\linkedSubsection{da:oprec}
\begin{center}
	\includegraphics[width=\textwidth]{assets/visualParadigm/attivita/mostraRecensioni}
\end{center}

\section{Analisi nomi verbi} 
\jsognrepu In questo documento verrà descritto nel dettaglio il sistema nel suo complesso, in modo più approfondito rispetto al \docref{cha:documento_di_visione}.

\subsection{Analisi della specifica del sistema} 
Il portale che si vuole realizzare interagirà con diverse \grupporuolo{figure pubbliche}, che si distinguono tra autenticate e non autenticate. Fra le \grupporuolo{figure pubbliche autenticate} troviamo gli \ruolo{utenti} e i \ruolo{produttori}, mentre i \ruolo{visitatori} sono l'unica \grupporuolo{figura pubblica non autenticata}.
Sono presenti inoltre diverse \grupporuolo{figure amministrative}, quali i \ruolo{redattori}, gli \ruolo{assistenti}, i \ruolo{moderatori} e gli \ruolo{amministratori}.
Le diverse figure presenti si distinguono a seconda di quali funzionalità possono utilizzare. 

\bigskip
\noindent
Nel sistema sono presente diversi \emph{tipi di account}.
Ogni \emph{account} presente nel sistema possiede almeno un tipo e, a seconda dei tipi che possiede, gli vengono associati determinati privilegi, che gli permettono l'accesso alle funzionalità che ogni privilegio fornisce. In questo modo un singolo account può ricoprire più ruoli, è ad esempio possibile che un utente sia anche un redattore anche se quest'ultimo di base non è un utente.

I \ruolo{visitatori} possono effettuare l'iscrizione, diversa a seconda della tipologia di account che si vuole creare, e l'autenticazione. La creazione di un account \ruolo{utente} può essere effettuata tramite l'integrazione con Facebook.

Gli \ruolo{utenti} possono valutare oppure effettuare una recensione di un prodotto, la quale aggiunge una descrizione con eventuali immagini alla valutazione. Una volta inserita una recensione o una valutazione è possibile modificarla, oppure è possibile eliminare la propria recensione. 
Per permettere di visualizzare velocemente le recensioni più utili, gli \ruolo{utenti} possono dare un giudizio positivo oppure negativo alle recensioni presenti. Hanno inoltre la possibilità di commentare le recensioni, per permettere una maggiore interazione fra gli \ruolo{utenti}.
Per facilitare l'inserimento di informazioni nel portale sono state inoltre pensate due particolari funzionalità: l'\ruolo{utente} può suggerire un prodotto mancante ad un \ruolo{produttore} iscritto, e tale \ruolo{produttore} potrà scegliere se accettare o rifiutare la richiesta di inserimento, eventualmente modificando le informazioni inserite dall'\ruolo{utente}; inoltre l'\ruolo{utente} potrà suggerire al sistema un prodotto mancante di un \ruolo{produttore} esterno non ancora iscritto al portale.

I \ruolo{produttori} hanno a disposizione uno spazio per pubblicizzare i loro prodotti, che da questo momento in avanti chiameremo \emph{vetrina}. Essi potranno personalizzare la loro vetrina, inserendo una descrizione e un immagine. Potranno inserire prodotti o modificare quelli già inseriti. L'inserimento di prodotti seguirà una rigida procedura per garantire la consistenza dei dati e facilitare la ricerca dei prodotti. Potranno inoltre commentare le recensioni dei prodotti.

\bigskip
Le \grupporuolo{figure pubbliche autenticate} possono usufruire del sistema di follower/followed, possono quindi seguire altri account per ricevere aggiornamenti da essi, rimuovere account che si stanno seguendo oppure visualizzare gli aggiornamenti dagli account seguiti. Possono inoltre segnalare recensioni/commenti con contenuti non appropriati e aprire tickets per richiedere assistenza.

I \ruolo{visitatori} e gli \ruolo{utenti} possono ricevere consigli sui prodotti o sulle notizie simili a quelli che stanno visualizzando. 

Tutte le \grupporuolo{figure pubbliche} possono effettuare ricerche nel portale, in base a diversi criteri, di prodotti, di notizie e di profili pubblici.
Esse possono inoltre gestire il proprio account, con la possibilità di visualizzare il profilo (pubblico), accedere alle impostazioni (private), effettuare il logout oppure rimuovere l'account. Da notare che l'eventuale rimozione dell'account non comporterà la rimozione dei contenuti inseriti nel portale, ma solo la rimozione dello stesso. Possono inoltre visualizzare i contenuti pubblici del portale, quali le vetrine dei produttori, con eventuali statistiche, i singoli prodotti con le recensioni associate, i profili pubblici altrui con le statistiche associate a quell'account e le notizie presenti nel portale.

\bigskip
\noindent
Segue una descrizione delle funzionalità specifiche delle \grupporuolo{figure amministrative}.

I \ruolo{redattori} possono aggiungere, modificare o rimuovere notizie provenienti dal mondo dolciario.

Gli \ruolo{assistenti} possono rispondere ai tickets per risolvere i problemi degli \ruolo{utenti}. Hanno inoltre il compito di gestire le iscrizioni dei produttori effettuate direttamente nel portale, controllando la veridicità delle informazioni inserite, ed in caso creare l'account richiesto.

I \ruolo{moderatori} possono eliminare contenuti non appropriati presenti nel sistema, che possono essere segnalati dalle \grupporuolo{figure pubbliche autenticate}.

Gli \ruolo{amministratori} possono accedere al pannello di controllo del \gls{cms}, attraverso il quale possono configurare il portale in ogni suo aspetto. Può inoltre gestire gli account, creandone di nuovi, modificando i privilegi o rimuovendo quelli già presenti. 

\bigskip
\noindent
Nel portale che si vuole realizzare le informazioni sui prodotti sono di fondamentale importanza, perciò elenchiamo di seguito quelle contenute nel sistema:
\begin{itemize}
	\item \infoProdotto{Nome prodotto.}
	\item \infoProdotto{Descrizione di presentazione.}
	\item \infoProdotto{Almeno un'immagine del prodotto.}
	\item \infoProdotto{Marca.}
	\item \infoProdotto{Categoria.}
	\item \infoProdotto{Consistenza.}
	\item \infoProdotto{Particolarità.}
	\item \infoProdotto{Ingredienti.}
\end{itemize}
L'inserimento di queste informazioni è compito del \ruolo{produttore}, il quale seguirà una rigida procedura che sarà descritta nel dettaglio durante la specifica delle funzionalità.


\section{Carte CRC}
Le carte \gls{crc} sono uno strumento usato per impostare un progetto software object-oriented attraverso un processo di brainstorming.
Ciascuna carta descrive una classe in modo sommario, indicando:
\begin{itemize}
	\item Il nome della classe
 	\item La sua superclassi
 	\item Le sue sottoclassi
 	\item Le sue responsabilità
 	\item Le sue collaborazioni
\end{itemize}
Dove le sue collaborazioni sono altre classi con cui essa collabora per svolgere i compiti di cui è responsabile
Di seguito sono elencate le carte \gls{crc}:
\begin{itemize}%{key}{nome}{superclassi}{sottoclassi}
	\item \newClassItem{class:Account}{Account}{\noClasse}{\noClasse}
	\item \newClassItem{class:TipoAccount}{Tipo Account}{\noClasse}{\noClasse}
	\item \newClassItem{class:Privilegi}{Privilegi}{\noClasse}{\noClasse}
	\item \newClassItem{class:Contenuto}{Contenuto}{\noClasse}{\getTitletodesc{class:Commento}, \getTitletodesc{class:Recensione}, \getTitletodesc{class:Giudizio}, \getTitletodesc{class:Scheda Prodotto}, \getTitletodesc{class:Valutazione}, \getTitletodesc{class:Notizia}}
	\item \newClassItem{class:Commento}{Commento}{\getTitletodesc{class:Contenuto}}{\noClasse}
	\item \newClassItem{class:Recensione}{Recensione}{\getTitletodesc{class:Contenuto}}{\noClasse}
	\item \newClassItem{class:Giudizio}{Giudizio}{\getTitletodesc{class:Contenuto}}{\noClasse}
	\item \newClassItem{class:Scheda Prodotto}{Scheda Prodotto}{\getTitletodesc{class:Contenuto}}{\noClasse}
	\item \newClassItem{class:Valutazione}{Valutazione}{\getTitletodesc{class:Contenuto}}{\noClasse}
	\item \newClassItem{class:Notizia}{Notizia}{\getTitletodesc{class:Contenuto}}{\noClasse}
	\item \newClassItem{class:ConsistenzaProdotto}{Consistenza Prodotto}{\noClasse}{\noClasse}
	\item \newClassItem{class:CategoriaProdotto}{Categoria Prodotto}{\noClasse}{\noClasse}
	\item \newClassItem{class:Ticket}{Ticket}{\noClasse}{\getTitletodesc{class:RichiestaProduttore}}
	\item \newClassItem{class:RichiestaProduttore}{Richiesta Produttore}{\getTitletodesc{class:Ticket}}{\noClasse}
	\item \newClassItem{class:RispostaTicket}{Risposta Ticket}{\noClasse}{\noClasse}
	\item \newClassItem{class:RichiestaProdotto}{Richiesta Prodotto}{\noClasse}{\noClasse}
	\item \newClassItem{class:GestioneSegnalazioni}{Gestione Segnalazioni}{\noClasse}{\noClasse}
	\item \newClassItem{class:SegnalazioneContenuto}{Segnalazione Contenuto}{\noClasse}{\noClasse}
	\item \newClassItem{class:Profilo}{Profilo}{\noClasse}{\getTitletodesc{class:ProfiloProduttore}}
	\item \newClassItem{class:ProfiloProduttore}{Profilo Produttore}{\getTitletodesc{class:Profilo}}{\noClasse}
	\item \newClassItem{class:Vetrina}{Vetrina}{\noClasse}{\noClasse}
	\item \newClassItem{class:Gestione Accessi}{Gestione Accessi}{\noClasse}{\noClasse}
	\item \newClassItem{class:Accesso Iscritto}{Accesso Iscritto}{\noClasse}{\noClasse}
	\item \newClassItem{class:Accesso Ospite}{Accesso Ospite}{\noClasse}{\noClasse}
	\item \newClassItem{class:Utility}{Utility}{\noClasse}{\noClasse}
	\item \newClassItem{class:TipoRicerca}{Tipo Ricerca}{\noClasse}{\noClasse}
	\item \newClassItem{class:Anteprima}{Anteprima}{\noClasse}{\getTitletodesc{class:AnteprimaProdotto}, \getTitletodesc{class:AnteprimaProfilo}, \getTitletodesc{class:AnteprimaNotizia}}
	\item \newClassItem{class:AnteprimaProdotto}{Anteprima Prodotto}{\getTitletodesc{class:Anteprima}}{\noClasse}
	\item \newClassItem{class:AnteprimaProfilo}{Anteprima Profilo}{\getTitletodesc{class:Anteprima}}{\getTitletodesc{class:AnteprimaProfiloProduttore}}
	\item \newClassItem{class:AnteprimaProfiloProduttore}{Anteprima Profilo Produttore}{\getTitletodesc{class:AnteprimaProfilo}}{\noClasse}
	\item \newClassItem{class:AnteprimaNotizia}{Anteprima Notizia}{\getTitletodesc{class:Anteprima}}{\noClasse}
\end{itemize}

\crcTab{class:Account}
{\begin{itemWork}
	\item Memorizzazione delle informazioni account
\end{itemWork}}
{\begin{itemWork}
	\item Gestione Accessi, Segnalazione Contenuto, Ticket, Risposta Ticket, Profilo, Contenuto
\end{itemWork}}

\vspaceTab

\crcTab{class:TipoAccount}
{\begin{itemWork}
	\item Enumerazione delle possibili tipologie di account presenti nel sistema
\end{itemWork}}
{\begin{itemWork}
	\item Account, Privilegi
\end{itemWork}}

\vspaceTab

\crcTab{class:Privilegi}
{\begin{itemWork}
	\item Enumerazione dei possibili privilegi, per avere accesso alle diverse funzionalità, che un account può avere
\end{itemWork}}
{\begin{itemWork}
	\item Tipo account
\end{itemWork}}

\vspaceTab

\crcTab{class:Contenuto}
{\begin{itemWork}
	\item Generalizzazione dei vari tipi di contenuti presenti nel sistema
\end{itemWork}}
{\begin{itemWork}
	\item Segnalazione Contenuto, Account
\end{itemWork}}

\vspaceTab

\crcTab{class:Commento}
{\begin{itemWork}
	\item Memorizzazione delle informazioni necessarie per commentare una recensione
\end{itemWork}}
{\begin{itemWork}
	\item Recensione
\end{itemWork}}

\vspaceTab

\crcTab{class:Recensione}
{\begin{itemWork}
	\item Memorizzazione delle informazioni necessarie per recensire una scheda di un prodotto
	\item Possibilità di effettuare un commento
	\item Possibilità di effettuare un giudizio
\end{itemWork}}
{\begin{itemWork}
	\item Commento, Giudizio, Scheda Prodotto
\end{itemWork}}

\vspaceTab

\crcTab{class:Giudizio}
{\begin{itemWork}
	\item Memorizzazione delle informazioni necessarie per giudicare una recensione
\end{itemWork}}
{\begin{itemWork}
	\item Recensione
\end{itemWork}}

\vspaceTab

\crcTab{class:Scheda Prodotto}
{\begin{itemWork}
	\item Memorizzazione di tutte le informazioni relative al prodotto
	\item Possibilità di effettuare una recensione
	\item Possibilità di effettuare una valutazione
\end{itemWork}}
{\begin{itemWork}
	\item Recensione, Valutazione, Consistenza prodotto, Categoria prodotto, Richiesta produttore, Richiesta prodotto, Vetrina, Anteprima prodotto
\end{itemWork}}

\vspaceTab

\crcTab{class:Valutazione}
{\begin{itemWork}
	\item Memorizzazione delle informazioni necessarie per valutare una scheda prodotto
\end{itemWork}}
{\begin{itemWork}
	\item Scheda Prodotto
\end{itemWork}}

\vspaceTab

\crcTab{class:Notizia}
{\begin{itemWork}
	\item Memorizzazione delle informazioni necessarie per l'inserimento di una notizia
\end{itemWork}}
{\begin{itemWork}
	\item AnteprimaNotizia
\end{itemWork}}

\vspaceTab

\crcTab{class:ConsistenzaProdotto}
{\begin{itemWork}
	\item Enumerazione delle possibili consistenze di un prodotto
\end{itemWork}}
{\begin{itemWork}
	\item Scheda Prodotto
\end{itemWork}}

\vspaceTab

\crcTab{class:CategoriaProdotto}
{\begin{itemWork}
	\item Enumerazione delle possibili categorie di un prodotto
\end{itemWork}}
{\begin{itemWork}
	\item Scheda Prodotto
\end{itemWork}}

\vspaceTab

\crcTab{class:Ticket}
{\begin{itemWork}
	\item Memorizzazione delle informazioni necessarie per la creazione di un ticket
\end{itemWork}}
{\begin{itemWork}
	\item Risposta Ticket, Account
\end{itemWork}}

\vspaceTab

\crcTab{class:RichiestaProduttore}
{\begin{itemWork}
	\item Memorizzazione delle informazioni necessarie per la creazione di una richiesta di inserimento di un produttore mancante
\end{itemWork}}
{\begin{itemWork}
	\item Scheda Prodotto
\end{itemWork}}

\vspaceTab


\crcTab{class:RispostaTicket}
{\begin{itemWork}
	\item Memorizzazione delle informazioni necessarie per rispondere ad un ticket aperto
\end{itemWork}}
{\begin{itemWork}
	\item Ticket, Account
\end{itemWork}}

\vspaceTab

\crcTab{class:RichiestaProdotto}
{\begin{itemWork}
	\item Memorizzazione delle informazioni necessarie per la creazione di una richiesta di inserimento di un prodotto mancante
\end{itemWork}}
{\begin{itemWork}
	\item Scheda Prodotto, Profilo Produttore
\end{itemWork}}

\vspaceTab

\crcTab{class:GestioneSegnalazioni}
{\begin{itemWork}
	\item Singleton necessario per raccogliere tutte le segnalazioni di contenuti presenti nel sistema
\end{itemWork}}
{\begin{itemWork}
	\item Segnalazione contenuto
\end{itemWork}}

\vspaceTab

\crcTab{class:SegnalazioneContenuto}
{\begin{itemWork}
	\item Memorizzazione delle informazioni necessarie per la creazione di una segnalazione di un contenuto
\end{itemWork}}
{\begin{itemWork}
	\item Gestione Segnalazioni, Account, Contenuto
\end{itemWork}}

\vspaceTab

\crcTab{class:Profilo}
{\begin{itemWork}
	\item Memorizzazione di tutte le informazioni del profilo
	\item Memorizzazione della lista dei follower/followed del profilo
\end{itemWork}}
{\begin{itemWork}
	\item Account, AnteprimaProfilo
\end{itemWork}}

\vspaceTab

\crcTab{class:ProfiloProduttore}
{\begin{itemWork}
	\item Memorizzazione delle informazioni relative ad un profilo produttore
	\item Possibilità di visualizzare la vetrina associata
\end{itemWork}}
{\begin{itemWork}
	\item Richiesta Prodotto, Scheda Prodotto, Vetrina
\end{itemWork}}

\vspaceTab

\crcTab{class:Vetrina}
{\begin{itemWork}
    \item Memorizzazione delle informazioni e dei prodotti presenti nella vetrina di un dato produttore
	\item Possibilità di inserire prodotti
\end{itemWork}}
{\begin{itemWork}
	\item Profilo Produttore, Anteprima Profilo Produttore, Scheda Prodotto
\end{itemWork}}

\vspaceTab

\crcTab{class:Gestione Accessi}
{\begin{itemWork}
	\item Singleton necessario per raccogliere le informazioni relative agli accessi degli iscritti e dei visitatori
\end{itemWork}}
{\begin{itemWork}
	\item Account, Accesso Iscritto, Accesso Ospite
\end{itemWork}}

\vspaceTab


\crcTab{class:Accesso Iscritto}
{\begin{itemWork}
	\item Memorizzazione delle informazioni relative ad un accesso di un iscritto
\end{itemWork}}
{-}


\vspaceTab

\crcTab{class:Accesso Ospite}
{\begin{itemWork}
	\item Memorizzazione delle informazioni relative ad un accesso di un ospite
\end{itemWork}}
{-}

\vspaceTab

\crcTab{class:Utility}
{\begin{itemWork}
	\item Classe necessaria per raccogliere metodi statici utili al sistema
	\item Gestione ricerca
\end{itemWork}}
{-}

\vspaceTab

\crcTab{class:TipoRicerca}
{\begin{itemWork}
	\item Enumerazione delle tipologie di ricerca presenti nel sistema
\end{itemWork}}
{-}

\vspaceTab

\crcTab{class:Anteprima}
{\begin{itemWork}
	\item Generalizzazione delle possibili anteprime dei contenuti presenti nel sistema
\end{itemWork}}
{-}

\vspaceTab

\crcTab{class:AnteprimaProdotto}
{\begin{itemWork}
	\item Memorizzazione delle informazioni necessarie per mostrare l'anteprima di un dato prodotto
	\item Possibilità di mostrare l'anteprima del prodotto associato
\end{itemWork}}
{\begin{itemWork}
	\item Prodotto
\end{itemWork}}

\vspaceTab

\crcTab{class:AnteprimaProfilo}
{\begin{itemWork}
	\item Memorizzazione delle informazioni necessarie per mostrare l'anteprima di un dato profilo
	\item Possibilità di mostrare l'anteprima del profilo associato
\end{itemWork}}
{\begin{itemWork}
	\item Profilo
\end{itemWork}}

\vspaceTab

\crcTab{class:AnteprimaProfiloProduttore}
{\begin{itemWork}
	\item Memorizzazione delle informazioni necessarie per mostrare l'anteprima di un dato profilo produttore
	\item Possibilità di mostrare l'anteprima del profilo associato, che comprenderà la possibilità di andare sia al profilo del produtture che alla vetrina dello stesso
\end{itemWork}}
{\begin{itemWork}
	\item Vetrina
\end{itemWork}}

\vspaceTab

\crcTab{class:AnteprimaNotizia}
{\begin{itemWork}
	\item Memorizzazione delle informazioni necessarie per mostrare l'anteprima di una data notizia
	\item Possibilità di mostrare l'anteprima della notizia associata
\end{itemWork}}
{\begin{itemWork}
	\item Notizia
\end{itemWork}}


\begin{landscape}
\section{Diagramma dei package}  \label{cha:package}
Il diagramma dei package sono diagrammi di struttura che descrive i package del sistema e le relazioni tra essi.
Un Package è uno spazio dei nomi utilizzato per raggruppare elementi che sono in relazione semantica tra loro.
\begin{center}
			\includegraphics[width=\linewidth]{assets/visualParadigm/package/DiagrammaPacakage}
\end{center}
\end{landscape}

\section{Diagrammi delle classi}  \label{cha:classi}
I diagrammi delle classi servono a modellare la relazione tra le entità del sistema, rappresentate come classi.
Una classe è composta da delle caratteristiche che la descrivono (attributi) e delle elaborazioni che vengono eseguite (metodi). 
Le classi servono ad identificare chi fa che cosa e come all’interno del sistema. Queste possono avere relazioni tra loro, le quali sono rappresentate con le associazioni. La cardinalità di un’associazione esprime il numero di oggetti di una certa classe che prendono parte all’associazione. Di seguito sono elencati i diagrammi delle classi:
\begin{itemize}
	\item \newPKG{pkg:GestioneAnteprime}{\formattaPKG}{Gestione anteprima}
	\item \newPKG{pkg:GestioneAutenticazione}{\formattaPKG}{Gestione autenticazione}
	\item \newPKG{pkg:GestioneIscritti}{\formattaPKG}{Gestione iscritti}
	\item \newPKG{pkg:GestioneTicket}{\formattaPKG}{Gestione ticket}
	\item \newPKG{pkg:GestioneProfilo}{\formattaPKG}{Gestione profilo}
	\item \newPKG{pkg:GestioneContenuti}{\formattaPKG}{Gestione contenuti}
	\item \newPKG{pkg:GestioneSistema}{\formattaPKG}{Gestione sistema}
\end{itemize}

\linkedSubsection{pkg:GestioneAnteprime}
\begin{center}
			\includegraphics[width=\textwidth]{assets/visualParadigm/classi/GestioneAnteprime}
\end{center}

\linkedSubsection{pkg:GestioneAutenticazione}
\begin{center}
			\includegraphics[width=\textwidth]{assets/visualParadigm/classi/GestioneAutenticazione}
\end{center}

\linkedSubsection{pkg:GestioneIscritti}
\begin{center}
			\includegraphics[width=\textwidth]{assets/visualParadigm/classi/GestioneIscritti}
\end{center}

\linkedSubsection{pkg:GestioneTicket}
\begin{center}
			\includegraphics[width=\textwidth]{assets/visualParadigm/classi/GestioneTicket}
\end{center}

\begin{landscape}
\linkedSubsection{pkg:GestioneProfilo}
\begin{center}
			\includegraphics[width=\linewidth]{assets/visualParadigm/classi/GestioneProfilo}
\end{center}
\end{landscape}

\begin{landscape}
\linkedSubsection{pkg:GestioneContenuti}
\begin{center}
			\includegraphics[width=1.1\linewidth]{assets/visualParadigm/classi/GestioneContenuti}
\end{center}
\end{landscape}

\linkedSubsection{pkg:GestioneSistema}
\begin{center}
			\includegraphics[width=\textwidth]{assets/visualParadigm/classi/GestioneSistema}
\end{center}


\section{Diagrammi di sequenza}  \label{cha:sequenza}
I diagrammi di sequenza descrivono le interazioni tra gli oggetti organizzate in sequenza temporale. Ogni caso d’uso contiene al suo interno diversi diagrammi di sequenza. Rappresenteremo solo le sequenze principali.
Un diagramma di sequenza è costituito da:
\begin{itemize}
	\item Gli oggetti
	\item I messaggi attraverso cui essi interagiscono
\end{itemize}
Di seguito sono elencati i diagrammi di sequenza, non verranno riportati tutti i diagrammi di sequenza, ma solo quelli corrispondenti ai casi d'uso principali.
\begin{itemize}
	\item \newSequenza{seq:login}{\formattaSEQ}{Login}
	\item \newSequenza{seq:logout}{\formattaSEQ}{Logout}
	\item \newSequenza{seq:iscrizioneProduttore}{\formattaSEQ}{Iscrizione Produttore}
	\item \newSequenza{seq:iscrizioneUtente}{\formattaSEQ}{Iscrizione Utente}
	\item \newSequenza{seq:approvazioneIscrizione}{\formattaSEQ}{Approvazione iscrizione}
	\item \newSequenza{seq:inserimentoSchedaProdotto}{\formattaSEQ}{Inserimento scheda prodotto}
	\item \newSequenza{seq:inserimentoValutazione}{\formattaSEQ}{Inserimento valutazione}
	\item \newSequenza{seq:inserimentoRecensione}{\formattaSEQ}{Inserimento recensione}
	\item \newSequenza{seq:rimuoviRecensione}{\formattaSEQ}{Rimuovi recensione}
	\item \newSequenza{seq:commentoRecensione}{\formattaSEQ}{Commento recensione}
	\item \newSequenza{seq:giudicaRecensione}{\formattaSEQ}{Giudica recensione}
	\item \newSequenza{seq:ricerca}{\formattaSEQ}{Ricerca}
\end{itemize}

\linkedSubsection{seq:login}
\begin{center}
			\includegraphics[width=\textwidth]{assets/visualParadigm/sequenza/login}
\end{center}

\linkedSubsection{seq:logout}
\begin{center}
			\includegraphics[width=\textwidth]{assets/visualParadigm/sequenza/logout}
\end{center}

\linkedSubsection{seq:iscrizioneProduttore}
\begin{center}
			\includegraphics[width=\textwidth]{assets/visualParadigm/sequenza/iscrizioneProduttore}
\end{center}

\linkedSubsection{seq:iscrizioneUtente}
\begin{center}
			\includegraphics[width=\textwidth]{assets/visualParadigm/sequenza/iscrizioneUtente}
\end{center}

\linkedSubsection{seq:approvazioneIscrizione}
\begin{center}
			\includegraphics[width=\textwidth]{assets/visualParadigm/sequenza/approvazioneIscrizione}
\end{center}

\linkedSubsection{seq:inserimentoSchedaProdotto}
\begin{center}
			\includegraphics[width=\textwidth]{assets/visualParadigm/sequenza/inserimentoSchedaProdotto}
\end{center}

\linkedSubsection{seq:inserimentoValutazione}
\begin{center}
			\includegraphics[width=\textwidth]{assets/visualParadigm/sequenza/inserimentoValutazione}
\end{center}

\linkedSubsection{seq:inserimentoRecensione}
\begin{center}
			\includegraphics[width=\textwidth]{assets/visualParadigm/sequenza/inserimentoRecensione}
\end{center}

\linkedSubsection{seq:rimuoviRecensione}
\begin{center}
			\includegraphics[width=\textwidth]{assets/visualParadigm/sequenza/rimuoviRecensione}
\end{center}

\linkedSubsection{seq:commentoRecensione}
\begin{center}
			\includegraphics[width=\textwidth]{assets/visualParadigm/sequenza/commentoRecensione}
\end{center}

\linkedSubsection{seq:giudicaRecensione}
\begin{center}
			\includegraphics[width=\textwidth]{assets/visualParadigm/sequenza/giudicaRecensione}
\end{center}

\linkedSubsection{seq:ricerca}
\begin{center}
			\includegraphics[width=\textwidth]{assets/visualParadigm/sequenza/ricerca}
\end{center}


\begin{comment}
-	CU.A.1 - Login
-	CU.A.3 - Logout
-	CU.B.1 - Iscrizione tramite modulo
-	CU.B.2 - Iscrizione tramite Social Network
-	CU.B.3 - Iscrizione tramite approvazione
-	CU.B.4 - Approvazione iscrizione
-	CU.C.5 - Inserimento scheda prodotto

	CU.G.1 - Inserimento valutazione
	CU.H.1 - Inserimento recensione
	CU.H.4 - Commento a recensione
	CU.H.5 - Giudizio recensione

-	CU.K.1 - Ricerca prodotto
-	CU.K.2 - Ricerca profilo
-	CU.K.3 - Ricerca notizia
-	CU.M.1 - Mostra vetrina
-	CU.M.2 - Mostra scheda prodotto
-	CU.M.4 - Mostra recensione
-	CU.M.5 - Mostra profilo pubblico
-	CU.M.6 - Mostra notizia
\end{comment}
