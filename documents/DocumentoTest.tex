\chapter{Piano dei test}
\section{Introduzione}
In questo documento vengono definiti i test funzionali, necessari per garantire il corretto comportamento del sistema.
L'utilizzo di un \gls{cms} commerciale e dei suoi plugin, di cui assumiamo corretto il funzionamento (in quanto prodotti ampiamente diffusi e utilizzati negli ambiti più disparati con successo), ci permette di ridurre il numero di test necessari.
Infatti i \docref{sub:requisiti_non_funzionali} possono essere considerati come soddisfatti, così come alcuni dei \docref{sub:requisiti_funzionali}.

\section{Elenco test funzionali}
I test funzionali garantiscono il rispetto dei casi d'uso da parte del sistema.
Vengono di seguito elencati i test funzionali:
\begin{itemize}
	\item \newTestItem{cu:login}
	\item \newTestItem{cu:logout}
	\item \newTestItem{cu:iscrizionePortale}
	\item \newTestItem{cu:iscrizioneApprovazione}
	\item \newTestItem{cu:approvazioneIscrizione}
	\item \newTestItem{cu:personalizzaVetrinaInsProd}
	\item \newTestItem{cu:inserisciRecensioneProdotto}
	\item \newTestItem{cu:eliminaRecensioneProdotto}
	\item \newTestItem{cu:commentoRecensione}
	\item \newTestItem{cu:giudizioRecensione}
	\item \newTestItem{cu:ricercaProdotto}
	\item \newTestItem{cu:ricercaProfilo}
	\item \newTestItem{cu:ricercaNotizia}
\end{itemize}

\section{Descrizione test funzionali}
%{Obiettivo}{Prerequisiti}{Azioni}
\testTab{cu:login}
{Testare il corretto funzionamento della funzionalità di Login, controllando anche che in caso di inserimento di dati errati venga restituito l'errore previsto}
{Le credenziali (username1, password1) sono presenti nel sistema, le credenziali (username2, passoword2) non sono presenti nel sistema. La persona non è autenticata nel sistema}
{\begin{enumCU}
	\item Inserire le credenziali (username1, password1)
	\item Effettuare il Login
	\item Verificare che il Login abbia successo
	\item Effettuare il Logout
	\item Inserire le credenziali (username2, passoword2)
	\item Effettuare il Login
	\item Verificare che il Login non abbia successo e che venga mostrato l'errore di credenziali errate
\end{enumCU}
}

\vspaceTab

\testTab{cu:logout}
{Testare il corretto funzionamento della funzionalità di Logout}
{La persona è autenticata nel sistema}
{\begin{enumCU}
	\item Effettuare il Logout
	\item Verificare che si è disconnesi dal sistema
\end{enumCU}
}

\vspaceTab

\testTab{cu:iscrizionePortale}
{Testare il corretto funzionamento della funzionalità di registrazione al sistema per creare account di tipo Utente}
{Nelle credenziali (username1, password1, email1) né l'username1 né l'email1 sono presenti nel sistema. Nelle (username2, password2, email2) l'username2 è presente nel sistema, mentre l'email2 non è presente nel sistema. Nelle (username3, password3, email3) l'username3 non è presente nel sistema, mentre l'email3 è presente nel sistema. La persona non è utenticata nel sistema}
{\begin{enumCU}
	\item Avviare la procedura di registrazione per creare un account Utente
	\item Compilare il form con le credenziali (username1, password1, email1)
	\item Confermare l'operazione di registrazione e verificare che avvenga con successo
	\item Avviare la procedura di registrazione per creare un account Utente
	\item Compilare il form con le credenziali (username2, password2, email2)
	\item Confermare l'operazione di registrazione e verificare che venga mostrato l'errore ``Errore username già esistente''
	\item Avviare la procedura di registrazione per creare un account Utente
	\item Compilare il form con le credenziali (username3, password3, email3)
	\item Confermare l'operazione di registrazione e verificare che venga mostrato l'errore ``Errore email già esistente''
\end{enumCU}}

\vspaceTab

\testTab{cu:iscrizioneApprovazione}
{Testare il corretto funzionamento della funzionalità di registrazione al sistema per creare richieste di creazione account di tipo Produttore}
{Nelle credenziali (username1, password1, email1, richiesta1) né l'username1 né l'email1 sono presenti nel sistema. Nelle (username2, password2, email2, richiesta2) l'username2 è presente nel sistema, mentre l'email2 non è presente nel sistema. Nelle (username3, password3, email3, richiesta3) l'username3 non è presente nel sistema, mentre l'email3 è presente nel sistema. La persona non è utenticata nel sistema}
{\begin{enumCU}
	\item Avviare la procedura di registrazione per creare un account Produttore
	\item Compilare il form con le credenziali (username1, password1, email1, richiesta1)
	\item Confermare l'operazione di registrazione e Verificare che la richiesta di creazione account sia avvenuta con successo
	\item Avviare la procedura di registrazione per creare un account Produttore
	\item Compilare il form con le credenziali (username2, password2, email2, richiesta2)
	\item Confermare l'operazione di registrazione e verificare che venga mostrato l'errore ``Errore username già esistente''
	\item Avviare la procedura di registrazione per creare un account Produttore
	\item Compilare il form con le credenziali (username3, password3, email3, richiesta3)
	\item Confermare l'operazione di registrazione e verificare che venga mostrato l'errore ``Errore email già esistente''
\end{enumCU}}

\vspaceTab

\testTab{cu:approvazioneIscrizione}
{Testare il corretto funzionamento della funzionalità di approvazione delle iscrizioni per creare account di tipo Produttore}
{È presente una richiesta di iscrizione (username1, password1, email1, richiesta1) in cui richiesta1 dimostri la validità del produttore che si vuole iscrivere. È presente una richiesta di iscrizione (username2, password2, email2, richiesta2) in cui richiesta2 non dimostra la validità del produttore che si vuole iscrivere. La persona è autenticata nel sistema ed ha i privilegi di Assistente}
{\begin{enumCU}
    \item Aprire la richiesta (username1, *****, email1, richiesta1)
    \item Confermare la richiesta di creazione dell'account
    \item Verificare che la creazione dell'account sia avvenuta con successo, controllando che sia presente un account con credenziali (username1, *****, email1)
    \item Verificare che la richiesta (username1, *****, email1, richiesta1) non sia più presente
    \item Aprire la richiesta (username2, *****, email2, richiesta2)
    \item Rifiutare la richiesta di creazione dell'account 
    \item Verificare che la richiesta non sia più presente, controllando che non sia presente un account con credenziali (username2, *****, email2)
\end{enumCU}}

\vspaceTab

\testTab{cu:personalizzaVetrinaInsProd}
{Testare il corretto funzionamento della funzionalità di inserimento di una scheda prodotto}
{La persona è autenticata nel sistema ed ha i privilegi di Produttore. I dati (nome, descrizionbe, immagini, marca, categoria, consistenza, paricolarità, ingredienti) sono relativi ad una possibile scheda prodotto}
{\begin{enumCU}
    \item Selezionare inserimento scheda prodotto
    \item Compilare i campi con i dati (nome, descrizionbe, immagini, marca, categoria, consistenza, paricolarità, ingredienti)
    \item Annullare l'operazione di inserimento
    \item Verificare che il prodotto non sia presente nella vetrina del produttore
    \item Selezionare inserimento scheda prodotto
    \item Compilare i campi con i dati (nome, descrizionbe, immagini, marca, categoria, consistenza, paricolarità, ingredienti)
    \item Confermare l'operazione di inserimento
    \item Verificare che il prodotto inserito sia presente nella vetrina del produttore
\end{enumCU}}

\vspaceTab

\testTab{cu:inserisciRecensioneProdotto}
{Testare il corretto funzionamento della funzionalità di inserimeto di una recensione di un prodotto}
{La persona è autenticata nel sistema ed ha i privilegi di Utente. La persona sta visualizzando la scheda del prodotto A. I dati (testo, immagini) sono relativi ad una possibile recensione}
{\begin{enumCU}
	\item Iniziare la procedura per inserire la recensione del prodotto A
	\item Compilare i campi con i dati (testo, immagini)
	\item Annullare l'operazione
	\item Verificare che la recensione non sia presente
	\item Iniziare la procedura per inserire la recensione del prodotto A
	\item Compilare i campi con i dati (testo, immagini)
	\item Confermare l'operazione
	\item Verificare che la recensione sia presente
\end{enumCU}}

\vspaceTab

\testTab{cu:eliminaRecensioneProdotto}
{Testare il corretto funzionamento della funzionalità di rimozione recensione di un prodotto}
{La persona è autenticata nel sistema ed ha i privilegi di Utente. La persona sta visualizzando la recensione R che ha inserito per un prodotto A presente nel sistema}
{\begin{enumCU}
	\item Iniziare la procedura per rimuovere la recensione R
	\item Annullare l'operazione
	\item Verificare che la recensione sia ancora presente
	\item Iniziare la procedura per rimuovere la recensione R
	\item Confermare l'operazione
	\item Verificare che la recensione non sia più presente
\end{enumCU}}

\vspaceTab

\testTab{cu:commentoRecensione}
{Testare il corretto funzionamento della funzionalità di inserimento commento a recensione}
{La persona è autenticata nel sistema ed ha i privilegi di Utente oppure Produttore. La persona sta visualizzando una recensione R di un prodotto presente nel sistema. I dati (testo, immagini) sono relativi ad un possibile commento}
{\begin{enumCU}
	\item Iniziare la procedura di inserimento commento per la recensione R
	\item Compilare i campi con i dati (testo, immagini)
	\item Annullare l'operazione 
	\item Verificare che il commento per la recensione R non sia presente
	\item Iniziare la procedura di inserimento commento per la recensione R
	\item Compilare i campi con i dati (testo, immagini)
	\item Confermare l'operazione 
	\item Verificare che il commento per la recensione R sia presente
\end{enumCU}}

\vspaceTab

\testTab{cu:giudizioRecensione}
{Testare il corretto funzionamento della funzionalità di inserimento giudizio a recensione}
{La persona è autenticata nel sistema ed ha i privilegi di Utente oppure Produttore. La persona sta visualizzando una recensione R di un prodotto presente nel sistema}
{\begin{enumCU}
	\item Inserire un giudizio positivo o negativo per la recensione R
	\item Verificare il giudizio inserito per la recensione R sia presente
\end{enumCU}}

\vspaceTab

\testTab{cu:ricercaProdotto}
{Testare il corretto funzionamento della funzionalità di ricerca prodotto}
{La persona è connessa al sistema. Il ``testo'' è la keyword che si vuole utilizzare per ricercare prodotti}
{\begin{enumCU}
	\item Inserire il ``testo'' nell'apposito campo per la ricerca
	\item Selezionare la tipologia di ricerca ``Prodotto''
	\item Annullare la ricerca
	\item Verificare che non vengano mostrati le anteprime dei risultati della ricerca
	\item Inserire il ``testo'' nell'apposito campo per la ricerca
	\item Selezionare la tipologia di ricerca ``Prodotto''
	\item Confermare la ricerca
	\item Verificare che siano mostrati come risultato di ricerca solamente le anteprime di schede prodotti
\end{enumCU}}

\vspaceTab

\testTab{cu:ricercaProfilo}
{Testare il corretto funzionamento della funzionalità di ricerca profilo}
{La persona è connessa al sistema. Il ``testo'' è la keyword che si vuole utilizzare per ricercare prodotti}
{\begin{enumCU}
	\item Inserire il ``testo'' nell'apposito campo per la ricerca
	\item Selezionare la tipologia di ricerca ``Profilo''
	\item Annullare la ricerca
	\item Verificare che non vengano mostrati le anteprime dei risultati della ricerca
	\item Inserire il ``testo'' nell'apposito campo per la ricerca
	\item Selezionare la tipologia di ricerca ``Profilo''
	\item Confermare la ricerca
	\item Verificare che siano mostrati come risultato di ricerca solamente le anteprime di profili
\end{enumCU}}

\vspaceTab

\testTab{cu:ricercaNotizia}
{Testare il corretto funzionamento della funzionalità di ricerca notizia}
{La persona è connessa al sistema. Il ````testo'''' è la keyword che si vuole utilizzare per ricercare prodotti}
{\begin{enumCU}
	\item Inserire il ``testo'' nell'apposito campo per la ricerca
	\item Selezionare la tipologia di ricerca ``Notizia''
	\item Annullare la ricerca
	\item Verificare che non vengano mostrati le anteprime dei risultati della ricerca
	\item Inserire il ``testo'' nell'apposito campo per la ricerca
	\item Selezionare la tipologia di ricerca ``Notizia''
	\item Confermare la ricerca
	\item Verificare che siano mostrati come risultato di ricerca solamente le anteprime di notizie
\end{enumCU}}

\newdate{testuno}{25}{10}{2016}
\section{Revisioni}
\begin{center}
	\begin{tabular}
	{lll}
		\toprule
		\tabhead{Versione} & \tabhead{Data} & \tabhead{Descrizione} \\
		\cmidrule(l{\cmidrulekern}r{\cmidrulekern}){1-3}
		1.0 & \displaydate{testuno} & Prima versione \\        
		\bottomrule
	\end{tabular}
\end{center}