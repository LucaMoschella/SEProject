\chapter{Documento di visione} 
\label{cha:documento_di_visione}

\section{Introduzione}
\label{sec:introduzione}
Dopo un'attenta analisi della richiesta dell'associazione ChocoClub, abbiamo stilato il seguente documento di visione che ha l'obbiettivo di descrivere ad alto livello il sistema che si vuole realizzare ed il suo scopo, e di introdurre le principali figure che interagiranno con esso, assieme alle loro necessità. Inoltre viene effettuato uno studio della fattibilità per stabilire se tale sistema è realizzabile con le attuali tecnologia a disposizione.

\section{Obbiettivi sistema} % (fold)
\label{sec:descrizione_sistema}
I principali obbiettivi del sistema che si intende realizzare sono riassumibili nei seguenti punti:
\begin{itemize}
	\item Fornire un portale capace di unificare le informazioni del mondo dolciario focalizzandosi sulla cioccolata.
	\item Fornire un punto di incontro fra produttori e clienti
	\item 
\end{itemize}

\section{Descrizione figure} 
\label{sec:descrizionefigure}
Le figure che potranno utilizzare il portale si dividono in tre categorie:
\begin{itemize}
	\item \ruolo{Visitatori}
	\item \ruolo{Utenza}
	\item \ruolo{Figure amministrative}
\end{itemize}

\subsection{Visitatori} % (fold)
\label{sub:}
I visitatori, coloro che non sono stati identificati dal portale, potranno usufruire dei suoi servizi pubblici.
\begin{descriptionInd}
    \item[Visitatori] useranno il portale per ricercare prodotti e per leggere notizie presenti nel sito.
\end{descriptionInd}



\subsection{Utenza}
\label{sub:utenza}
L'utenza è rappresentata da coloro che, dopo aver effettuato la procedura di autenticazione, utilizzerà il portale per usufruire dei suoi servizi. Essi si dividono in due profili differenti:
\begin{descriptionInd}
    \item[Produttori] useranno il portale per pubblicizzare i propri prodotti.

    \item[Utenti] useranno il portale per ricercare prodotti, con la possibilità di recensirli e/o valutarli; per seguire altra utenza e ricevere aggiornamenti da essa e per leggere notizie presenti nel sito.
\end{descriptionInd}




\subsection{Figure amministrative}
\label{sub:figureamministrative}
Le figure amministrative sono coloro che utilizzeranno il portale con lo scopo di fornire servizi all'utenza.
\begin{descriptionInd}
   % \item[Web Admin] avrà il compito di configurare e mantenere il portale, sia lato backend che frontend.
    \item[Redattori] useranno il portale per pubblicare notizie e articoli inerenti al mondo dolciario.   
    \item[Assistenti] useranno il portale per fornire supporto all'utenza.
    \item[Moderatori] useranno il portale per gestire segnalazioni di utenti.
\end{descriptionInd}


\section{Tabelle necessità o esigenze} % (fold)
\label{sec:tabelle_necessita_o_esigenze}
Nella seguente tabella riassumiamo le necessità o esigenze delle diverse figure:
\begin{center}
	\begin{tabularx}{0.8\textwidth}{l X}
	\toprule 
		\tabhead{Figura} & \tabhead{Necessità o Esigenze} \\
	\midrule
		\ruolo{Visitatori} & Ricercare prodotti, visualizzare prodotti, visualizzare notizie, autenticarsi o iscriversi.  \\
		\addlinespace[0.75em]
		\ruolo{Produttori} & Inserire prodotti nello spazio dedicato, rimuovere prodotti dallo spazio dedicato, modificare i prodotti inseriti.  \\
		\ruolo{Utenti} & Cercare prodotti, visualizzare prodotti, valutare/recensire prodotti, visualizzare notizie, seguire altra utenza. \\
		\addlinespace[0.75em]
		\ruolo{Redattori} & Pubblicare notizie, modificare notizie, eliminare notizie.  \\
		\ruolo{Assistenti} & Rispondere alle richieste dell`utenza  \\
		\ruolo{Moderatori} & Eliminare contenuti inappropriati, modificare contenuti inappropriati.  \\
	\bottomrule
	\end{tabularx}
\end{center}


\section{Studio di fattiblità}
\label{sec:studio_di_fatt}

\newdate{visuno}{22}{04}{2016}
\section{Revisioni}
\begin{center}
    \begin{tabular}{lll}
        \toprule
        Versione & Data & Descrizione \\
        %   \cmidrule{2-3} 
        \midrule
        1.0 & \displaydate{visuno} & Prima versione \\
        \bottomrule
    \end{tabular}
\end{center}


