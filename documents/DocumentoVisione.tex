\chapter{Documento di visione} 
\label{cha:documento_di_visione}

\section{Introduzione}
\label{sec:introduzione}
Dopo un'attenta analisi della richiesta dell'associazione ChocoClub ed un esteso studio del contesto di utilizzo, abbiamo stilato il seguente documento di visione, che ha lo scopo di introdurre le principali figure del sistema proposto assieme alle loro necessità e studiare la fattibilità dello stesso.

\section{Descrizione figure} 
\label{sec:descrizionefigure}
Le figure che potranno utilizzare il portale si dividono in tre categorie:
\begin{itemize}
	\item \ruolo{Visitatori}
	\item \ruolo{Utenza}
	\item \ruolo{Figure amministrative}
\end{itemize}

\subsection{Visitatori} % (fold)
\label{sub:}
I visitatori, coloro che non sono stati identificati dal portale, utilizzeranno il portale per usufruire dei suoi servizi pubblici.
\begin{descriptionInd}
    \item[Visitatori] useranno il portale per ricercare prodotti, con la possibilità di recensirli e/o valutarli; per seguire altre utenze e ricevere aggiornamenti da esse e per leggere notizie presenti nel sito.
\end{descriptionInd}


% subsection subsection_name (end)
\subsection{Utenza}
\label{sub:utenza}
L'utenza è rappresentata da coloro che, dopo aver effettuato la procedura di autenticazione, utilizzerà il portale per usufruire dei suoi servizi. Essi si dividono in due profili differenti:
\begin{descriptionInd}
    \item[Produttori] useranno il portale per pubblicizzare i propri prodotti.

    \item[Utenti] useranno il portale per ricercare prodotti, con la possibilità di recensirli e/o valutarli; per seguire altre utenze e ricevere aggiornamenti da esse e per leggere notizie presenti nel sito.
\end{descriptionInd}

\noindent
Nella seguente tabella riassumiamo le necessità o esigenze delle utenze:
\begin{center}
	\begin{tabularx}{0.8\textwidth}{l X}
	\toprule 
		\tabhead{Utenze} & \tabhead{Necessità o Esigenze} \\
	\midrule
		\ruolo{Produttori} & Pubblicizzare se stessi e i loro prodotti.  \\ \\
		\ruolo{Utenti} & Cercare prodotti, visualizzare prodotti, valutare/recensire prodotti, visualizzare notizie, seguire altre utenze. \\
	\bottomrule
	\end{tabularx}
\end{center}


\subsection{Figure amministrative}
\label{sub:figureamministrative}
Le figure amministrative sono coloro che utilizzeranno il portale con lo scopo di fornire servizi all'utenz.
\begin{descriptionInd}
   % \item[Web Admin] avrà il compito di configurare e mantenere il portale, sia lato backend che frontend.

    \item[Newsers] useranno il portale per pubblicare notizie e articoli inerenti al mondo dolciario.
    
    \item[Supports] useranno il portale per fornire supporto all'utenza.
    
    \item[Moderatori] useranno il portale per gestire segnalazioni di utenti.
\end{descriptionInd}

\noindent
Nella seguente tabella riassumiamo le necessità o esigenze delle figure amministrative:
\begin{center}
	\begin{tabularx}{0.8\textwidth}{l X}
	\toprule 
		\tabhead{Figure Amministrative} & \tabhead{Necessità o Esigenze} \\
	\midrule
		%\ruolo{Web Admin} & Pubblicizzare se stessi e i loro prodotti.  \\ \\
		\ruolo{Newsers} & pubblicare notizie, modificare notizie, eliminare notizie.  \\ \\
		\ruolo{Supports} & rispondere alle richieste dell`utenza'  \\ \\
		\ruolo{Moderatori} & eliminare contenuti inappropriati, modificare contenuti inappropriati.  \\ \\
	\bottomrule
	\end{tabularx}
\end{center}


\section{Studio di fattiblità}
\label{sec:studio_di_fatt}


