\chapter{Documento di visione} 
\label{cha:documento_di_visione}

\section{Introduzione}
\label{sec:introduzione}
Dopo un'attenta analisi della richiesta dell'associazione ChocoClub, abbiamo stilato il seguente documento di visione, che ha lo scopo di introdurre le principali figure del sistema proposto, assieme alle loro necessità e studiare la fattibilità dello stesso.

\section{Descrizione figure} 
\label{sec:descrizionefigure}
Le figure che potranno utilizzare il portale si dividono in tre categorie:
\begin{itemize}
	\item \ruolo{Visitatori}
	\item \ruolo{Utenza}
	\item \ruolo{Figure amministrative}
\end{itemize}

\subsection{Visitatori} % (fold)
\label{sub:}
I visitatori, coloro che non sono stati identificati dal portale, potranno usufruire dei suoi servizi pubblici.
\begin{descriptionInd}
    \item[Visitatori] useranno il portale per ricercare prodotti e per leggere notizie presenti nel sito.
\end{descriptionInd}



% subsection subsection_name (end)
\subsection{Utenza}
\label{sub:utenza}
L'utenza è rappresentata da coloro che, dopo aver effettuato la procedura di autenticazione, utilizzerà il portale per usufruire dei suoi servizi. Essi si dividono in due profili differenti:
\begin{descriptionInd}
    \item[Produttori] useranno il portale per pubblicizzare i propri prodotti.

    \item[Utenti] useranno il portale per ricercare prodotti, con la possibilità di recensirli e/o valutarli; per seguire altra utenza e ricevere aggiornamenti da essa e per leggere notizie presenti nel sito.
\end{descriptionInd}




\subsection{Figure amministrative}
\label{sub:figureamministrative}
Le figure amministrative sono coloro che utilizzeranno il portale con lo scopo di fornire servizi all'utenza.
\begin{descriptionInd}
   % \item[Web Admin] avrà il compito di configurare e mantenere il portale, sia lato backend che frontend.
    \item[Redattori] useranno il portale per pubblicare notizie e articoli inerenti al mondo dolciario.   
    \item[Assistenti] useranno il portale per fornire supporto all'utenza.
    \item[Moderatori] useranno il portale per gestire segnalazioni di utenti.
\end{descriptionInd}


\section{Tabelle necessità o esigenze} % (fold)
\label{sec:tabelle_necessita_o_esigenze}
Nella seguente tabella riassumiamo le necessità o esigenze delle diverse figure:
\begin{center}
	\begin{tabularx}{0.8\textwidth}{l X}
	\toprule 
		\tabhead{Figura} & \tabhead{Necessità o Esigenze} \\
	\midrule
		\ruolo{Visitatori} & ricercare prodotti, visualizzare prodotti, visualizzare notizie, autenticarsi o iscriversi.  \\
		\addlinespace
		\ruolo{Produttori} & inserire prodotti nello spazio dedicato, rimuovere prodotti dallo spazio dedicato, modificare i prodotti inseriti.  \\
		\ruolo{Utenti} & Cercare prodotti, visualizzare prodotti, valutare/recensire prodotti, visualizzare notizie, seguire altra utenza. \\
		\addlinespace
		\ruolo{Redattori} & pubblicare notizie, modificare notizie, eliminare notizie.  \\
		\ruolo{Assistenti} & rispondere alle richieste dell`utenza  \\
		\ruolo{Moderatori} & eliminare contenuti inappropriati, modificare contenuti inappropriati.  \\
	\bottomrule
	\end{tabularx}
\end{center}


\section{Studio di fattiblità}
\label{sec:studio_di_fatt}

\newdate{visuno}{22}{04}{2016}
\section{Revisioni}
\begin{center}
    \begin{tabular}{lll}
        \toprule
        Versione & Data & Descrizione \\
        %   \cmidrule{2-3} 
        \midrule
        1.0 & \displaydate{visuno} & Prima versione \\
        \bottomrule
    \end{tabular}
\end{center}


