\chapter{Documento di visione} 
\label{cha:documento_di_visione}

\section{Introduzione}
\label{sec:introduzione}
Dopo un'attenta analisi della richiesta dell'associazione ChocoClub, abbiamo stilato il seguente documento di visione che ha l'obiettivo di descrivere ad alto livello il portale che si vuole realizzare ed il suo scopo, e di introdurre le principali figure che interagiranno con esso, assieme alle loro necessità. Inoltre viene effettuato uno studio della fattibilità per stabilire se tale portale è realizzabile con le attuali tecnologie a disposizione.

\section{Obiettivi portale} 
\label{sec:descrizione_portale}
I principali obiettivi del portale che si intende realizzare sono riassumibili nei seguenti punti:
\begin{itemize}
	\item Fornire un portale capace di unificare le informazioni dal mondo dolciario, focalizzandosi sui prodotti contenenti cioccolata.
	\item Fornire un punto di incontro fra produttori e utenti.
	\item Fornire alle figure pubbliche autenticate metodi di interazione.
\end{itemize}

\section{Descrizione portale}
\label{sec:descrizione_portale}
Il portale offrirà ai produttori la possibilità di pubblicizzare i propri prodotti in uno spazio dedicato; mentre agli utenti e ai visitatori la possibilità di ricercarli e visualizzarli. 

Offrirà inoltre, solo per gli utenti, un sistema di recensione e valutazione prodotti e un modo per restare aggiornati sulle attività altrui, sia dei produttori che degli utenti.

È inoltre prevista l'integrazione con social network.

\section{Descrizione figure} 
\label{sec:descrizione_figure}
Le figure che potranno utilizzare il portale si dividono in due categorie:
\begin{itemize}
%	\item \ruolo{Visitatori}
	\item \ruolo{Figure pubbliche}
	\item \ruolo{Figure amministrative}
\end{itemize}

%\subsection{Visitatori} % (fold)
%\label{sub:}
%I visitatori, coloro che non sono stati identificati dal portale, potranno usufruire dei suoi servizi pubblici.
%\begin{descriptionInd}
%    \item[Visitatori] useranno il portale per ricercare prodotti e per leggere notizie presenti nel sito.
%\end{descriptionInd}

\subsection{Figure pubbliche}
\label{sub:figure_pubbliche}
Le figure pubbliche sono coloro che usufruiscono dei servizi del portale. Si distinguono in base al loro stato (figure \stato{autenticate} oppure \stato{non autenticate} all'interno del portale) e nelle loro funzionalità. 
\begin{itemize}
	\item \stato{Non autenticate}:
					\begin{descriptionInd}
					    \item[Visitatori] useranno il portale per ricercare prodotti, per leggere notizie presenti nel sito e per iscriversi o autenticarsi.
					\end{descriptionInd}

	\item \stato{Autenticate}:
					\begin{descriptionInd}
					    \item[Produttori] useranno il portale per pubblicizzare i propri prodotti.

					    \item[Utenti] useranno il portale per ricercare prodotti, con la possibilità di recensirli e/o valutarli; per seguire altre figure pubbliche autenticate per ricevere aggiornamenti da esse e per leggere notizie presenti nel sito.
					\end{descriptionInd}
\end{itemize}

\subsection{Figure amministrative}
\label{sub:figureamministrative}
Le figure amministrative sono coloro che utilizzeranno il portale con lo scopo di fornire servizi alle figure pubbliche.
\begin{descriptionInd}
   % \item[Web Admin] avrà il compito di configurare e mantenere il portale, sia lato backend che frontend.
    \item[Redattori] useranno il portale per pubblicare notizie e articoli inerenti al mondo dolciario.   
    \item[Assistenti] useranno il portale per fornire supporto alle figure pubbliche.
    \item[Moderatori] useranno il portale per gestire segnalazioni di utenti.
\end{descriptionInd}


\section{Tabella necessità o esigenze} % (fold)
\label{sec:tabelle_necessita_o_esigenze}
Nella seguente tabella riassumiamo le necessità o esigenze delle diverse figure:
\begin{center}
	\begin{tabularx}{0.8\textwidth}{l X}
	\toprule 
		\tabhead{Figura} & \tabhead{Necessità o Esigenze} \\
	\midrule
		\ruolo{Visitatori} & Ricercare prodotti, visualizzare prodotti, visualizzare notizie, autenticarsi o iscriversi.  \\
		\addlinespace[1em]
		\ruolo{Produttori} & Inserire prodotti nello spazio dedicato, modificare i prodotti inseriti.  \\
		\ruolo{Utenti} & Cercare prodotti, visualizzare prodotti, valutare/recensire prodotti, visualizzare notizie, seguire altre figure pubbliche autenticate. \\
		\addlinespace[1em]
		\ruolo{Redattori} & Pubblicare notizie, modificare notizie, eliminare notizie.  \\
		\ruolo{Assistenti} & Rispondere alle richieste delle figure pubbliche attraverso il sistema di tickets  \\
		\ruolo{Moderatori} & Eliminare contenuti inappropriati, modificare contenuti inappropriati.  \\
	\bottomrule
	\end{tabularx}
\end{center}

\newdate{visuno}{30}{04}{2016}
\section{Revisioni}
\begin{center}
    \begin{tabular}{lll}
        \toprule
        Versione & Data & Descrizione \\
        %   \cmidrule{2-3} 
        \midrule
        1.0 & \displaydate{visuno} & Prima versione \\
        \bottomrule
    \end{tabular}
\end{center}


