\chapter{Pianificazione del progetto} 
\label{cha:pianificazione_del_progetto}	

\section{Introduzione}
\label{sec:introduzione}
In questo documento verrà descritto il processo di sviluppo utilizzato e verrà effettuata la pianificazione delle iterazioni e delle attività da svolgere, fornendone i tempi previsti. %da modificare 

\section{Processo di sviluppo}
\label{sec:processo_di_sviluppo}
Il progetto sarà realizzato secondo la metodologia di sviluppo \gls{rup}.
Tale motodologia prevede quattro fasi sequenziali che possono essere iterate:
\begin{enumerate}
	\item \fase{Inizio}: fase in cui vengono raccolti i requisiti principali.

	\item \fase{Elaborazione}: fase in cui vengono corretti i requisiti, viene effettuata una analisi ordinata, la progettazione e le prime implementazioni.

	\item \fase{Costruzione}: fase in cui viene completata la progettazione (con i dettagli), viene implementato il sistema e vengono effettuati verifiche e test di aggregazione.

	\item \fase{Transizione}: fase in cui vengono effettuati test finali di pre--produzione, messa in opera, manutenzione, archiviazione documentale e attivazione procedure di gestione.
\end{enumerate}

\noindent
In ognuna di queste fasi possono essere svolte le seguenti attività:
\begin{itemize}
	\item \workflow{Raccolta dei requisiti}
	\item \workflow{Analisi}
	\item \workflow{Progettazione}
	\item \workflow{Implementazione}
	\item \workflow{Test}
\end{itemize}

A seconda della fase in cui ci si trova verranno svolte più o meno attività. 
Un tipico esempio di distrubuzione delle attività nelle varie fasi è il seguente:
\begin{center}
   \includegraphics[width= \textwidth]{assets/fasiworkflow}
\end{center}