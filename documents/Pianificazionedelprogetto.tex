\chapter{Pianificazione del progetto} 
\label{cha:pianificazione_del_progetto}	

\section{Introduzione}
In questo documento verrà descritto il processo di sviluppo utilizzato e verrà effettuata la pianificazione delle iterazioni e delle attività da svolgere.
Sarà inoltre fornita una pianificazione temporale del lavoro e sarà possibile verificare lo stato attuale dei lavori consultando questo documento. Infatti manterremo aggiornato questo documento tenendo traccia dell'avanzamento dello sviluppo del sistema.

\section{Processo di sviluppo}
\label{sec:processo_di_sviluppo}
Il progetto sarà realizzato secondo la metodologia di sviluppo \gls{rup}.
Tale metodologia prevede quattro fasi sequenziali che possono essere iterate:
\begin{enumerate}
	\item \fase{Inizio}: fase in cui vengono raccolti i requisiti principali.

	\item \fase{Elaborazione}: fase in cui vengono corretti i requisiti, viene effettuata una analisi ordinata, la progettazione e le prime implementazioni.

	\item \fase{Costruzione}: fase in cui viene completata la progettazione (con i dettagli), viene implementato il sistema e vengono effettuati verifiche e test di aggregazione.

	\item \fase{Transizione}: fase in cui vengono effettuati test finali di pre--produzione, messa in opera, manutenzione, archiviazione documentale e attivazione procedure di gestione.
\end{enumerate}

\noindent
In ognuna di queste fasi possono essere svolte le seguenti attività:
\begin{itemize}
	\item \workflow{Raccolta dei requisiti}
	\item \workflow{Analisi}
	\item \workflow{Progettazione}
	\item \workflow{Implementazione}
	\item \workflow{Test}
\end{itemize}

\noindent
A seconda della fase in cui ci si trova verranno svolte più o meno attività. 
Un tipico esempio di distribuzione delle attività nelle varie fasi è il seguente:
\begin{center}
   \includegraphics[width=\textwidth]{assets/fasiworkflow}
\end{center}

\section{Pianificazione iterazioni}
\label{sec:pianificazione_iterazioni}
Saranno di seguito elencate le iterazioni svolte durante il progetto, durante lo sviluppo compariranno le iterazioni già effettuate e la pianificazione dell'iterazione successiva.

\begin{center}
	\begin{tabularx}{\tabwidthiter}{ l  X } 
		\toprule
			\multicolumn{2}{c}{\tabtitleiter{Iterazione 01}}  \\
		\cmidrule(l{\cmidrulekern}r{\cmidrulekern}){1-2}
			\tabheaditer{Fase} & Inizio \\ 
			\addlinespace[1em] 
			\tabheaditer{Milestones} & 
			    \begin{enumWork}
				        \item Stesura del documento di richiesta
				        \item Stesura del documento di visione
				        \item Stesura dello studio di fattibilità
				        \item Stesura del contratto
				        \item Inizio stesura del glossario e degli acronimi
				        \item Inizio stesura del documento di pianificazione del progetto
			    \end{enumWork} \\
			\addlinespace[1em]
			\tabheaditer{Stato} &  Effettuata \\
		\bottomrule
	\end{tabularx}
\end{center}

\bigskip

\begin{center}
	\begin{tabularx}{\tabwidthiter}{ l  X } 
		\toprule
			\multicolumn{2}{c}{\tabtitleiter{Iterazione 02}}  \\
		\cmidrule(l{\cmidrulekern}r{\cmidrulekern}){1-2}
			\tabheaditer{Fase} & Elaborazione \\ 
			\addlinespace[1em] 
			\tabheaditer{Milestones} & 
			    \begin{enumWork}
				        \item Stesura del documento di specifica del sistema
				        \item Stesura del documento di specifica dei requisiti
				        \item Stesura del documento dei casi d'uso
				        \item Aggiornamento del glossario e degli acronimi
				        \item Aggiornamento del documento di pianificazione
			    \end{enumWork} \\
			\addlinespace[1em]
			\tabheaditer{Stato} &  In corso \\
		\bottomrule
	\end{tabularx}
\end{center}


\begin{landscape}
\section{Diagramma di Gantt}
\label{sec:diagramma_di_gantt}
Nel seguente diagramma di Gantt sono riportati i periodi di svolgimento di ogni attività effettuata per la realizzazione del sistema.

\begin{center}
	\includegraphics[width=\linewidth]{diagrammaGantt/DiagrammaGantt}
\end{center}
\end{landscape}

\newdate{pianuno}{3}{05}{2016}
\newdate{piandue}{4}{05}{2016}
\section{Revisioni}
\begin{center}
    \begin{tabular}{lll}
        \toprule
        	\tabhead{Versione} & \tabhead{Data} & \tabhead{Descrizione} \\
		\cmidrule(l{\cmidrulekern}r{\cmidrulekern}){1-3}
	        1.0 & \displaydate{pianuno} & Prima versione \\
	        1.1 & \displaydate{piandue} & Aggiunta iterazione 02 \\
        \bottomrule
    \end{tabular}
\end{center}
