\chapter{Specifica requisiti} 
\label{cha:specifica_requisiti}

\section{Introduzione} 
In questo documento sono specificati nel dettaglio tutti i requisiti del sistema, suddivisi in funzionali e non funzionali.
Per organizzare in modo corretto le informazioni è stato adottato un sistema di codifica dei requisiti, descritto nella prossima sezione.

\section{Struttura del documento}
\label{sec:struttura_del_documento}

I requisiti mostrati sono prima elencati, per una più facile consultazione, e in seguito sono descritti in apposite tabelle. 
Ogni tabella contiene diverse informazioni:
\begin{itemize}
	\item Un \campiTabReq{titolo} per semplificare il riconoscimento del requisito.
	\item Un \campiTabReq{id} per identificare univocamente il requisito.
	\item Una \campiTabReq{descrizione} per indicare in cosa consiste la funzionalità rappresentata dal requisito.
	\item Un \campiTabReq{tipo} per indicare se il requisito è funzionale o non funzionale
	\item Una \campiTabReq{priorità} per indicare il grado di importanza del requisito.
	\item Lo \campiTabReq{stato} per indicare l'idoneità del requisito.
\end{itemize}

\subsection{Identificativo dei requisiti}
\label{subsec:identificativo_dei_requisiti}

Di seguito è spiegato come interpretare l'identificativo dei requisiti.
Un identificativo è della forma:
\begin{center}
	\subsubreq{tipo}{contesto}{categoria}{req}{[sottoreq]}{[\dots]}
\end{center}

\noindent
Segue una descrizione di ogni campo utilizzato nell'identificatore:
\begin{itemize}
	\item Il \campiIdReq{tipo} può essere:
	\begin{descriptionInd}
		\item[RF] indica un requisito funzionale.
		\item[RNF] indica un requisito non funzionale.
	\end{descriptionInd}

	\item Il \campiIdReq{contesto} serve ad identificare a quali figure è rivolta una data funzionalità, è un numero composto da 7 cifre ognuna delle quali può assumere un valore 0 oppure 1. La posizione della cifra si riferisce ad una specifica figura, se la cifra i-esima vale 1 significa che l'i-esima figura è presente nel contesto del requisitio, altrimenti no.
	Di seguito elenchiamo ogni posizione a quale figura riferisce: 
	\begin{descriptionInd}
		\item[1\textdegree\ bit] Visitatore.
		\item[2\textdegree\ bit] Utente.
		\item[3\textdegree\ bit] Produttore.
		\item[4\textdegree\ bit] Redattore.
		\item[5\textdegree\ bit] Assistente.
		\item[6\textdegree\ bit] Moderatore.
		\item[7\textdegree\ bit] Amministratore.
	\end{descriptionInd}		

	\item La \campiIdReq{categoria} è una lettera utilizzata per raggruppare i requisiti con scopi simili.
	Le categorie per i requisiti funzionali sono le seguenti:
	\begin{enumerateIndLabel}{\textsf{\Alph*}}{\displayReq{\Alph*}}
		\item Autenticazione. \label{cat:autenticazione}
		\item Gestione iscrizione. \label{cat:iscrizione}
		\item Gestione vetrina. \label{cat:vetrina}
		\item Gestione notizie. \label{cat:notizie}
		\item Gestioni suggerimenti. \label{cat:suggerimenti}
		\item Gestione account. \label{cat:account}
		\item Interazione fra figure. \label{cat:interazione}
		\item Ricerca contenuti. \label{cat:ricerca}
		\item Visualizzazione contenuti. \label{cat:visualizzazione}
	\end{enumerateIndLabel}
	Le categorie per i requisiti non funzionali sono le seguenti:
	\begin{enumerateIndLabel}{\textsf{\Alph*}}{\displayReq{\Alph*}}
		\item Sicurezza. \label{cat:sicurezza}
		\item Usabilità. \label{cat:usabilita}
		\item Portabilità. \label{cat:portabilita}
		%\item[D] Implementazione.
		%\item[E] Organizzazione.
	\end{enumerateIndLabel}	

	\item Il campo \campiIdReq{req} è un numero progressivo utilizzato per identificare in modo univoco un requisito all'interno della sua categoria.

	\item Il campo \campiIdReq{sottoreq} (opzionale) è un numero progressivo utilizzato per identificare in modo univoco un sotto--requisito di un dato requisito. Ci possono essere diversi campi sottoreq consecutivi, per esplicitare una nidificazione dei requisiti più dettagliata.
\end{itemize}

\subsection{Priorità}
\label{subsec:priorita}

La priorità di un requisito identifica quanto un requisito sia importante ai fini del completamento del sistema.
Qui di seguito sono elencati in breve i ``gradi di giudizio'' della priorità di un requisito:
\begin{descriptionInd}
	\item[Priorità Alta] Requisiti che il sistema deve necessariamente soddisfare.
	\item[Priorità Media] Requisiti la cui assenza non comporta grossi problemi per la soddisfacibilità del progetto: i requisiti base riescono comunque ad essere soddisfatti, con qualche sacrificio in termini di funzionalità. 
	\item[Priorità Bassa] Requisiti la cui assenza non comporta alcun tipo di problema per la soddisfacibilità del progetto. Questi requisiti possono anche visti come superflui o non strettamente necessari. La loro inclusione comporta un miglioramento della qualità complessiva dell’offerta dei servizi e delle funzionalità del progetto.
\end{descriptionInd}

\section{Elenco dei requisiti}
\label{sec:elenco_dei_requisiti}
Di seguito sono elencati tutti i requisiti, suddivisi in funzionali e non funzionali.

\subsection{Requisiti funzionali}
Vengono di seguito elencati i requisiti funzionali:
%USAGE: {keyrequisito}{\ereq{tipo}{contesto}}{titolo} - doublelink automatico con \refDef{keyrequisito}
\begin{itemize} 
	\setCat{\ref*{cat:autenticazione}}%A
	\item \reqItem{req:login}{\ereq{RF}{1000000}}{Login}
	\item \reqItem{req:logout}{\ereq{RF}{0111111}}{Logout}

	\setCat{\ref*{cat:iscrizione}}%B
	\item \reqItem{req:iscrizione}{\ereq{RF}{1000000}}{Iscrizione}
	\item \reqItem{req:iscrizionePortale}{\esubreq{RF}{1000000}}{Iscrizione tramite modulo del portale}
	\item \reqItem{req:iscrizioneSocial}{\esubreq{RF}{1000000}}{Iscrizione tramite Social Network}
	\item \reqItem{req:iscrizioneSocialVer}{\esubsubreq{RF}{1000000}}{Iscrizione tramite Social Network verificati}
	\item \reqItem{req:iscrizioneSocialNotVer}{\esubsubreq{RF}{1000000}}{Iscrizione tramite Social Network non verificati}
	\item \reqItem{req:iscrizioneApprovazione}{\esubreq{RF}{1000000}}{Iscrizione tramite approvazione}
	
	\item \reqItem{req:approvazioneIscrizione}{\ereq{RF}{0000101}}{Approvazione iscrizione}
	
	\item \reqItem{req:rimozioneAccountAltrui}{\ereq{RF}{0000011}}{Rimozione account altrui}

	\setCat{\ref*{cat:vetrina}}%C
	\item \reqItem{req:personalizzaVetrina}{\ereq{RF}{0010000}}{Personalizzione vetrina}
	\item \reqItem{req:personalizzaVetrinaInsDesc}{\esubreq{RF}{0010000}}{Inserimento descrizione}
	\item \reqItem{req:personalizzaVetrinaInsImg}{\esubreq{RF}{0010000}}{Inserimento immagine}
	\item \reqItem{req:personalizzaVetrinaInsProd}{\esubreq{RF}{0010000}}{Inserimento prodotto}
	\item \reqItem{req:personalizzaVetrinaModProd}{\esubreq{RF}{0010000}}{Modifica prodotto}
	
	\item \reqItem{req:statistichePrivateVetrina}{\ereq{RF}{0010000}}{Visualizzazione statistiche private della vetrina}
	
	\setCat{\ref*{cat:notizie}}%D
	\item \reqItem{req:inserimentoNotizia}{\ereq{RF}{0001001}}{Inserimento notizia}
	
	\item \reqItem{req:modificaNotizia}{\ereq{RF}{0001001}}{Modifica notizia}
	
	\item \reqItem{req:rimozioneNotizia}{\ereq{RF}{0001001}}{Rimozione notizia}
	
	\setCat{\ref*{cat:suggerimenti}}%E
	\item \reqItem{req:suggerimentoProdotti}{\ereq{RF}{0111111}}{Suggerimento prodotti intelligente}
	
	\item \reqItem{req:prodottiSimili}{\ereq{RF}{0111111}}{Suggerimento prodotti simili}

	\item \reqItem{req:notizieSimili}{\ereq{RF}{0111111}}{Suggerimento notizie simili}
	
	\setCat{\ref*{cat:account}}%F
	\item \reqItem{req:accessoProfilo}{\ereq{RF}{0111111}}{Accesso al proprio profilo}
	
	\item \reqItem{req:accessoImpostazioni}{\ereq{RF}{0111111}}{Accesso alle proprie impostazioni}

	\item \reqItem{req:rimozioneAccountProprio}{\ereq{RF}{0111111}}{Rimozione account proprio}
	
	\setCat{\ref*{cat:interazione}}%G
	\item \reqItem{req:valutazioneProdotto}{\ereq{RF}{0101111}}{Valutazione prodotto}
	\item \reqItem{req:inserisciValutazioneProdotto}{\esubreq{RF}{0101111}}{Inserimento valutazione}
	\item \reqItem{req:modificaValutazioneProdotto}{\esubreq{RF}{0101111}}{Modifica valutazione inserita}
	\item \reqItem{req:eliminaValutazioneProdotto}{\esubreq{RF}{0101111}}{Rimozione valutazione inserita}

	\item \reqItem{req:recensioneProdotto}{\ereq{RF}{0101111}}{Recensione prodotto}
	\item \reqItem{req:inserisciRecensioneProdotto}{\esubreq{RF}{0101111}}{Inserimento recensione}
	\item \reqItem{req:modificaRecensioneProdotto}{\esubreq{RF}{0101111}}{Modifica valutazione inserita}
	\item \reqItem{req:eliminaRecensioneProdotto}{\esubreq{RF}{0101111}}{Rimozione recensione inserita}

	\item \reqItem{req:commentoRecensione}{\ereq{RF}{0111111}}{Commento a recensione}
	
	\item \reqItem{req:valutaRecensione}{\ereq{RF}{0101111}}{Valutazione recensione}
	
	\item \reqItem{req:segnalazioneContenutiInap}{\ereq{RF}{0111111}}{Segnalazione contenuti inappropriati}
	
	\item \reqItem{req:rimozioneContenutiInap}{\ereq{RF}{0000011}}{Rimozione contenuti inappropriati}

	\item \reqItem{req:followAccount}{\ereq{RF}{0111111}}{Follow account}
	
	\item \reqItem{req:unFollowAccount}{\ereq{RF}{0111111}}{Unfollow account}

	\item \reqItem{req:ticketComunicazione}{\ereq{RF}{1111111}}{Comunicazione tramite tickets}
	\item \reqItem{req:ticketInvio}{\esubreq{RF}{1111111}}{Invio di un ticket}
	\item \reqItem{req:ticketRisposta}{\esubreq{RF}{1111111}}{Risposta ad un ticket}
	\item \reqItem{req:ticketChiudi}{\esubreq{RF}{0000101}}{Chiusura di un ticket}

	\setCat{\ref*{cat:ricerca}}%H
	\item \reqItem{req:ricercaInfo}{\ereq{RF}{1111111}}{Ricerca contenuti}
	\item \reqItem{req:ricercaProdotto}{\esubreq{RF}{1111111}}{Ricerca prodotto}
	\item \reqItem{req:ricercaProfilo}{\esubreq{RF}{1111111}}{Ricerca profilo}
	\item \reqItem{req:ricercaNotizia}{\esubreq{RF}{1111111}}{Ricerca notizia}

	\item \reqItem{req:richiestaInsProdotto}{\ereq{RF}{0101111}}{Richiesta inserimento prodotto a un produttore iscritto}

	\item \reqItem{req:richiestaInsProduttore}{\ereq{RF}{0101111}}{Richiesta inserimento prodotto e del suo produttore}
	
	\setCat{\ref*{cat:visualizzazione}}%I
	\item \reqItem{req:visualizzazioneVetrina}{\ereq{RF}{1111111}}{Visualizzazione vetrina}
	\item \reqItem{req:visualizzazioneStatistichePubVetrina}{\esubreq{RF}{1111111}}{Visualizzazione statische pubbliche vetrina}
	\item \reqItem{req:visualizzazioneProdottiVetrina}{\esubreq{RF}{1111111}}{Visualizzazione prodotti esposti in vetrina}

	\item \reqItem{req:visualizzazioneProdotto}{\ereq{RF}{1111111}}{Visualizzazione prodotto}
	\item \reqItem{req:visualizzazioneInfoProdotto}{\esubreq{RF}{1111111}}{Visualizzazione informazioni del prodotto}
	\item \reqItem{req:visualizzazioneRecProdotto}{\esubreq{RF}{1111111}}{Visualizzazione recensioni associate al prodotto}
	\item \reqItem{req:visualizzazioneStatProdotto}{\esubreq{RF}{1111111}}{Visualizzazione statistiche del prodotto}

	\item \reqItem{req:visualizzazioneProfilo}{\ereq{RF}{1111111}}{Visualizzazione profilo pubblico}
	\item \reqItem{req:visualizzazioneInfoProfilo}{\esubreq{RF}{1111111}}{Visualizzazione informazioni di un profilo pubblico}
	\item \reqItem{req:visualizzazioneStatProfilo}{\esubreq{RF}{1111111}}{Visualizzazione statistiche di un profilo pubblico}

	\item \reqItem{req:visualizzazioneNotizie}{\ereq{RF}{1111111}}{Visualizzazione notizie}
	
	\item \reqItem{req:visualizzazioneAggF}{\ereq{RF}{0111111}}{Visualizzazione aggiornamenti dai followed}
\end{itemize}	


\subsection{Requisiti non funzionali}
\label{sub:requisiti_non_funzionali}
Vengono di seguito elencati i requisiti non funzionali:
%USAGE: {keyrequisito}{\ereq{tipo}{contesto}{categoria}}{titolo} - doublelink automatico con \refDef{keyrequisito}
\begin{itemize}
	\setCat{\ref*{cat:sicurezza}}%A
	\item  \reqItem{req:sicurezzaHttps}{\ereq{RNF}{1111111}}{Protollo \texttt{HTTPS}}
	\item  \reqItem{req:sicurezzaDati}{\ereq{RNF}{1111111}}{Protezione dei dati}
	
	\setCat{\ref*{cat:usabilita}}%B
	\item  \reqItem{req:usabilitaInterfaccia}{\ereq{RNF}{1111111}}{Interfaccia intuitiva del portale}
	\item  \reqItem{req:usabilitaRicerca}{\ereq{RNF}{1111111}}{Semplicità ricerca contenuti}
	
	\setCat{\ref*{cat:portabilita}}%C
	\item  \reqItem{req:portabilitaBrowser}{\ereq{RNF}{1111111}}{Sito web portabile fra browser}
	\item  \reqItem{req:portabilitaMobile}{\ereq{RNF}{1111111}}{Sito web responsive}

	%\item  \reqItem{req:implementazioneDDF}{\ereq{RNF}{1111111}}{ }
	%\countReset

	%\item  \reqItem{req:standards}{\ereq{RNF}{0000000}}{Rispetto degli standard}
	%\countReset
\end{itemize}

\section{Descrizione dei requisiti}
\label{sec:descrizione_dei_requisiti}
Di seguito sono presenti tutte le descrizioni dei requisiti, suddivise in funzionali e non funzionali.

\subsection{Requisiti funzionali}
Vengono di seguito riportate le tabelle che descrivono nel dettaglio ogni requisito funzionale.
%usage: \reqTabRF{reqkey}{descrizione}{priorità}{stato}
%
%\prioritaH - alta
%\prioritaM - media
%\prioritaL - bassa
%
%\statoOK - approvato
%\statoNOK - non approvato
%\statoUn - da approvare

%%%%%%%%%%%%%%%%%%%%%%%%%%%%%%%%%%%%% A - AUTENTICAZIONE %%%%%%%%%%%%%%%%%%%%%%%%%%%%%%%%%%%%%
\reqTabRF{req:login}{Il portale deve permettere ai visitatori di effettuare il login con i dati inseriti nella fase di iscrizione}{\prioritaH}{\statoUn}

\tabreqvspace

\reqTabRF{req:logout}{Il portale deve permettere di effettuare il logout}{\prioritaH}{\statoUn}

\tabreqvspace

%%%%%%%%%%%%%%%%%%%%%%%%%%%%%%%%%%%%% B - GESTIONE ISCRIZIONE %%%%%%%%%%%%%%%%%%%%%%%%%%%%%%%%%%%%%
\reqTabRF{req:iscrizione}{Il portale deve permettere l'iscrizione ai visitatori.}{\prioritaH}{\statoUn}

\tabreqvspace

\reqTabRF{req:iscrizionePortale}{L'iscrizione dei visitatori deve essere possibile tramite il portale, compilando un apposito modulo e fornendo le informazione necessarie. Tramite questo metodo di iscrizione è possibile creare solo un account utente}{\prioritaH}{\statoUn}

\tabreqvspace

\reqTabRF{req:iscrizioneSocial}{L'iscrizione dei visitatori deve essere possibile con account terzi di social network}{\prioritaM}{\statoUn}

\tabreqvspace

\reqTabRF{req:iscrizioneSocialVer}{Deve essere possibile l'iscrizione con account verificati di social network. Tramite questo metodo di iscrizione è possibile creare un account utente oppure un account produttore}{\prioritaM}{\statoUn}

\tabreqvspace

\reqTabRF{req:iscrizioneSocialNotVer}{Deve essere possibile l'iscrizione con account non verificati di social network. Tramite questo metodo di iscrizione è possibile creare solo un account utente}{\prioritaM}{\statoUn}

\tabreqvspace

\reqTabRF{req:iscrizioneApprovazione}{L'iscrizione dei visitatori deve essere possibile richiedendo l'approvazione di un account, con apposita procedura. Tramite questo metodo di iscrizione è possibile creare solo un account produttore}{\prioritaH}{\statoUn}

\tabreqvspace

\reqTabRF{req:approvazioneIscrizione}{Il portale deve permettere agli assistenti di rispondere alle richieste di iscrizione effettuate tramite la procedura che richiede approvazione}{\prioritaH}{\statoUn}

\tabreqvspace

\reqTabRF{req:rimozioneAccountAltrui}{Il portale deve permettere agli amministratori di rimuovere un account}{\prioritaL}{\statoUn}

\tabreqvspace

%%%%%%%%%%%%%%%%%%%%%%%%%%%%%%%%%%%%% C - GESTIONE VETRINA %%%%%%%%%%%%%%%%%%%%%%%%%%%%%%%%%%%%%
\reqTabRF{req:personalizzaVetrina}{Il portale deve permettere la personalizzazione della vetrina ad ogni produttore}{\prioritaH}{\statoUn}

\tabreqvspace

\reqTabRF{req:personalizzaVetrinaInsDesc}{Deve essere possibile per il produttore inserire una descrizione nella propria vetrina}{\prioritaM}{\statoUn}

\tabreqvspace

\reqTabRF{req:personalizzaVetrinaInsImg}{Deve essere possibile per il produttore inserire un'immagine nella propria vetrina}{\prioritaM}{\statoUn}

\tabreqvspace

%Vedi se va qui la descrizione dettagliata dell'inserimento!
\reqTabRF{req:personalizzaVetrinaInsProd}{Deve essere possibile per il produttore inserire un prodotto nella propria vetrina}{\prioritaH}{\statoUn}

\tabreqvspace

\reqTabRF{req:personalizzaVetrinaModProd}{Deve essere possibile per il produttore modificare un prodotto già inserito nella propria vetrina}{\prioritaH}{\statoUn}

\tabreqvspace

\reqTabRF{req:statistichePrivateVetrina}{Il portale deve permettere al produttore l'accesso alle statistiche private della sua vetrina}{\prioritaM}{\statoUn}

\tabreqvspace

%%%%%%%%%%%%%%%%%%%%%%%%%%%%%%%%%%%%% H - GESTIONE NOTIZIE %%%%%%%%%%%%%%%%%%%%%%%%%%%%%%%%%%%%%
\reqTabRF{req:inserimentoNotizia}{Il portale deve permettere ai redattori di inserire una notizia}{\prioritaH}{\statoUn}

\tabreqvspace

\reqTabRF{req:modificaNotizia}{Il portale deve permettere ai redattori di modificare una notizia}{\prioritaH}{\statoUn}

\tabreqvspace

\reqTabRF{req:rimozioneNotizia}{Il portale deve permettere ai redattori di eliminare una notizia}{\prioritaH}{\statoUn}

\tabreqvspace

%%%%%%%%%%%%%%%%%%%%%%%%%%%%%%%%%%%%% H - GESTIONE SUGGERIMENTI %%%%%%%%%%%%%%%%%%%%%%%%%%%%%%%%%%%%%
\reqTabRF{req:suggerimentoProdotti}{Il portale deve permettere di ricevere suggerimenti riguardo quali altri prodotti potrebbero piacere}{\prioritaL}{\statoUn}

\tabreqvspace

\reqTabRF{req:prodottiSimili}{Il portale deve permettere di ricevere suggerimenti riguardo quali altri prodotti sono simili a quello che si sta visualizzando}{\prioritaL}{\statoUn}

\tabreqvspace

\reqTabRF{req:notizieSimili}{Il portale deve permettere di ricevere suggerimenti riguardo quali altre notizie sono simili a quella che si sta visualizzando}{\prioritaL}{\statoUn}

\tabreqvspace

%%%%%%%%%%%%%%%%%%%%%%%%%%%%%%%%%%%%% H - GESTIONE ACCOUNT %%%%%%%%%%%%%%%%%%%%%%%%%%%%%%%%%%%%%
\reqTabRF{req:accessoProfilo}{Il portale deve permette l'accesso al proprio profilo per visualizzarne i contenuti}{\prioritaM}{\statoUn}

\tabreqvspace

\reqTabRF{req:accessoImpostazioni}{Il portale deve permettere l'accesso alle impostazioni e la modifica di queste ultime}{\prioritaM}{\statoUn}

\tabreqvspace

\reqTabRF{req:rimozioneAccountProprio}{Il portale deve permettere l'eliminazione del proprio account. L'eliminazione di un account non comporterà l'eliminazione dei contenuti inseriti da esso.}{\prioritaL}{\statoUn}

\tabreqvspace

%%%%%%%%%%%%%%%%%%%%%%%%%%%%%%%%%%%%% H - INTERAZIONE FRA FIGURE %%%%%%%%%%%%%%%%%%%%%%%%%%%%%%%%%%%%%
\reqTabRF{req:valutazioneProdotto}{Il portale deve permettere di valutare un prodotto}{\prioritaH}{\statoUn}

\tabreqvspace

\reqTabRF{req:inserisciValutazioneProdotto}{Deve essere possibile inserire una valutazione di un dato prodotto, cioè un voto in voto in una scala limitata}{\prioritaH}{\statoUn}

\tabreqvspace

\reqTabRF{req:modificaValutazioneProdotto}{Deve essere possibile modificare una propria valutazione}{\prioritaM}{\statoUn}

\tabreqvspace

\reqTabRF{req:eliminaValutazioneProdotto}{Deve essere possibile rimuovere una propria valutazione}{\prioritaM}{\statoUn}

\tabreqvspace

\reqTabRF{req:recensioneProdotto}{Il portale deve permettere di recensire di un prodotto}{\prioritaH}{\statoUn}

\tabreqvspace

\reqTabRF{req:inserisciRecensioneProdotto}{Deve essere possibile inserire una recensione di un dato prodotto, cioè una valutazione e una descrizione testuale}{\prioritaH}{\statoUn}

\tabreqvspace

\reqTabRF{req:modificaRecensioneProdotto}{Deve essere possibile modificare una propria recensione}{\prioritaM}{\statoUn}

\tabreqvspace

\reqTabRF{req:eliminaRecensioneProdotto}{Deve essere possibile rimuovere una propria recensione}{\prioritaM}{\statoUn}

\tabreqvspace

\reqTabRF{req:commentoRecensione}{Il portale deve permettere l'inserimento di un commento in una recensione}{\prioritaM}{\statoUn}

\tabreqvspace

\reqTabRF{req:valutaRecensione}{Il portale deve permettere di esprimere una valutazione, positiva o negativa, di una recensione}{\prioritaM}{\statoUn}

\tabreqvspace

\reqTabRF{req:segnalazioneContenutiInap}{Il portale deve permettere di segnalare contenuti inappropriati}{\prioritaL}{\statoUn}

\tabreqvspace

\reqTabRF{req:rimozioneContenutiInap}{Il portale deve permettere ai moderatori la rimozione di contenuti inappropriati}{\prioritaL}{\statoUn}

\tabreqvspace

\reqTabRF{req:followAccount}{Il portale deve permettere di seguire un account per ricevere aggiornamenti da esso}{\prioritaH}{\statoUn}

\tabreqvspace

\reqTabRF{req:unFollowAccount}{Il portale deve permettere di cessare di seguire un account che si segue}{\prioritaH}{\statoUn}

\tabreqvspace

\reqTabRF{req:ticketComunicazione}{Il portale deve permettere alle figure di comunicare con gli assistenti tramite tickets}{\prioritaM}{\statoUn}

\tabreqvspace

\reqTabRF{req:ticketInvio}{Deve essere possibile creare ed inviare diversi tipi di tickets}{\prioritaM}{\statoUn}

\tabreqvspace

\reqTabRF{req:ticketRisposta}{Deve essere possibile rispondere ad un ticket. Le risposte ad un ticket possono essere effettuate da un assistente o dal creatore del ticket}{\prioritaM}{\statoUn}

\tabreqvspace

\reqTabRF{req:ticketChiudi}{Deve essere possibile per gli assistenti chiudere un ticket}{\prioritaM}{\statoUn}

\tabreqvspace

%%%%%%%%%%%%%%%%%%%%%%%%%%%%%%%%%%%%% H - RICERCA CONTENUTI %%%%%%%%%%%%%%%%%%%%%%%%%%%%%%%%%%%%%
\reqTabRF{req:ricercaInfo}{Il portale deve permettere di effettuare ricerche in base a diversi criteri}{\prioritaH}{\statoUn}

\tabreqvspace

\reqTabRF{req:ricercaProdotto}{Deve essere possibili cercare prodotti in base a diversi criteri}{\prioritaH}{\statoUn}

\tabreqvspace

\reqTabRF{req:ricercaProfilo}{Deve essere possibili cercare profili pubblici in base a diversi criteri}{\prioritaM}{\statoUn}

\tabreqvspace

\reqTabRF{req:ricercaNotizia}{Deve essere possibili cercare notizie in base a diversi criteri}{\prioritaM}{\statoUn}

\tabreqvspace

\reqTabRF{req:richiestaInsProdotto}{Deve essere possibile segnalare ad un produttore la mancanza di un suo prodotto tra quelli presenti nel sistema}{\prioritaL}{\statoUn}

\tabreqvspace

\reqTabRF{req:richiestaInsProduttore}{Deve essere possibile segnalare al sistema l’assenza di un produttore ed allo stesso tempo di uno suo specifico prodotto}{\prioritaL}{\statoUn}

\tabreqvspace

%%%%%%%%%%%%%%%%%%%%%%%%%%%%%%%%%%%%% H - VISUALIZZIONE CONTENUTI %%%%%%%%%%%%%%%%%%%%%%%%%%%%%%%%%%%%%
\reqTabRF{req:visualizzazioneVetrina}{Il portale deve permettere di visualizzare una data vetrina}{\prioritaH}{\statoUn}

\tabreqvspace

\reqTabRF{req:visualizzazioneStatistichePubVetrina}{Deve essere possibile visualizzare le statistiche pubbliche di una data vetrina}{\prioritaL}{\statoUn}

\tabreqvspace

\reqTabRF{req:visualizzazioneProdottiVetrina}{Deve essere possibile visualizzare i prodotti presenti in una data vetrina}{\prioritaH}{\statoUn}

\tabreqvspace

\reqTabRF{req:visualizzazioneProdotto}{Il portale deve permettere di visualizzare un dato prodotto}{\prioritaH}{\statoUn}

\tabreqvspace

\reqTabRF{req:visualizzazioneInfoProdotto}{Deve essere possibile visualizzare le informazioni di un dato prodotto}{\prioritaH}{\statoUn}

\tabreqvspace

\reqTabRF{req:visualizzazioneRecProdotto}{Deve essere possibile visualizzare le recensioni di un dato prodotto}{\prioritaH}{\statoUn}

\tabreqvspace

\reqTabRF{req:visualizzazioneStatProdotto}{Deve essere possibile visualizzare le statistiche di un dato prodotto}{\prioritaM}{\statoUn}

\tabreqvspace

\reqTabRF{req:visualizzazioneProfilo}{Il portale deve permettere di visualizzare un dato pro}{\prioritaM}{\statoUn}

\tabreqvspace

\reqTabRF{req:visualizzazioneInfoProfilo}{Deve essere possibile visualizzare le informazioni di un dato profilo}{\prioritaM}{\statoUn}

\tabreqvspace

\reqTabRF{req:visualizzazioneStatProfilo}{Deve essere possibile visualizzare le statistiche di un dato profilo}{\prioritaM}{\statoUn}

\tabreqvspace

\reqTabRF{req:visualizzazioneNotizie}{Il portale deve permettere di visualizzare le notizie}{\prioritaH}{\statoUn}

\tabreqvspace

\reqTabRF{req:visualizzazioneAggF}{Il portale deve permettere di visualizzare gli aggiornamenti ricevuti dagli account che si stanno seguendo}{\prioritaH}{\statoUn}

\subsection{Requisiti non funzionali}
Vengono di seguito riportate le tabelle che descrivono nel dettaglio ogni requisito non funzionale.

%%%%%%%%%%%%%%%%%%%%%%%%%%%%%%%%%%%%% SICUREZZA %%%%%%%%%%%%%%%%%%%%%%%%%%%%%%%%%%%%%
\reqTabRNF{req:sicurezzaHttps}{Il portale garantirà l'autenticità e la confidenzialità delle comunicazioni attraverso l'utilizzo del protocollo \texttt{HTTPS}}{\prioritaH}{\statoUn}

\tabreqvspace

\reqTabRNF{req:sicurezzaDati}{Il portale garantirà la protezione dei dati sensibili di tutte le figure che interagiscono con esso}{\prioritaH}{\statoUn}

\tabreqvspace

%%%%%%%%%%%%%%%%%%%%%%%%%%%%%%%%%%%%% USABILITA %%%%%%%%%%%%%%%%%%%%%%%%%%%%%%%%%%%%%
\reqTabRNF{req:usabilitaInterfaccia}{Il portale disporrà di un interfaccia semplice e intuitiva in modo tale da avere una curva di apprendimento bassa, affinché chiunque sia in un grado di utilizzarlo appieno}{\prioritaM}{\statoUn}

\tabreqvspace

\reqTabRNF{req:usabilitaRicerca}{Il portale disporrà di metodi di ricerca intuitivi per permettere di effettuare in poco tempo ricerche affidabili}{\prioritaH}{\statoUn}

\tabreqvspace

%%%%%%%%%%%%%%%%%%%%%%%%%%%%%%%%%%%%% PORTABILITA %%%%%%%%%%%%%%%%%%%%%%%%%%%%%%%%%%%%%
\reqTabRNF{req:portabilitaBrowser}{Il sito web del portale sarà portabile fra i principali browser web presenti nel mercato}{\prioritaH}{\statoUn}

\tabreqvspace

\reqTabRNF{req:portabilitaMobile}{Il sito web del portale sarà \emph{responsive} in modo tale da poter essere consultato agevolmente da qualsiasi dispositivo}{\prioritaH}{\statoUn}

\newdate{requno}{08}{05}{2016}
\section{Revisioni}
\begin{center}
    \begin{tabular}{lll}
        \toprule
        	\tabhead{Versione} & \tabhead{Data} & \tabhead{Descrizione} \\
		\cmidrule(l{\cmidrulekern}r{\cmidrulekern}){1-3}
        	1.0 & \displaydate{requno} & Prima versione \\
        \bottomrule
    \end{tabular}
\end{center}
