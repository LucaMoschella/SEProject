\chapter{Specifica requisiti} 
\label{cha:specifica_sistema}

\section{Introduzione} 
In questo documento sono elencati e poi specificati nel dettaglio tutti i requisiti del sistema, suddivisi in funzionali e non funzionali.
Per organizzare in modo corretto le informazioni è stato adottato un sistema di codifica dei requisiti, descritto nella prossima sezione.

\section{Struttura del documento}
\label{cha:struttura_del_documento}

I requisiti mostrati in questo documento sono prima elencati per una più facile consultazione, in seguito sono descritte in apposite tabelle. 
Ogni tabella contiene diverse informazioni:
\begin{itemize}
	\item Un \descCTab{id} per identificare unvicamente il requisito.
	\item Un \descCTab{titolo} per semplificare il riconoscimento del requisito.
	\item Una \descCTab{descrizione} per indicare in cosa consiste la funzionalità rappresentata dal requisito.
	\item Un \descCTab{tipo} per indicare se il requisito è funzionale o non funzionale
	\item Una \descCTab{priorità} per indicare il grado di importanza del requisito.
	\item Lo \descCTab{stato} per indicare l'idoneità del requisito.
\end{itemize}

\subsection{Identificativo dei requisiti}
\label{cha:identificativo_dei_requisiti}

Di seguito è spiegato come interpretare l'identificativo dei requisiti.
Un identificativo è della forma:
\begin{center}
	\subsubreq{tipo}{contesto}{categoria}{req}{[sottoreq]}{[\dots]}
\end{center}

\noindent
Segue una descrizione di ogni campo utilizzato nell'identificatore:
\begin{itemize}
	\item Il \descCID{tipo} può essere:
	\begin{descriptionInd}
		\item[RF] indica un requisito funzionale.
		\item[RNF] indica un requisito non funzionale.
	\end{descriptionInd}

	\item Il \descCID{contesto} serve ad identificare a quali figure è rivolta una data funzionalità è un numero composto da 6 cifre, ognuna delle quali può assumere un valore 0 oppure 1. La posizione della cifra si riferisce ad una specifica figura, se la cifra i-esima vale 1 significa che l'i-esima figura è presente nel contesto del requisitio, altrimenti no.
	Di seguito elenchiamo ogni posizione a quale figura riferisce: 
	\begin{descriptionInd}
		\item[1\textdegree\ bit] Visitatore.
		\item[2\textdegree\ bit] Utente.
		\item[3\textdegree\ bit] Produttore.
		\item[4\textdegree\ bit] Redattore.
		\item[5\textdegree\ bit] Assistente.
		\item[6\textdegree\ bit] Moderatore.
	\end{descriptionInd}		

	\item La \descCID{categoria} è un numero progressivo utilizzato per raggruppare i requisiti con scopi simili.
	Le categorie per i requisiti funzionali sono le seguenti:
	\begin{descriptionInd}
		\item[01] Iscrizione e autenticazione.
		\item[02] Gestione iscrizione.
		\item[03] Gestione vetrina.
		\item[04] Ricerca.
		\item[05] Gestione recensioni.
		\item[06] Gestione notizie.
		\item[07] Gestione follower/followed.
		\item[08] Gestioni suggerimenti.
		\item[09] Gestione account.
		\item[10] Visualizzazione.
	\end{descriptionInd}	
	Le categorie per i requisiti non funzionali sono le seguenti:
	\begin{descriptionInd}
		\item[01] Sicurezza.
		\item[02] Portabilità.
		\item[03] Usabilità.
		\item[04] Implementazione.
	\end{descriptionInd}	

	\item Il campo \descCID{req} è il numero progressivo utilizzato per identificare in modo univoco un requisito all'interno della sua categoria.

	\item Il campo \descCID{sottoreq} (opzionale) è il numero progressivo utilizzato per identificare in modo univoco un sotto--requisito di un dato requisito.
\end{itemize}

\subsection{Priorità}
\label{cha:priorita}