\chapter{Specifica requisiti} 
\label{cha:specifica_requisiti}

\section{Introduzione} 
In questo documento sono elencati e poi specificati nel dettaglio tutti i requisiti del sistema, suddivisi in funzionali e non funzionali.
Per organizzare in modo corretto le informazioni è stato adottato un sistema di codifica dei requisiti, descritto nella prossima sezione.

\section{Struttura del documento}
\label{sec:struttura_del_documento}

I requisiti mostrati in questo documento sono prima elencati per una più facile consultazione, in seguito sono descritte in apposite tabelle. 
Ogni tabella contiene diverse informazioni:
\begin{itemize}
	\item Un \descCTab{id} per identificare unvicamente il requisito.
	\item Un \descCTab{titolo} per semplificare il riconoscimento del requisito.
	\item Una \descCTab{descrizione} per indicare in cosa consiste la funzionalità rappresentata dal requisito.
	\item Un \descCTab{tipo} per indicare se il requisito è funzionale o non funzionale
	\item Una \descCTab{priorità} per indicare il grado di importanza del requisito.
	\item Lo \descCTab{stato} per indicare l'idoneità del requisito.
\end{itemize}

\subsection{Identificativo dei requisiti}
\label{subsec:identificativo_dei_requisiti}

Di seguito è spiegato come interpretare l'identificativo dei requisiti.
Un identificativo è della forma:
\begin{center}
	\subsubreq{tipo}{contesto}{categoria}{req}{[sottoreq]}{[\dots]}
\end{center}

\noindent
Segue una descrizione di ogni campo utilizzato nell'identificatore:
\begin{itemize}
	\item Il \descCID{tipo} può essere:
	\begin{descriptionInd}
		\item[RF] indica un requisito funzionale.
		\item[RNF] indica un requisito non funzionale.
	\end{descriptionInd}

	\item Il \descCID{contesto} serve ad identificare a quali figure è rivolta una data funzionalità, è un numero composto da 6 cifre ognuna delle quali può assumere un valore 0 oppure 1. La posizione della cifra si riferisce ad una specifica figura, se la cifra i-esima vale 1 significa che l'i-esima figura è presente nel contesto del requisitio, altrimenti no.
	Di seguito elenchiamo ogni posizione a quale figura riferisce: 
	\begin{descriptionInd}
		\item[1\textdegree\ bit] Visitatore.
		\item[2\textdegree\ bit] Utente.
		\item[3\textdegree\ bit] Produttore.
		\item[4\textdegree\ bit] Redattore.
		\item[5\textdegree\ bit] Assistente.
		\item[6\textdegree\ bit] Moderatore.
	\end{descriptionInd}		

	\item La \descCID{categoria} è un numero progressivo utilizzato per raggruppare i requisiti con scopi simili.
	Le categorie per i requisiti funzionali sono le seguenti:
	\begin{descriptionInd}
		\item[A] Iscrizione e autenticazione.
		\item[B] Gestione iscrizione.
		\item[C] Gestione vetrina.
		\item[D] Ricerca.
		\item[E] Gestione recensioni.
		\item[F] Gestione notizie.
		\item[G] Gestione follower/followed.
		\item[H] Gestioni suggerimenti.
		\item[I] Gestione account.
		\item[L] Visualizzazione.
	\end{descriptionInd}	
	Le categorie per i requisiti non funzionali sono le seguenti:
	\begin{descriptionInd}
		\item[A] Sicurezza.
		\item[B] Portabilità.
		\item[C] Usabilità.
		\item[D] Implementazione.
	\end{descriptionInd}	

	\item Il campo \descCID{req} è il numero progressivo utilizzato per identificare in modo univoco un requisito all'interno della sua categoria.

	\item Il campo \descCID{sottoreq} (opzionale) è il numero progressivo utilizzato per identificare in modo univoco un sotto--requisito di un dato requisito.
\end{itemize}

\subsection{Priorità}
\label{subsec:priorita}

La priorità di un requisito identifica quanto un requisito sia importante ai fini del completamento del sistema.
Qui di seguito sono elencati in breve i ``gradi di giudizio'' della priorità di un requisito:

\begin{descriptionInd}
	\item[Priorità Alta] Requisiti che il sistema deve necessariamente soddisfare.
	\item[Priorità Media] Requisiti la cui assenza non comporta grossi problemi per la soddisfacibilità del progetto: i requisiti base riescono comunque ad essere soddisfatti, con qualche sacrificio in termini di funzionalità. 
	\item[Priorità Bassa] Requisiti la cui assenza non comporta alcun tipo di problema per la soddisfacibilità del progetto. Questi requisiti possono anche visti come superflui o non strettamente necessari. La loro inclusione comporta un miglioramento della qualità complessiva dell’offerta dei servizi e delle funzionalità del progetto.
\end{descriptionInd}

\section{Elenco dei requisiti}
\label{sec:elenco_dei_requisiti}
Di seguito sono elencati tutti i requisiti, suddivisi in funzionali e non funzionali.

\subsection{Requisiti funzionali}


\label{sub:requisiti_funzionali}
Vengono di seguito elencati i requisiti funzionali:
\begin{itemize}
	\item \textLabel{elencoisc]{\ereq{RF}{100000}{A}} - Iscrizione
	\item \textLabel[elencoiscportale]{\esubreq{RF}{100000}{A}} - Iscrizione tramite modulo del portale
	\item \textLabel{elencoissocial}{\esubreq{RF}{100000}{A}} - Iscrizione tramite Social Network
	\item \esubsubreq{RF}{100000}{A} - Iscrizione tramite Social Network verificati \label{elenco:issocialVer}
	\item \esubsubreq{RF}{100000}{A} - Iscrizione tramite Social Network non verificati \label{elenco:issocialNonVer}
	\item \esubreq{RF}{100000}{A} - Iscrizione tramite approvazione
	\item \textLabel{elencologin}{\ereq{RF}{100000}{A}} - Login
	\countReset
	\item \ereq{RF}{000010}{B} - Approvazione iscrizione
	\item \textLabel{elencorimozioneaccount}{\ereq{RF}{000001}{B}} - Rimozione account altrui
	\countReset
\end{itemize}	


\subsection{Requisiti non funzionali}
\label{sub:requisiti_non_funzionali}


\section{Descrizione dei requisiti}
\label{sec:descrizione_dei_requisiti}

elencoisc\\
\elencoiscportale\\
\elencoissocial\\
\elencologin\\
\elencorimozioneaccount
