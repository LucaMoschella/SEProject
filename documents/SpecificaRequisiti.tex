\chapter{Specifica requisiti} 
\label{cha:specifica_requisiti}

\section{Introduzione} 
In questo documento sono specificati nel dettaglio tutti i requisiti del sistema, suddivisi in funzionali e non funzionali.
Per organizzare in modo corretto le informazioni è stato adottato un sistema di codifica dei requisiti, descritto nella prossima sezione.

\section{Struttura del documento}
\label{sec:struttura_del_documento}

I requisiti mostrati sono prima elencati, per una più facile consultazione, e in seguito sono descritti in apposite tabelle. 
Ogni tabella contiene diverse informazioni:
\begin{itemize}
	\item Un \campiTabReq{titolo} per semplificare il riconoscimento del requisito.
	\item Un \campiTabReq{id} per identificare univocamente il requisito.
	\item Una \campiTabReq{descrizione} per indicare in cosa consiste la funzionalità rappresentata dal requisito.
	\item Un \campiTabReq{tipo} per indicare se il requisito è funzionale o non funzionale
	\item Una \campiTabReq{priorità} per indicare il grado di importanza del requisito.
	\item Lo \campiTabReq{stato} per indicare l'idoneità del requisito.
\end{itemize}

\subsection{Identificativo dei requisiti}
\label{subsec:identificativo_dei_requisiti}

Di seguito è spiegato come interpretare l'identificativo dei requisiti.
Un identificativo è della forma:
\begin{center}
	\req{tipo}{contesto}{categoria}{req}%{[sottoreq]}{[\dots]}
\end{center}

\noindent
Segue una descrizione di ogni campo utilizzato nell'identificatore:
\begin{itemize}
	\item Il \campiIdReq{tipo} può essere:
	\begin{descriptionInd}
		\item[RF] indica un requisito funzionale.
		\item[RNF] indica un requisito non funzionale.
	\end{descriptionInd}

	\item Il \campiIdReq{contesto} serve ad identificare a quali figure è rivolta una data funzionalità. Le lettere che compaiono in questo campo individuano le figure a cui il requisito è rivolto. Ad ogni figura è associata una lettera, come segue:
	\begin{descriptionInd}
		\item[\setTextToLabel{V}{con:visitatore}] Visitatore.
		\item[\setTextToLabel{U}{con:utente}] Utente. 
		\item[\setTextToLabel{P}{con:produttore}] Produttore. 
		\item[\setTextToLabel{R}{con:redattore}] Redattore. 
		\item[\setTextToLabel{S}{con:assistente}] Assistente. 
		\item[\setTextToLabel{M}{con:moderatore}] Moderatore. 
		\item[\setTextToLabel{A}{con:amministratore}] Amministratore. 
	\end{descriptionInd}		

	\item La \campiIdReq{categoria} è una lettera utilizzata per raggruppare i requisiti con scopi simili.
	Le categorie per i requisiti funzionali sono le seguenti:
	\begin{enumerateIndLabel}{\textsf{\Alph*}}{\Alph*}
		\item Autenticazione. \label{cat:autenticazione}
		\item Gestione iscrizioni. \label{cat:iscrizione}
		%\item Gestione iscrizioni. \label{cat:gestioneiscrizione}
		\item Gestione vetrina. \label{cat:vetrina} 
		\item Gestione notizie. \label{cat:notizie}
		\item Gestione suggerimenti. \label{cat:suggerimenti}
		\item Gestione account. \label{cat:account}
		\item Gestione valutazioni. \label{cat:gestionevalutazione}
		\item Gestione recensioni. \label{cat:gestionerecensione}
		\item Interazione tra figure. \label{cat:interazione}
		\item Gestione ticket. \label{cat:gestioneticket}
		\item Ricerca contenuti. \label{cat:ricerca}
		\item Gestione prodotti mancanti. \label{cat:prodottimancanti}
		\item Visualizzazione contenuti. \label{cat:visualizzazione}
	\end{enumerateIndLabel}
	Le categorie per i requisiti non funzionali sono le seguenti:
	\begin{enumerateIndLabel}{\textsf{\Alph*}}{\Alph*}
		\item Sicurezza. \label{cat:sicurezza}
		\item Usabilità. \label{cat:usabilita}
		\item Portabilità. \label{cat:portabilita}
		%\item[D] Implementazione.
		%\item[E] Organizzazione.
	\end{enumerateIndLabel}	

	\item Il campo \campiIdReq{req} è un numero progressivo utilizzato per identificare in modo univoco un requisito all'interno della sua categoria.

	%\item Il campo \campiIdReq{sottoreq} (opzionale) è un numero progressivo utilizzato per identificare in modo univoco un sotto--requisito di un dato requisito. Ci possono essere diversi campi sottoreq consecutivi, per esplicitare una nidificazione dei requisiti più dettagliata.
\end{itemize}

\subsection{Priorità}
\label{subsec:priorita}

La priorità di un requisito identifica quanto un requisito sia importante ai fini del completamento del sistema.
Qui di seguito sono elencati in breve i ``gradi di giudizio'' della priorità di un requisito:
\begin{descriptionInd}
	\item[Priorità Alta] Requisiti che il sistema deve necessariamente soddisfare.

	\item[Priorità Media] Requisiti la cui assenza non comporta grossi problemi per la soddisfacibilità del progetto: i requisiti base riescono comunque ad essere soddisfatti, con qualche sacrificio in termini di funzionalità. 

	\item[Priorità Bassa] Requisiti la cui assenza non comporta alcun tipo di problema per la soddisfacibilità del progetto. Questi requisiti possono anche visti come superflui o non strettamente necessari. La loro inclusione comporta un miglioramento della qualità complessiva dell’offerta dei servizi e delle funzionalità del progetto.
\end{descriptionInd}

\newpage
\section{Elenco dei requisiti}
\label{sec:elenco_dei_requisiti}
Di seguito sono elencati tutti i requisiti, suddivisi in funzionali e non funzionali.

\subsection{Requisiti funzionali}
Vengono di seguito elencati i requisiti funzionali:
%USAGE: {keyrequisito}{\ereq{tipo}{contesto}}{titolo} - doublelink automatico con \refDef{keyrequisito}
\begin{itemize} 
	\setCatR{\ref*{cat:autenticazione}}%A
	\item \newListItem{req:login}{\ereq{RF}{\visitatore\noF\noF\noF\noF\noF\noF}}{Login}
	\item \newListItem{req:logout}{\ereq{RF}{\noF\utente\produttore\noF\noF\noF\noF}}{Logout}

	\setCatR{\ref*{cat:iscrizione}}%B
	%\item \newListItem{req:iscrizione}{\ereq{RF}{\visitatore\noF\noF\noF\noF\noF\noF}}{Iscrizione}
	\item \newListItem{req:iscrizionePortale}{\ereq{RF}{\visitatore\noF\noF\noF\noF\noF\noF}}{Iscrizione tramite modulo del portale}
	%\item \newListItem{req:iscrizioneSocial}{\esubreq{RF}{\visitatore\noF\noF\noF\noF\noF\noF}}{Iscrizione tramite Social Network}
	\item \newListItem{req:iscrizioneSocialVer}{\ereq{RF}{\visitatore\noF\noF\noF\noF\noF\noF}}{Iscrizione tramite Social Network verificati}
	\item \newListItem{req:iscrizioneSocialNotVer}{\ereq{RF}{\visitatore\noF\noF\noF\noF\noF\noF}}{Iscrizione tramite Social Network non verificati}
	\item \newListItem{req:iscrizioneApprovazione}{\ereq{RF}{\visitatore\noF\noF\noF\noF\noF\noF}}{Iscrizione tramite approvazione}
	
	%\setCatR{\ref*{cat:gestioneiscrizione}}%B
	\item \newListItem{req:approvazioneIscrizione}{\ereq{RF}{\noF\noF\noF\noF\assistente\noF\noF}}{Approvazione iscrizione}


	\setCatR{\ref*{cat:vetrina}}%C
	%\item \newListItem{req:personalizzaVetrina}{\ereq{RF}{\noF\noF\produttore\noF\noF\noF\noF}}{Personalizzione vetrina}
	\item \newListItem{req:personalizzaVetrinaInsDesc}{\ereq{RF}{\noF\noF\produttore\noF\noF\noF\noF}}{Inserimento descrizione}
	\item \newListItem{req:personalizzaVetrinaInsImg}{\ereq{RF}{\noF\noF\produttore\noF\noF\noF\noF}}{Inserimento immagine}
	\item \newListItem{req:personalizzaVetrinaInsProd}{\ereq{RF}{\noF\noF\produttore\noF\noF\noF\noF}}{Inserimento prodotto}
	\item \newListItem{req:personalizzaVetrinaModProd}{\ereq{RF}{\noF\noF\produttore\noF\noF\noF\noF}}{Modifica prodotto}
	\item \newListItem{req:statistichePrivateVetrina}{\ereq{RF}{\noF\noF\produttore\noF\noF\noF\noF}}{Visualizzazione statistiche private della vetrina}
	
	\setCatR{\ref*{cat:notizie}}%D
	\item \newListItem{req:inserimentoNotizia}{\ereq{RF}{\noF\noF\noF\redattore\noF\noF\noF}}{Inserimento notizia}
	\item \newListItem{req:modificaNotizia}{\ereq{RF}{\noF\noF\noF\redattore\noF\noF\noF}}{Modifica notizia}
	\item \newListItem{req:rimozioneNotizia}{\ereq{RF}{\noF\noF\noF\redattore\noF\noF\noF}}{Rimozione notizia}
	
	\setCatR{\ref*{cat:suggerimenti}}%E
	\item \newListItem{req:suggerimentoProdotti}{\ereq{RF}{\noF\utente\produttore\noF\noF\noF\noF}}{Suggerimento prodotti intelligente}
	\item \newListItem{req:prodottiSimili}{\ereq{RF}{\noF\utente\produttore\noF\noF\noF\noF}}{Suggerimento prodotti simili}
	\item \newListItem{req:notizieSimili}{\ereq{RF}{\noF\utente\produttore\noF\noF\noF\noF}}{Suggerimento notizie simili}
	
	\setCatR{\ref*{cat:account}}%F
	\item \newListItem{req:accessoProfilo}{\ereq{RF}{\noF\utente\produttore\noF\noF\noF\noF}}{Accesso al proprio profilo}
	\item \newListItem{req:modificaImpostazioni}{\ereq{RF}{\noF\utente\produttore\noF\noF\noF\noF}}{Modifica delle proprie impostazioni}
	\item \newListItem{req:rimozioneAccountProprio}{\ereq{RF}{\noF\utente\produttore\noF\noF\noF\noF}}{Rimozione account proprio}
	
	\item \newListItem{req:rimozioneAccountAltrui}{\ereq{RF}{\noF\noF\noF\noF\noF\noF\amministratore}}{Rimozione account altrui}
	\item \newListItem{req:aggiungiAccountAFigura}{\ereq{RF}{\noF\noF\noF\noF\noF\noF\amministratore}}{Aggiungi funzionalità di una figura ad account}
	\item \newListItem{req:rimuoviAccountDaFigura}{\ereq{RF}{\noF\noF\noF\noF\noF\noF\amministratore}}{Rimuovi funzionalità di una figura da account}


	\setCatR{\ref*{cat:gestionevalutazione}}%G
	%\item \newListItem{req:valutazioneProdotto}{\ereq{RF}{\noF\utente\noF\redattore\assistente\moderatore\amministratore}}{Valutazione prodotto}
	\item \newListItem{req:inserisciValutazioneProdotto}{\ereq{RF}{\noF\utente\noF\noF\noF\noF\noF}}{Inserimento valutazione}
	\item \newListItem{req:modificaValutazioneProdotto}{\ereq{RF}{\noF\utente\noF\noF\noF\noF\noF}}{Modifica valutazione}
	%\item \newListItem{req:eliminaValutazioneProdotto}{\ereq{RF}{\noF\utente\noF\noF\noF\noF\noF}}{Rimozione valutazione}

	\setCatR{\ref*{cat:gestionerecensione}}%H
	%\item \newListItem{req:recensioneProdotto}{\ereq{RF}{\noF\utente\noF\redattore\assistente\moderatore\amministratore}}{Recensione prodotto}
	\item \newListItem{req:inserisciRecensioneProdotto}{\ereq{RF}{\noF\utente\noF\noF\noF\noF\noF}}{Inserimento recensione}
	\item \newListItem{req:modificaRecensioneProdotto}{\ereq{RF}{\noF\utente\noF\noF\noF\noF\noF}}{Modifica recensione propria}
	\item \newListItem{req:eliminaRecensioneProdotto}{\ereq{RF}{\noF\utente\noF\noF\noF\noF\noF}}{Rimozione recensione propria}
	\item \newListItem{req:commentoRecensione}{\ereq{RF}{\noF\utente\produttore\noF\noF\noF\noF}}{Commento a recensione}
	\item \newListItem{req:giudizioRecensione}{\ereq{RF}{\noF\utente\noF\noF\noF\noF\noF}}{Giudizio recensione}
	\item \newListItem{req:modificaGiudizioRecensione}{\ereq{RF}{\noF\utente\noF\noF\noF\noF\noF}}{Modifica giudizio recensione}
	\item \newListItem{req:segnalazioneContenutiInap}{\ereq{RF}{\noF\utente\produttore\noF\noF\noF\noF}}{Segnalazione contenuti inappropriati}
	\item \newListItem{req:visualizzazioneSegnContenutiInap}{\ereq{RF}{\noF\noF\noF\noF\noF\moderatore\noF}}{Visualizzazione segnalazioni contenuti inappropriati}
	\item \newListItem{req:rimozioneContenutiInap}{\ereq{RF}{\noF\noF\noF\noF\noF\moderatore\noF}}{Rimozione contenuti inappropriati}

	\setCatR{\ref*{cat:interazione}}%I
	\item \newListItem{req:followAccount}{\ereq{RF}{\noF\utente\produttore\noF\noF\noF\noF}}{Follow account}
	\item \newListItem{req:unFollowAccount}{\ereq{RF}{\noF\utente\produttore\noF\noF\noF\noF}}{Unfollow account}

	\setCatR{\ref*{cat:gestioneticket}}%L
	%\item \newListItem{req:ticketComunicazione}{\ereq{RF}{\visitatore\utente\produttore\redattore\assistente\moderatore\amministratore}}{Comunicazione tramite ticket}
	\item \newListItem{req:ticketInvio}{\ereq{RF}{\noF\utente\produttore\noF\noF\noF\noF}}{Invio di un ticket}
	\item \newListItem{req:ticketRisposta}{\ereq{RF}{\noF\utente\produttore\noF\assistente\noF\noF}}{Risposta ad un ticket}
	\item \newListItem{req:ticketChiudi}{\ereq{RF}{\noF\noF\noF\noF\assistente\noF\noF}}{Chiusura di un ticket}
	\item \newListItem{req:ticketLettura}{\ereq{RF}{\noF\utente\produttore\noF\assistente\noF\noF}}{Lettura di un ticket}

	\setCatR{\ref*{cat:ricerca}}%M
	%\item \newListItem{req:ricercaInfo}{\ereq{RF}{\visitatore\utente\produttore\redattore\assistente\moderatore\amministratore}}{Ricerca contenuti}
	\item \newListItem{req:ricercaProdotto}{\ereq{RF}{\visitatore\utente\produttore\noF\noF\noF\noF}}{Ricerca prodotto}
	\item \newListItem{req:ricercaProfilo}{\ereq{RF}{\visitatore\utente\produttore\noF\noF\noF\noF}}{Ricerca profilo}
	\item \newListItem{req:ricercaNotizia}{\ereq{RF}{\visitatore\utente\produttore\noF\noF\noF\noF}}{Ricerca notizia}

	\setCatR{\ref*{cat:prodottimancanti}}%N
	\item \newListItem{req:richiestaInsProdotto}{\ereq{RF}{\noF\utente\noF\noF\noF\noF\noF}}{Richiesta inserimento prodotto a un produttore iscritto}
	\item \newListItem{req:visualizzazioneRichiesteInsProdotto}{\ereq{RF}{\noF\noF\produttore\noF\noF\noF\noF}}{Visualizzazione richieste inserimento prodotto}
	\item \newListItem{req:richiestaInsProduttore}{\ereq{RF}{\noF\utente\noF\noF\noF\noF\noF}}{Richiesta inserimento di un produttore}
	
	\setCatR{\ref*{cat:visualizzazione}}%O
	\item \newListItem{req:visualizzazioneVetrina}{\ereq{RF}{\visitatore\utente\produttore\noF\noF\noF\noF}}{Visualizzazione vetrina}
	\item \newListItem{req:visualizzazioneStatistichePubVetrina}{\ereq{RF}{\visitatore\utente\produttore\noF\noF\noF\noF}}{Visualizzazione statische pubbliche vetrina}
	%\item %\newListItem{req:visualizzazioneProdottiVetrina}{\ereq{RF}{\visitatore\utente\produttore\noF\noF\noF\noF}}{Visualizzazione prodotti esposti in vetrina}
	\item \newListItem{req:visualizzazioneProdotto}{\ereq{RF}{\visitatore\utente\produttore\noF\noF\noF\noF}}{Visualizzazione prodotto}
	\item \newListItem{req:visualizzazioneInfoProdotto}{\ereq{RF}{\visitatore\utente\produttore\noF\noF\noF\noF}}{Visualizzazione informazioni del prodotto}
	\item \newListItem{req:visualizzazioneRecProdotto}{\ereq{RF}{\visitatore\utente\produttore\noF\noF\noF\noF}}{Visualizzazione recensioni associate al prodotto}
	\item \newListItem{req:visualizzazioneStatProdotto}{\ereq{RF}{\visitatore\utente\produttore\noF\noF\noF\noF}}{Visualizzazione statistiche del prodotto}
	\item \newListItem{req:visualizzazioneProfilo}{\ereq{RF}{\visitatore\utente\produttore\noF\noF\noF\noF}}{Visualizzazione profilo pubblico}
	\item \newListItem{req:visualizzazioneInfoProfilo}{\ereq{RF}{\visitatore\utente\produttore\noF\noF\noF\noF}}{Visualizzazione informazioni di un profilo pubblico}
	\item \newListItem{req:visualizzazioneStatProfilo}{\ereq{RF}{\visitatore\utente\produttore\noF\noF\noF\noF}}{Visualizzazione statistiche di un profilo pubblico}
	\item \newListItem{req:visualizzazioneNotizie}{\ereq{RF}{\visitatore\utente\produttore\noF\noF\noF\noF}}{Visualizzazione notizie}
	\item \newListItem{req:visualizzazioneAggF}{\ereq{RF}{\noF\utente\produttore\noF\noF\noF\noF}}{Visualizzazione aggiornamenti dai followed}
\end{itemize}	


\subsection{Requisiti non funzionali}
\label{sub:requisiti_non_funzionali}
Vengono di seguito elencati i requisiti non funzionali:
%USAGE: {keyrequisito}{\ereq{tipo}{contesto}{categoria}}{titolo} - doublelink automatico con \refDef{keyrequisito}
\begin{itemize}
	\setCatR{\ref*{cat:sicurezza}}%A
	\item  \newListItem{req:sicurezzaHttps}{\ereq{RNF}{\visitatore\utente\produttore\noF\noF\noF\noF}}{Protollo \texttt{HTTPS}}
	\item  \newListItem{req:sicurezzaDati}{\ereq{RNF}{\visitatore\utente\produttore\noF\noF\noF\noF}}{Protezione dei dati}
	
	\setCatR{\ref*{cat:usabilita}}%B
	\item  \newListItem{req:usabilitaInterfaccia}{\ereq{RNF}{\visitatore\utente\produttore\noF\noF\noF\noF}}{Interfaccia intuitiva del portale}
	\item  \newListItem{req:usabilitaRicerca}{\ereq{RNF}{\visitatore\utente\produttore\noF\noF\noF\noF}}{Semplicità ricerca contenuti}
	
	\setCatR{\ref*{cat:portabilita}}%C
	\item  \newListItem{req:portabilitaBrowser}{\ereq{RNF}{\visitatore\utente\produttore\noF\noF\noF\noF}}{Sito web portabile fra browser}
	\item  \newListItem{req:portabilitaMobile}{\ereq{RNF}{\visitatore\utente\produttore\noF\noF\noF\noF}}{Sito web responsive}

	%\item  \newListItem{req:implementazioneDDF}{\ereq{RNF}{\visitatore\utente\produttore\redattore\assistente\moderatore\amministratore}}{ }
	%\countReset

	%\item  \newListItem{req:standards}{\ereq{RNF}{\noF\noF\noF\noF\noF\noF\noF}}{Rispetto degli standard}
	%\countReset
\end{itemize}

\section{Descrizione dei requisiti}
\label{sec:descrizione_dei_requisiti}
Di seguito sono presenti tutte le descrizioni dei requisiti, suddivise in funzionali e non funzionali.

\subsection{Requisiti funzionali}
Vengono di seguito riportate le tabelle che descrivono nel dettaglio ogni requisito funzionale.
%usage: \reqTabRF{reqkey}{descrizione}{priorità}{stato}
%
%\prioritaH - alta
%\prioritaM - media
%\prioritaL - bassa
%
%\statoOK - approvato
%\statoNOK - non approvato
%\statoUn - da approvare

%%%%%%%%%%%%%%%%%%%%%%%%%%%%%%%%%%%%% A - AUTENTICAZIONE %%%%%%%%%%%%%%%%%%%%%%%%%%%%%%%%%%%%%
\reqTabRF{req:login}{Il portale deve permettere ai visitatori di effettuare il login con i dati inseriti nella fase di iscrizione}{\prioritaH}{\statoUn}

\tabreqvspace

\reqTabRF{req:logout}{Il portale deve permettere di effettuare il logout}{\prioritaH}{\statoUn}

\tabreqvspace

%%%%%%%%%%%%%%%%%%%%%%%%%%%%%%%%%%%%% B - GESTIONE ISCRIZIONE %%%%%%%%%%%%%%%%%%%%%%%%%%%%%%%%%%%%%
%\reqTabRF{req:iscrizione}{Il portale deve permettere l'iscrizione ai visitatori}{\prioritaH}{\statoUn}

\tabreqvspace

\reqTabRF{req:iscrizionePortale}{L'iscrizione dei visitatori deve essere possibile tramite il portale, compilando un apposito modulo e fornendo le informazione necessarie. Tramite questo metodo di iscrizione è possibile creare solo un account utente}{\prioritaH}{\statoUn}

\tabreqvspace

%\reqTabRF{req:iscrizioneSocial}{L'iscrizione dei visitatori deve essere possibile con account terzi di social network}{\prioritaM}{\statoUn}

\tabreqvspace

\reqTabRF{req:iscrizioneSocialVer}{Deve essere possibile l'iscrizione con account verificati di social network. Tramite questo metodo di iscrizione è possibile creare un account utente oppure un account produttore. Per quest'ultimo verrà chiesto di selezionare la tipologia di produttore: \textsc{Industriale} o \textsc{Artigianale}.}{\prioritaM}{\statoUn}

\tabreqvspace

\reqTabRF{req:iscrizioneSocialNotVer}{Deve essere possibile l'iscrizione con account non verificati di social network. Tramite questo metodo di iscrizione è possibile creare solo un account utente}{\prioritaM}{\statoUn}

\tabreqvspace

\reqTabRF{req:iscrizioneApprovazione}{L'iscrizione dei visitatori deve essere possibile richiedendo l'approvazione di un account, con apposita procedura. Tramite questo metodo di iscrizione è possibile creare solo un account produttore. Verrà chiesto di selezionare la tipologia di produttore: \textsc{Industriale} o \textsc{Artigianale}}{\prioritaH}{\statoUn}

\tabreqvspace

\reqTabRF{req:approvazioneIscrizione}{Il portale deve permettere agli assistenti di rispondere alle richieste di iscrizione effettuate tramite la procedura che richiede approvazione}{\prioritaH}{\statoUn}

\tabreqvspace


%%%%%%%%%%%%%%%%%%%%%%%%%%%%%%%%%%%%% C - GESTIONE VETRINA %%%%%%%%%%%%%%%%%%%%%%%%%%%%%%%%%%%%%
%\reqTabRF{req:personalizzaVetrina}{Il portale deve permettere la personalizzazione della vetrina ad ogni produttore}{\prioritaH}{\statoUn}

\reqTabRF{req:personalizzaVetrinaInsDesc}{Deve essere possibile per il produttore inserire una descrizione nella propria vetrina}{\prioritaM}{\statoUn}

\tabreqvspace

\reqTabRF{req:personalizzaVetrinaInsImg}{Deve essere possibile per il produttore inserire un'immagine nella propria vetrina}{\prioritaM}{\statoUn}

\tabreqvspace

%Vedi se va qui la descrizione dettagliata dell'inserimento!
\reqTabRF{req:personalizzaVetrinaInsProd}{Deve essere possibile per il produttore inserire un prodotto nella propria vetrina}{\prioritaH}{\statoUn}

\tabreqvspace

\reqTabRF{req:personalizzaVetrinaModProd}{Deve essere possibile per il produttore modificare un prodotto già inserito nella propria vetrina}{\prioritaH}{\statoUn}

\tabreqvspace

\reqTabRF{req:statistichePrivateVetrina}{Il portale deve permettere al produttore l'accesso alle statistiche private della sua vetrina}{\prioritaM}{\statoUn}

\tabreqvspace

%%%%%%%%%%%%%%%%%%%%%%%%%%%%%%%%%%%%% H - GESTIONE NOTIZIE %%%%%%%%%%%%%%%%%%%%%%%%%%%%%%%%%%%%%
\reqTabRF{req:inserimentoNotizia}{Il portale deve permettere ai redattori di inserire una notizia}{\prioritaH}{\statoUn}

\tabreqvspace

\reqTabRF{req:modificaNotizia}{Il portale deve permettere ai redattori di modificare una notizia}{\prioritaH}{\statoUn}

\tabreqvspace

\reqTabRF{req:rimozioneNotizia}{Il portale deve permettere ai redattori di eliminare una notizia}{\prioritaH}{\statoUn}

\tabreqvspace

%%%%%%%%%%%%%%%%%%%%%%%%%%%%%%%%%%%%% H - GESTIONE SUGGERIMENTI %%%%%%%%%%%%%%%%%%%%%%%%%%%%%%%%%%%%%
\reqTabRF{req:suggerimentoProdotti}{Il portale deve permettere di ricevere suggerimenti riguardo quali altri prodotti potrebbero piacere}{\prioritaL}{\statoUn}

\tabreqvspace

\reqTabRF{req:prodottiSimili}{Il portale deve permettere di ricevere suggerimenti riguardo quali altri prodotti sono simili a quello che si sta visualizzando}{\prioritaL}{\statoUn}

\tabreqvspace

\reqTabRF{req:notizieSimili}{Il portale deve permettere di ricevere suggerimenti riguardo quali altre notizie sono simili a quella che si sta visualizzando}{\prioritaL}{\statoUn}

\tabreqvspace

%%%%%%%%%%%%%%%%%%%%%%%%%%%%%%%%%%%%% H - GESTIONE ACCOUNT %%%%%%%%%%%%%%%%%%%%%%%%%%%%%%%%%%%%%
\reqTabRF{req:accessoProfilo}{Il portale deve permette l'accesso al proprio profilo per visualizzarne i contenuti}{\prioritaM}{\statoUn}

\tabreqvspace

\reqTabRF{req:modificaImpostazioni}{Il portale deve permettere l'accesso alle impostazioni e la modifica di queste ultime}{\prioritaM}{\statoUn}

\tabreqvspace

\reqTabRF{req:rimozioneAccountProprio}{Il portale deve permettere l'eliminazione del proprio account. L'eliminazione di un account non comporterà l'eliminazione dei contenuti inseriti da esso.}{\prioritaL}{\statoUn}

\tabreqvspace

\reqTabRF{req:rimozioneAccountAltrui}{Il portale deve permettere agli amministratori di rimuovere un account}{\prioritaL}{\statoUn}

\tabreqvspace

\reqTabRF{req:aggiungiAccountAFigura}{Il portale deve permette agli amministratori di aggiungere un account presente nel sistema al gruppo associato ad una figura. 
Ciò significa che gli amministratori potranno aggiungere un account al gruppo degli account che hanno accesso alle funzionalità di una data figura, fornendogli tali funzionalità aggiuntive.}{\prioritaL}{\statoUn}

\tabreqvspace

\reqTabRF{req:rimuoviAccountDaFigura}{Il portale deve permettere agli amministratori di rimuovere un account presente nel sistema dal gruppo associato ad una figura. Ciò significa che gli amministratori potranno rimuovere un account dal gruppo degli account che hanno accesso alle funzionalità di una data figura, rimuovendogli l'accesso a tali funzionalità, se non presenti in altre figure associate all'account.}{\prioritaL}{\statoUn}

%%%%%%%%%%%%%%%%%%%%%%%%%%%%%%%%%%%%% H - INTERAZIONE TRA FIGURE %%%%%%%%%%%%%%%%%%%%%%%%%%%%%%%%%%%%%
%\reqTabRF{req:valutazioneProdotto}{Il portale deve permettere di valutare un prodotto}{\prioritaH}{\statoUn}

\tabreqvspace

\reqTabRF{req:inserisciValutazioneProdotto}{Deve essere possibile inserire una valutazione di un dato prodotto, cioè un voto in voto in una scala limitata}{\prioritaH}{\statoUn}

\tabreqvspace

\reqTabRF{req:modificaValutazioneProdotto}{Deve essere possibile modificare una propria valutazione}{\prioritaM}{\statoUn}

%\tabreqvspace

%\reqTabRF{req:eliminaValutazioneProdotto}{Deve essere possibile rimuovere una propria valutazione}{\prioritaM}{\statoUn}

%\tabreqvspace

%\reqTabRF{req:recensioneProdotto}{Il portale deve permettere di recensire di un prodotto}{\prioritaH}{\statoUn}

\tabreqvspace

\reqTabRF{req:inserisciRecensioneProdotto}{Deve essere possibile inserire una recensione di un dato prodotto, cioè una valutazione e una descrizione testuale}{\prioritaH}{\statoUn}

\tabreqvspace

\reqTabRF{req:modificaRecensioneProdotto}{Deve essere possibile modificare una propria recensione}{\prioritaM}{\statoUn}

\tabreqvspace

\reqTabRF{req:eliminaRecensioneProdotto}{Deve essere possibile rimuovere una propria recensione}{\prioritaM}{\statoUn}

\tabreqvspace

\reqTabRF{req:commentoRecensione}{Il portale deve permettere l'inserimento di un commento in una recensione}{\prioritaM}{\statoUn}

\tabreqvspace

\reqTabRF{req:giudizioRecensione}{Il portale deve permettere di esprimere un giudizio, positivo o negativo, di una recensione}{\prioritaM}{\statoUn}

\tabreqvspace

\reqTabRF{req:modificaGiudizioRecensione}{Il portale deve permettere di modificare un giudizio, positivo o negativo, espresso su una recensione}{\prioritaM}{\statoUn}

\tabreqvspace

\reqTabRF{req:segnalazioneContenutiInap}{Il portale deve permettere di segnalare contenuti inappropriati}{\prioritaL}{\statoUn}

\tabreqvspace

\reqTabRF{req:visualizzazioneSegnContenutiInap}{Il portale deve permettere ai moderatori la visualizzazione delle segnalazioni dei contenuti inappropriati}{\prioritaL}{\statoUn}

\tabreqvspace

\reqTabRF{req:rimozioneContenutiInap}{Il portale deve permettere ai moderatori la rimozione di contenuti inappropriati}{\prioritaL}{\statoUn}

\tabreqvspace

\reqTabRF{req:followAccount}{Il portale deve permettere di seguire un account per ricevere aggiornamenti da esso}{\prioritaH}{\statoUn}

\tabreqvspace

\reqTabRF{req:unFollowAccount}{Il portale deve permettere di cessare di seguire un account che si segue}{\prioritaH}{\statoUn}

%\tabreqvspace

%\reqTabRF{req:ticketComunicazione}{Il portale deve permettere alle figure di comunicare con gli assistenti tramite ticket}{\prioritaM}{\statoUn}

\tabreqvspace

\reqTabRF{req:ticketInvio}{Deve essere possibile creare ed inviare diversi tipi di ticket per comunicare con gli assistenti}{\prioritaM}{\statoUn}

\tabreqvspace

\reqTabRF{req:ticketRisposta}{Deve essere possibile rispondere ad un ticket. Le risposte ad un ticket possono essere effettuate da un assistente o dal creatore dello stesso}{\prioritaM}{\statoUn}

\tabreqvspace

\reqTabRF{req:ticketChiudi}{Deve essere possibile per gli assistenti chiudere un ticket}{\prioritaM}{\statoUn}

\tabreqvspace

\reqTabRF{req:ticketLettura}{Deve essere possibile visualizzare il ticket dal suo creatore per controllare lo status e dagli assistenti per gestirli}{\prioritaM}{\statoUn}

\tabreqvspace



%%%%%%%%%%%%%%%%%%%%%%%%%%%%%%%%%%%%% H - RICERCA CONTENUTI %%%%%%%%%%%%%%%%%%%%%%%%%%%%%%%%%%%%%
%\reqTabRF{req:ricercaInfo}{Il portale deve permettere di effettuare ricerche in base a diversi criteri}{\prioritaH}{\statoUn}

\tabreqvspace

\reqTabRF{req:ricercaProdotto}{Deve essere possibili cercare prodotti in base a diversi criteri}{\prioritaH}{\statoUn}

\tabreqvspace

\reqTabRF{req:ricercaProfilo}{Deve essere possibili cercare profili pubblici in base a diversi criteri}{\prioritaM}{\statoUn}

\tabreqvspace

\reqTabRF{req:ricercaNotizia}{Deve essere possibili cercare notizie in base a diversi criteri}{\prioritaM}{\statoUn}

\tabreqvspace

\reqTabRF{req:richiestaInsProdotto}{Deve essere possibile segnalare ad un produttore la mancanza di un suo prodotto tra quelli presenti nel sistema. La segnalazione conterrà una scheda prodotto parzialmente compilata dall'utente per favorirne l'aggiunta da parte del produttore}{\prioritaL}{\statoUn}

\tabreqvspace

\reqTabRF{req:visualizzazioneRichiesteInsProdotto}{Il produttore potrà visualizzare le richieste di inserimento prodotto indirizzate a lui, con la possibilità di inserire direttamente il prodotto, modificarne la scheda o rifiutare la richiesta.}{\prioritaL}{\statoUn}

\tabreqvspace

\reqTabRF{req:richiestaInsProduttore}{Deve essere possibile segnalare al sistema l’assenza di un produttore}{\prioritaL}{\statoUn}

\tabreqvspace

%%%%%%%%%%%%%%%%%%%%%%%%%%%%%%%%%%%%% H - VISUALIZZIONE CONTENUTI %%%%%%%%%%%%%%%%%%%%%%%%%%%%%%%%%%%%%
\reqTabRF{req:visualizzazioneVetrina}{Il portale deve permettere di visualizzare una data vetrina}{\prioritaH}{\statoUn}

\tabreqvspace

\reqTabRF{req:visualizzazioneStatistichePubVetrina}{Deve essere possibile visualizzare le statistiche pubbliche di una data vetrina}{\prioritaL}{\statoUn}

%\tabreqvspace
%
%\reqTabRF{req:visualizzazioneProdottiVetrina}{Deve essere possibile visualizzare i prodotti presenti in una data vetrina}{\prioritaH}{\statoUn}

\tabreqvspace

\reqTabRF{req:visualizzazioneProdotto}{Il portale deve permettere di visualizzare un dato prodotto}{\prioritaH}{\statoUn}

\tabreqvspace

\reqTabRF{req:visualizzazioneInfoProdotto}{Deve essere possibile visualizzare le informazioni di un dato prodotto}{\prioritaH}{\statoUn}

\tabreqvspace

\reqTabRF{req:visualizzazioneRecProdotto}{Deve essere possibile visualizzare le recensioni di un dato prodotto}{\prioritaH}{\statoUn}

\tabreqvspace

\reqTabRF{req:visualizzazioneStatProdotto}{Deve essere possibile visualizzare le statistiche di un dato prodotto}{\prioritaM}{\statoUn}

\tabreqvspace

\reqTabRF{req:visualizzazioneProfilo}{Il portale deve permettere di visualizzare un dato profilo}{\prioritaM}{\statoUn}

\tabreqvspace

\reqTabRF{req:visualizzazioneInfoProfilo}{Deve essere possibile visualizzare le informazioni di un dato profilo}{\prioritaM}{\statoUn}

\tabreqvspace

\reqTabRF{req:visualizzazioneStatProfilo}{Deve essere possibile visualizzare le statistiche di un dato profilo}{\prioritaM}{\statoUn}

\tabreqvspace

\reqTabRF{req:visualizzazioneNotizie}{Il portale deve permettere di visualizzare le notizie}{\prioritaH}{\statoUn}

\tabreqvspace

\reqTabRF{req:visualizzazioneAggF}{Il portale deve permettere di visualizzare gli aggiornamenti ricevuti dagli account che si stanno seguendo}{\prioritaH}{\statoUn}

\subsection{Requisiti non funzionali}
Vengono di seguito riportate le tabelle che descrivono nel dettaglio ogni requisito non funzionale.

%%%%%%%%%%%%%%%%%%%%%%%%%%%%%%%%%%%%% SICUREZZA %%%%%%%%%%%%%%%%%%%%%%%%%%%%%%%%%%%%%
\reqTabRNF{req:sicurezzaHttps}{Il portale garantirà l'autenticità e la confidenzialità delle comunicazioni attraverso l'utilizzo del protocollo \texttt{HTTPS}}{\prioritaH}{\statoUn}

\tabreqvspace

\reqTabRNF{req:sicurezzaDati}{Il portale garantirà la protezione dei dati sensibili di tutte le figure che interagiscono con esso}{\prioritaH}{\statoUn}

\tabreqvspace

%%%%%%%%%%%%%%%%%%%%%%%%%%%%%%%%%%%%% USABILITA %%%%%%%%%%%%%%%%%%%%%%%%%%%%%%%%%%%%%
\reqTabRNF{req:usabilitaInterfaccia}{Il portale disporrà di un interfaccia semplice e intuitiva in modo tale da avere una curva di apprendimento bassa, affinché chiunque sia in un grado di utilizzarlo appieno}{\prioritaM}{\statoUn}

\tabreqvspace

\reqTabRNF{req:usabilitaRicerca}{Il portale disporrà di metodi di ricerca intuitivi per permettere di effettuare in poco tempo ricerche affidabili}{\prioritaH}{\statoUn}

\tabreqvspace

%%%%%%%%%%%%%%%%%%%%%%%%%%%%%%%%%%%%% PORTABILITA %%%%%%%%%%%%%%%%%%%%%%%%%%%%%%%%%%%%%
\reqTabRNF{req:portabilitaBrowser}{Il sito web del portale sarà portabile fra i principali browser web presenti nel mercato}{\prioritaH}{\statoUn}

\tabreqvspace

\reqTabRNF{req:portabilitaMobile}{Il sito web del portale sarà \emph{responsive} in modo tale da poter essere consultato agevolmente da qualsiasi dispositivo}{\prioritaH}{\statoUn}

\newdate{requno}{09}{05}{2016}
\newdate{reqdue}{13}{05}{2016}
\section{Revisioni}
\begin{center}
    \begin{tabular}{lll}
        \toprule
        	\tabhead{Versione} & \tabhead{Data} & \tabhead{Descrizione} \\
		\cmidrule(l{\cmidrulekern}r{\cmidrulekern}){1-3}
        	1.0 & \displaydate{requno} & Prima versione \\
        	1.1 & \displaydate{reqdue} & Riorganizzati i requisiti \\
        \bottomrule
    \end{tabular}
\end{center}
