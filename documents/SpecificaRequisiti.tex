\chapter{Specifica requisiti} 
\label{cha:specifica_requisiti}

\section{Introduzione} 
In questo documento sono specificati nel dettaglio tutti i requisiti del sistema, suddivisi in funzionali e non funzionali.
Per organizzare in modo corretto le informazioni è stato adottato un sistema di codifica dei requisiti, descritto nella prossima sezione.

\section{Struttura del documento}
\label{sec:struttura_del_documento}

I requisiti mostrati sono prima elencati, per una più facile consultazione, e in seguito sono descritti in apposite tabelle. 
Ogni tabella contiene diverse informazioni:
\begin{itemize}
	\item Un \campiTabReq{id} per identificare univocamente il requisito.
	\item Un \campiTabReq{titolo} per semplificare il riconoscimento del requisito.
	\item Una \campiTabReq{descrizione} per indicare in cosa consiste la funzionalità rappresentata dal requisito.
	\item Un \campiTabReq{tipo} per indicare se il requisito è funzionale o non funzionale
	\item Una \campiTabReq{priorità} per indicare il grado di importanza del requisito.
	\item Lo \campiTabReq{stato} per indicare l'idoneità del requisito.
\end{itemize}

\subsection{Identificativo dei requisiti}
\label{subsec:identificativo_dei_requisiti}

Di seguito è spiegato come interpretare l'identificativo dei requisiti.
Un identificativo è della forma:
\begin{center}
	\subsubreq{tipo}{contesto}{categoria}{req}{[sottoreq]}{[\dots]}
\end{center}

\noindent
Segue una descrizione di ogni campo utilizzato nell'identificatore:
\begin{itemize}
	\item Il \campiIdReq{tipo} può essere:
	\begin{descriptionInd}
		\item[RF] indica un requisito funzionale.
		\item[RNF] indica un requisito non funzionale.
	\end{descriptionInd}

	\item Il \campiIdReq{contesto} serve ad identificare a quali figure è rivolta una data funzionalità, è un numero composto da 6 cifre ognuna delle quali può assumere un valore 0 oppure 1. La posizione della cifra si riferisce ad una specifica figura, se la cifra i-esima vale 1 significa che l'i-esima figura è presente nel contesto del requisitio, altrimenti no.
	Di seguito elenchiamo ogni posizione a quale figura riferisce: 
	\begin{descriptionInd}
		\item[1\textdegree\ bit] Visitatore.
		\item[2\textdegree\ bit] Utente.
		\item[3\textdegree\ bit] Produttore.
		\item[4\textdegree\ bit] Redattore.
		\item[5\textdegree\ bit] Assistente.
		\item[6\textdegree\ bit] Moderatore.
	\end{descriptionInd}		

	\item La \campiIdReq{categoria} è una lettera utilizzata per raggruppare i requisiti con scopi simili.
	Le categorie per i requisiti funzionali sono le seguenti:
	\begin{descriptionInd}
		\item[A] Iscrizione e autenticazione.
		\item[B] Gestione iscrizione.
		\item[C] Gestione vetrina.
		\item[D] Ricerca.
		\item[E] Gestione recensioni.
		\item[F] Gestione notizie.
		\item[G] Gestione follower/followed.
		\item[H] Gestioni suggerimenti.
		\item[I] Gestione account.
		\item[L] Visualizzazione contenuti.
	\end{descriptionInd}	
	Le categorie per i requisiti non funzionali sono le seguenti:
	\begin{descriptionInd}
		\item[A] Sicurezza.
		\item[B] Portabilità.
		\item[C] Usabilità.
		\item[D] Implementazione.
	\end{descriptionInd}	

	\item Il campo \campiIdReq{req} è un numero progressivo utilizzato per identificare in modo univoco un requisito all'interno della sua categoria.

	\item Il campo \campiIdReq{sottoreq} (opzionale) è un numero progressivo utilizzato per identificare in modo univoco un sotto--requisito di un dato requisito. Ci possono essere diversi campi sottoreq consecutivi, per esplicitare una nidificazione dei requisiti più dettagliata.
\end{itemize}

\subsection{Priorità}
\label{subsec:priorita}

La priorità di un requisito identifica quanto un requisito sia importante ai fini del completamento del sistema.
Qui di seguito sono elencati in breve i ``gradi di giudizio'' della priorità di un requisito:

\begin{descriptionInd}
	\item[Priorità Alta] Requisiti che il sistema deve necessariamente soddisfare.
	\item[Priorità Media] Requisiti la cui assenza non comporta grossi problemi per la soddisfacibilità del progetto: i requisiti base riescono comunque ad essere soddisfatti, con qualche sacrificio in termini di funzionalità. 
	\item[Priorità Bassa] Requisiti la cui assenza non comporta alcun tipo di problema per la soddisfacibilità del progetto. Questi requisiti possono anche visti come superflui o non strettamente necessari. La loro inclusione comporta un miglioramento della qualità complessiva dell’offerta dei servizi e delle funzionalità del progetto.
\end{descriptionInd}

\section{Elenco dei requisiti}
\label{sec:elenco_dei_requisiti}
Di seguito sono elencati tutti i requisiti, suddivisi in funzionali e non funzionali.
\nopagebreak
\subsection{Requisiti funzionali}
\label{sub:requisiti_funzionali}
Vengono di seguito elencati i requisiti funzionali:
%USAGE: {tipo}{contesto}{categoria}{keyrequisito} - doublelink automatico con \refElencoFromDef{keyrequisito}
\begin{itemize} 
	\item \reqLabel{req:iscrizione}{\ereq{RF}{100000}{A}} - Iscrizione
	\item \reqLabel{req:iscrizionePortale}{\esubreq{RF}{100000}{A}} - Iscrizione tramite modulo del portale 	
	\item \reqLabel{req:iscrizioneSocial}{\esubreq{RF}{100000}{A}} - Iscrizione tramite Social Network	
	\item \reqLabel{req:iscrizioneSocialVer}{\esubsubreq{RF}{100000}{A}} - Iscrizione tramite Social Network verificati
	\item \reqLabel{req:iscrizioneSocialNotVer}{\esubsubreq{RF}{100000}{A}} - Iscrizione tramite Social Network non verificati	
	\item \reqLabel{req:iscrizioneApprovazione}{\esubreq{RF}{100000}{A}} - Iscrizione tramite approvazione
	
	\item \reqLabel{req:login}{\ereq{RF}{100000}{A}} - Login 
	\countReset 
	
	\item \reqLabel{req:approvazioneIscrizione}{\ereq{RF}{000010}{B}} - Approvazione iscrizione
	
	\item \reqLabel{req:rimozioneAccountAltrui}{\ereq{RF}{000001}{B}} - Rimozione account altrui 
	\countReset
	
	\item \reqLabel{req:personalizzaVetrina}{\ereq{RF}{001000}{C}} - Personalizzione vetrina	
	\item \reqLabel{req:personalizzaVetrinaInsDesc}{\esubreq{RF}{001000}{C}} - Inserimento descrizione	
	\item \reqLabel{req:personalizzaVetrinaInsImg}{\esubreq{RF}{001000}{C}} - Inserimento immagine
	\item \reqLabel{req:personalizzaVetrinaInsProd}{\esubreq{RF}{001000}{C}} - Inserimento prodotto
	\item \reqLabel{req:personalizzaVetrinaModProd}{\esubreq{RF}{001000}{C}} - Modifica prodotto
	
	\item \reqLabel{req:statistichePrivateVetrina}{\ereq{RF}{001000}{C}} - Visualizzazione statistiche private della vetrina
	\countReset
	
	\item \reqLabel{req:ricercaInfo}{\ereq{RF}{111111}{D}} - Ricerca informazioni	
	\item \reqLabel{req:ricercaProdotto}{\esubreq{RF}{111111}{D}} - Ricerca prodotto
	\item \reqLabel{req:ricercaProfilo}{\esubreq{RF}{111111}{D}} - Ricerca profilo	
	\item \reqLabel{req:ricercaNotizia}{\esubreq{RF}{111111}{D}} - Ricerca notizia

	\item \reqLabel{req:richiestaInsProdotto}{\ereq{RF}{010000}{D}} - Richiesta inserimento di un prodotto a un produttore iscritto

	\item \reqLabel{req:richiestaInsProduttore}{\ereq{RF}{010000}{D}} - Richiesta inserimento di un prodotto e del suo produttore
	\countReset

	\item \reqLabel{req:valutazioneProdotto}{\ereq{RF}{010111}{E}} - Valutazione prodotto
	\item \reqLabel{req:inserisciValutazioneProdotto}{\esubreq{RF}{010111}{E}} - Inserimento valutazione
	\item \reqLabel{req:modificaValutazioneProdotto}{\esubreq{RF}{010111}{E}} - Modifica valutazione inserita
	\item \reqLabel{req:eliminaValutazioneProdotto}{\esubreq{RF}{010111}{E}} - Rimozione valutazione inserita

	\item \reqLabel{req:recensioneProdotto}{\ereq{RF}{010111}{E}} - Recensione prodotto
	\item \reqLabel{req:inserisciRecensioneProdotto}{\esubreq{RF}{010111}{E}} - Inserimento recensione
	\item \reqLabel{req:modificaRecensioneProdotto}{\esubreq{RF}{010111}{E}} - Modifica valutazione inserita
	\item \reqLabel{req:eliminaRecensioneProdotto}{\esubreq{RF}{010111}{E}} - Rimozione recensione inserita

	\item \reqLabel{req:commentoRecensione}{\ereq{RF}{011111}{E}} - Commento a recensione
	
	\item \reqLabel{req:valutaRecensione}{\ereq{RF}{010111}{E}} - Valutazione recensione
	
	\item \reqLabel{req:segnalazioneContenuti}{\ereq{RF}{011111}{E}} - Segnalazione contenuti inappropriati
	
	\item \reqLabel{req:rimozioneRecensioneInap}{\ereq{RF}{000001}{E}} - Rimozione recensione inappropriata
	\countReset

	\item \reqLabel{req:inserimentoNotizia}{\ereq{RF}{000100}{F}} - Inserimento notizia
	
	\item \reqLabel{req:modificaNotizia}{\ereq{RF}{000100}{F}} - Modifica notizia
	
	\item \reqLabel{req:rimozioneNotizia}{\ereq{RF}{000100}{F}} - Rimozione notizia
	\countReset
	
	\item \reqLabel{req:followAccount}{\ereq{RF}{011111}{G}} - Follow account
	
	\item \reqLabel{req:unFollowAccount}{\ereq{RF}{011111}{G}} - Unfollow account
	\countReset

	\item \reqLabel{req:suggerimentoProdotti}{\ereq{RF}{011111}{H}} - Suggerimento prodotti intelligente
	
	\item \reqLabel{req:prodottiSimili}{\ereq{RF}{011111}{H}} - Suggerimento prodotti simili

	\item \reqLabel{req:notizieSimili}{\ereq{RF}{011111}{H}} - Suggerimento notizie simili
	\countReset

	\item \reqLabel{req:accessoProfilo}{\ereq{RF}{011111}{I}} - Accesso al proprio profilo
	
	\item \reqLabel{req:accessoImpostazioni}{\ereq{RF}{011111}{I}} - Accesso alle proprie impostazioni

	\item \reqLabel{req:rimozioneAccountProprio}{\ereq{RF}{011111}{I}} - Rimozione account proprio 
	
	\item \reqLabel{req:logout}{\ereq{RF}{011111}{I}} - Logout
	\countReset
	
	\item \reqLabel{req:visualizzazioneVetrina}{\ereq{RF}{111111}{L}} - Visualizzazione vetrina
	\item \reqLabel{req:visualizzazioneStatistichePubVetrina}{\esubreq{RF}{111111}{L}} - Visualizzazione statische pubbliche vetrina
	\item \reqLabel{req:visualizzazioneProdottiVetrina}{\esubreq{RF}{111111}{L}} - Visualizzazione prodotti esposti in vetrina

	\item \reqLabel{req:visualizzazioneProdotto}{\ereq{RF}{111111}{L}} - Visualizzazione prodotto
	\item \reqLabel{req:visualizzazioneInfoProdotto}{\esubreq{RF}{111111}{L}} - Visualizzazione le informazioni del prodotto
	\item \reqLabel{req:visualizzazioneRecProdotto}{\esubreq{RF}{111111}{L}} - Visualizzazione le recensioni associate al prodotto
	\item \reqLabel{req:visualizzazioneStatProdotto}{\esubreq{RF}{111111}{L}} - Visualizzazione le statistiche del prodotto

	\item \reqLabel{req:visualizzazioneProfilo}{\ereq{RF}{111111}{L}} - Visualizzazione un profilo pubblico
	\item \reqLabel{req:visualizzazioneInfoProfilo}{\esubreq{RF}{111111}{L}} - Visualizzazione informazioni di un profilo pubblico
	\item \reqLabel{req:visualizzazioneStatProfilo}{\esubreq{RF}{111111}{L}} - Visualizzazione statistiche di un profilo pubblico

	\item \reqLabel{req:visualizzazioneNotizie}{\ereq{RF}{111111}{L}} - Visualizzazione notizie
	
	\item \reqLabel{req:visualizzazioneAggF}{\ereq{RF}{011111}{L}} - Visualizzazione aggiornamenti dai followed
\end{itemize}	


\subsection{Requisiti non funzionali}
\label{sub:requisiti_non_funzionali}


\section{Descrizione dei requisiti}
\label{sec:descrizione_dei_requisiti}
Questi verranno linkati dall'elenco, sono le ``definizioni'' dei sequisiti:\\
\reqDef{req:iscrizione} \\
\reqDef{req:iscrizionePortale} \\
\reqDef{req:iscrizioneSocial} \\
\reqDef{req:iscrizioneSocialVer} \\
\reqDef{req:iscrizioneSocialNotVer} \\
\reqDef{req:iscrizioneApprovazione}\\
\reqDef{req:login}\\
\reqDef{req:approvazioneIscrizione}\\
\reqDef{req:rimozioneAccountAltrui}\\

Ma possiamo riusare i nomi, linkandoli all'elenco, senza creare problemi al link elenco -> def!\\
\reqID{req:iscrizione} \\
\reqID{req:iscrizionePortale} \\
\reqID{req:iscrizioneSocial} \\
\reqID{req:iscrizioneSocialVer} \\
\reqID{req:iscrizioneSocialNotVer} \\
\reqID{req:iscrizioneApprovazione}\\
\reqID{req:login}\\
\reqID{req:approvazioneIscrizione}\\
\reqID{req:rimozioneAccount}\\

Provaidsvuiosbvcosneui: \\
prova: \reqDef{req:prova}