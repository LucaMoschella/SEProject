\chapter{Stima dei costi}

\section{Introduzione}
In questo documento utilizzeremo il metodo \gls{ucp} per calcolare la stima della dimensione del progetto. Tale metodo si basa sul calcolo dei seguenti elementi:
\begin{itemize}
	\item \gls{uaw} una stima che tiene del numero e della complessità degli attori
	\item \gls{uucw} una stima che tiene del numero e della complessità dei casi d'uso
	\item \gls{tcf} un fattore che è usato per aggiustare la stima basandosi su considerazioni tecniche
	\item \gls{ecf} un fattore che è usato per aggiustare la stima basandosi su considerazioni ambientali
\end{itemize}

\section{Stima della complessità degli attori}
Gli attori vengono classificati in una scala di tre valori, in base alla loro complessità. Il metodo \gls{ucp} definisce le seguenti direttive per catalogare gli attori:
\begin{center}
	\begin{tabularx}{\widthTab}{ l  X  l} 
		\toprule
			\formattaTitoloTab{Classificazione} & \formattaTitoloTab{Tipo di attore} & \formattaTitoloTab{Peso} \\
		\cmidrule(l{\cmidrulekern}r{\cmidrulekern}){1-3}
			\formattaCampiTab{Simple} & L'attore è un sistema esterno che utilizza una \gls{api} & 1 \\ 
			\addlinespace[1em] 
			\formattaCampiTab{Average} & L'attore è un sistema esterno che utilizza un protocollo (ad esempio TCP/IP, FTP, HTTP) & 2 \\ 
			\addlinespace[1em] 
			\formattaCampiTab{Complex} & L'attore è un utente che utilizza una \gls{gui} & 3 \\ 
		\bottomrule
	\end{tabularx}
\end{center}
La stima del costo degli attori (\gls{uaw}) è data dalla seguente formula:
\begin{displaymath}
UAW = (NoSimpleActors \times 1) + (NoAvgActors \times 2) + (NoComplexActors \times 3)
\end{displaymath}
Di seguito riportiamo la tabella con indicati pesi di ogni attore:
\begin{center}
	\begin{tabularx}{\widthTab}{ l  X  l} 
		\toprule
			\formattaTitoloTab{ID} & \formattaTitoloTab{Attore} & \formattaTitoloTab{Peso} \\
		\cmidrule(l{\cmidrulekern}r{\cmidrulekern}){1-3}
			\rowIDTitle{att:visitatore} & 3 \\ 
			\addlinespace[1em] 
			\rowIDTitle{att:figuraPubblicaAutenticata} & 3 \\ 
			\addlinespace[1em] 
			\rowIDTitle{att:figuraAmministrativa} & 3 \\ 
			\addlinespace[1em] 
			\rowIDTitle{att:utente} & 3 \\ 
			\addlinespace[1em] 
			\rowIDTitle{att:produttore} & 3 \\ 
			\addlinespace[1em] 
			\rowIDTitle{att:redattore} & 3 \\ 
			\addlinespace[1em] 
			\rowIDTitle{att:assistente} & 3 \\ 
			\addlinespace[1em] 
			\rowIDTitle{att:moderatore} & 3 \\ 
			\addlinespace[1em] 
			\rowIDTitle{att:amministratore} & 3 \\ 
			\addlinespace[1em] 
			\rowIDTitle{att:cms} & 2 \\ 
		\cmidrule(l{\cmidrulekern}r{\cmidrulekern}){1-3}
			\formattaCampiTab{\gls{uaw}}  & & 29\\
		\bottomrule
	\end{tabularx}
\end{center}

\section{Stima della complessità dei casi d'uso}
I casi d'uso vengono classificati in una scala di tre valori, in base alla loro complessità. Il metodo \gls{ucp} definisce le seguenti direttive per catalogare i casi d'uso:
\begin{center}
	\begin{tabularx}{\widthTab}{ l  X  l} 
		\toprule
			\formattaTitoloTab{Classificazione} & \formattaTitoloTab{Numero di transazioni} & \formattaTitoloTab{Peso} \\
		\cmidrule(l{\cmidrulekern}r{\cmidrulekern}){1-3}
			\formattaCampiTab{Simple} & Un numero di transazioni compreso fra 1 e 3 & 5 \\ 
			\addlinespace[1em] 
			\formattaCampiTab{Average} & Un numero di transazioni compreso fra 4 e 7 & 10 \\ 
			\addlinespace[1em] 
			\formattaCampiTab{Complex} & Un numero di transazioni pari a 8 o maggiore & 15 \\ 
		\bottomrule
	\end{tabularx}
\end{center}
La stima del costo dei casi d'uso (\gls{uucw}) è data dalla seguente formula:
\begin{displaymath}
UUCW = (NoSimpleUC \times 5) + (NoAvgUC \times 10) + (NoComplexUC \times 15)
\end{displaymath}
Di seguito riportiamo la tabella con indicati pesi di ogni attore: 
\begin{center}
	\begin{longtable}{ p{0.125\textwidth} p{0.7\textwidth} p{0.075\textwidth} } 
		\toprule
			\formattaTitoloTab{ID} & \formattaTitoloTab{Caso d'uso} & \formattaTitoloTab{Peso} \\
		\cmidrule(l{\cmidrulekern}r{\cmidrulekern}){1-3}
		\endfirsthead

		\toprule
			\formattaTitoloTab{ID} & \formattaTitoloTab{Attore} & \formattaTitoloTab{Peso} \\
		\cmidrule(l{\cmidrulekern}r{\cmidrulekern}){1-3}
		\endhead

		\midrule
		\multicolumn{3}{r}{\footnotesize\itshape continua nella prossima pagina} \\
		\endfoot

		\bottomrule
		\endlastfoot

			\rowIDTitle{cu:login} & 5 \\ 
			\addlinespace[1em] 
			\rowIDTitle{cu:loginAmm} & 5 \\ 
			\addlinespace[1em] 
			\rowIDTitle{cu:logout} & 5 \\ 
			\addlinespace[1em] 
			\rowIDTitle{cu:iscrizionePortale} & 5 \\ 
			\addlinespace[1em] 
			\rowIDTitle{cu:iscrizioneSocial} & 5 \\ 
			\addlinespace[1em] 
			\rowIDTitle{cu:iscrizioneApprovazione} & 5 \\ 
			\addlinespace[1em] 
			\rowIDTitle{cu:approvazioneIscrizione} & 5 \\ 
			\addlinespace[1em] 
			\rowIDTitle{cu:personalizzaVetrinaInsDesc} & 10 \\ 
			\addlinespace[1em] 
			\rowIDTitle{cu:personalizzaVetrinaModDesc} & 10 \\ 
			\addlinespace[1em] 
			\rowIDTitle{cu:personalizzaVetrinaInsImg} & 10 \\ 			
			\addlinespace[1em] 
			\rowIDTitle{cu:personalizzaVetrinaDelImg} & 10 \\ 
			\addlinespace[1em] 
			\rowIDTitle{cu:personalizzaVetrinaInsProd} & 10 \\ 
			\addlinespace[1em] 
			\rowIDTitle{cu:personalizzaVetrinaModProd} & 10 \\ 
			\addlinespace[1em] 
			\rowIDTitle{cu:statistichePrivateVetrina} & 10 \\ 
			\addlinespace[1em] 
			\rowIDTitle{cu:inserimentoNotizia} & 5 \\ 
			\addlinespace[1em] 
			\rowIDTitle{cu:modificaNotizia} & 10 \\ 
			\addlinespace[1em] 
			\rowIDTitle{cu:rimozioneNotizia} & 10 \\ 
			\addlinespace[1em] 
			\rowIDTitle{cu:suggerimentoProdotti} & 5 \\ 
			\addlinespace[1em] 
			\rowIDTitle{cu:notizieSimili} & 5 \\ 
			\addlinespace[1em] 
			\rowIDTitle{cu:accessoProfilo} & 5 \\ 
			\addlinespace[1em] 
			\rowIDTitle{cu:modificaImpostazioni} & 5 \\ 
			\addlinespace[1em] 
			\rowIDTitle{cu:rimozioneAccountProprio} & 5 \\ 
			\addlinespace[1em] 
			\rowIDTitle{cu:rimozioneAccountAltrui} & 10 \\ 
			\addlinespace[1em] 
			\rowIDTitle{cu:modificaPrivilegiAccount} & 10 \\ 
			\addlinespace[1em] 
			\rowIDTitle{cu:inserisciValutazioneProdotto} & 10 \\ 
			\addlinespace[1em] 
			\rowIDTitle{cu:modificaValutazioneProdotto} & 10 \\ 
			\addlinespace[1em] 
			\rowIDTitle{cu:inserisciRecensioneProdotto} & 15 \\ 
			\addlinespace[1em] 
			\rowIDTitle{cu:modificaRecensioneProdotto} & 10 \\ 			
			\addlinespace[1em] 
			\rowIDTitle{cu:eliminaRecensioneProdotto} & 10 \\ 
			\addlinespace[1em] 
			\rowIDTitle{cu:commentoRecensione} & 15 \\ 
			\addlinespace[1em] 
			\rowIDTitle{cu:giudizioRecensione} & 10 \\ 
			\addlinespace[1em] 
			\rowIDTitle{cu:modificaGiudizioRecensione} & 10 \\ 
			\addlinespace[1em] 
			\rowIDTitle{cu:segnalazioneContenutiInap} & 10 \\ 
			\addlinespace[1em] 
			\rowIDTitle{cu:mostraSegnContenutiInap} & 5 \\ 
			\addlinespace[1em] 
			\rowIDTitle{cu:rimozioneContenutiInap} & 10 \\ 
			\addlinespace[1em] 
			\rowIDTitle{cu:followAccount} & 10 \\ 
			\addlinespace[1em] 
			\rowIDTitle{cu:unFollowAccount} & 10 \\ 
			\addlinespace[1em] 
			\rowIDTitle{cu:ticketInvio} & 5 \\ 
			\addlinespace[1em] 
			\rowIDTitle{cu:ticketRisposta} & 10 \\ 
			\addlinespace[1em] 
			\rowIDTitle{cu:ticketChiudi} & 10 \\ 
			\addlinespace[1em] 
			\rowIDTitle{cu:ticketLettura} & 5 \\ 
			\addlinespace[1em] 
			\rowIDTitle{cu:ricercaProdotto} & 5 \\ 
			\addlinespace[1em] 
			\rowIDTitle{cu:ricercaProfilo} & 5 \\ 
			\addlinespace[1em] 
			\rowIDTitle{cu:ricercaNotizia} & 5 \\ 
			\addlinespace[1em] 
			\rowIDTitle{cu:richiestaInsProdotto} & 10 \\ 
			\addlinespace[1em] 
			\rowIDTitle{cu:mostraRichiestaInsProdotto} & 10 \\ 			
			\addlinespace[1em] 
			\rowIDTitle{cu:richiestaInsProduttore} & 10 \\ 
			\addlinespace[1em] 
			\rowIDTitle{cu:mostraVetrina} & 5 \\ 
			\addlinespace[1em] 
			\rowIDTitle{cu:mostraProdotto} & 10 \\ 
			\addlinespace[1em] 
			\rowIDTitle{cu:mostraRecProdotto} & 10 \\ 
			\addlinespace[1em] 
			\rowIDTitle{cu:mostraProfilo} & 5 \\ 
			\addlinespace[1em] 
			\rowIDTitle{cu:mostraNotizia} & 10 \\ 
			\addlinespace[1em] 
			\rowIDTitle{cu:mostraAggF} & 5 \\ 
		\cmidrule(l{\cmidrulekern}r{\cmidrulekern}){1-3}
			\formattaCampiTab{\gls{uucw}}  & & 460\\
	\end{longtable}
\end{center}

\section{Fattore di aggiustamento tecnico}
Il \gls{tcf} è uno dei fattori che viene applicato alla stima della dimensione del progetto in modo tale da tener conto delle considerazioni tecniche.
Questo fattore è determinato assegnando un valore compreso fra 0 (irrilevante) e 5 (essenziale) ai 13 aspetti tecnici, ad ognuno dei quali è associato un peso. 
Il \gls{tcf} è dato dalla seguente formula:
\begin{displaymath}
	TCF = 0.6 + (TF \div 100)
\end{displaymath}
Dove il \gls{tf} viene così calcolato:
\begin{displaymath}
	TF = \sum_{i=1}^{13}FactorWeight_i \times FactorScore_i 
\end{displaymath}
Nella seguente tabella riportiamo gli aspetti tecnici che vengono presi in considerazione nel calcolo del \gls{tcf}, il peso di ognuno di essi e il valore che gli viene assegnato:
\begin{center}
	\begin{tabularx}{\widthTab}{l X l l} 
		\toprule
			\formattaTitoloTab{Fattore} & \formattaTitoloTab{Descrizione} & \formattaTitoloTab{Peso} & \formattaTitoloTab{Valore} \\
		\cmidrule(l{\cmidrulekern}r{\cmidrulekern}){1-4}
			\formattaCampiTab{T1} & Distributed system & 2.0 & 0.0\\ 
			\addlinespace[1em] 
			\formattaCampiTab{T2} & Response time/performance objectives & 1.0 & 5.0\\ 
			\addlinespace[1em] 
			\formattaCampiTab{T3} & End-user efficiency & 1.0 & 5.0\\ 
			\addlinespace[1em] 
			\formattaCampiTab{T4} & Internal processing complexity & 1.0 & 1.0\\ 
			\addlinespace[1em] 
			\formattaCampiTab{T5} & Code reusability & 1.0 & 3.0\\ 
			\addlinespace[1em] 
			\formattaCampiTab{T6} & Easy to install & 0.5 & 1.0\\ 
			\addlinespace[1em] 
			\formattaCampiTab{T7} & Easy to use & 0.5 & 5.0\\ 
			\addlinespace[1em] 
			\formattaCampiTab{T8} & Portability to other platforms & 2.0 & 5.0\\ 
			\addlinespace[1em] 
			\formattaCampiTab{T9} & System maintenance & 1.0 & 3.0 \\ 
			\addlinespace[1em] 
			\formattaCampiTab{T10} & Concurrent/parallel processing & 1.0 & 1.0 \\ 
			\addlinespace[1em] 
			\formattaCampiTab{T11} & Security features & 1.0 & 4.0 \\ 
			\addlinespace[1em] 
			\formattaCampiTab{T12} & Access for third parties & 1.0 & 0.0 \\ 
			\addlinespace[1em] 
			\formattaCampiTab{T13} & End user training & 1.0 & 0.0 \\ 
		\cmidrule(l{\cmidrulekern}r{\cmidrulekern}){1-4}
			\formattaCampiTab{TF} & & & 35.0 \\ 
			\formattaCampiTab{TCF} & & & 0.95 \\ 
		\bottomrule
	\end{tabularx}
\end{center}

\section{Fattore di aggiustamento ambientale}
Il \gls{ecf} è l'altro fattore che viene applicato alla stima della dimensione del progetto in modo tale da tener conto delle considerazioni ambientali.
Questo fattore è determinato assegnando un valore compreso fra 0 (nessuna esperienza) e 5 (esperto) agli 8 aspetti ambientali, ad ognuno dei quali è associato un peso. 
Il \gls{ecf} è dato dalla seguente formula:
\begin{displaymath}
	ECF = 1.4 + (-0.03 \times EF)
\end{displaymath}
Dove il \gls{ef} viene così calcolato:
\begin{displaymath}
	EF = \sum_{i=1}^{8}FactorWeight_i \times FactorScore_i 
\end{displaymath}
Nella seguente tabella riportiamo gli aspetti ambientali che vengono presi in considerazione nel calcolo del \gls{ecf}, il peso di ognuno di essi e il valore che gli viene assegnato:
\begin{center}
	\begin{tabularx}{\widthTab}{l X l l} 
		\toprule
			\formattaTitoloTab{Fattore} & \formattaTitoloTab{Descrizione} & \formattaTitoloTab{Peso} & \formattaTitoloTab{Valore} \\
		\cmidrule(l{\cmidrulekern}r{\cmidrulekern}){1-4}
			\formattaCampiTab{E1} & Familiarity with development process used & 1.5 & 0.5\\ 
			\addlinespace[1em] 
			\formattaCampiTab{E2} & Application experience & 0.5 & 0.0\\ 
			\addlinespace[1em] 
			\formattaCampiTab{E3} & Object-oriented experience of team & 1.0 & 5.0\\ 
			\addlinespace[1em] 
			\formattaCampiTab{E4} & Lead analyst capability & 0.5 & 2.5\\ 
			\addlinespace[1em] 
			\formattaCampiTab{E5} & Motivation of the team & 1.0 & 5.0\\ 
			\addlinespace[1em] 
			\formattaCampiTab{E6} & Stability of requirements & 2.0 & 3.0\\ 
			\addlinespace[1em] 
			\formattaCampiTab{E7} & Part-time staff & -1.0 & 0.0\\ 
			\addlinespace[1em] 
			\formattaCampiTab{E8} & Difficult programming language & -1.0 & 2.0\\ 
		\cmidrule(l{\cmidrulekern}r{\cmidrulekern}){1-4}
			\formattaCampiTab{EF} & & & 16.0 \\ 
			\formattaCampiTab{ECF} & & & 0.92 \\ 
		\bottomrule
	\end{tabularx}
\end{center}

\section{Calcolo UCP}
Il valore finale degli \acrfull{ucp} è dato dalla seguente formula:
\begin{displaymath}
	UCP = (UAW + UUCW) \times TCF \times ECF = (29 + 460) \times 0.95 \times 0.92 = 427.386 
\end{displaymath}

\section{Stima dello sforzo}
Una volta nota la stima della dimensione del progetto è possibile stimare lo sforzo necessario per comopletarlo.
Si assume, non avendo esperienze passate, che per ogni \gls{ucp} siano necessarie 30 ore/uomo.
Lo sforzo totale per il completamento del progetto, \gls{ee}, è dato dalla seguente formula:
\begin{displaymath}
	EE = UCP \times (Hours Per UCP) = 427.386 \times 30 = 12821.58\, ore/uomo
\end{displaymath}
In un giorno si hanno generalmente 8 ore lavorative, quindi, dividendo la quantità precedente per 8 si ottiene lo sforzo in giorni/uomo:
\begin{displaymath}
	EE = 12821.58 \div 8 = 1602.70\, giorni/uomo
\end{displaymath}
In un mese si hanno generalmente 22 giorni lavorativi, quindi, dividendo la quantità precedente per 22 si ottiene lo sforzo in mesi/uomo:
\begin{displaymath}
	EE = 1602.6975 \div 22 = 72.85\, mesi/uomo
\end{displaymath}
Quindi lo sforzo in anni/uomo è il seguente:
\begin{displaymath}
	EE = 72.85 \div 12 = 6\, anni/uomo
\end{displaymath}
Con un team di tre persone si stima il completamento del progetto entro 2 anni.

\section{Stima del costo}
Supponendo uno stipendio medio di \(4000\, euro\) si stima il seguente costo del progetto:
\begin{displaymath}
	COSTO = 72.85 \times 4000 = 291400\, euro
\end{displaymath}

\newdate{costiuno}{03}{10}{2016}
\section{Revisioni}
\begin{center}
    \begin{tabular}{lll}
        \toprule
        	\tabhead{Versione} & \tabhead{Data} & \tabhead{Descrizione} \\
		\cmidrule(l{\cmidrulekern}r{\cmidrulekern}){1-3}
        	1.0 & \displaydate{costiuno} & Prima versione \\
        \bottomrule
    \end{tabular}
\end{center}