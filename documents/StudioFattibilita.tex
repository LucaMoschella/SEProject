\chapter{Studio della fattibilità} % (fold)
\label{cha:studio_della_fattibilita}

\section{Introduzione} 
\label{sec:realizzabilita_sistema}
ChocoClub intende affrontare la realizzazione di un portale per adeguarsi alle nuove esigenze del mercato in 
termini di comunicazione fra utenti, integrazione con altri sistemi e con i socialmedia.

\section{Obiettivo} 
\label{sec:obiettivo}
Obiettivo di questo progetto è la definizione e realizzazione di una piattaforma tecnologica Web Based
che realizzi una vetrina e catalogo multimediale per la consultazione di informazioni provenienti dal mondo dolciario.

\section{Requisiti} 
\label{sec:requisiti}
I principali requisiti del sistema che si vuole realizzare sono i seguenti:
\begin{itemize}
	\item Verrà utilizzato un \gls{cms} commerciale, con eventuali estensioni e plugin.

	\item Ogni produttore potrà gestire la sua vetrina, inserendo e modificando i suoi prodotti.

	\item Le figure pubbliche potranno effettuare vari tipi di ricerche all'interno del portale. 

	\item Le figure pubbliche avranno accesso ad un sistema soacial di follower/followed.

	\item Gli utenti potranno recensire prodotti.

	\item Le figure pubbliche autenticate potranno commentare altre recensioni.

%	\item Verranno, opzionalmente, messe a disposizione dei servizi in tecnologia Rest API per lo scambio di informazioni dal portale verso eventuali partner esterni.

%	\item Il sistema sarà realizzato con responsive design e dunque fruibile da dispositivi mobili quali smartphones e tablet.

	\item Si potranno gestire dal portale le informazioni inviate ai social come Facebook, Twitter, Linkedin.
\end{itemize}



\section{Architettura del sistema} 
\label{sec:architettura_sistema}

\section{Competenze} 
\label{sec:competenze}

\section{Tecnologie}
\label{sec:tecnologie}

\section{Attività} 
\label{sec:attivita}

\section{Stima Tempi} 
\label{sec:tempi}

\section{Stima Costi} 
\label{sec:costi}

\section{Vantaggi} 
\label{sec:vantaggi}



\section{Realizzabilità del sistema} % (fold)
\label{sec:realizzabilita_sistema}
Le tecnologie richieste per lo sviluppo del portale sono attualmente presenti nel mercato, e utilizzate da sistemi simili, possiamo quindi convenire che esso sia effettivamente realizzabile.

Il portale che si intende realizzare offre funzionalità abbastanza comuni e già messe a dispozione da sofware commerciali; intendiamo quindi utilizzare un \gls{cms} commerciale che inglobi le funzionalità richieste. 

Ciò permetterà di abbattere i costi di sviluppo e implementazione in quanto non sarà necessario re--implementare tutte le funzionalità necessarie, poiché già disponibili nel \gls{cms}.

Inoltre la maggior parte dei \gls{cms} commerciali mette a disposizione una grandissima quantità di estensioni e plugin, che permette di usufruire di un notevole numero di funzionalità; le quali potranno essere utilizzate in futuro per ampliare il portale con nuovi servizi, ad esempio esistono plugin per l'intregazione con i principali social network o per la gestione del sistema di recensioni.

Per la fase iniziale di inserimento informazioni è stato deciso di non utilizzare i contenuti attualmente presenti sul sito dell’associazione, in quanto il sito è statico e di difficile consultazione; oltretutto, la maggior parte dei contenuti non è aggiornata e non riguarda principalmente i prodotti, ma le pasticcerie come attività commerciali, quindi non è utile per gli obiettivi del nuovo portale.

%\section{Quantità di risorse necessarie}
%\label{sec:risorse_necessarie}%

%Lo studio di fattibilit`a serve a
%
%☛ stabilire se un dato prodotto pu`o essere realizzato
%
%☛ se `e conveniente realizzarlo
%
%☛ a valutare, per la realizzazione
%
%➠ la quantit`a di risorse necessarie
%
%➠ i tempi necessari
%
%➠ i costi relativi