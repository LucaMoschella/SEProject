\chapter{Studio della fattibilità} % (fold)
\label{cha:studio_della_fattibilita}

\section{Introduzione} 
\label{sec:realizzabilita_sistema}
ChocoClub intende affrontare la realizzazione di un portale per adeguarsi alle nuove esigenze del mercato in 
termini di comunicazione fra utenti, integrazione con altri sistemi e con i socialmedia.

\section{Obiettivo} 
\label{sec:obiettivo}
Obiettivo di questo progetto è la definizione e realizzazione di una piattaforma tecnologica Web Based
che realizzi una vetrina e catalogo multimediale per la consultazione di informazioni provenienti dal mondo dolciario e che fornisca alle figure pubbliche autenticate metodi di interazione.

\section{Requisiti} 
\label{sec:requisiti}
I principali requisiti del sistema che si vuole realizzare sono i seguenti:
\begin{itemize}
	\item Verrà utilizzato un \gls{cms} commerciale, con eventuali estensioni e plugin.

	\item Ogni produttore potrà gestire la sua vetrina, inserendo e modificando i suoi prodotti.

	\item Le figure pubbliche potranno effettuare vari tipi di ricerche all'interno del portale. 

	\item Le figure pubbliche avranno accesso ad un sistema social di follower/followed.

	\item Gli utenti potranno recensire prodotti.

	\item Gli utenti potranno suggerire prodotti e/o produttori mancanti.

	\item Le figure pubbliche autenticate potranno commentare altre recensioni.

	\item Si potranno gestire dal portale le informazioni inviate ai social come Facebook, Twitter, Linkedin.
\end{itemize}



\section{Architettura del sistema} 
\label{sec:architettura_sistema}
Il portale che si intende realizzare offre funzionalità abbastanza comuni e già messe a dispozione da molti sofware, intendiamo quindi utilizzare un \gls{cms} commerciale che inglobi le funzionalità richieste.

Il \gls{cms} verrà installato su un server di proprietà di ChocoClub.
I fruitori del servizio potranno connettersi al server, il quale darà loro accesso alle funzionalità presenti nel portale, messe a disposizione dal \gls{cms}.
L'architettura del sistema è quindi riassumibile dalla seguente figura:
%METTERE FIGURA

\section{Competenze} 
\label{sec:competenze}
Le competenze necessarie per portare a termine con successo il progetto sono:
\begin{itemize}
	\item Progettazione di applicativi Web.
	\item Gestione infrastrutture informatiche.
	\item Gestione di un \gls{cms} e di eventuali estensioni e plugin per esso.
\end{itemize}

\section{Tecnologie}
\label{sec:tecnologie}
Le tecnologie richieste per lo sviluppo del portale sono attualmente presenti nel mercato, e utilizzate da sistemi simili, possiamo quindi convenire che esso sia effettivamente realizzabile.
Sarà necessario l'utilizzo di un \tecnologia{\gls{cms}} e la conoscenza di \tecnologia{CSS} per eventuali personalizzazioni dell'interfaccia.
Per quanto riguarda l'infrastruttura, visto l'approccio centralizzato, basterà un \tecnologia{server} per rendere disponibile l'accesso al servizio.

\section{Vantaggi} 
\label{sec:vantaggi}
L'uso di un \gls{cms} permetterà di abbattere i costi di sviluppo e implementazione in quanto non sarà necessario re--implementare tutte le funzionalità necessarie, poiché già disponibili nel \gls{cms}.

Inoltre la maggior parte dei \gls{cms} commerciali mette a disposizione una grandissima quantità di estensioni e plugin, che permette di usufruire di un notevole numero di funzionalità; le quali potranno essere utilizzate in futuro per ampliare il portale con nuovi servizi, ad esempio esistono plugin per l'intregazione con i principali social network o per la gestione del sistema di recensioni.

Grazie alla soluzione proposta, Chococlub potrà disporre di un portale più fruibile rispetto a quello attuale, con contenuti sempre aggiornati in quanto delegato ai produttori associati ed iscritti al portale. Disporrà inoltre di un portale con una completa integrazione con i sistemi Social. 

Chococlub inoltre, dopo un periodo di assestamento del portale, potrà valutare: 
\begin{itemize}
	\item La ricezione di compenso per la pubblicità garantita ai produttori iscritti al proprio portale.
	\item L'introduzione di un produttore delegato che, sotto compenso, potrà gestire le piccole realtà locali che non fanno utilizzo delle nuove tecnologie.
\end{itemize}

Il sistema è vantaggioso quindi sia per i produttori, che potranno pubblicizzare i propri prodotti, sia per gli utenti, che potranno interagire direttamente con i produttori, scoprire nuovi prodotti e comunicare con altri utenti e sia per Chococlub che, oltre al beneficio di un portale unificante, potrà valutare diversi modi per introdurre nuove fonti di guadagno.


\newdate{fattuno}{1}{05}{2016}
\section{Revisioni}
\begin{center}
    \begin{tabular}{lll}
        \toprule
        \tabhead{Versione} & \tabhead{Data} & \tabhead{Descrizione} \\
        %   \cmidrule{2-3} 
        \midrule
        1.0 & \displaydate{fattuno} & Prima versione \\
        \bottomrule
    \end{tabular}
\end{center}



% In documento: Project plan
%\section{Attività} 
%\label{sec:attivita}

% In documento: Project plan
%\section{Stima Tempi} 
%\label{sec:tempi}

% In documento: Project plan
%\section{Stima Costi} 
%\label{sec:costi}


%? Probabilmente non serve più, non abbiamo parlato di "rinnovare portali" ne di informazioni guà presenti:
%Per la fase iniziale di inserimento informazioni è stato deciso di non utilizzare i contenuti attualmente presenti sul sito dell’associazione, in quanto il sito è statico e di difficile consultazione; oltretutto, la maggior parte dei contenuti non è aggiornata e non riguarda principalmente i prodotti, ma le pasticcerie come attività commerciali, quindi non è utile per gli obiettivi del nuovo portale.
