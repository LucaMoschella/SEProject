\chapter{Specifica dei casi d'uso}

\section{Introduzione}
In questo documento sono analizzati ed elencati gli attori ed i casi d'uso del sistema.
Scopo del documento è la dettagliata descrizione delle modalità di interazione tra le diverse entità presenti ed il sistema.
Ciò permetterà di mettere in evidenza eventuali conflitti tra i requisiti del sistema stesso.
Tale analisi e descrizione dei casi d'uso verrà presentata tramite grafici che seguono lo standard UML.

\noindent
Si utilizzeranno le seguenti convenzioni:
\begin{itemize}
	\item Nei casi in cui più attori possono accedere ad un determinato caso d’uso, ma quest'ultimo è rivolto principalmente ad un solo attore, non verranno elencati tutti gli altri.

	\item Non verranno riportati tutti gli attori che hanno accesso ad un determinato caso d'uso, ma solo l'attore più generico.
\end{itemize}

\section{Elenco degli attori}
Gli attori che interagiscono con il portale sono elencati di seguito:
\begin{itemize}
	\item \newListItem{att:visitatore}{\formattaAtt}{Visitatore}
	\item \newListItem{att:utente}{\formattaAtt}{Utente}
	\item \newListItem{att:produttore}{\formattaAtt}{Produttore}
	\item \newListItem{att:redattore}{\formattaAtt}{Redattore}
	\item \newListItem{att:assistente}{\formattaAtt}{Assistente}
	\item \newListItem{att:moderatore}{\formattaAtt}{Moderatore}
	\item \newListItem{att:amministratore}{\formattaAtt}{Amministratore}
	\item \newListItem{att:cms}{\formattaAtt}{CMS}
\end{itemize}
La gerarchia degli attori è mostrata nel seguente diagramma UML:
\begin{center}
   \includegraphics[width=\textwidth]{assets/visualParadigm/SchemaAttori}
\end{center}

\section{Specifica degli attori}
Gli attori che interagiscono con il portale sono descritti di seguito:
\attTab{att:visitatore}{-}{\glsdesc*{visitatore}}

\tabattvspace

\attTab{att:utente}{-}{\glsdesc*{utente}}

\tabattvspace

\attTab{att:produttore}{-}{\glsdesc*{produttore}}

\tabattvspace

\attTab{att:redattore}{\getTitletodesc{att:utente}}{\glsdesc*{redattore}}

\tabattvspace

\attTab{att:assistente}{\getTitletodesc{att:utente}}{\glsdesc*{assistente}}

\tabattvspace

\attTab{att:moderatore}{\getTitletodesc{att:utente}}{\glsdesc*{moderatore}}

\tabattvspace

\attTab{att:amministratore}{\getTitletodesc{att:utente}}{\glsdesc*{amministratore}}

\tabattvspace

\attTab{att:cms}{-}{\glsdesc*{cmsdef}}

\section{Identificativo dei casi d'uso} %(vengono usati per definire l'ID, non ha senso metterli dopo la struttura ID)
Di seguito è spiegato come interpretare l'identificativo dei casi d'uso:
\begin{center}
	\use{categoria}{caso}%{[sottoreq]}{[\dots]}
\end{center}

\noindent
Segue una descrizione di ogni campo utilizzato nell'identificatore:
\begin{itemize}
	\item La \campiIdReq{categoria} è una lettera utilizzata per raggruppare i casi d'uso con scopi simili.
	Le categorie sono le seguenti:
	\begin{enumerateIndLabel}{\textsf{\Alph*}}{\Alph*}
		\item Autenticazione. \label{pkg:autenticazione}
		\item Gestione iscrizioni. \label{pkg:iscrizione}
		\item Gestione vetrina. \label{pkg:vetrina} 
		\item Gestione notizie. \label{pkg:notizie}
		\item Gestione suggerimenti. \label{pkg:suggerimenti}
		\item Gestione account. \label{pkg:account}
		\item Gestione valutazioni. \label{pkg:gestionevalutazione}
		\item Gestione recensioni. \label{pkg:gestionerecensione}
		\item Interazione tra figure. \label{pkg:interazione}
		\item Gestione ticket. \label{pkg:gestioneticket}
		\item Ricerca contenuti. \label{pkg:ricerca}
		\item Gestione prodotti mancanti. \label{pkg:prodottimancanti}
		\item Visualizzazione contenuti pubblici. \label{pkg:visualizzazione}
	\end{enumerateIndLabel}

	\item Il campo \campiIdReq{caso} è un numero progressivo utilizzato per identificare in modo univoco un caso d'uso all'interno della sua categoria.
\end{itemize}

\section{Elenco dei casi d'uso}
Vengono di seguito elencati i casi d'uso:
\begin{itemize} 
	\setCatCU{\ref*{pkg:autenticazione}}%A
	\item \newListItem{cu:login}{\formattaCU}{Login}
	\item \newListItem{cu:logout}{\formattaCU}{Logout}

	\setCatCU{\ref*{pkg:iscrizione}}%B
	\item \newListItem{cu:iscrizionePortale}{\formattaCU}{Iscrizione tramite modulo}
	\item \newListItem{cu:iscrizioneSocial}{\formattaCU}{Iscrizione tramite Social Network}
	\item \newListItem{cu:iscrizioneApprovazione}{\formattaCU}{Iscrizione tramite approvazione}
	\item \newListItem{cu:approvazioneIscrizione}{\formattaCU}{Approvazione iscrizione}

	\setCatCU{\ref*{pkg:vetrina}}%C
	\item \newListItem{cu:personalizzaVetrinaInsDesc}{\formattaCU}{Inserimento descrizione}
	\item \newListItem{cu:personalizzaVetrinaInsImg}{\formattaCU}{Inserimento immagine}
	\item \newListItem{cu:personalizzaVetrinaInsProd}{\formattaCU}{Inserimento prodotto}
	\item \newListItem{cu:personalizzaVetrinaModProd}{\formattaCU}{Modifica prodotto}
	\item \newListItem{cu:statistichePrivateVetrina}{\formattaCU}{Visualizzazione statistiche private della vetrina}


	\setCatCU{\ref*{pkg:notizie}}%D
	\item \newListItem{cu:inserimentoNotizia}{\formattaCU}{Inserimento notizia}
	\item \newListItem{cu:modificaNotizia}{\formattaCU}{Modifica notizia}
	\item \newListItem{cu:rimozioneNotizia}{\formattaCU}{Rimozione notizia}

	\setCatCU{\ref*{pkg:suggerimenti}}%E
	\item \newListItem{cu:suggerimentoProdotti}{\formattaCU}{Suggerimento prodotti}
	\item \newListItem{cu:notizieSimili}{\formattaCU}{Suggerimento notizie simili}
	
	\setCatCU{\ref*{pkg:account}}%F
	\item \newListItem{cu:accessoProfilo}{\formattaCU}{Accesso al proprio profilo}
	\item \newListItem{cu:accessoImpostazioni}{\formattaCU}{Accesso alle proprie impostazioni}
	\item \newListItem{cu:rimozioneAccountProprio}{\formattaCU}{Rimozione account proprio}

	\item \newListItem{cu:rimozioneAccountAltrui}{\formattaCU}{Rimozione account altrui}
	\item \newListItem{cu:aggiungiAccountAFigura}{\formattaCU}{Aggiungi funzionalità di una figura ad account}
	\item \newListItem{cu:rimuoviAccountDaFigura}{\formattaCU}{Rimuovi funzionalità di una figura da account}

	\setCatCU{\ref*{pkg:gestionevalutazione}}%G
	\item \newListItem{cu:inserisciValutazioneProdotto}{\formattaCU}{Inserimento valutazione}
	\item \newListItem{cu:modificaValutazioneProdotto}{\formattaCU}{Modifica valutazione}

	\setCatCU{\ref*{pkg:gestionerecensione}}%H
	\item \newListItem{cu:inserisciRecensioneProdotto}{\formattaCU}{Inserimento recensione}
	\item \newListItem{cu:modificaRecensioneProdotto}{\formattaCU}{Modifica recensione propria}
	\item \newListItem{cu:eliminaRecensioneProdotto}{\formattaCU}{Rimozione recensione propria}
	\item \newListItem{cu:commentoRecensione}{\formattaCU}{Commento a recensione}
	\item \newListItem{cu:giudizioRecensione}{\formattaCU}{Giudizio recensione}
	\item \newListItem{cu:segnalazioneContenutiInap}{\formattaCU}{Segnalazione contenuti inappropriati}
	\item \newListItem{cu:rimozioneContenutiInap}{\formattaCU}{Rimozione contenuti inappropriati}

	\setCatCU{\ref*{pkg:interazione}}%I
	\item \newListItem{cu:followAccount}{\formattaCU}{Follow account}
	\item \newListItem{cu:unFollowAccount}{\formattaCU}{Unfollow account}

	\setCatCU{\ref*{pkg:gestioneticket}}%J
	\item \newListItem{cu:ticketInvio}{\formattaCU}{Invio di un ticket}
	\item \newListItem{cu:ticketRisposta}{\formattaCU}{Risposta ad un ticket}
	\item \newListItem{cu:ticketChiudi}{\formattaCU}{Chiusura di un ticket}
	\item \newListItem{cu:ticketLettura}{\formattaCU}{Lettura di un ticket}

	\setCatCU{\ref*{pkg:ricerca}}%K
	\item \newListItem{cu:ricercaProdotto}{\formattaCU}{Ricerca prodotto}
	\item \newListItem{cu:ricercaProfilo}{\formattaCU}{Ricerca profilo}
	\item \newListItem{cu:ricercaNotizia}{\formattaCU}{Ricerca notizia}


	\setCatCU{\ref*{pkg:prodottimancanti}}%L
	\item \newListItem{cu:richiestaInsProdotto}{\formattaCU}{Richiesta inserimento prodotto a un produttore iscritto}
	\item \newListItem{cu:richiestaInsProduttore}{\formattaCU}{Richiesta inserimento prodotto e del suo produttore}

	\setCatCU{\ref*{pkg:visualizzazione}}%M
	\item \newListItem{cu:visualizzazioneVetrina}{\formattaCU}{Visualizzazione vetrina}
	\item \newListItem{cu:visualizzazioneInfoVetrina}{\formattaCU}{Visualizzazione informazioni vetrina}
%%	\item \newListItem{cu:visualizzazioneProdottiVetrina}{\formattaCU}{Visualizzazione prodotti esposti in vetrina}
	\item \newListItem{cu:visualizzazioneProdotto}{\formattaCU}{Visualizzazione prodotto}
	\item \newListItem{cu:visualizzazioneInfoProdotto}{\formattaCU}{Visualizzazione informazioni del prodotto}
	\item \newListItem{cu:visualizzazioneRecProdotto}{\formattaCU}{Visualizzazione recensioni associate al prodotto}
%	\item \newListItem{cu:visualizzazioneStatProdotto}{\formattaCU}{Visualizzazione statistiche del prodotto}
	\item \newListItem{cu:visualizzazioneProfilo}{\formattaCU}{Visualizzazione profilo pubblico}
	\item \newListItem{cu:visualizzazioneInfoProfilo}{\formattaCU}{Visualizzazione informazioni di un profilo pubblico}
%	\item \newListItem{cu:visualizzazioneStatProfilo}{\formattaCU}{Visualizzazione statistiche di un profilo pubblico}
	\item \newListItem{cu:visualizzazioneNotizie}{\formattaCU}{Visualizzazione notizie}
	\item \newListItem{cu:visualizzazioneAggF}{\formattaCU}{Visualizzazione aggiornamenti dai followed}
\end{itemize}	

%{id}{attori}{pre}{post}{flusso}
\section{Specifica dei casi d'uso}
Vengono di seguito elencate le tabelle che descrivono i casi d'uso, suddivisi per categoria:

\subsection{Autenticazione}
\begin{center}
   \includegraphics[width=\textwidth]{assets/visualParadigm/Autenticazione}
\end{center}
\cuTab{cu:login}{\getTitletodesc{att:visitatore}}{La persona connessa al sito è inizialmente riconosciuta come Visitatore. Il sistema mette a disposizione una funzionalità per effettuare il Login}{La persona connessa al sito è ora autenticata, come Utente o Produttore}{}

\tabcuvspace

<<<<<<< HEAD
\cuTab{cu:logout}{\getTitletodesc{att:utente}, \getTitletodesc{att:produttore}}{La persona connessa al sito è autenticata}{La persona connessa al sito non è più autenticata, ed è ora un Visitatore}{}
=======
\cuTab{cu:logout}{\getTitletodesc{att:utente}, \getTitletodesc{att:produttore}, \getTitletodesc{att:redattore}, \getTitletodesc{att:assistente}, \getTitletodesc{att:moderatore}, \getTitletodesc{att:amministratore}}{La persona connessa al sito è autenticata}{La persona connessa al sito non è autenticata, ed è ora un Visitatore}{}
>>>>>>> origin/master


\subsection{Gestione iscrizioni}
\begin{center}
   \includegraphics[width=\textwidth]{assets/visualParadigm/GestioneIscrizione}
\end{center}
\cuTab{cu:iscrizionePortale}{\getTitletodesc{att:visitatore}}{La persona è connessa al sito come Visitatore}{La persona ha ora un account proprio, creato con i dati da essa inseriti}{}

\tabcuvspace

%Accorpa verificati e non verificati
\cuTab{cu:iscrizioneSocial}{\getTitletodesc{att:visitatore}}{La persona è connessa al sito come Visitatore}{La persona ha ora un account proprio, creato tramite le API messe a disposizione dal Social Network utilizzato durante l'iscrizione}{}

\tabcuvspace

\cuTab{cu:iscrizioneApprovazione}{\getTitletodesc{att:visitatore}}{La persona è connessa al sito come Visitatore}{La persona ha inviato correttamente una richiesta per la creazione di un account di tipo produttore ad essa associato}{}

\tabcuvspace

\cuTab{cu:approvazioneIscrizione}{\getTitletodesc{att:assistente}}{La persona è autenticata come utente e ha i privilegi di assistente. E' stata inviata una richiesta di creazione di un account di tipo produttore da parte di un Visitatore}{Il Visitatore che ha effettuato la richiesta ha ora un account di tipo produttore}{}


\subsection{Gestione vetrina}
\begin{center}
   \includegraphics[width=\textwidth]{assets/visualParadigm/GestioneVentrina}
\end{center}
\cuTab{cu:personalizzaVetrinaInsDesc}{\getTitletodesc{att:produttore}}{La persona è autenticata al sito come Produttore}{Il Produttore inserisce correttamente una descrizione nella vetrina}{}

\tabcuvspace

\cuTab{cu:personalizzaVetrinaInsImg}{\getTitletodesc{att:produttore}}{La persona è autenticata al sito come Produttore}{Il Produttore inserisce correttamente un'immagine nella vetrina}{}

\tabcuvspace

\cuTab{cu:personalizzaVetrinaInsProd}{\getTitletodesc{att:produttore}}{La persona è autenticata al sito come Produttore}{Il Produttore inserisce correttamente un prodotto nella vetrina}{}

\tabcuvspace

\cuTab{cu:personalizzaVetrinaModProd}{\getTitletodesc{att:produttore}}{La persona è autenticata al sito come Produttore. Il prodotto è presente nella sua vetrina}{Il Produttore modifica correttamente un prodotto nella sua vetrina}{} %%SETTARE UN PRODOTTO COME FUORI PRODUZIONE&&

\tabcuvspace

\cuTab{cu:statistichePrivateVetrina}{\getTitletodesc{att:produttore}}{La persona è autenticata al sito come Produttore}{Il Produttore accede alle statistiche private della propria vetrina}{}


\subsection{Gestione notizie}
\begin{center}
   \includegraphics[width=\textwidth]{assets/visualParadigm/GestioneNotizie}
\end{center}
\cuTab{cu:inserimentoNotizia}{\getTitletodesc{att:redattore}}{La persona è autenticata come utente e ha i privilegi di Redattore}{Il Redattore inserisce correttamente la notizia nel sistema}{}

\tabcuvspace

\cuTab{cu:modificaNotizia}{\getTitletodesc{att:redattore}}{La persona è autenticata come utente e ha i privilegi di Redattore. La notizia è presente nel sistema}{Il Redattore modifica correttamente la notizia}{}

\tabcuvspace

\cuTab{cu:rimozioneNotizia}{\getTitletodesc{att:redattore}}{La persona è autenticata come utente e ha i privilegi di Redattore. La notizia è presente nel sistema}{Il Redattore rimuove correttamente la notizia dal sistema}{}



\subsection{Gestione suggerimenti}
\begin{center}
   \includegraphics[width=\textwidth]{assets/visualParadigm/GestioneSuggerimenti}
\end{center}

%Accorpa suggerimento intelligente e simili
\cuTab{cu:suggerimentoProdotti}{\getTitletodesc{att:utente}, \getTitletodesc{att:produttore}}{}{}{}

\tabcuvspace

\cuTab{cu:notizieSimili}{\getTitletodesc{att:utente}, \getTitletodesc{att:produttore}}{}{}{}


\subsection{Gestione account}
\begin{center}
   \includegraphics[width=\textwidth]{assets/visualParadigm/GestioneAccount}
\end{center}
\cuTab{cu:accessoProfilo}{\getTitletodesc{att:utente}, \getTitletodesc{att:produttore}}{}{}{}

\tabcuvspace

\cuTab{cu:accessoImpostazioni}{\getTitletodesc{att:utente}, \getTitletodesc{att:produttore}}{}{}{}

\tabcuvspace

\cuTab{cu:rimozioneAccountProprio}{\getTitletodesc{att:utente}, \getTitletodesc{att:produttore}}{}{}{}

\tabcuvspace

\cuTab{cu:rimozioneAccountAltrui}{\getTitletodesc{att:amministratore}}{}{}{}

\tabcuvspace

\cuTab{cu:aggiungiAccountAFigura}{\getTitletodesc{att:amministratore}}{}{}{}

\tabcuvspace

\cuTab{cu:rimuoviAccountDaFigura}{\getTitletodesc{att:amministratore}}{}{}{}


\subsection{Gestione valutazioni}
\begin{center}
   \includegraphics[width=\textwidth]{assets/visualParadigm/GestioneValutazioni}
\end{center}
\cuTab{cu:inserisciValutazioneProdotto}{\getTitletodesc{att:utente}}{}{}{}

\tabcuvspace

\cuTab{cu:modificaValutazioneProdotto}{\getTitletodesc{att:utente}}{}{}{}


\subsection{Gestione recensioni}
\begin{center}
   \includegraphics[width=\textwidth]{assets/visualParadigm/GestioneRecensioni}
\end{center}
\cuTab{cu:inserisciRecensioneProdotto}{\getTitletodesc{att:utente}}{}{}{}

\tabcuvspace

\cuTab{cu:modificaRecensioneProdotto}{\getTitletodesc{att:utente}}{}{}{}

\tabcuvspace

\cuTab{cu:eliminaRecensioneProdotto}{\getTitletodesc{att:utente}}{}{}{}

\tabcuvspace

\cuTab{cu:commentoRecensione}{\getTitletodesc{att:utente}, \getTitletodesc{att:produttore}}{}{}{}

\tabcuvspace

\cuTab{cu:giudizioRecensione}{\getTitletodesc{att:utente}}{}{}{}

\tabcuvspace

\cuTab{cu:segnalazioneContenutiInap}{\getTitletodesc{att:utente}, \getTitletodesc{att:produttore}}{}{}{}

\tabcuvspace

\cuTab{cu:rimozioneContenutiInap}{\getTitletodesc{att:moderatore}}{}{}{}


\subsection{Interazione tra figure}
\begin{center}
   \includegraphics[width=\textwidth]{assets/visualParadigm/InterazioneTraFigure}
\end{center}
\cuTab{cu:followAccount}{\getTitletodesc{att:utente}, \getTitletodesc{att:produttore}}{}{}{}

\tabcuvspace

\cuTab{cu:unFollowAccount}{\getTitletodesc{att:utente}, \getTitletodesc{att:produttore}}{}{}{}


\subsection{Gestione ticket}
\begin{center}
   \includegraphics[width=\textwidth]{assets/visualParadigm/GestioneTicket}
\end{center}
\cuTab{cu:ticketInvio}{\getTitletodesc{att:utente}, \getTitletodesc{att:produttore}}{}{}{}

\tabcuvspace

\cuTab{cu:ticketRisposta}{\getTitletodesc{att:utente}, \getTitletodesc{att:produttore}, \getTitletodesc{att:assistente}}{}{}{}

\tabcuvspace

\cuTab{cu:ticketChiudi}{\getTitletodesc{att:assistente}}{}{}{}

\tabcuvspace

\cuTab{cu:ticketLettura}{\getTitletodesc{att:utente}, \getTitletodesc{att:produttore}, \getTitletodesc{att:assistente}}{}{}{}


\subsection{Ricerca contenuti}
\begin{center}
   \includegraphics[width=\textwidth]{assets/visualParadigm/RicercaContenuti}
\end{center}
\cuTab{cu:ricercaProdotto}{\getTitletodesc{att:visitatore}, \getTitletodesc{att:utente}, \getTitletodesc{att:produttore}}{}{}{}

\tabcuvspace

\cuTab{cu:ricercaProfilo}{\getTitletodesc{att:visitatore}, \getTitletodesc{att:utente}, \getTitletodesc{att:produttore}}{}{}{}

\tabcuvspace

\cuTab{cu:ricercaNotizia}{\getTitletodesc{att:visitatore}, \getTitletodesc{att:utente}, \getTitletodesc{att:produttore}}{}{}{}


\subsection{Gestione prodotti mancanti}
\begin{center}
   \includegraphics[width=\textwidth]{assets/visualParadigm/GestioneProdottiMancanti}
\end{center}
\cuTab{cu:richiestaInsProdotto}{\getTitletodesc{att:utente}}{}{}{}

\tabcuvspace

\cuTab{cu:richiestaInsProduttore}{\getTitletodesc{att:utente}}{}{}{}


\subsection{Visualizzazione contenuti pubblici}
\begin{center}
   \includegraphics[width=\textwidth]{assets/visualParadigm/Visualizzazione}
\end{center}
%accorpa visalizza prodotti in vetrina e visualizza vetrina
\cuTab{cu:visualizzazioneVetrina}{\getTitletodesc{att:visitatore}, \getTitletodesc{att:utente}, \getTitletodesc{att:produttore}}{}{}{}

\tabcuvspace

%Cambiato da: statistiche pubbliche vetrina
\cuTab{cu:visualizzazioneInfoVetrina}{\getTitletodesc{att:visitatore}, \getTitletodesc{att:utente}, \getTitletodesc{att:produttore}}{}{}{}

%\tabcuvspace
%
%\cuTab{cu:visualizzazioneProdottiVetrina}{\getTitletodesc{att:visitatore}, \getTitletodesc{att:utente}, \getTitletodesc{att:produttore}}{}{}{}

\tabcuvspace

\cuTab{cu:visualizzazioneProdotto}{\getTitletodesc{att:visitatore}, \getTitletodesc{att:utente}, \getTitletodesc{att:produttore}}{}{}{}

\tabcuvspace

%Accorpa statistiche prodotto e informazioni prodotto
\cuTab{cu:visualizzazioneInfoProdotto}{\getTitletodesc{att:visitatore}, \getTitletodesc{att:utente}, \getTitletodesc{att:produttore}}{}{}{}

\tabcuvspace

\cuTab{cu:visualizzazioneRecProdotto}{\getTitletodesc{att:visitatore}, \getTitletodesc{att:utente}, \getTitletodesc{att:produttore}}{}{}{}

\tabcuvspace

%\cuTab{cu:visualizzazioneStatProdotto}{\getTitletodesc{att:visitatore}, \getTitletodesc{att:utente}, \getTitletodesc{att:produttore}}{}{}{}
%
%\tabcuvspace

\cuTab{cu:visualizzazioneProfilo}{\getTitletodesc{att:visitatore}, \getTitletodesc{att:utente}, \getTitletodesc{att:produttore}}{}{}{}

\tabcuvspace

%Accorpa informazioni profilo e statistiche profilo
\cuTab{cu:visualizzazioneInfoProfilo}{\getTitletodesc{att:visitatore}, \getTitletodesc{att:utente}, \getTitletodesc{att:produttore}}{}{}{}

%\tabcuvspace
%
%\cuTab{cu:visualizzazioneStatProfilo}{\getTitletodesc{att:visitatore}, \getTitletodesc{att:utente}, \getTitletodesc{att:produttore}}{}{}{}

\tabcuvspace

\cuTab{cu:visualizzazioneNotizie}{\getTitletodesc{att:visitatore}, \getTitletodesc{att:utente}, \getTitletodesc{att:produttore}}{}{}{}

\tabcuvspace

\cuTab{cu:visualizzazioneAggF}{\getTitletodesc{att:utente}, \getTitletodesc{att:produttore}}{}{}{}




























%Struttura dei casi d'uso (ID) T_T
%Esempi:
%Lumiere -> Ogni caso `e identificato da una stringa del tipo CU.entit`a.sottoentit`a.caso.
%Airbnb.it -> package = PKG_<numero package>_<numero_sottopackage> | caso d'uso = UC_<numero package>_<numero caso d’uso> | attore = ATT_<num. Attore generico>_<num. Attore specifico>

%A me la struttura a package piace, possiamo racchiudere molti casi d'uso in categorie così.

%Scopiazzerei il grafico di Airbnb.it a pagina 9, utilizzando quella figura per descrivere il sistema a grandi linee, con i package. Mostrata questa, procederei con l'analisi
%dei package uno ad uno e quindi dei singoli casi d'uso.
%Per ogni caso d'uso, entrambi gli esempi fanno la descrizione stile Basi 2 con precondizioni e postcondizioni (oltre che attori coinvolti, nome, descrizione e flusso).
%A me piace


