 \begin{comment}
Introduzione
In questo documento sono analizzati ed elencati i casi d'uso del sistema.
Scopo del documento è la dettagliata descrizione delle modalità di interazione tra le diverse entità possibili ed il sistema.
Ciò permetterà di mettere in evidenza eventuali conflitti tra i requisiti del sistema stesso.
Tale analisi e descrizione dei casi d'uso verrà presentata tramite grafici che seguono lo standard UML, con relativo specifico commento per ognuno.

Elenco attori (vengono usati per definire l'ID, non ha senso metterli dopo la struttura ID)
[L'elenco attori mi pare tanto una ripetizione di roba già scritta più volte xD]
Amministratore
CMS ?
Produttore
Utente
Visitatore


Struttura dei casi d'uso (ID) T_T
Esempi:
Lumiere -> Ogni caso `e identificato da una stringa del tipo CU.entit`a.sottoentit`a.caso.
Airbnb.it -> package = PKG_<numero package>_<numero_sottopackage> | caso d'uso = UC_<numero package>_<numero caso d’uso> | attore = ATT_<num. Attore generico>_<num. Attore specifico>

A me la struttura a package piace, possiamo racchiudere molti casi d'uso in categorie così.

Scopiazzerei il grafico di Airbnb.it a pagina 9, utilizzando quella figura per descrivere il sistema a grandi linee, con i package. Mostrata questa, procederei con l'analisi
dei package uno ad uno e quindi dei singoli casi d'uso.
Per ogni caso d'uso, entrambi gli esempi fanno la descrizione stile Basi 2 con precondizioni e postcondizioni (oltre che attori coinvolti, nome, descrizione e flusso).
A me piace


 \end{comment}