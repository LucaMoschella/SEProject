\chapter{Valutazione dei rischi} 

\section{Introduzione}
In questo documento verrà effettuata l'analisi dei rischi, per gestire eventuali imprevisti e limitarne i danni, i quali comporterebbero possibili aumenti del costo, ritardi rispetto alla \docref{cha:pianificazione_del_progetto} o, nei casi più gravi, l'annullamento dell'intero progetto.

Di seguito sono mostrate delle tabelle con alcuni dei possibili rischi, in cui verrà indicata la probabilità del rischio e la tipologia dell'effetto negativo che il rischio comporta. I rischi più probabili e quelli con i maggiori effetti negativi dovranno essere oggetto di maggiori attenzioni durante lo sviluppo del progetto.

\section{Classificazione probabilità}
Le probabilità del verificarsi di un rischio vengono, per semplicità, così classificate:
\begin{center}
	\begin{tabularx}{\widthTab}{ l  X } 
		\toprule
			\formattaTitoloTab{Classificazione} & \formattaTitoloTab{Probabilità} \\
		\cmidrule(l{\cmidrulekern}r{\cmidrulekern}){1-2}
			Molta alta & maggiore del 75\% \\ 
			\addlinespace[1em] 
			Alta & compresa fra il 50\% e il 75\%  \\ 
			\addlinespace[1em] 
			Media & compresa fra il 25\% e il 50\%  \\ 
			\addlinespace[1em] 
			Bassa & compresa fra il 10\% e il 25\%  \\ 
			\addlinespace[1em] 
			Molto bassa & minore del 10\% \\ 
		\bottomrule
	\end{tabularx}
\end{center}

\section{Classificazione effetti}
I possibili eventi che un rischio può provocare vengono così suddivisi:
\begin{center}
	\begin{tabularx}{\widthTab}{ l  X } 
		\toprule
			\formattaTitoloTab{Classificazione} & \formattaTitoloTab{Effetti} \\
		\cmidrule(l{\cmidrulekern}r{\cmidrulekern}){1-2}
			Catastrofici & Il verificarsi del rischio probabilmente comporta il fallimento del progetto \\ 
			\addlinespace[1em] 
			Seri & Il verificarsi del rischio, se non correttamente gestito, potrebbe comportare il fallimento del progetto o ad ingenti aumenti di costo e del tempo richiesto per il completamento dello stesso \\ 
			\addlinespace[1em] 
			Tollerabili & Il verificarsi del rischio comporta un aumento dei costi e un ritardo nel completamento del progetto  \\ 
			\addlinespace[1em] 
			Insignificanti & Il verificarsi del rischio non comporta aumenti di costo o ritardi \\
		\bottomrule
	\end{tabularx}
\end{center}

\section{Elenco dei rischi}
Vengono di seguito elencati i possibili rischi che possono verificarsi:
\begin{itemize}
	\item \newListItem{rk:ritardo}{\formattaRK}{Ritardi}
	\item \newListItem{rk:reqerrati}{\formattaRK}{Requisiti errati}
	\item \newListItem{rk:reqincompleti}{\formattaRK}{Requisiti incompleti o cambiati}
	\item \newListItem{rk:ucerrati}{\formattaRK}{Use Case errati}
	\item \newListItem{rk:ucincompleti}{\formattaRK}{Use Case incompleti o cambiati}
	\item \newListItem{rk:docerrata}{\formattaRK}{Documentazione incompleta o errata}
	\item \newListItem{rk:complessita}{\formattaRK}{Complessità sottovalutata}
\end{itemize}

\section{Specifica tabelle RMMM}
Le tabelle del \gls{rmmm} servono per gestire i rischi, la loro specifica indica tre diverse fasi per ogni rischio:
\begin{descriptionInd}
	\item[Mitigation] Come cercare di impedire il verificarsi del rischio adottando alcune tecniche o strategie
	\item[Monitoring] Come sorvegliare il lavoro progettuale per accorgersi del rischio in tempo utile
	\item[Management] Come gestire il rischio nel caso in cui esso si verifichi
\end{descriptionInd}
Di seguito elenchiamo le tabelle \gls{rmmm}:

\rkTab{rk:ritardo}{Alta}{Seri}{Presenza di ritardi causati da una cattiva gestione del tempo o da imprevisti che rallentano lo sviluppo del progetto}{Anticipare le scadenze in base al calendario stabilito nella pianificazione temporale del progetto, in modo da avere del tempo poter gestire eventuali ritardi. Informare i clienti di eventuali ritardi}{Controllare periodicamente che i tempi stabiliti nella pianificazione temporale del progetto siano rispettati}{Qualora non fosse possibile rispettare i tempi stabiliti per un'iterazione, cercare di posticipare, se possibile. alcune attività nelle iterazioni successive e pianificare nuovamente queste iterazioni in modo tale da rispettare le scadenze.  Solo se  strettamente necessario contrattare con il committente una nuova data  di rilascio}
\rkTab{rk:reqerrati}{Bassa}{Catastrofici}{I requisiti del progetto risultano errati. Questo può essere dovuto ad un'esposizione superficiale del problema da parte del committente o ad un'interpretazione sbagliata delle richieste da parte degli analisti}{Evitare ogni tipo di ambiguità nei requisiti e far controllare periodicamente al committente il documento relativo alla specifica dei requisiti, in modo tale da correggere preventivamente eventuali errori presenti nei requisiti}{Far controllare al committente ogni revisione o cambiamento effettuato ai requisiti}{Correggere i requisiti errati senza introdurre nuovi errori}
\rkTab{rk:reqincompleti}{Moderata}{Seri}{I requisiti specificati non ricoprono la proposta di progetto in ogni sua funzionalità oppure vengono modificati durante lo sviluppo}{Discutere con il committente le richieste meno chiare per cercare di rendere espliciti i requisiti necessari, evitando così di dover aggiungere o modificare requisiti nei successivi incontri}{Pianificare continue revisioni con il committente per rilevare tempestivamente eventuali difetti nei requisiti}{Effettuare le modifiche ai requisiti cercando di non introdurre nuovi problemi}
\rkTab{rk:ucerrati}{Molto bassa}{Catastrofici}{I casi d'uso del progetto risultano errati. Quuesto può essere dovuto ad un'interpretazione sbagliata dei requisiti}{Cercare di essere precisi nella descrizione dei casi d'uso e far controllare il documento dei casi d'uso al committente, in modo tale da correggere eventuali errori}{Far controllare al committente ogni revisione o cambiamento effettuato ai casi d'uso}{Correggere i casi d'uso errati senza introdurre nuovi errori}
\rkTab{rk:ucincompleti}{Bassa}{Seri}{I casi d'uso specificati non ricoprono i requisiti proposti oppure vengono modificati durante lo sviluppo}{Discutere con il committente i requisiti in modo da rappresentare al meglio i casi d'uso associati ad essi, evitando di dover aggiungere o modificare casi d'uso nei successivi incontri}{Pianificare continue revisioni con il committente per rilevare eventuali difetti nei casi d'uso}{Effettuare le modifiche necessarie ai casi d'uso, cercando di non introdurre nuovi problemi}
\rkTab{rk:docerrata}{Moderata}{Tollerabili}{La documentazione del progetto potrebbe essere poco dettagliata o contenere errori}{Utilizzare il modello di documentazione specifico per il processo di sviluppo software che si sta utilizzando o basarsi sulla struttura dei documenti di progetti simili}{Revisionare periodicamente la documentazione per individuare tempestivamente eventuali errori o mancanze}{Riscrivere la documentazione mancante/errata}
\rkTab{rk:complessita}{Bassa}{Catastrofici}{Una sottovalutazione durante la raccolta dei requisiti potrebbe portare a sottostimare la complessità del sistema}{Porre particolare attenzione durante l'attività di analisi in modo che essa non sia troppo superficiale e che rispetti la complessità del sistema}{Revisionare periodicamente il progetto per individuare eventuali complessità e poterle gestire in tempo}{In caso di imprevisti, concentrare l'attenzione dell'intero team nella loro gestione, in particolare durante l'attività di analisi}
