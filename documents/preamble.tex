%IMPORT PACKAGE - l'ordine ha importanza
\usepackage[italian]{babel}						%imposta la lingua
\usepackage[T1]{fontenc}  						%compatibilità di output delle cose
\usepackage[utf8]{inputenc}						%compatibilità di input delle cose
\usepackage{lmodern, microtype} 				%rende più bello il font
\usepackage{titlesec} 							%roba per personalizzare gli header (capitolo, sezioni, etc.)
\usepackage{titling}							%roba per personalizzare il titolo e la pagina inziale
\usepackage[italian]{varioref} 					%ref intelligente
\usepackage[inline]{enumitem} 					%gestione liste avanzata
\usepackage{parskip} 							%spazio vericale invece di indentazione nei paragrafi
\usepackage{graphicx, xcolor} 					%gestione delle immagini e dei colori
\usepackage{amsmath, amsthm, amssymb} 			%roba per matematica
\usepackage{imakeidx}							%gestione indice analitico
\usepackage{hyperref} 							%gestione dei link
\usepackage[xindy,toc]{glossaries} 				%glossario



%INIZIALIZZAZIONE - setup package e impostazioni varie

%toglie lo spazio allungato dopo i punti di fine frase.
\frenchspacing 

%toglie la parola capitolo ad ogni capitolo
\titleformat{\chapter}{\normalfont\bfseries\huge}{\thechapter.}{15pt}{\normalfont\bfseries\huge} %toglie la scritta "Capitolo N"

%crea l'indice analitico, e lo metta nella table-of-content
\makeindex[intoc]

%crea il glossario ed imposta lo stile
\makeglossaries
\glossarystyle{altlist}

%usando le macro possiamo mantenere la coerenza del documento e modificare TUTTO contemporaneamente
%MACRO COMANDI

\newcommand{\wemph}[1]{\emph{#1}} %strong emphatize 
\newcommand{\semph}[1]{\textbf{#1}} %weak emphatize

\newcommand{\iindex}[1]{#1\index{#1}} %Indiicizzare e inserire contemporaneamente

% forse da togleire
\newcommand{\funref}[1]{funzionalità descritta nel punto \texttt{\ref{#1}}}
\newcommand{\funrefs}[2]{funzionalità descritte nei punti \texttt{\ref{#1}} e \texttt{\ref{#2}} }
\newcommand{\bloccofunref}[1]{blocco funzionalità \texttt{\ref{#1}.}}

%MACRO AMBIENTI

%lista descrption dove la descrizione va a capo.
\newenvironment{descriptionNext}{\begin{description}[style=nextline]}{\end{description}}
%Def: \newenvironment{name}[num]{before}{after}