%PACKAGE ESSENZIALI - anche se non usati così mi ricordo della loro esistenza
\usepackage[italian]{babel} %imposta la lingua
\usepackage[T1]{fontenc}  %compatibilità di output delle cose
\usepackage[utf8]{inputenc}  %compatibilità di input delle cose

\usepackage{lmodern} %rende più bello il font
\usepackage{microtype}   %rende più bello il font

\usepackage{index} %roba per personalizzare l'index
\usepackage{titlesec} %roba per personalizzare gli header (capitolo, sezioni, etc.)
\usepackage{titling} %roba per personalizzare il titolo e la pagina inziale

\usepackage[italian]{varioref} %ref intelligente

\usepackage[inline]{enumitem} %gestione liste avanzata
\usepackage{parskip} %spazio vericale invece di indentazione nei paragrafi

\usepackage{natbib} %roba per bibliografia
\usepackage{tocbibind} %roba per inserire la bibliografia nella toc (table of contents)

\usepackage{graphicx, xcolor} %gestione delle immagini e dei colori

\usepackage{amsmath, amsthm, amssymb} %roba per matematica

\usepackage{hyperref} %gestione dei link (da personalizzar eper togliere i quadrettoni)

\usepackage{glossaries}

%PERSONALIZZAZIONE
%Va tolto la scritta capitolo ogni volta
%Vanno sistemati i link (forse)
%Va sistemato il titolo principale

%\setlist[description]{itemindent=0.5cm}
%\titleformat{\chapter}[display]{\normalfont\bfseries}{}{0pt}{\Huge}

\titleformat{\chapter}{\normalfont\bfseries\huge}{\thechapter.}{15pt}{\normalfont\bfseries\huge} %toglie la scritta "Capitolo N"

\frenchspacing 

%DEFINIZIONE COMANDI PERSONALI - usando questi comandi invece di quelli di default possiamo mantenere la coerenza del documento e modificare TUTTO contemporaneamente
\newcommand{\wemph}[1]{\emph{#1}} %strong emphatize 
\newcommand{\semph}[1]{\textbf{#1}} %weak emphatize

% forse da togleire
\newcommand{\funref}[1]{funzionalità descritta nel punto \texttt{\ref{#1}}}
\newcommand{\funrefs}[2]{funzionalità descritte nei punti \texttt{\ref{#1}} e \texttt{\ref{#2}} }
\newcommand{\bloccofunref}[1]{blocco funzionalità \texttt{\ref{#1}.}}

%DEFINIZIONE AMBIENTI PERSONALI
%Def: \newenvironment{name}[num]{before}{after}
\newenvironment{descriptionNext}{\begin{description}[style=nextline]}{\end{description}}
