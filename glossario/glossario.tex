%Glossario

\newglossaryentry{contenuto}
{
    name={Contenuto},
    description={è qualsiasi oggetto inserito nel sistema dagli utilizzatori dello stesso}
}

\newglossaryentry{riferimento}
{
    name={Riferimento},
    description={è una porzione di testo, un immagine o un elemento composto che contiene un link ad un altro oggetto presente nel sistema}
}

\newglossaryentry{cmsdef}
{
    name={Content management system},
    description={è uno strumento software, installato su un server web, il cui
        compito è facilitare la gestione dei contenuti di siti web, svincolando il
        webmaster da conoscenze tecniche specifiche di programmazione Web}
}


\newglossaryentry{account}
{
    name={Account},
    description={individua un singolo fruitore del servizio, si assume che ogni utilizzatore abbia un singolo account. Un account può appartenere a diverse figure del portale, in funzione delle figure a cui appartiene può accedere alle rispettive funzionalità}
}

\newglossaryentry{visitatore}
{
    name={Visitatore},
    description={è una figura pubblica non autenticata nel portale}
}

\newglossaryentry{figuraPubblica}
{
    name={Figura pubblica},
    description={è il fruitore dei servizi del portale}
}

\newglossaryentry{figuraAmministrativa}
{
    name={Figure amministrativa},
    description={sono coloro che utilizzeranno il portale con lo scopo di fornire servizi alle figure pubbliche}
}

\newglossaryentry{figuraPubblicaAutenticata}
{
    name={Figura pubblica autenticata},
    description={è una figura pubblica autenticata nel portale}
}

\newglossaryentry{produttore}
{
    name={Produttore},
    description={è una figura pubblica autenticata nel portale che rappresenta colui che produce prodotti dolciari a base di cioccolata, sia a livello industriale sia a livello artigianale}
}

\newglossaryentry{utente}
{
    name={Utente},
    description={è una figura pubblica autenticata nel portale che rappresenta colui che consuma prodotti dolciari a base di cioccolata}
}

\newglossaryentry{redattore}
{
    name={Redattore},
    description={è una figura amministrativa addetta alla gestione delle notizie presenti nel portale}
}

\newglossaryentry{assistente}
{
    name={Assistente},
    description={è una figura amministrativa addetta alla comunicazione con le figure pubbliche}
}

\newglossaryentry{moderatore}
{
    name={Moderatore},
    description={è una figura amministrativa addetta al controllo dei contenuti inseriti dalle figure pubbliche, garantendo la conformità degli stessi al regolamento della piattaforma}
}

\newglossaryentry{amministratore}
{
    name={Amministratore},
    description={è una figura amministrativa addetta alla gestione del portale e delle sue figure}
}

\newglossaryentry{follower}
{
    name={Follower},
    description={di \emph{X}, sono le figure pubblice autenticate che ricevono aggiornamenti da \emph{X}}
}

\newglossaryentry{followed}
{
    name={Followed},
    description={ di \emph{X}, sono le figure pubblice autenticate dalle quali \emph{X} riceve aggiornamenti}
}

\newglossaryentry{vetrina}
{
    name={Vetrina},
    description={è lo spazio di ogni produttore dedicato all'esposizione dei suoi prodotti}
}

\newglossaryentry{valutazione}
{
    name={Valutazione},
    description={di un prodotto è un voto in una scala limitata per esprimere la bontà dello stesso}
}

\newglossaryentry{giudizio}
{
    name={Giudizio},
    description={di una recensione è un voto che può essere positivo o negativo}
}

\newglossaryentry{recensione}
{
    name={Recensione},
    description={di un prodotto è una valutazione a cui è associata una descrizione}
}

\newglossaryentry{prodotto}
{
    name={Prodotto},
    description={indica un dolce a base di cioccolata}
}

\newglossaryentry{notizia}
{
    name={Notizia},
    description={indica un articolo o una novità proveniente dal mondo dolciario}
}
%Acronimi

\newacronym{cms}{CMS}{Content Management System}

\newacronym{rup}{RUP}{Rational Unified Process}
