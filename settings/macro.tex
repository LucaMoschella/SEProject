%PERCHE' USARE LE MACRO
%1) Usando le macro possiamo stabili la formattazione "logica" del documento,
%   rendendo possibile cambiare le convenzioni adottate mantenendo la coerenza
%2) Definire comandi abbreviati per semplificare comandi con opzioni complese ripetute più volte
%3) Senza esagerare, definire testo da inserire in automatico senza riscriverlo ogni volta

%MACRO COMANDI 

%%%%%%%%%%%%%%%%%% FORMATTAZIONE LOGICA %%%%%%%%%%%%%%%%%%

%tecnologie necessarie (fattibilità)
\newcommand{\tecnologia}[1]{\emph{#1}} 

%autenticate o non autenticate (documenti di visione)
\newcommand{\stato}[1]{\emph{#1}} 

%ruolo nel sistema visitatori, etc. (visione, specifica sistema)
\newcommand{\ruolo}[1]{\emph{#1}} 

%gruppo di ruoli (visione, specifica sistema)
\newcommand{\grupporuolo}[1]{\textsl{#1}} 

%campo di informazione nel prodotto
\newcommand{\campo}[1]{\emph{#1}} 

%possibili valori dei campi
\newcommand{\dominiocampo}[1]{\emph{#1}} 

%intestazioni nelle tabelle generiche
\newcommand{\tabhead}[1]{\textsc{#1}} 

%funzionalità specifiche - deprecated
\newcommand{\fun}[1]{\emph{#1}}

%blocco di funzionalità - deprecated
\newcommand{\bloccofun}[1]{\emph{#1}}

%funzionalità da implementare
\newcommand{\funfutura}[1]{\emph{#1}}

%matricola 
\newcommand{\matricola}[1]{\textit{#1}}

%email
\newcommand{\email}[1]{\texttt{#1}}

%fase di sviluppo (inception, etc.)
\newcommand{\fase}[1]{\textsf{#1}} 

%workflow di sviluppo (raccolta requisiti, etc.)
\newcommand{\workflow}[1]{\emph{#1}} 


%formattazione tabelle iterazioni ( pianificazione progetto)
\newcommand{\tabtitleiter}[1]{\textbf{#1}} 
\newcommand{\tabheaditer}[1]{\textsc{#1}} 

%%%%%%%%%%%%%%%%%% INDICE ANALITICO %%%%%%%%%%%%%%%%%%
%indiicizzare e inserire contemporaneamente
\newcommand{\iindex}[1]{#1\index{#1}} 

%%%%%%%%%%%%%%%%%% RIFERIMENTI %%%%%%%%%%%%%%%%%%
%a funzioni
\newcommand{\funref}[1]{\texttt{\ref{#1}}}
%a documenti
\newcommand{\docref}[1]{\nameref{#1} \vpageref{#1}} 


%%%%%%%%%%%%%%%%%% MATEMATICA %%%%%%%%%%%%%%%%%%

%insiemi numerici
\newcommand{\numberset}{\mathbb}
\newcommand{\N}{\numberset{N}}
\newcommand{\R}{\numberset{R}}

%lettere strane
\renewcommand{\epsilon}{\varepsilon}
\renewcommand{\theta}{\vartheta}
\renewcommand{\rho}{\varrho}
\renewcommand{\phi}{\varphi}

%%%%%%%%%%%%%%%%%% TABELLE %%%%%%%%%%%%%%%%%%
%larghezza tabelle iterazione
\newcommand{\tabwidthiter}{\textwidth} 

%larghezza tabelle revisione - non usata (se con poco testo viene brutta)
\newcommand{\tabwidthrev}{0.8\textwidth} 



%%%%%%%%%%%%%%%%%% MACRO AMBIENTI %%%%%%%%%%%%%%%%%%

%lista descrption dove c'è una leggera indentazione.
\newenvironment{descriptionInd}{\begin{description}[leftmargin=1.5cm,labelindent=1cm]}{\end{description}}

%lista descrption dove la descrizione va a capo.
\newenvironment{descriptionNext}{\begin{description}[style=nextline]}{\end{description}}
        %signatura: \newenvironment{name}[num]{before}{after}

%lista lavoro da fare dentro iterazione
\newenvironment{enumWork}
{
	\begin{minipage}[t]{\linewidth}
		\begin{enumerate*}[itemjoin={\newline}]
		}
		{
		\end{enumerate*}
	\end{minipage}
}


\newenvironment{itemWork}
{
	\begin{minipage}[t]{\linewidth}
		\begin{itemize}[nosep, leftmargin=*]
		}
		{
		\end{itemize}
	\end{minipage}
}
