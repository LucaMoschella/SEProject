%PERCHE' USARE LE MACRO
%1) Usando le macro possiamo stabili la formattazione "logica" del documento,
%		rendendo possibile cambiare le convenzioni adottate mantenendo la coerenza
%2) Definire comandi abbreviati per semplificare comandi con opzioni complese ripetute più volte
%3) Senza esagerare, definire testo da inserire in automatico senza riscriverlo ogni volta

%MACRO COMANDI

\newcommand{\wemph}[1]{\emph{#1}} %strong emphatize 
\newcommand{\semph}[1]{\textbf{#1}} %weak emphatize

\newcommand{\iindex}[1]{#1\index{#1}} %Indiicizzare e inserire contemporaneamente

% forse da togleire - DA TOGLIRE SICURAMENTE, NON USARE ( meglio scrivere e usare ref/vref)
\newcommand{\funref}[1]{funzionalità descritta nel punto \texttt{\ref{#1}}}
\newcommand{\funrefs}[2]{funzionalità descritte nei punti \texttt{\ref{#1}} e \texttt{\ref{#2}} }
\newcommand{\bloccofunref}[1]{blocco funzionalità \texttt{\ref{#1}.}}


%Matematica

%insiemi numerici
\newcommand{\numberset}{\mathbb}
\newcommand{\N}{\numberset{N}}
\newcommand{\R}{\numberset{R}}

%lettere strane
\renewcommand{\epsilon}{\varepsilon}
\renewcommand{\theta}{\vartheta}
\renewcommand{\rho}{\varrho}
\renewcommand{\phi}{\varphi}

%MACRO AMBIENTI

%lista descrption dove la descrizione va a capo.
\newenvironment{descriptionNext}{\begin{description}[style=nextline]}{\end{description}}
%Def: \newenvironment{name}[num]{before}{after}