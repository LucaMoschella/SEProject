%PERCHE' USARE LE MACRO
%1) Usando le macro possiamo stabili la formattazione "logica" del documento,
%   rendendo possibile cambiare le convenzioni adottate mantenendo la coerenza
%2) Definire comandi abbreviati per semplificare comandi con opzioni complese ripetute più volte
%3) Senza esagerare, definire testo da inserire in automatico senza riscriverlo ogni volta

%MACRO COMANDI 

%formattazione logica
\newcommand{\tecnologia}[1]{\emph{#1}} 

\newcommand{\stato}[1]{\emph{#1}} 

\newcommand{\ruolo}[1]{\emph{#1}} 

\newcommand{\campo}[1]{\textsc{#1}} 

\newcommand{\tabhead}[1]{\textsc{#1}} 

\newcommand{\dominiocampo}[1]{\emph{#1}} 

\newcommand{\fun}[1]{\emph{#1}}

\newcommand{\bloccofun}[1]{\emph{#1}}

\newcommand{\funfutura}[1]{\emph{#1}}

\newcommand{\matricola}[1]{\textit{#1}}

\newcommand{\email}[1]{\texttt{#1}}

\newcommand{\fase}[1]{\textsf{#1}} 

\newcommand{\workflow}[1]{\emph{#1}} 


%Indiicizzare e inserire contemporaneamente
\newcommand{\iindex}[1]{#1\index{#1}} 

%Riferimenti vari
\newcommand{\funref}[1]{\texttt{\ref{#1}}}

\newcommand{\docref}[1]{\nameref{#1} \vpageref{#1}} 


%Matematica

%insiemi numerici
\newcommand{\numberset}{\mathbb}
\newcommand{\N}{\numberset{N}}
\newcommand{\R}{\numberset{R}}

%lettere strane
\renewcommand{\epsilon}{\varepsilon}
\renewcommand{\theta}{\vartheta}
\renewcommand{\rho}{\varrho}
\renewcommand{\phi}{\varphi}

%MACRO AMBIENTI


%lista descrption dove c'è una leggera indentazione.
\newenvironment{descriptionInd}{\begin{description}[leftmargin=1.5cm,labelindent=1cm]}{\end{description}}

%lista descrption dove la descrizione va a capo.
\newenvironment{descriptionNext}{\begin{description}[style=nextline]}{\end{description}}
        %signatura: \newenvironment{name}[num]{before}{after}