%PERCHE' USARE LE MACRO
%1) Usando le macro possiamo stabili la formattazione "logica" del documento,
%   rendendo possibile cambiare le convenzioni adottate mantenendo la coerenza
%2) Definire comandi abbreviati per semplificare comandi con opzioni complese ripetute più volte
%3) Senza esagerare, definire testo da inserire in automatico senza riscriverlo ogni volta

%%%%%%%%%%%%%%%%%%%%%%%%%%%%%%%%%%%%%%%%%%%%%%%%%%%%%%%%%%%%%%%%%%%%%%%%%%%%%%%%%%%%%%%%%%%%%%
%%%%%%%%%%%%%%%%%%%%%%%%%%%%%%%%%%%% FUNZIONI COMUNI %%%%%%%%%%%%%%%%%%%%%%%%%%%%%%%%%%%%%%%%%
%%%%%%%%%%%%%%%%%%%%%%%%%%%%%%%%%%%%%%%%%%%%%%%%%%%%%%%%%%%%%%%%%%%%%%%%%%%%%%%%%%%%%%%%%%%%%%


\newcommand{\iindex}[1]{#1\index{#1}} %indiicizzare e inserire contemporaneamente
\newcommand{\docref}[1]{\nameref{#1} \vpageref{#1}} %a documenti
\newcommand{\formattaTitoloTab}[1]{\texttt{#1}} 
\newcommand{\formattaCampiTab}[1]{\textsc{#1}} 
\newcommand{\widthTab}{\textwidth} %larghezza tabelle iterazione
\newcommand{\vspaceTab}{\bigskip} %spazio fra tabelle iterazioni 

%%%%% funzioni per la creazione automatica degli identificatori %%%%
%Macro per linkare requisiti
%separatore id titolo
\newcommand{\sepIDTitle}{ - }

%Modifica il contenuto della prossima \label
\makeatletter
\newcommand{\currentLabelText}[1]{%
    \def\@currentlabel{#1}%
}
\makeatother

%macro per creare un nuovo item in lista, necessità di una descrizione \label{desc:key}
%usage: \newListItem{key}{id}{titolo} 
\newcommand\newListItem[3]{\hyperref[desc:#1]{\phantomsection\currentLabelText{#2}#2\label{id:#1}}\sepIDTitle\currentLabelText{#3}#3\label{title:#1}}

\newcommand\linkedSubsection[1]{
	\subsection{\getIDTitle{#1}}\label{desc:#1}%
}

%macro per stampare (e linkare, tranne il titolo) informazioni del req.
%usage: \reqGet..{reqkey}
\newcommand\getID[1]{\ref{id:#1}}
\newcommand\getIDnolink[1]{\ref*{id:#1}}
\newcommand\getIDtodesc[1]{\hyperref[desc:#1]{\ref*{id:#1}}}
\newcommand\getTitle[1]{\ref*{title:#1}}
\newcommand\getTitletodesc[1]{\hyperref[desc:#1]{\ref*{title:#1}}}
\newcommand\getIDTitle[1]{\getID{#1}\sepIDTitle\getTitle{#1}}
\newcommand\getIDTitletodesc[1]{\getIDtodesc{#1}\sepIDTitle\getTitle{#1}}
\newcommand\getIDTitlenolink[1]{\getIDnolink{#1}\sepIDTitle\getTitle{#1}}

%Stampa il testo, imposta una label con la key passata, facendo il ref ti stampa il testo.
\makeatletter
\newcommand{\setTextToLabel}[2]{%
	#1
    \def\@currentlabel{#1}%
    \phantomsection\label{#2}
}
\makeatother


%%%%%%%%%%%%%%%%%%%%%%%%%%%%%%%%%%%%%%%%%%%%%%%%%%%%%%%%%%%%%%%%%%%%%%%%%%%%%%%%%%%%%%%%%%%%%%
%%%%%%%%%%%%%%%%%%%%%%%%%%%%%%%%%%%% FORMATTAZIONE LOGICA %%%%%%%%%%%%%%%%%%%%%%%%%%%%%%%%%%%%
%%%%%%%%%%%%%%%%%%%%%%%%%%%%%%%%%%%%%%%%%%%%%%%%%%%%%%%%%%%%%%%%%%%%%%%%%%%%%%%%%%%%%%%%%%%%%%

%Il documento indicato è quello in cui sono state definite per la prima volta le macro, poi riutilizzate anche in altri documenti. 
%(Un minimo di organizzazione senno impazzisco)

%%%%%%%%%%%%%%%%%%%%%%%%%%%%%%%%%%%% documento di richiesta %%%%%%%%%%%%%%%%%%%%%%%%%%%%%%%%%%%%

%%%%%%%%%%%%%%%%%%%%%%%%%%%%%%%%%%%% documento di visione %%%%%%%%%%%%%%%%%%%%%%%%%%%%%%%%%%%%

%ruolo nel sistema visitatori, etc. (visione, specifica sistema)
\newcommand{\ruolo}[1]{\emph{#1}} 

%autenticate o non autenticate (documenti di visione)
\newcommand{\stato}[1]{\emph{#1}} 

%intestazioni nelle tabelle generiche
\newcommand{\tabhead}[1]{\textsc{#1}} 



%%%%%%%%%%%%%%%%%%%%%%%%%%%%%%%%%%%% documento di fattibilità %%%%%%%%%%%%%%%%%%%%%%%%%%%%%%%%%%%%

%tecnologie necessarie
\newcommand{\tecnologia}[1]{\emph{#1}} 



%%%%%%%%%%%%%%%%%%%%%%%%%%%%%%%%%%%% documento di contratto %%%%%%%%%%%%%%%%%%%%%%%%%%%%%%%%%%%%

%matricola (e titolo)
\newcommand{\matricola}[1]{\textit{#1}}

%email (e titolo)
\newcommand{\email}[1]{\texttt{#1}}



%%%%%%%%%%%%%%%%%%%%%%%%%%%%%%%%%%%% documento di pianificazione %%%%%%%%%%%%%%%%%%%%%%%%%%%%%%%%%%%%

%fase di sviluppo (inception, etc.)
\newcommand{\fase}[1]{\textsf{#1}} 

%workflow di sviluppo (raccolta requisiti, etc.)
\newcommand{\workflow}[1]{\emph{#1}} 

%tabelle iterazioni
\newcommand{\tabiter}[5]{
\begin{center}
	\begin{minipage}[t]{\linewidth}
		\begin{tabularx}{\widthTab}{ l  X } 
			\toprule
				\multicolumn{2}{c}{\formattaTitoloTab{#1}}  \\
			\cmidrule(l{\cmidrulekern}r{\cmidrulekern}){1-2}
				\formattaCampiTab{Fase} & #2 \\ 
				\addlinespace[1em] 
				\formattaCampiTab{Obiettivi} & #3 \\
				\addlinespace[1em]
				\formattaCampiTab{Milestones} & #4 \\
				\addlinespace[1em]
				\formattaCampiTab{Stato} &  #5 \\
			\bottomrule
		\end{tabularx}
	\end{minipage}
\end{center}
} 

%campo vuoto
\newcommand{\emptycell}{-} 


%%%%%%%%%%%%%%%%%%%%%%%%%%%%%%%%%%%% documento di specifica sistema %%%%%%%%%%%%%%%%%%%%%%%%%%%%%%%%%%%%

%gruppi di ruoli (figure pubbliche, etc.)
\newcommand{\grupporuolo}[1]{\textsl{#1}} 

%campo di informazione nel prodotto
\newcommand{\infoProdotto}[1]{\emph{#1}} 



%%%%%%%%%%%%%%%%%%%%%%%%%%%%%%%%%%%% documento di specifica requisiti %%%%%%%%%%%%%%%%%%%%%%%%%%%%%%%%%%%%

%descrizione campi delle tabelle dei requisiti
\newcommand{\campiTabReq}[1]{\emph{#1}} 

%descrizione campi degli identificatori dei requisiti
\newcommand{\campiIdReq}[1]{\emph{#1}} 

%formattazione requisiti - non mette link
\newcommand{\reqsep}{.} 
\newcommand{\displayReq}[1]{\texttt{#1}}
\newcommand{\req}[4]{\displayReq{#1\reqsep#2\reqsep#3\reqsep#4}} 
\newcommand{\subreq}[5]{\displayReq{#1\reqsep#2\reqsep#3\reqsep#4\reqsep#5}} 
\newcommand{\subsubreq}[6]{\displayReq{#1\reqsep#2\reqsep#3\reqsep#4\reqsep#5\reqsep#6}} 

%Creazione dei contatori per la gestione automatica dei requisiti
\newcounter{req}
\newcounter{sottoreq}[req]
\newcounter{sottosottoreq}[sottoreq]

%Funzioni per gestire le categorie
\newcommand{\getCatR}{
}

\newcommand{\setCatR}[1]{%
	\renewcommand{\getCatR}{#1}
	\resetCounterR
}

%Funzione per resettare i contatori al cambio di categoria
\newcommand{\resetCounterR}{
\setcounter{req}{0}%
\setcounter{sottoreq}{0}%
\setcounter{sottosottoreq}{0}%
}

%macro per gestire i requisiti, e sotto requisiti vari.
%Oltre a formattare il requisito con \req (è in pratica un wrapper di req), 
%fa in modo che anche la label contenga la giusta formattazione del requisito.
\newcommand{\ereq}[2]{%
    \protect\stepcounter{req}%
    \req{#1}{#2}{\getCatR}{\arabic{req}}%
}

\newcommand{\esubreq}[2]{%
    \protect\stepcounter{sottoreq}%
	\subreq{#1}{#2}{\getCatR}{\arabic{req}}{\arabic{sottoreq}}%
}

\newcommand{\esubsubreq}[2]{%
    \protect\stepcounter{sottosottoreq}%
	\subsubreq{#1}{#2}{\getCatR}{\arabic{req}}{\arabic{sottoreq}}{\arabic{sottosottoreq}}%
}


%Tabella per descrizione requisiti funzionali 
%usage: {reqkey}{descrizione}{priorità}{stato}
\newcommand{\reqTabRF}[4]{\reqTab{#1}{#2}{Requisito funzionale}{#3}{#4}}

%Tabella per descrizione requisiti non funzionali
%usage: {reqkey}{descrizione}{priorità}{stato}
\newcommand{\reqTabRNF}[4]{\reqTab{#1}{#2}{Requisito non funzionale}{#3}{#4}}

%Tabella per descrizione requisiti
%usage: {reqkey}{descrizione}{tipo}{priorità}{stato}
\newcommand{\reqTab}[5]{%
	\begin{center}
		\begin{minipage}[t]{\linewidth}
			\phantomsection\label{desc:#1}
			\begin{tabularx}{\widthTab}{ l  X }
				\toprule
				\multicolumn{2}{c}{\formattaTitoloTab{\getTitle{#1}}}  \\
				\cmidrule(l{\cmidrulekern}r{\cmidrulekern}){1-2}
				\formattaCampiTab{ID} & \getID{#1} \\ 
				\addlinespace[0.5em] 
				\formattaCampiTab{Descrizione} &  #2 \\		
				\addlinespace[0.5em]
				\formattaCampiTab{Tipo} &  #3 \\
				\addlinespace[0.5em]
				\formattaCampiTab{Priorità} &  #4 \\
				\addlinespace[0.5em]
				\formattaCampiTab{Stato} &  #5 \\
				\bottomrule
			\end{tabularx}
		\end{minipage}
	\end{center}
}

\newcommand{\prioritaH}{Alta} 
\newcommand{\prioritaM}{Media} 
\newcommand{\prioritaL}{Media}

\newcommand{\statoOK}{Approvato} 
\newcommand{\statoNOK}{Non approvato} 
\newcommand{\statoUn}{Da approvare} 

\newcommand{\visitatore}{\ref*{con:visitatore}} 
\newcommand{\utente}{\ref*{con:utente}} 
\newcommand{\produttore}{\ref*{con:produttore}} 
\newcommand{\redattore}{\ref*{con:redattore}} 
\newcommand{\assistente}{\ref*{con:assistente}} 
\newcommand{\moderatore}{\ref*{con:moderatore}} 
\newcommand{\amministratore}{\ref*{con:amministratore}} 

\newcommand{\noF}{}%\textemdash} 

%%%%%%%%%%%%%%%%%%%%%%%%%%%%%%%%%%%% documento dei casi d'uso %%%%%%%%%%%%%%%%%%%%%%%%%%%%%%%%%%%%

%formattazione requisiti - non mette link
\newcommand{\usesep}{.} 
\newcommand{\displayUse}[1]{\texttt{#1}}
\newcommand{\use}[2]{\displayUse{CU\usesep#1\usesep#2}} 

 %%%%%%%%%%%%%%%%%%%%%%%%%%%%%%%%ATTORI
\newcounter{counterAtt}

\newcommand{\resetCounterAtt}{
\setcounter{counterAtt}{0}%
}

%formattazione requisiti - non mette link
\newcommand{\attSep}{.} 
\newcommand{\displayAtt}[1]{\texttt{#1}}
\newcommand{\att}[2]{\displayAtt{#1\attSep#2}}


\newcommand{\formattaAtt}{%
    \protect\stepcounter{counterAtt}%
	\att{ATT}{\arabic{counterAtt}}%
}

%Tabella per descrizione requisiti
%usage: {reqkey}{descrizione}{tipo}{priorità}{stato}
\newcommand{\attTab}[3]{%
	\begin{center}
		\begin{minipage}[t]{\linewidth}
			\phantomsection\label{desc:#1}
			\begin{tabularx}{\textwidth}{ l  X }
				\toprule
				\multicolumn{2}{c}{\formattaTitoloTab{\getTitle{#1}}}  \\
				\cmidrule(l{\cmidrulekern}r{\cmidrulekern}){1-2}
				\formattaCampiTab{ID} & \getID{#1} \\ 
				\addlinespace[0.5em] 
				\formattaCampiTab{Genitore} &  #2 \\		
				\addlinespace[0.5em]
				\formattaCampiTab{Descrizione} &  #3 \\
				\bottomrule
			\end{tabularx}
		\end{minipage}
	\end{center}
}


 %%%%%%%%%%%%%%%%%%%%%%%%%%%%%%%% casi d'uso


%Creazione dei contatori per la gestione automatica dei requisiti
\newcounter{counterCU}

%Funzioni per gestire le categorie
\newcommand{\getCatCU}{
}

\newcommand{\setCatCU}[1]{%
	\renewcommand{\getCatCU}{#1}
	\resetCounterCU
}

%Funzione per resettare i contatori al cambio di categoria
\newcommand{\resetCounterCU}{
\setcounter{counterCU}{0}%
}

%formattazione requisiti - non mette link
\newcommand{\cuSep}{.} 
\newcommand{\displayCU}[1]{\texttt{#1}}
\newcommand{\cu}[3]{\displayCU{#1\cuSep#2\cuSep#3}}

%macro per gestire i requisiti, e sotto requisiti vari.
%Oltre a formattare il requisito con \req (è in pratica un wrapper di req), 
%fa in modo che anche la label contenga la giusta formattazione del requisito.
\newcommand{\formattaCU}{%
    \protect\stepcounter{counterCU}%
	\cu{CU}{\getCatCU}{\arabic{counterCU}}%
}


%Tabella per descrizione requisiti
%usage: {reqkey}{descrizione}{tipo}{priorità}{stato}
\newcommand{\cuTab}[5]{%
	\begin{center}
		\begin{minipage}[t]{\linewidth}
			\phantomsection\label{desc:#1}
			\begin{tabularx}{\textwidth}{ l  X }
				\toprule
				\multicolumn{2}{c}{\formattaTitoloTab{\getTitle{#1}}}  \\
				\cmidrule(l{\cmidrulekern}r{\cmidrulekern}){1-2}
				\formattaCampiTab{ID} & \getID{#1} \\ 
				\addlinespace[0.5em] 
				\formattaCampiTab{Attori} &  #2 \\		
				\addlinespace[0.5em]
				\formattaCampiTab{Precondizioni} &  #3 \\
				\addlinespace[0.5em]
				\formattaCampiTab{Flusso principale} & \begin{NoHyper}#5\end{NoHyper} \\
				\addlinespace[0.5em]
				\formattaCampiTab{Postcondizioni} &  #4 \\
				\bottomrule
			\end{tabularx}
		\end{minipage}
	\end{center}
}

\newcommand{\tabtitleflussosec}[1]{\textit{#1}} 
\newcommand{\postNulle}{-} 

%key #flusso nomeFLusso pre post flusso
\newcommand{\cuAlternativo}[6]{%
	\begin{center}
		\begin{minipage}[t]{\linewidth}
			%\phantomsection
			\begin{tabularx}{\textwidth}{ l  X }
				\toprule
				\multicolumn{2}{c}{\formattaTitoloTab{\getTitle{#1}} - \tabtitleflussosec{#2} }  \\
				\cmidrule(l{\cmidrulekern}r{\cmidrulekern}){1-2}
				\formattaCampiTab{ID} & \getIDtodesc{#1} \\ 
				\addlinespace[0.5em] 
				\formattaCampiTab{Nome} & #3 \\ 
				\addlinespace[0.5em] 
				\formattaCampiTab{Precondizioni} &  #4 \\
				\addlinespace[0.5em]
				\formattaCampiTab{Flusso alternativo} &  \begin{NoHyper}#6\end{NoHyper} \\
				\addlinespace[0.5em]
				\formattaCampiTab{Postcondizioni} &  #5 \\
				\bottomrule
			\end{tabularx}
		\end{minipage}
	\end{center}
}


\newcommand{\inc}[1]{%
\textbf{include}( \getTitletodesc{#1} )%
}
\newcommand{\ext}[1]{%
\newline
<\getTitletodesc{#1}>%
}

%%%%%%%%%%%%%%%%%%%%%%%%%%%%%%%%%%%%%%%%%%%%%%%%%%%%%%%%%%%%%%%%%%%%%%%%%%%%%%%%%%%%%%%%%%
%%%%%%%%%%%%%%%%%%%%%%%%%%%%%% VALUTAZIONE DEI RISCHI %%%%%%%%%%%%%%%%%%%%%%%%%%%%%%%%%%%%
%%%%%%%%%%%%%%%%%%%%%%%%%%%%%%%%%%%%%%%%%%%%%%%%%%%%%%%%%%%%%%%%%%%%%%%%%%%%%%%%%%%%%%%%%%

\newcounter{counterRK}

%Funzione per resettare i contatori al cambio di categoria
\newcommand{\resetCounterRK}{
\setcounter{counterRK}{0}%
}

%formattazione requisiti - non mette link
\newcommand{\rkSep}{.} 
\newcommand{\displayRK}[1]{\texttt{#1}}
\newcommand{\rk}[2]{\displayRK{#1\rkSep#2}}

%macro per gestire i requisiti, e sotto requisiti vari.
%Oltre a formattare il requisito con \req (è in pratica un wrapper di req), 
%fa in modo che anche la label contenga la giusta formattazione del requisito.
\newcommand{\formattaRK}{%    
	\protect\stepcounter{counterRK}%
	\rk{RK}{\arabic{counterRK}}%
}

%Tabella per rischi
%usage: {rkkey}{prob}{effetto}{desc}{mitig}{monit}{manag}
\newcommand{\rkTab}[7]{%
	\begin{center}
		\begin{minipage}[t]{\linewidth}
			\phantomsection\label{desc:#1}
			\begin{tabularx}{\textwidth}{ l  X }
				\toprule
					\multicolumn{2}{c}{\formattaTitoloTab{\getTitle{#1}}}  \\
				\cmidrule(l{\cmidrulekern}r{\cmidrulekern}){1-2}
					\formattaCampiTab{ID} & \getID{#1} \\ 
					\addlinespace[0.5em] 
					\formattaCampiTab{Probabilità} &  #2 \\		
					\addlinespace[0.5em]
					\formattaCampiTab{Effetti} &  #3 \\
					\addlinespace[0.5em]
					\formattaCampiTab{Descrizione} &  #4 \\
					\addlinespace[0.5em]
					\formattaCampiTab{Mitigation} &  #5 \\					
					\addlinespace[0.5em]
					\formattaCampiTab{Monitoring} &  #6 \\
					\addlinespace[0.5em]
					\formattaCampiTab{Management} &  #7 \\
				\bottomrule
			\end{tabularx}
		\end{minipage}
	\end{center}
}

%%%%%%%%%%%%%%%%%%%%%%%%%%%%%%%%%%%%%%%%%%%%%%%%%%%%%%%%%%%%%%%%%%%%%%%%%%%%%%%%%%%%%%%%%%
%%%%%%%%%%%%%%%%%%%%%%%%%%%%%%%%% STIMA DEI COSTI %%%%%%%%%%%%%%%%%%%%%%%%%%%%%%%%%%%%%%%%
%%%%%%%%%%%%%%%%%%%%%%%%%%%%%%%%%%%%%%%%%%%%%%%%%%%%%%%%%%%%%%%%%%%%%%%%%%%%%%%%%%%%%%%%%%

\newcommand{\rowIDTitle}[1]{%
	\getIDtodesc{#1} & \getTitle{#1}%
}

%%%%%%%%%%%%%%%%%%%%%%%%%%%%%%%%%%%%%%%%%%%%%%%%%%%%%%%%%%%%%%%%%%%%%%%%%%%%%%%%%%%%%%%%%%
%%%%%%%%%%%%%%%%%%%%%%%%%%%%%%%%%%%%% ATTIVITA %%%%%%%%%%%%%%%%%%%%%%%%%%%%%%%%%%%%%%%%%%%
%%%%%%%%%%%%%%%%%%%%%%%%%%%%%%%%%%%%%%%%%%%%%%%%%%%%%%%%%%%%%%%%%%%%%%%%%%%%%%%%%%%%%%%%%%

\newcounter{counterAT}

%Funzione per resettare i contatori al cambio di categoria
\newcommand{\resetCounterAT}{
\setcounter{counterAT}{0}%
}

%formattazione requisiti - non mette link
\newcommand{\attivitaSep}{.} 
\newcommand{\displayAT}[1]{\texttt{#1}}
\newcommand{\attivita}[2]{\displayAT{#1\attivitaSep#2}}

%macro per gestire i requisiti, e sotto requisiti vari.
%Oltre a formattare il requisito con \req (è in pratica un wrapper di req), 
%fa in modo che anche la label contenga la giusta formattazione del requisito.
\newcommand{\formattaAT}{%    
	\attivita{DA}{\arabic{counterAT}}%
}

\newcommand{\newAttivita}[3]{%
	\protect\stepcounter{counterAT}\newListItem{#1}{#2}{#3}
}


			

%%%%%%%%%%%%%%%%%%%%%%%%%%%%%%%%%%%%%%%%%%%%%%%%%%%%%%%%%%%%%%%%%%%%%%%%%%%%%%%%%%%%%%%%
%%%%%%%%%%%%%%%%%%%%%%%%%%%%%%%%%%%%% ATTIVITA %%%%%%%%%%%%%%%%%%%%%%%%%%%%%%%%%%%%%%%%%
%%%%%%%%%%%%%%%%%%%%%%%%%%%%%%%%%%%%%%%%%%%%%%%%%%%%%%%%%%%%%%%%%%%%%%%%%%%%%%%%%%%%%%%%


\newcounter{counterSEQ}

%Funzione per resettare i contatori al cambio di categoria
\newcommand{\resetCounterSEQ}{
\setcounter{counterSEQ}{0}%
}

%formattazione requisiti - non mette link
\newcommand{\sequenzaSep}{.} 
\newcommand{\displaySEQ}[1]{\texttt{#1}}
\newcommand{\sequenza}[2]{\displaySEQ{#1\sequenzaSep#2}}

%macro per gestire i requisiti, e sotto requisiti vari.
%Oltre a formattare il requisito con \req (è in pratica un wrapper di req), 
%fa in modo che anche la label contenga la giusta formattazione del requisito.
\newcommand{\formattaSEQ}{%    
	\sequenza{SEQ}{\arabic{counterSEQ}}%
}

\newcommand{\newSequenza}[3]{%
	\protect\stepcounter{counterSEQ}\newListItem{#1}{#2}{#3}
}

%%%%%%%%%%%%%%%%%%%%%%%%%%%%%%%%%%%%%%%%%%%%%%%%%%%%%%%%%%%%%%%%%%%%%%%%%%%%%%%%%%%%%%%%
%%%%%%%%%%%%%%%%%%%%%%%%%%%%%%%%%%%%  AMBIENTI %%%%%%%%%%%%%%%%%%%%%%%%%%%%%%%%%%%%%%%%%
%%%%%%%%%%%%%%%%%%%%%%%%%%%%%%%%%%%%%%%%%%%%%%%%%%%%%%%%%%%%%%%%%%%%%%%%%%%%%%%%%%%%%%%%

%toglie il bold dalle label delle descrizioni
\renewcommand{\descriptionlabel}[1]{\textsf{#1}}

%lista descrption dove c'è una leggera indentazione.
\newenvironment{descriptionInd}{\begin{description}[leftmargin=1.5cm,labelindent=1cm]}{\end{description}}

\newenvironment{descriptionIndLabel}[1]{\begin{description}[leftmargin=1.5cm,labelindent=1cm,ref=#1]}{\end{description}}


%lista enumerate dove c'è una leggera indentazione.
\newenvironment{enumerateInd}[1]{\begin{enumerate}[leftmargin=1.5cm,labelindent=1cm,label=#1]}{\end{enumerate}}

%lista enumerate dove c'è una leggera indentazione.
\newenvironment{enumerateIndLabel}[2]{\begin{enumerate}[leftmargin=1.5cm,labelindent=1cm,label=#1,ref=#2]}{\end{enumerate}}


%lista descrption dove la descrizione va a capo.
\newenvironment{descriptionNext}{\begin{description}[style=nextline]}{\end{description}}
        %signatura: \newenvironment{name}[num]{before}{after}

%lista numerata inseribile dentro tabelle
\newenvironment{enumWork}
{
	\noindent
	\begin{minipage}[t]{\linewidth}
		\begin{enumerate*}[itemjoin={\newline}]
		}
		{
		\end{enumerate*}
	\end{minipage}
}

%lista itemize inseribile dentro tabelle
\newenvironment{itemWork}
{
	\noindent
	\begin{minipage}[t]{\linewidth}
		\begin{itemize}[nosep, leftmargin=*]
		}
		{
		\end{itemize}
	\end{minipage}
}

\newenvironment{enumCU}
{
	\noindent
	\begin{minipage}[t]{\linewidth}
		\begin{enumerate}[nosep, leftmargin=*]
		}
		{
		\end{enumerate}
	\end{minipage}
}

%%%%%%%%%%%%%%%%%%%%%%%%%%%%%%%%%%%%%%%%%%%%%%%%%%%%%%%%%%%%%%%%%%%%%%%%%%%%%%%%%%%%
%%%%%%%%%%%%%%%%%%%%%%%%%%%%%%%%%%%% MATEMATICA %%%%%%%%%%%%%%%%%%%%%%%%%%%%%%%%%%%%
%%%%%%%%%%%%%%%%%%%%%%%%%%%%%%%%%%%%%%%%%%%%%%%%%%%%%%%%%%%%%%%%%%%%%%%%%%%%%%%%%%%%

%insiemi numerici
\newcommand{\numberset}{\mathbb}
\newcommand{\N}{\numberset{N}}
\newcommand{\R}{\numberset{R}}

%lettere strane
\renewcommand{\epsilon}{\varepsilon}
\renewcommand{\theta}{\vartheta}
\renewcommand{\rho}{\varrho}
\renewcommand{\phi}{\varphi}
