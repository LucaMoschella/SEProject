%IMPORT PACKAGE & INIZIALIZZAZIONI & PERSONALIZZAZIONI PACKAGE- l'ordine ha importanza
%Compatibili input, output e lingua italiana
\usepackage[T1]{fontenc}  					
\usepackage[utf8]{inputenc}					
\usepackage[italian]{babel}					

%Miglioramenti visivi - sempre necessario
\usepackage{microtype} 						

%Riferimenti intelligenti
\usepackage[italian]{varioref} 				

%Gestione liste avanzata e semplificata
\usepackage[inline]{enumitem} 			

%spazio vericale invece di indentazione nei paragrafi
%\usepackage{parskip} 						

%Gestione matematica
\usepackage{amsmath, amsthm, amssymb} 	

%gestione indice analitico
\usepackage{makeidx}						
\makeindex

%Gestione inserimente di codice
\usepackage{listings}

%Gestione tabelle avanzata
\usepackage{booktabs}					
\usepackage{tabularx}
\usepackage{multicol}

%gestione delle immagini e dei colori
\usepackage{xcolor}
\usepackage{graphicx} 	
					%Inserire le immagini  dim relative a \textwidth
					%es: 0.8\textwidth
				
%Gestione figura con sottofigura
\usepackage{subfig}		

%Gestione avvolgimento figure dal testo
\usepackage{wrapfig} 	

%Gestione didascalie avanzata
\usepackage{caption}
\captionsetup{tableposition=top,figureposition=bottom,font=small,format=hang,labelfont={sf,bf}}

%Gestione didascalie a lato di una figura
\usepackage{sidecap}

%Gestione date
\usepackage{datetime}

%Miglioramento quoting						
\usepackage{quoting}
\quotingsetup{font=small}

%Permette i commenti
\usepackage{comment}

%Gestione grafici
%\usepackage{pgfplots}
%\pgfplotsset{/pgf/number format/use comma,compat=newest}

%Gestione unità di misura
\usepackage[output-decimal-marker={,}]{siunitx}
					%Per inserire numeri va usato \num{...}
	
%Gestione bibliografia e citazioni
\usepackage[autostyle,italian=guillemets]{csquotes}
\usepackage[backend=biber]{biblatex}
\addbibresource{bibliografia/bibliografia.bib}	%estensione necessaria (unico caso!)

%Gestione dei link
\usepackage{hyperref} 						

%Crea il glossario ed imposta lo stile - dopo hyperref
\usepackage[xindy,toc]{glossaries} 	
\makeglossaries
\setglossarystyle{altlist}
%Glossario
\newglossaryentry{anteprima}
{
    name={Anteprima},
    description={è una vista ridotta di un Contenuto. A seconda del tipo di contenuto cambia ciò che viene visualizzato},
    plural=Anteprime
}


\newglossaryentry{contenuto}
{
    name={Contenuto},
    description={è qualsiasi oggetto inserito nel sistema dagli utilizzatori dello stesso}
}

\newglossaryentry{riferimento}
{
    name={Riferimento},
    description={è una porzione di testo, un immagine o un elemento composto che contiene un link ad un altro oggetto presente nel sistema}
}

\newglossaryentry{cmsdef}
{
    name={Content management system},
    description={è uno strumento software, installato su un server web, il cui
        compito è facilitare la gestione dei contenuti di siti web, svincolando il
        webmaster da conoscenze tecniche specifiche di programmazione Web}
}


\newglossaryentry{account}
{
    name={Account},
    description={individua un singolo profilo all'interno del sistema, si assume che ogni utilizzatore abbia un singolo account. Un account ha dei privilegi di base, che gli permettono di accedere ad alcune funzionalità; l'Amministratore può aggiungere o rimuovere privilegi da un account}
}

\newglossaryentry{visitatore}
{
    name={Visitatore},
    description={è una Figura Pubblica non autenticata nel portale}
}

\newglossaryentry{figuraPubblica}
{
    name={Figura Pubblica},
    description={è il fruitore dei servizi del portale: Visitatore, Utente oppure Produttore}
}

\newglossaryentry{figuraAmministrativa}
{
    name={Figure Amministrativa},
    description={è colui che utilizza il portale con lo scopo di fornire servizi alle Figure Pubbliche}
}

\newglossaryentry{figuraPubblicaAutenticata}
{
    name={Figura Pubblica Autenticata},
    description={è una Figura Pubblica che ha effettuato il login nel portale: Utente oppure Produttore}
}

\newglossaryentry{produttore}
{
    name={Produttore},
    description={è una Figura Pubblica Autenticata. Modella un negozio o produttore di dolci a base di cioccolata (sia a livello industriale sia a livello artigianale) che ha una propria vetrina per l'esposizione}
}

\newglossaryentry{utente}
{
    name={Utente},
    description={è una Figura Pubblica Autenticata. Modella una persona interessata ai dolci a base di cioccolata}
}

\newglossaryentry{redattore}
{
    name={Redattore},
    description={è una Figura Amministrativa addetta alla gestione delle notizie presenti nel portale}
}

\newglossaryentry{assistente}
{
    name={Assistente},
    description={è una Figura Amministrativa addetta alla comunicazione con le Figure Pubbliche}
}

\newglossaryentry{moderatore}
{
    name={Moderatore},
    description={è una Figura Amministrativa addetta al controllo dei contenuti inseriti dalle Figure Pubbliche, garantendo la conformità degli stessi al regolamento della piattaforma}
}

\newglossaryentry{amministratore}
{
    name={Amministratore},
    description={è una Figura Amministrativa addetta alla gestione del portale e delle sue figure}
}

\newglossaryentry{follower}
{
    name={Follower},
    description={di \emph{X}, sono le figure pubblice autenticate che ricevono aggiornamenti da \emph{X}}
}

\newglossaryentry{followed}
{
    name={Followed},
    description={di \emph{X}, sono le figure pubblice autenticate dalle quali \emph{X} riceve aggiornamenti}
}

\newglossaryentry{vetrina}
{
    name={Vetrina},
    description={è uno spazio privato di ogni Produttore dedicato all'esposizione dei suoi prodotti}
}

\newglossaryentry{valutazione}
{
    name={Valutazione},
    description={di un prodotto è un voto in una scala limitata per esprimere la bontà dello stesso}
}

\newglossaryentry{giudizio}
{
    name={Giudizio},
    description={di una recensione è un voto che può essere positivo o negativo}
}

\newglossaryentry{recensione}
{
    name={Recensione},
    description={di un prodotto è una valutazione (obbligatoria) a cui è associata una descrizione}
}

\newglossaryentry{prodotto}
{
    name={Prodotto},
    description={indica un dolce a base di cioccolata}
}

\newglossaryentry{notizia}
{
    name={Notizia},
    description={indica un articolo o una novità proveniente dal mondo dolciario inserita da un Redattore}
}

%Acronimi
\newacronym{cms}{CMS}{Content Management System}

\newacronym{uml}{UML}{Unified Modeling Language}

\newacronym{rup}{RUP}{Rational Unified Process}

\newacronym{rmmm}{RMMM}{Risk Mitigation, Monitoring, Management}

\newacronym{ucp}{UCP}{Use Case Points}

\newacronym{api}{API}{Application Programming Interface}

\newacronym{gui}{GUI}{Graphical User Interface}

\newacronym{uaw}{UAW}{Unadjusted Actor Weight}
\newacronym{uucw}{UUCW}{Unadjusted Use Case Weight}
\newacronym{tcf}{TCF}{Technical Complexity Factor}
\newacronym{tf}{TF}{Technical Factor}
\newacronym{ecf}{ECF}{Environmental Complexity Factor}
\newacronym{ef}{EF}{Environmental Factor}
\newacronym{ee}{EE}{Estimated Effort}

\newacronym{crc}{CRC}{Classe Responsabilità Collaborazione}
\newacronym{ecb}{ECB}{Entity Control Boundary}
	%importiamo il glossario, va usato con \gls{<label>}


%ULTERIORI PERSONALIZZAZIONI

%cambia Chapter in Documento
\addto\captionsitalian{\renewcommand{\chaptername}{Documento}}

%cambia Bibliografia in Opere consultate - al maledetto non piace il comando sopra -___-
\DefineBibliographyStrings{italian}{%
	bibliography = {Opere consultate},
}


%toglie lo spazio allungato dopo i punti di fine frase.
\frenchspacing 

%non stiracchia la pagina per riempirla
\raggedbottom

%toglie la parola capitolo ad ogni capitolo
%\titleformat{\chapter}{\normalfont\bfseries\huge}{\thechapter.}{15pt}{\normalfont\bfseries\huge} %toglie la scritta "Capitolo N
