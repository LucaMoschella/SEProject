%IMPORT PACKAGE & INIZIALIZZAZIONI & PERSONALIZZAZIONI PACKAGE- l'ordine ha importanza
%Compatibili input, output e lingua italiana
\usepackage[T1]{fontenc}  					
\usepackage[utf8]{inputenc}					
\usepackage[italian]{babel}					

%Miglioramenti visivi
\usepackage{microtype} 						

%Riferimenti intelligenti
\usepackage[italian]{varioref} 				

%Gestione liste avanzata e semplificata
\usepackage[inline]{enumitem} 			

%spazio vericale invece di indentazione nei paragrafi
%\usepackage{parskip} 						

%Gestione matematica
\usepackage{amsmath, amsthm, amssymb} 	

%gestione indice analitico
\usepackage{makeidx}						
\makeindex

%Gestione inserimente di codice
\usepackage{listings}

%Gestione tabelle avanzata
\usepackage{booktabs}					
\usepackage{tabularx}
\usepackage{multicol}

%gestione delle immagini e dei colori
\usepackage{xcolor}
\usepackage{graphicx} 	
					%Inserire le immagini con dim relative a \textwidth
					%es: 0.8\textwidth

%Gestione didascalie avanzara
\usepackage{caption}
\captionsetup{tableposition=top,figureposition=bottom,font=small, format=hang,labelfont={sf,bf}}

%Gestione date
\usepackage{datetime}

%Miglioramento quoting						
\usepackage{quoting}
\quotingsetup{font=small}

%Permette i commenti
\usepackage{comment}

%Gestione grafici
%\usepackage{pgfplots}
%\pgfplotsset{/pgf/number format/use comma,compat=newest}

%Gestione note a margine di una figura
\usepackage{sidecap}

%Gestione figura con sottofigura
\usepackage{subfig}		

%Gestione avvolgimento figure dal testo
\usepackage{wrapfig} 	 

%Gestione unità di misura
\usepackage[output-decimal-marker={,}]{siunitx}
					%Per inserire numeri va usato \num{...}
	
%Gestione bibliografia e citazioni
\usepackage[autostyle,italian=guillemets]{csquotes}
\usepackage[backend=biber]{biblatex}
\addbibresource{bibliografia/bibliografia.bib}	

%Gestione dei link
\usepackage{hyperref} 						

%Crea il glossario ed imposta lo stile - dopo hyperref
\usepackage[xindy,toc]{glossaries} 	
\makeglossaries
\setglossarystyle{altlist}



%ULTERIORI PERSONALIZZAZIONI

%cambia Chapter in Documento
\addto\captionsitalian{\renewcommand{\chaptername}{Documento}}

%toglie lo spazio allungato dopo i punti di fine frase.
\frenchspacing 

%non stiracchia la pagina per riempirla
\raggedbottom

%toglie la parola capitolo ad ogni capitolo
%\titleformat{\chapter}{\normalfont\bfseries\huge}{\thechapter.}{15pt}{\normalfont\bfseries\huge} %toglie la scritta "Capitolo N
